\book{The Adventures of Huckleberry Fin}

ADVENTURES OF HUCKLEBERRY FINN
By Mark Twain

NOTICE

PERSONS attempting to find a motive in this narrative will be prosecuted;
persons attempting to find a moral in it will be banished; persons
attempting to find a plot in it will be shot.

BY ORDER OF THE AUTHOR, Per G.G., Chief of Ordnance.


EXPLANATORY

IN this book a number of dialects are used, to wit:  the Missouri negro
dialect; the extremest form of the backwoods Southwestern dialect; the
ordinary "Pike County" dialect; and four modified varieties of this last.
The shadings have not been done in a haphazard fashion, or by guesswork;
but painstakingly, and with the trustworthy guidance and support of
personal familiarity with these several forms of speech.

I make this explanation for the reason that without it many readers would
suppose that all these characters were trying to talk alike and not
succeeding.

THE AUTHOR.

ADVENTURES OF HUCKLEBERRY FINN

Scene:  The Mississippi Valley Time:  Forty to fifty years ago

CHAPTER I.

YOU don't know about me without you have read a book by the name of The
Adventures of Tom Sawyer; but that ain't no matter.  That book was made
by Mr. Mark Twain, and he told the truth, mainly.  There was things which
he stretched, but mainly he told the truth.  That is nothing.  I never
seen anybody but lied one time or another, without it was Aunt Polly, or
the widow, or maybe Mary.  Aunt Polly--Tom's Aunt Polly, she is--and
Mary, and the Widow Douglas is all told about in that book, which is
mostly a true book, with some stretchers, as I said before.

Now the way that the book winds up is this:  Tom and me found the money
that the robbers hid in the cave, and it made us rich.  We got six
thousand dollars apiece--all gold.  It was an awful sight of money when
it was piled up.  Well, Judge Thatcher he took it and put it out at
interest, and it fetched us a dollar a day apiece all the year round
--more than a body could tell what to do with.  The Widow Douglas she took
me for her son, and allowed she would sivilize me; but it was rough
living in the house all the time, considering how dismal regular and
decent the widow was in all her ways; and so when I couldn't stand it no
longer I lit out.  I got into my old rags and my sugar-hogshead again,
and was free and satisfied.  But Tom Sawyer he hunted me up and said he
was going to start a band of robbers, and I might join if I would go back
to the widow and be respectable.  So I went back.

The widow she cried over me, and called me a poor lost lamb, and she
called me a lot of other names, too, but she never meant no harm by it.
She put me in them new clothes again, and I couldn't do nothing but sweat
and sweat, and feel all cramped up.  Well, then, the old thing commenced
again.  The widow rung a bell for supper, and you had to come to time.
When you got to the table you couldn't go right to eating, but you had to
wait for the widow to tuck down her head and grumble a little over the
victuals, though there warn't really anything the matter with them,--that
is, nothing only everything was cooked by itself.  In a barrel of odds
and ends it is different; things get mixed up, and the juice kind of
swaps around, and the things go better.

After supper she got out her book and learned me about Moses and the
Bulrushers, and I was in a sweat to find out all about him; but by and by
she let it out that Moses had been dead a considerable long time; so then
I didn't care no more about him, because I don't take no stock in dead
people.

Pretty soon I wanted to smoke, and asked the widow to let me.  But she
wouldn't.  She said it was a mean practice and wasn't clean, and I must
try to not do it any more.  That is just the way with some people.  They
get down on a thing when they don't know nothing about it.  Here she was
a-bothering about Moses, which was no kin to her, and no use to anybody,
being gone, you see, yet finding a power of fault with me for doing a
thing that had some good in it.  And she took snuff, too; of course that
was all right, because she done it herself.

Her sister, Miss Watson, a tolerable slim old maid, with goggles on,
had just come to live with her, and took a set at me now with a
spelling-book. She worked me middling hard for about an hour, and then
the widow made her ease up.  I couldn't stood it much longer.  Then for
an hour it was deadly dull, and I was fidgety.  Miss Watson would say,
"Don't put your feet up there, Huckleberry;" and "Don't scrunch up like
that, Huckleberry--set up straight;" and pretty soon she would say,
"Don't gap and stretch like that, Huckleberry--why don't you try to
behave?"  Then she told me all about the bad place, and I said I wished I
was there. She got mad then, but I didn't mean no harm.  All I wanted was
to go somewheres; all I wanted was a change, I warn't particular.  She
said it was wicked to say what I said; said she wouldn't say it for the
whole world; she was going to live so as to go to the good place.  Well,
I couldn't see no advantage in going where she was going, so I made up my
mind I wouldn't try for it.  But I never said so, because it would only
make trouble, and wouldn't do no good.

Now she had got a start, and she went on and told me all about the good
place.  She said all a body would have to do there was to go around all
day long with a harp and sing, forever and ever.  So I didn't think much
of it. But I never said so.  I asked her if she reckoned Tom Sawyer would
go there, and she said not by a considerable sight.  I was glad about
that, because I wanted him and me to be together.

Miss Watson she kept pecking at me, and it got tiresome and lonesome.  By
and by they fetched the niggers in and had prayers, and then everybody
was off to bed.  I went up to my room with a piece of candle, and put it
on the table.  Then I set down in a chair by the window and tried to
think of something cheerful, but it warn't no use.  I felt so lonesome I
most wished I was dead.  The stars were shining, and the leaves rustled
in the woods ever so mournful; and I heard an owl, away off, who-whooing
about somebody that was dead, and a whippowill and a dog crying about
somebody that was going to die; and the wind was trying to whisper
something to me, and I couldn't make out what it was, and so it made the
cold shivers run over me. Then away out in the woods I heard that kind of
a sound that a ghost makes when it wants to tell about something that's
on its mind and can't make itself understood, and so can't rest easy in
its grave, and has to go about that way every night grieving.  I got so
down-hearted and scared I did wish I had some company.  Pretty soon a
spider went crawling up my shoulder, and I flipped it off and it lit in
the candle; and before I could budge it was all shriveled up.  I didn't
need anybody to tell me that that was an awful bad sign and would fetch
me some bad luck, so I was scared and most shook the clothes off of me.
I got up and turned around in my tracks three times and crossed my breast
every time; and then I tied up a little lock of my hair with a thread to
keep witches away.  But I hadn't no confidence.  You do that when you've
lost a horseshoe that you've found, instead of nailing it up over the
door, but I hadn't ever heard anybody say it was any way to keep off bad
luck when you'd killed a spider.

I set down again, a-shaking all over, and got out my pipe for a smoke;
for the house was all as still as death now, and so the widow wouldn't
know. Well, after a long time I heard the clock away off in the town go
boom--boom--boom--twelve licks; and all still again--stiller than ever.
Pretty soon I heard a twig snap down in the dark amongst the trees
--something was a stirring.  I set still and listened.  Directly I could
just barely hear a "me-yow! me-yow!" down there.  That was good!  Says I,
"me-yow! me-yow!" as soft as I could, and then I put out the light and
scrambled out of the window on to the shed.  Then I slipped down to the
ground and crawled in among the trees, and, sure enough, there was Tom
Sawyer waiting for me.




CHAPTER II.

WE went tiptoeing along a path amongst the trees back towards the end of
the widow's garden, stooping down so as the branches wouldn't scrape our
heads. When we was passing by the kitchen I fell over a root and made a
noise.  We scrouched down and laid still.  Miss Watson's big nigger,
named Jim, was setting in the kitchen door; we could see him pretty
clear, because there was a light behind him.  He got up and stretched his
neck out about a minute, listening.  Then he says:

"Who dah?"

He listened some more; then he come tiptoeing down and stood right
between us; we could a touched him, nearly.  Well, likely it was minutes
and minutes that there warn't a sound, and we all there so close
together.  There was a place on my ankle that got to itching, but I
dasn't scratch it; and then my ear begun to itch; and next my back, right
between my shoulders.  Seemed like I'd die if I couldn't scratch.  Well,
I've noticed that thing plenty times since.  If you are with the quality,
or at a funeral, or trying to go to sleep when you ain't sleepy--if you
are anywheres where it won't do for you to scratch, why you will itch all
over in upwards of a thousand places. Pretty soon Jim says:

"Say, who is you?  Whar is you?  Dog my cats ef I didn' hear sumf'n.
Well, I know what I's gwyne to do:  I's gwyne to set down here and listen
tell I hears it agin."

So he set down on the ground betwixt me and Tom.  He leaned his back up
against a tree, and stretched his legs out till one of them most touched
one of mine.  My nose begun to itch.  It itched till the tears come into
my eyes.  But I dasn't scratch.  Then it begun to itch on the inside.
Next I got to itching underneath.  I didn't know how I was going to set
still. This miserableness went on as much as six or seven minutes; but it
seemed a sight longer than that.  I was itching in eleven different
places now.  I reckoned I couldn't stand it more'n a minute longer, but I
set my teeth hard and got ready to try.  Just then Jim begun to breathe
heavy; next he begun to snore--and then I was pretty soon comfortable
again.

Tom he made a sign to me--kind of a little noise with his mouth--and we
went creeping away on our hands and knees.  When we was ten foot off Tom
whispered to me, and wanted to tie Jim to the tree for fun.  But I said
no; he might wake and make a disturbance, and then they'd find out I
warn't in. Then Tom said he hadn't got candles enough, and he would slip
in the kitchen and get some more.  I didn't want him to try.  I said Jim
might wake up and come.  But Tom wanted to resk it; so we slid in there
and got three candles, and Tom laid five cents on the table for pay.
Then we got out, and I was in a sweat to get away; but nothing would do
Tom but he must crawl to where Jim was, on his hands and knees, and play
something on him.  I waited, and it seemed a good while, everything was
so still and lonesome.

As soon as Tom was back we cut along the path, around the garden fence,
and by and by fetched up on the steep top of the hill the other side of
the house.  Tom said he slipped Jim's hat off of his head and hung it on
a limb right over him, and Jim stirred a little, but he didn't wake.
Afterwards Jim said the witches be witched him and put him in a trance,
and rode him all over the State, and then set him under the trees again,
and hung his hat on a limb to show who done it.  And next time Jim told
it he said they rode him down to New Orleans; and, after that, every time
he told it he spread it more and more, till by and by he said they rode
him all over the world, and tired him most to death, and his back was all
over saddle-boils.  Jim was monstrous proud about it, and he got so he
wouldn't hardly notice the other niggers.  Niggers would come miles to
hear Jim tell about it, and he was more looked up to than any nigger in
that country.  Strange niggers would stand with their mouths open and
look him all over, same as if he was a wonder.  Niggers is always talking
about witches in the dark by the kitchen fire; but whenever one was
talking and letting on to know all about such things, Jim would happen in
and say, "Hm!  What you know 'bout witches?" and that nigger was corked
up and had to take a back seat.  Jim always kept that five-center piece
round his neck with a string, and said it was a charm the devil give to
him with his own hands, and told him he could cure anybody with it and
fetch witches whenever he wanted to just by saying something to it; but
he never told what it was he said to it.  Niggers would come from all
around there and give Jim anything they had, just for a sight of that
five-center piece; but they wouldn't touch it, because the devil had had
his hands on it.  Jim was most ruined for a servant, because he got stuck
up on account of having seen the devil and been rode by witches.

Well, when Tom and me got to the edge of the hilltop we looked away down
into the village and could see three or four lights twinkling, where
there was sick folks, maybe; and the stars over us was sparkling ever so
fine; and down by the village was the river, a whole mile broad, and
awful still and grand.  We went down the hill and found Jo Harper and Ben
Rogers, and two or three more of the boys, hid in the old tanyard.  So we
unhitched a skiff and pulled down the river two mile and a half, to the
big scar on the hillside, and went ashore.

We went to a clump of bushes, and Tom made everybody swear to keep the
secret, and then showed them a hole in the hill, right in the thickest
part of the bushes.  Then we lit the candles, and crawled in on our hands
and knees.  We went about two hundred yards, and then the cave opened up.
Tom poked about amongst the passages, and pretty soon ducked under a wall
where you wouldn't a noticed that there was a hole.  We went along a
narrow place and got into a kind of room, all damp and sweaty and cold,
and there we stopped.  Tom says:

"Now, we'll start this band of robbers and call it Tom Sawyer's Gang.
Everybody that wants to join has got to take an oath, and write his name
in blood."

Everybody was willing.  So Tom got out a sheet of paper that he had wrote
the oath on, and read it.  It swore every boy to stick to the band, and
never tell any of the secrets; and if anybody done anything to any boy in
the band, whichever boy was ordered to kill that person and his family
must do it, and he mustn't eat and he mustn't sleep till he had killed
them and hacked a cross in their breasts, which was the sign of the band.
And nobody that didn't belong to the band could use that mark, and if he
did he must be sued; and if he done it again he must be killed.  And if
anybody that belonged to the band told the secrets, he must have his
throat cut, and then have his carcass burnt up and the ashes scattered
all around, and his name blotted off of the list with blood and never
mentioned again by the gang, but have a curse put on it and be forgot
forever.

Everybody said it was a real beautiful oath, and asked Tom if he got it
out of his own head.  He said, some of it, but the rest was out of
pirate-books and robber-books, and every gang that was high-toned had it.

Some thought it would be good to kill the FAMILIES of boys that told the
secrets.  Tom said it was a good idea, so he took a pencil and wrote it
in. Then Ben Rogers says:

"Here's Huck Finn, he hain't got no family; what you going to do 'bout
him?"

"Well, hain't he got a father?" says Tom Sawyer.

"Yes, he's got a father, but you can't never find him these days.  He
used to lay drunk with the hogs in the tanyard, but he hain't been seen
in these parts for a year or more."

They talked it over, and they was going to rule me out, because they said
every boy must have a family or somebody to kill, or else it wouldn't be
fair and square for the others.  Well, nobody could think of anything to
do--everybody was stumped, and set still.  I was most ready to cry; but
all at once I thought of a way, and so I offered them Miss Watson--they
could kill her.  Everybody said:

"Oh, she'll do.  That's all right.  Huck can come in."

Then they all stuck a pin in their fingers to get blood to sign with, and
I made my mark on the paper.

"Now," says Ben Rogers, "what's the line of business of this Gang?"

"Nothing only robbery and murder," Tom said.

"But who are we going to rob?--houses, or cattle, or--"

"Stuff! stealing cattle and such things ain't robbery; it's burglary,"
says Tom Sawyer.  "We ain't burglars.  That ain't no sort of style.  We
are highwaymen.  We stop stages and carriages on the road, with masks on,
and kill the people and take their watches and money."

"Must we always kill the people?"

"Oh, certainly.  It's best.  Some authorities think different, but mostly
it's considered best to kill them--except some that you bring to the cave
here, and keep them till they're ransomed."

"Ransomed?  What's that?"

"I don't know.  But that's what they do.  I've seen it in books; and so
of course that's what we've got to do."

"But how can we do it if we don't know what it is?"

"Why, blame it all, we've GOT to do it.  Don't I tell you it's in the
books?  Do you want to go to doing different from what's in the books,
and get things all muddled up?"

"Oh, that's all very fine to SAY, Tom Sawyer, but how in the nation are
these fellows going to be ransomed if we don't know how to do it to them?
--that's the thing I want to get at.  Now, what do you reckon it is?"

"Well, I don't know.  But per'aps if we keep them till they're ransomed,
it means that we keep them till they're dead."

"Now, that's something LIKE.  That'll answer.  Why couldn't you said that
before?  We'll keep them till they're ransomed to death; and a bothersome
lot they'll be, too--eating up everything, and always trying to get
loose."

"How you talk, Ben Rogers.  How can they get loose when there's a guard
over them, ready to shoot them down if they move a peg?"

"A guard!  Well, that IS good.  So somebody's got to set up all night and
never get any sleep, just so as to watch them.  I think that's
foolishness. Why can't a body take a club and ransom them as soon as they
get here?"

"Because it ain't in the books so--that's why.  Now, Ben Rogers, do you
want to do things regular, or don't you?--that's the idea.  Don't you
reckon that the people that made the books knows what's the correct thing
to do?  Do you reckon YOU can learn 'em anything?  Not by a good deal.
No, sir, we'll just go on and ransom them in the regular way."

"All right.  I don't mind; but I say it's a fool way, anyhow.  Say, do we
kill the women, too?"

"Well, Ben Rogers, if I was as ignorant as you I wouldn't let on.  Kill
the women?  No; nobody ever saw anything in the books like that.  You
fetch them to the cave, and you're always as polite as pie to them; and
by and by they fall in love with you, and never want to go home any
more."

"Well, if that's the way I'm agreed, but I don't take no stock in it.
Mighty soon we'll have the cave so cluttered up with women, and fellows
waiting to be ransomed, that there won't be no place for the robbers.
But go ahead, I ain't got nothing to say."

Little Tommy Barnes was asleep now, and when they waked him up he was
scared, and cried, and said he wanted to go home to his ma, and didn't
want to be a robber any more.

So they all made fun of him, and called him cry-baby, and that made him
mad, and he said he would go straight and tell all the secrets.  But Tom
give him five cents to keep quiet, and said we would all go home and meet
next week, and rob somebody and kill some people.

Ben Rogers said he couldn't get out much, only Sundays, and so he wanted
to begin next Sunday; but all the boys said it would be wicked to do it
on Sunday, and that settled the thing.  They agreed to get together and
fix a day as soon as they could, and then we elected Tom Sawyer first
captain and Jo Harper second captain of the Gang, and so started home.

I clumb up the shed and crept into my window just before day was
breaking. My new clothes was all greased up and clayey, and I was
dog-tired.




CHAPTER III.

WELL, I got a good going-over in the morning from old Miss Watson on
account of my clothes; but the widow she didn't scold, but only cleaned
off the grease and clay, and looked so sorry that I thought I would
behave awhile if I could.  Then Miss Watson she took me in the closet and
prayed, but nothing come of it.  She told me to pray every day, and
whatever I asked for I would get it.  But it warn't so.  I tried it.
Once I got a fish-line, but no hooks.  It warn't any good to me without
hooks.  I tried for the hooks three or four times, but somehow I couldn't
make it work.  By and by, one day, I asked Miss Watson to try for me, but
she said I was a fool.  She never told me why, and I couldn't make it out
no way.

I set down one time back in the woods, and had a long think about it.  I
says to myself, if a body can get anything they pray for, why don't
Deacon Winn get back the money he lost on pork?  Why can't the widow get
back her silver snuffbox that was stole?  Why can't Miss Watson fat up?
No, says I to my self, there ain't nothing in it.  I went and told the
widow about it, and she said the thing a body could get by praying for it
was "spiritual gifts."  This was too many for me, but she told me what
she meant--I must help other people, and do everything I could for other
people, and look out for them all the time, and never think about myself.
This was including Miss Watson, as I took it.  I went out in the woods
and turned it over in my mind a long time, but I couldn't see no
advantage about it--except for the other people; so at last I reckoned I
wouldn't worry about it any more, but just let it go.  Sometimes the
widow would take me one side and talk about Providence in a way to make a
body's mouth water; but maybe next day Miss Watson would take hold and
knock it all down again.  I judged I could see that there was two
Providences, and a poor chap would stand considerable show with the
widow's Providence, but if Miss Watson's got him there warn't no help for
him any more.  I thought it all out, and reckoned I would belong to the
widow's if he wanted me, though I couldn't make out how he was a-going to
be any better off then than what he was before, seeing I was so ignorant,
and so kind of low-down and ornery.

Pap he hadn't been seen for more than a year, and that was comfortable
for me; I didn't want to see him no more.  He used to always whale me
when he was sober and could get his hands on me; though I used to take to
the woods most of the time when he was around.  Well, about this time he
was found in the river drownded, about twelve mile above town, so people
said.  They judged it was him, anyway; said this drownded man was just
his size, and was ragged, and had uncommon long hair, which was all like
pap; but they couldn't make nothing out of the face, because it had been
in the water so long it warn't much like a face at all.  They said he was
floating on his back in the water.  They took him and buried him on the
bank.  But I warn't comfortable long, because I happened to think of
something.  I knowed mighty well that a drownded man don't float on his
back, but on his face.  So I knowed, then, that this warn't pap, but a
woman dressed up in a man's clothes.  So I was uncomfortable again.  I
judged the old man would turn up again by and by, though I wished he
wouldn't.

We played robber now and then about a month, and then I resigned.  All
the boys did.  We hadn't robbed nobody, hadn't killed any people, but
only just pretended.  We used to hop out of the woods and go charging
down on hog-drivers and women in carts taking garden stuff to market, but
we never hived any of them.  Tom Sawyer called the hogs "ingots," and he
called the turnips and stuff "julery," and we would go to the cave and
powwow over what we had done, and how many people we had killed and
marked.  But I couldn't see no profit in it.  One time Tom sent a boy to
run about town with a blazing stick, which he called a slogan (which was
the sign for the Gang to get together), and then he said he had got
secret news by his spies that next day a whole parcel of Spanish
merchants and rich A-rabs was going to camp in Cave Hollow with two
hundred elephants, and six hundred camels, and over a thousand "sumter"
mules, all loaded down with di'monds, and they didn't have only a guard
of four hundred soldiers, and so we would lay in ambuscade, as he called
it, and kill the lot and scoop the things.  He said we must slick up our
swords and guns, and get ready.  He never could go after even a
turnip-cart but he must have the swords and guns all scoured up for it,
though they was only lath and broomsticks, and you might scour at them
till you rotted, and then they warn't worth a mouthful of ashes more than
what they was before.  I didn't believe we could lick such a crowd of
Spaniards and A-rabs, but I wanted to see the camels and elephants, so I
was on hand next day, Saturday, in the ambuscade; and when we got the
word we rushed out of the woods and down the hill.  But there warn't no
Spaniards and A-rabs, and there warn't no camels nor no elephants.  It
warn't anything but a Sunday-school picnic, and only a primer-class at
that.  We busted it up, and chased the children up the hollow; but we
never got anything but some doughnuts and jam, though Ben Rogers got a
rag doll, and Jo Harper got a hymn-book and a tract; and then the teacher
charged in, and made us drop everything and cut.  I didn't see no
di'monds, and I told Tom Sawyer so.  He said there was loads of them
there, anyway; and he said there was A-rabs there, too, and elephants and
things.  I said, why couldn't we see them, then?  He said if I warn't so
ignorant, but had read a book called Don Quixote, I would know without
asking.  He said it was all done by enchantment.  He said there was
hundreds of soldiers there, and elephants and treasure, and so on, but we
had enemies which he called magicians; and they had turned the whole
thing into an infant Sunday-school, just out of spite.  I said, all
right; then the thing for us to do was to go for the magicians.  Tom
Sawyer said I was a numskull.

"Why," said he, "a magician could call up a lot of genies, and they would
hash you up like nothing before you could say Jack Robinson.  They are as
tall as a tree and as big around as a church."

"Well," I says, "s'pose we got some genies to help US--can't we lick the
other crowd then?"

"How you going to get them?"

"I don't know.  How do THEY get them?"

"Why, they rub an old tin lamp or an iron ring, and then the genies come
tearing in, with the thunder and lightning a-ripping around and the smoke
a-rolling, and everything they're told to do they up and do it.  They
don't think nothing of pulling a shot-tower up by the roots, and belting
a Sunday-school superintendent over the head with it--or any other man."

"Who makes them tear around so?"

"Why, whoever rubs the lamp or the ring.  They belong to whoever rubs the
lamp or the ring, and they've got to do whatever he says.  If he tells
them to build a palace forty miles long out of di'monds, and fill it full
of chewing-gum, or whatever you want, and fetch an emperor's daughter
from China for you to marry, they've got to do it--and they've got to do
it before sun-up next morning, too.  And more:  they've got to waltz that
palace around over the country wherever you want it, you understand."

"Well," says I, "I think they are a pack of flat-heads for not keeping
the palace themselves 'stead of fooling them away like that.  And what's
more--if I was one of them I would see a man in Jericho before I would
drop my business and come to him for the rubbing of an old tin lamp."

"How you talk, Huck Finn.  Why, you'd HAVE to come when he rubbed it,
whether you wanted to or not."

"What! and I as high as a tree and as big as a church?  All right, then;
I WOULD come; but I lay I'd make that man climb the highest tree there
was in the country."

"Shucks, it ain't no use to talk to you, Huck Finn.  You don't seem to
know anything, somehow--perfect saphead."

I thought all this over for two or three days, and then I reckoned I
would see if there was anything in it.  I got an old tin lamp and an iron
ring, and went out in the woods and rubbed and rubbed till I sweat like
an Injun, calculating to build a palace and sell it; but it warn't no
use, none of the genies come.  So then I judged that all that stuff was
only just one of Tom Sawyer's lies.  I reckoned he believed in the A-rabs
and the elephants, but as for me I think different.  It had all the marks
of a Sunday-school.




CHAPTER IV.

WELL, three or four months run along, and it was well into the winter
now. I had been to school most all the time and could spell and read and
write just a little, and could say the multiplication table up to six
times seven is thirty-five, and I don't reckon I could ever get any
further than that if I was to live forever.  I don't take no stock in
mathematics, anyway.

At first I hated the school, but by and by I got so I could stand it.
Whenever I got uncommon tired I played hookey, and the hiding I got next
day done me good and cheered me up.  So the longer I went to school the
easier it got to be.  I was getting sort of used to the widow's ways,
too, and they warn't so raspy on me.  Living in a house and sleeping in a
bed pulled on me pretty tight mostly, but before the cold weather I used
to slide out and sleep in the woods sometimes, and so that was a rest to
me.  I liked the old ways best, but I was getting so I liked the new
ones, too, a little bit. The widow said I was coming along slow but sure,
and doing very satisfactory.  She said she warn't ashamed of me.

One morning I happened to turn over the salt-cellar at breakfast.  I
reached for some of it as quick as I could to throw over my left shoulder
and keep off the bad luck, but Miss Watson was in ahead of me, and
crossed me off. She says, "Take your hands away, Huckleberry; what a mess
you are always making!"  The widow put in a good word for me, but that
warn't going to keep off the bad luck, I knowed that well enough.  I
started out, after breakfast, feeling worried and shaky, and wondering
where it was going to fall on me, and what it was going to be.  There is
ways to keep off some kinds of bad luck, but this wasn't one of them
kind; so I never tried to do anything, but just poked along low-spirited
and on the watch-out.

I went down to the front garden and clumb over the stile where you go
through the high board fence.  There was an inch of new snow on the
ground, and I seen somebody's tracks.  They had come up from the quarry
and stood around the stile a while, and then went on around the garden
fence.  It was funny they hadn't come in, after standing around so.  I
couldn't make it out.  It was very curious, somehow.  I was going to
follow around, but I stooped down to look at the tracks first.  I didn't
notice anything at first, but next I did.  There was a cross in the left
boot-heel made with big nails, to keep off the devil.

I was up in a second and shinning down the hill.  I looked over my
shoulder every now and then, but I didn't see nobody.  I was at Judge
Thatcher's as quick as I could get there.  He said:

"Why, my boy, you are all out of breath.  Did you come for your
interest?"

"No, sir," I says; "is there some for me?"

"Oh, yes, a half-yearly is in last night--over a hundred and fifty
dollars.  Quite a fortune for you.  You had better let me invest it along
with your six thousand, because if you take it you'll spend it."

"No, sir," I says, "I don't want to spend it.  I don't want it at all
--nor the six thousand, nuther.  I want you to take it; I want to give it
to you--the six thousand and all."

He looked surprised.  He couldn't seem to make it out.  He says:

"Why, what can you mean, my boy?"

I says, "Don't you ask me no questions about it, please.  You'll take it
--won't you?"

He says:

"Well, I'm puzzled.  Is something the matter?"

"Please take it," says I, "and don't ask me nothing--then I won't have to
tell no lies."

He studied a while, and then he says:

"Oho-o!  I think I see.  You want to SELL all your property to me--not
give it.  That's the correct idea."

Then he wrote something on a paper and read it over, and says:

"There; you see it says 'for a consideration.'  That means I have bought
it of you and paid you for it.  Here's a dollar for you.  Now you sign
it."

So I signed it, and left.

Miss Watson's nigger, Jim, had a hair-ball as big as your fist, which had
been took out of the fourth stomach of an ox, and he used to do magic
with it.  He said there was a spirit inside of it, and it knowed
everything.  So I went to him that night and told him pap was here again,
for I found his tracks in the snow.  What I wanted to know was, what he
was going to do, and was he going to stay?  Jim got out his hair-ball and
said something over it, and then he held it up and dropped it on the
floor.  It fell pretty solid, and only rolled about an inch.  Jim tried
it again, and then another time, and it acted just the same.  Jim got
down on his knees, and put his ear against it and listened.  But it
warn't no use; he said it wouldn't talk. He said sometimes it wouldn't
talk without money.  I told him I had an old slick counterfeit quarter
that warn't no good because the brass showed through the silver a little,
and it wouldn't pass nohow, even if the brass didn't show, because it was
so slick it felt greasy, and so that would tell on it every time.  (I
reckoned I wouldn't say nothing about the dollar I got from the judge.) I
said it was pretty bad money, but maybe the hair-ball would take it,
because maybe it wouldn't know the difference.  Jim smelt it and bit it
and rubbed it, and said he would manage so the hair-ball would think it
was good.  He said he would split open a raw Irish potato and stick the
quarter in between and keep it there all night, and next morning you
couldn't see no brass, and it wouldn't feel greasy no more, and so
anybody in town would take it in a minute, let alone a hair-ball.  Well,
I knowed a potato would do that before, but I had forgot it.

Jim put the quarter under the hair-ball, and got down and listened again.
This time he said the hair-ball was all right.  He said it would tell my
whole fortune if I wanted it to.  I says, go on.  So the hair-ball talked
to Jim, and Jim told it to me.  He says:

"Yo' ole father doan' know yit what he's a-gwyne to do.  Sometimes he
spec he'll go 'way, en den agin he spec he'll stay.  De bes' way is to
res' easy en let de ole man take his own way.  Dey's two angels hoverin'
roun' 'bout him.  One uv 'em is white en shiny, en t'other one is black.
De white one gits him to go right a little while, den de black one sail
in en bust it all up.  A body can't tell yit which one gwyne to fetch him
at de las'.  But you is all right.  You gwyne to have considable trouble
in yo' life, en considable joy.  Sometimes you gwyne to git hurt, en
sometimes you gwyne to git sick; but every time you's gwyne to git well
agin.  Dey's two gals flyin' 'bout you in yo' life.  One uv 'em's light
en t'other one is dark. One is rich en t'other is po'.  You's gwyne to
marry de po' one fust en de rich one by en by.  You wants to keep 'way
fum de water as much as you kin, en don't run no resk, 'kase it's down in
de bills dat you's gwyne to git hung."

When I lit my candle and went up to my room that night there sat pap--his
own self!




CHAPTER V.

I HAD shut the door to.  Then I turned around and there he was.  I used
to be scared of him all the time, he tanned me so much.  I reckoned I was
scared now, too; but in a minute I see I was mistaken--that is, after the
first jolt, as you may say, when my breath sort of hitched, he being so
unexpected; but right away after I see I warn't scared of him worth
bothring about.

He was most fifty, and he looked it.  His hair was long and tangled and
greasy, and hung down, and you could see his eyes shining through like he
was behind vines.  It was all black, no gray; so was his long, mixed-up
whiskers.  There warn't no color in his face, where his face showed; it
was white; not like another man's white, but a white to make a body sick,
a white to make a body's flesh crawl--a tree-toad white, a fish-belly
white.  As for his clothes--just rags, that was all.  He had one ankle
resting on t'other knee; the boot on that foot was busted, and two of his
toes stuck through, and he worked them now and then.  His hat was laying
on the floor--an old black slouch with the top caved in, like a lid.

I stood a-looking at him; he set there a-looking at me, with his chair
tilted back a little.  I set the candle down.  I noticed the window was
up; so he had clumb in by the shed.  He kept a-looking me all over.  By
and by he says:

"Starchy clothes--very.  You think you're a good deal of a big-bug, DON'T
you?"

"Maybe I am, maybe I ain't," I says.

"Don't you give me none o' your lip," says he.  "You've put on
considerable many frills since I been away.  I'll take you down a peg
before I get done with you.  You're educated, too, they say--can read and
write.  You think you're better'n your father, now, don't you, because he
can't?  I'LL take it out of you.  Who told you you might meddle with such
hifalut'n foolishness, hey?--who told you you could?"

"The widow.  She told me."

"The widow, hey?--and who told the widow she could put in her shovel
about a thing that ain't none of her business?"

"Nobody never told her."

"Well, I'll learn her how to meddle.  And looky here--you drop that
school, you hear?  I'll learn people to bring up a boy to put on airs
over his own father and let on to be better'n what HE is.  You lemme
catch you fooling around that school again, you hear?  Your mother
couldn't read, and she couldn't write, nuther, before she died.  None of
the family couldn't before THEY died.  I can't; and here you're
a-swelling yourself up like this.  I ain't the man to stand it--you hear?
Say, lemme hear you read."

I took up a book and begun something about General Washington and the
wars. When I'd read about a half a minute, he fetched the book a whack
with his hand and knocked it across the house.  He says:

"It's so.  You can do it.  I had my doubts when you told me.  Now looky
here; you stop that putting on frills.  I won't have it.  I'll lay for
you, my smarty; and if I catch you about that school I'll tan you good.
First you know you'll get religion, too.  I never see such a son."

He took up a little blue and yaller picture of some cows and a boy, and
says:

"What's this?"

"It's something they give me for learning my lessons good."

He tore it up, and says:

"I'll give you something better--I'll give you a cowhide."

He set there a-mumbling and a-growling a minute, and then he says:

"AIN'T you a sweet-scented dandy, though?  A bed; and bedclothes; and a
look'n'-glass; and a piece of carpet on the floor--and your own father
got to sleep with the hogs in the tanyard.  I never see such a son.  I
bet I'll take some o' these frills out o' you before I'm done with you.
Why, there ain't no end to your airs--they say you're rich.  Hey?--how's
that?"

"They lie--that's how."

"Looky here--mind how you talk to me; I'm a-standing about all I can
stand now--so don't gimme no sass.  I've been in town two days, and I
hain't heard nothing but about you bein' rich.  I heard about it away
down the river, too.  That's why I come.  You git me that money
to-morrow--I want it."

"I hain't got no money."

"It's a lie.  Judge Thatcher's got it.  You git it.  I want it."

"I hain't got no money, I tell you.  You ask Judge Thatcher; he'll tell
you the same."

"All right.  I'll ask him; and I'll make him pungle, too, or I'll know
the reason why.  Say, how much you got in your pocket?  I want it."

"I hain't got only a dollar, and I want that to--"

"It don't make no difference what you want it for--you just shell it
out."

He took it and bit it to see if it was good, and then he said he was
going down town to get some whisky; said he hadn't had a drink all day.
When he had got out on the shed he put his head in again, and cussed me
for putting on frills and trying to be better than him; and when I
reckoned he was gone he come back and put his head in again, and told me
to mind about that school, because he was going to lay for me and lick me
if I didn't drop that.

Next day he was drunk, and he went to Judge Thatcher's and bullyragged
him, and tried to make him give up the money; but he couldn't, and then
he swore he'd make the law force him.

The judge and the widow went to law to get the court to take me away from
him and let one of them be my guardian; but it was a new judge that had
just come, and he didn't know the old man; so he said courts mustn't
interfere and separate families if they could help it; said he'd druther
not take a child away from its father.  So Judge Thatcher and the widow
had to quit on the business.

That pleased the old man till he couldn't rest.  He said he'd cowhide me
till I was black and blue if I didn't raise some money for him.  I
borrowed three dollars from Judge Thatcher, and pap took it and got
drunk, and went a-blowing around and cussing and whooping and carrying
on; and he kept it up all over town, with a tin pan, till most midnight;
then they jailed him, and next day they had him before court, and jailed
him again for a week.  But he said HE was satisfied; said he was boss of
his son, and he'd make it warm for HIM.

When he got out the new judge said he was a-going to make a man of him.
So he took him to his own house, and dressed him up clean and nice, and
had him to breakfast and dinner and supper with the family, and was just
old pie to him, so to speak.  And after supper he talked to him about
temperance and such things till the old man cried, and said he'd been a
fool, and fooled away his life; but now he was a-going to turn over a new
leaf and be a man nobody wouldn't be ashamed of, and he hoped the judge
would help him and not look down on him.  The judge said he could hug him
for them words; so he cried, and his wife she cried again; pap said he'd
been a man that had always been misunderstood before, and the judge said
he believed it.  The old man said that what a man wanted that was down
was sympathy, and the judge said it was so; so they cried again.  And
when it was bedtime the old man rose up and held out his hand, and says:

"Look at it, gentlemen and ladies all; take a-hold of it; shake it.
There's a hand that was the hand of a hog; but it ain't so no more; it's
the hand of a man that's started in on a new life, and'll die before
he'll go back.  You mark them words--don't forget I said them.  It's a
clean hand now; shake it--don't be afeard."

So they shook it, one after the other, all around, and cried.  The
judge's wife she kissed it.  Then the old man he signed a pledge--made
his mark. The judge said it was the holiest time on record, or something
like that. Then they tucked the old man into a beautiful room, which was
the spare room, and in the night some time he got powerful thirsty and
clumb out on to the porch-roof and slid down a stanchion and traded his
new coat for a jug of forty-rod, and clumb back again and had a good old
time; and towards daylight he crawled out again, drunk as a fiddler, and
rolled off the porch and broke his left arm in two places, and was most
froze to death when somebody found him after sun-up.  And when they come
to look at that spare room they had to take soundings before they could
navigate it.

The judge he felt kind of sore.  He said he reckoned a body could reform
the old man with a shotgun, maybe, but he didn't know no other way.




CHAPTER VI.

WELL, pretty soon the old man was up and around again, and then he went
for Judge Thatcher in the courts to make him give up that money, and he
went for me, too, for not stopping school.  He catched me a couple of
times and thrashed me, but I went to school just the same, and dodged him
or outrun him most of the time.  I didn't want to go to school much
before, but I reckoned I'd go now to spite pap.  That law trial was a
slow business--appeared like they warn't ever going to get started on
it; so every now and then I'd borrow two or three dollars off of the
judge for him, to keep from getting a cowhiding.  Every time he got money
he got drunk; and every time he got drunk he raised Cain around town; and
every time he raised Cain he got jailed.  He was just suited--this kind
of thing was right in his line.

He got to hanging around the widow's too much and so she told him at last
that if he didn't quit using around there she would make trouble for him.
Well, WASN'T he mad?  He said he would show who was Huck Finn's boss.  So
he watched out for me one day in the spring, and catched me, and took me
up the river about three mile in a skiff, and crossed over to the
Illinois shore where it was woody and there warn't no houses but an old
log hut in a place where the timber was so thick you couldn't find it if
you didn't know where it was.

He kept me with him all the time, and I never got a chance to run off.
We lived in that old cabin, and he always locked the door and put the key
under his head nights.  He had a gun which he had stole, I reckon, and we
fished and hunted, and that was what we lived on.  Every little while he
locked me in and went down to the store, three miles, to the ferry, and
traded fish and game for whisky, and fetched it home and got drunk and
had a good time, and licked me.  The widow she found out where I was by
and by, and she sent a man over to try to get hold of me; but pap drove
him off with the gun, and it warn't long after that till I was used to
being where I was, and liked it--all but the cowhide part.

It was kind of lazy and jolly, laying off comfortable all day, smoking
and fishing, and no books nor study.  Two months or more run along, and
my clothes got to be all rags and dirt, and I didn't see how I'd ever got
to like it so well at the widow's, where you had to wash, and eat on a
plate, and comb up, and go to bed and get up regular, and be forever
bothering over a book, and have old Miss Watson pecking at you all the
time.  I didn't want to go back no more.  I had stopped cussing, because
the widow didn't like it; but now I took to it again because pap hadn't
no objections.  It was pretty good times up in the woods there, take it
all around.

But by and by pap got too handy with his hick'ry, and I couldn't stand
it. I was all over welts.  He got to going away so much, too, and locking
me in.  Once he locked me in and was gone three days.  It was dreadful
lonesome.  I judged he had got drowned, and I wasn't ever going to get
out any more.  I was scared.  I made up my mind I would fix up some way
to leave there.  I had tried to get out of that cabin many a time, but I
couldn't find no way.  There warn't a window to it big enough for a dog
to get through.  I couldn't get up the chimbly; it was too narrow.  The
door was thick, solid oak slabs.  Pap was pretty careful not to leave a
knife or anything in the cabin when he was away; I reckon I had hunted
the place over as much as a hundred times; well, I was most all the time
at it, because it was about the only way to put in the time.  But this
time I found something at last; I found an old rusty wood-saw without any
handle; it was laid in between a rafter and the clapboards of the roof.
I greased it up and went to work.  There was an old horse-blanket nailed
against the logs at the far end of the cabin behind the table, to keep
the wind from blowing through the chinks and putting the candle out.  I
got under the table and raised the blanket, and went to work to saw a
section of the big bottom log out--big enough to let me through.  Well,
it was a good long job, but I was getting towards the end of it when I
heard pap's gun in the woods.  I got rid of the signs of my work, and
dropped the blanket and hid my saw, and pretty soon pap come in.

Pap warn't in a good humor--so he was his natural self.  He said he was
down town, and everything was going wrong.  His lawyer said he reckoned
he would win his lawsuit and get the money if they ever got started on
the trial; but then there was ways to put it off a long time, and Judge
Thatcher knowed how to do it.  And he said people allowed there'd be
another trial to get me away from him and give me to the widow for my
guardian, and they guessed it would win this time.  This shook me up
considerable, because I didn't want to go back to the widow's any more
and be so cramped up and sivilized, as they called it.  Then the old man
got to cussing, and cussed everything and everybody he could think of,
and then cussed them all over again to make sure he hadn't skipped any,
and after that he polished off with a kind of a general cuss all round,
including a considerable parcel of people which he didn't know the names
of, and so called them what's-his-name when he got to them, and went
right along with his cussing.

He said he would like to see the widow get me.  He said he would watch
out, and if they tried to come any such game on him he knowed of a place
six or seven mile off to stow me in, where they might hunt till they
dropped and they couldn't find me.  That made me pretty uneasy again, but
only for a minute; I reckoned I wouldn't stay on hand till he got that
chance.

The old man made me go to the skiff and fetch the things he had got.
There was a fifty-pound sack of corn meal, and a side of bacon,
ammunition, and a four-gallon jug of whisky, and an old book and two
newspapers for wadding, besides some tow.  I toted up a load, and went
back and set down on the bow of the skiff to rest.  I thought it all
over, and I reckoned I would walk off with the gun and some lines, and
take to the woods when I run away.  I guessed I wouldn't stay in one
place, but just tramp right across the country, mostly night times, and
hunt and fish to keep alive, and so get so far away that the old man nor
the widow couldn't ever find me any more.  I judged I would saw out and
leave that night if pap got drunk enough, and I reckoned he would.  I got
so full of it I didn't notice how long I was staying till the old man
hollered and asked me whether I was asleep or drownded.

I got the things all up to the cabin, and then it was about dark.  While
I was cooking supper the old man took a swig or two and got sort of
warmed up, and went to ripping again.  He had been drunk over in town,
and laid in the gutter all night, and he was a sight to look at.  A body
would a thought he was Adam--he was just all mud.  Whenever his liquor
begun to work he most always went for the govment, this time he says:

"Call this a govment! why, just look at it and see what it's like.
Here's the law a-standing ready to take a man's son away from him--a
man's own son, which he has had all the trouble and all the anxiety and
all the expense of raising.  Yes, just as that man has got that son
raised at last, and ready to go to work and begin to do suthin' for HIM
and give him a rest, the law up and goes for him.  And they call THAT
govment!  That ain't all, nuther.  The law backs that old Judge Thatcher
up and helps him to keep me out o' my property.  Here's what the law
does:  The law takes a man worth six thousand dollars and up'ards, and
jams him into an old trap of a cabin like this, and lets him go round in
clothes that ain't fitten for a hog. They call that govment!  A man can't
get his rights in a govment like this. Sometimes I've a mighty notion to
just leave the country for good and all. Yes, and I TOLD 'em so; I told
old Thatcher so to his face.  Lots of 'em heard me, and can tell what I
said.  Says I, for two cents I'd leave the blamed country and never come
a-near it agin.  Them's the very words.  I says look at my hat--if you
call it a hat--but the lid raises up and the rest of it goes down till
it's below my chin, and then it ain't rightly a hat at all, but more like
my head was shoved up through a jint o' stove-pipe.  Look at it, says I
--such a hat for me to wear--one of the wealthiest men in this town if I
could git my rights.

"Oh, yes, this is a wonderful govment, wonderful.  Why, looky here.
There was a free nigger there from Ohio--a mulatter, most as white as a
white man.  He had the whitest shirt on you ever see, too, and the
shiniest hat; and there ain't a man in that town that's got as fine
clothes as what he had; and he had a gold watch and chain, and a
silver-headed cane--the awfulest old gray-headed nabob in the State.  And
what do you think?  They said he was a p'fessor in a college, and could
talk all kinds of languages, and knowed everything.  And that ain't the
wust. They said he could VOTE when he was at home.  Well, that let me
out. Thinks I, what is the country a-coming to?  It was 'lection day, and
I was just about to go and vote myself if I warn't too drunk to get
there; but when they told me there was a State in this country where
they'd let that nigger vote, I drawed out.  I says I'll never vote agin.
Them's the very words I said; they all heard me; and the country may rot
for all me--I'll never vote agin as long as I live.  And to see the cool
way of that nigger--why, he wouldn't a give me the road if I hadn't
shoved him out o' the way.  I says to the people, why ain't this nigger
put up at auction and sold?--that's what I want to know.  And what do you
reckon they said? Why, they said he couldn't be sold till he'd been in
the State six months, and he hadn't been there that long yet.  There,
now--that's a specimen.  They call that a govment that can't sell a free
nigger till he's been in the State six months.  Here's a govment that
calls itself a govment, and lets on to be a govment, and thinks it is a
govment, and yet's got to set stock-still for six whole months before it
can take a hold of a prowling, thieving, infernal, white-shirted free
nigger, and--"

Pap was agoing on so he never noticed where his old limber legs was
taking him to, so he went head over heels over the tub of salt pork and
barked both shins, and the rest of his speech was all the hottest kind of
language--mostly hove at the nigger and the govment, though he give the
tub some, too, all along, here and there.  He hopped around the cabin
considerable, first on one leg and then on the other, holding first one
shin and then the other one, and at last he let out with his left foot
all of a sudden and fetched the tub a rattling kick.  But it warn't good
judgment, because that was the boot that had a couple of his toes leaking
out of the front end of it; so now he raised a howl that fairly made a
body's hair raise, and down he went in the dirt, and rolled there, and
held his toes; and the cussing he done then laid over anything he had
ever done previous.  He said so his own self afterwards.  He had heard
old Sowberry Hagan in his best days, and he said it laid over him, too;
but I reckon that was sort of piling it on, maybe.

After supper pap took the jug, and said he had enough whisky there for
two drunks and one delirium tremens.  That was always his word.  I judged
he would be blind drunk in about an hour, and then I would steal the key,
or saw myself out, one or t'other.  He drank and drank, and tumbled down
on his blankets by and by; but luck didn't run my way.  He didn't go
sound asleep, but was uneasy.  He groaned and moaned and thrashed around
this way and that for a long time.  At last I got so sleepy I couldn't
keep my eyes open all I could do, and so before I knowed what I was about
I was sound asleep, and the candle burning.

I don't know how long I was asleep, but all of a sudden there was an
awful scream and I was up.  There was pap looking wild, and skipping
around every which way and yelling about snakes.  He said they was
crawling up his legs; and then he would give a jump and scream, and say
one had bit him on the cheek--but I couldn't see no snakes.  He started
and run round and round the cabin, hollering "Take him off! take him off!
he's biting me on the neck!"  I never see a man look so wild in the eyes.
Pretty soon he was all fagged out, and fell down panting; then he rolled
over and over wonderful fast, kicking things every which way, and
striking and grabbing at the air with his hands, and screaming and saying
there was devils a-hold of him.  He wore out by and by, and laid still a
while, moaning.  Then he laid stiller, and didn't make a sound.  I could
hear the owls and the wolves away off in the woods, and it seemed
terrible still.  He was laying over by the corner. By and by he raised up
part way and listened, with his head to one side.  He says, very low:

"Tramp--tramp--tramp; that's the dead; tramp--tramp--tramp; they're
coming after me; but I won't go.  Oh, they're here! don't touch me
--don't! hands off--they're cold; let go.  Oh, let a poor devil alone!"

Then he went down on all fours and crawled off, begging them to let him
alone, and he rolled himself up in his blanket and wallowed in under the
old pine table, still a-begging; and then he went to crying.  I could
hear him through the blanket.

By and by he rolled out and jumped up on his feet looking wild, and he
see me and went for me.  He chased me round and round the place with a
clasp-knife, calling me the Angel of Death, and saying he would kill me,
and then I couldn't come for him no more.  I begged, and told him I was
only Huck; but he laughed SUCH a screechy laugh, and roared and cussed,
and kept on chasing me up.  Once when I turned short and dodged under his
arm he made a grab and got me by the jacket between my shoulders, and I
thought I was gone; but I slid out of the jacket quick as lightning, and
saved myself. Pretty soon he was all tired out, and dropped down with his
back against the door, and said he would rest a minute and then kill me.
He put his knife under him, and said he would sleep and get strong, and
then he would see who was who.

So he dozed off pretty soon.  By and by I got the old split-bottom chair
and clumb up as easy as I could, not to make any noise, and got down the
gun.  I slipped the ramrod down it to make sure it was loaded, then I
laid it across the turnip barrel, pointing towards pap, and set down
behind it to wait for him to stir.  And how slow and still the time did
drag along.




CHAPTER VII.

"GIT up!  What you 'bout?"

I opened my eyes and looked around, trying to make out where I was.  It
was after sun-up, and I had been sound asleep.  Pap was standing over me
looking sour and sick, too.  He says:

"What you doin' with this gun?"

I judged he didn't know nothing about what he had been doing, so I says:

"Somebody tried to get in, so I was laying for him."

"Why didn't you roust me out?"

"Well, I tried to, but I couldn't; I couldn't budge you."

"Well, all right.  Don't stand there palavering all day, but out with you
and see if there's a fish on the lines for breakfast.  I'll be along in a
minute."

He unlocked the door, and I cleared out up the river-bank.  I noticed
some pieces of limbs and such things floating down, and a sprinkling of
bark; so I knowed the river had begun to rise.  I reckoned I would have
great times now if I was over at the town.  The June rise used to be
always luck for me; because as soon as that rise begins here comes
cordwood floating down, and pieces of log rafts--sometimes a dozen logs
together; so all you have to do is to catch them and sell them to the
wood-yards and the sawmill.

I went along up the bank with one eye out for pap and t'other one out for
what the rise might fetch along.  Well, all at once here comes a canoe;
just a beauty, too, about thirteen or fourteen foot long, riding high
like a duck.  I shot head-first off of the bank like a frog, clothes and
all on, and struck out for the canoe.  I just expected there'd be
somebody laying down in it, because people often done that to fool folks,
and when a chap had pulled a skiff out most to it they'd raise up and
laugh at him.  But it warn't so this time.  It was a drift-canoe sure
enough, and I clumb in and paddled her ashore.  Thinks I, the old man
will be glad when he sees this--she's worth ten dollars.  But when I
got to shore pap wasn't in sight yet, and as I was running her into a
little creek like a gully, all hung over with vines and willows, I struck
another idea:  I judged I'd hide her good, and then, 'stead of taking to
the woods when I run off, I'd go down the river about fifty mile and camp
in one place for good, and not have such a rough time tramping on foot.

It was pretty close to the shanty, and I thought I heard the old man
coming all the time; but I got her hid; and then I out and looked around
a bunch of willows, and there was the old man down the path a piece just
drawing a bead on a bird with his gun.  So he hadn't seen anything.

When he got along I was hard at it taking up a "trot" line.  He abused me
a little for being so slow; but I told him I fell in the river, and that
was what made me so long.  I knowed he would see I was wet, and then he
would be asking questions.  We got five catfish off the lines and went
home.

While we laid off after breakfast to sleep up, both of us being about
wore out, I got to thinking that if I could fix up some way to keep pap
and the widow from trying to follow me, it would be a certainer thing
than trusting to luck to get far enough off before they missed me; you
see, all kinds of things might happen.  Well, I didn't see no way for a
while, but by and by pap raised up a minute to drink another barrel of
water, and he says:

"Another time a man comes a-prowling round here you roust me out, you
hear? That man warn't here for no good.  I'd a shot him.  Next time you
roust me out, you hear?"

Then he dropped down and went to sleep again; but what he had been saying
give me the very idea I wanted.  I says to myself, I can fix it now so
nobody won't think of following me.

About twelve o'clock we turned out and went along up the bank.  The river
was coming up pretty fast, and lots of driftwood going by on the rise.
By and by along comes part of a log raft--nine logs fast together.  We
went out with the skiff and towed it ashore.  Then we had dinner.
Anybody but pap would a waited and seen the day through, so as to catch
more stuff; but that warn't pap's style.  Nine logs was enough for one
time; he must shove right over to town and sell.  So he locked me in and
took the skiff, and started off towing the raft about half-past three.  I
judged he wouldn't come back that night.  I waited till I reckoned he had
got a good start; then I out with my saw, and went to work on that log
again.  Before he was t'other side of the river I was out of the hole;
him and his raft was just a speck on the water away off yonder.

I took the sack of corn meal and took it to where the canoe was hid, and
shoved the vines and branches apart and put it in; then I done the same
with the side of bacon; then the whisky-jug.  I took all the coffee and
sugar there was, and all the ammunition; I took the wadding; I took the
bucket and gourd; I took a dipper and a tin cup, and my old saw and two
blankets, and the skillet and the coffee-pot.  I took fish-lines and
matches and other things--everything that was worth a cent.  I cleaned
out the place.  I wanted an axe, but there wasn't any, only the one out
at the woodpile, and I knowed why I was going to leave that.  I fetched
out the gun, and now I was done.

I had wore the ground a good deal crawling out of the hole and dragging
out so many things.  So I fixed that as good as I could from the outside
by scattering dust on the place, which covered up the smoothness and the
sawdust.  Then I fixed the piece of log back into its place, and put two
rocks under it and one against it to hold it there, for it was bent up at
that place and didn't quite touch ground.  If you stood four or five foot
away and didn't know it was sawed, you wouldn't never notice it; and
besides, this was the back of the cabin, and it warn't likely anybody
would go fooling around there.

It was all grass clear to the canoe, so I hadn't left a track.  I
followed around to see.  I stood on the bank and looked out over the
river.  All safe.  So I took the gun and went up a piece into the woods,
and was hunting around for some birds when I see a wild pig; hogs soon
went wild in them bottoms after they had got away from the prairie farms.
I shot this fellow and took him into camp.

I took the axe and smashed in the door.  I beat it and hacked it
considerable a-doing it.  I fetched the pig in, and took him back nearly
to the table and hacked into his throat with the axe, and laid him down
on the ground to bleed; I say ground because it was ground--hard packed,
and no boards.  Well, next I took an old sack and put a lot of big rocks
in it--all I could drag--and I started it from the pig, and dragged it
to the door and through the woods down to the river and dumped it in, and
down it sunk, out of sight.  You could easy see that something had been
dragged over the ground.  I did wish Tom Sawyer was there; I knowed he
would take an interest in this kind of business, and throw in the fancy
touches.  Nobody could spread himself like Tom Sawyer in such a thing as
that.

Well, last I pulled out some of my hair, and blooded the axe good, and
stuck it on the back side, and slung the axe in the corner.  Then I took
up the pig and held him to my breast with my jacket (so he couldn't drip)
till I got a good piece below the house and then dumped him into the
river.  Now I thought of something else.  So I went and got the bag of
meal and my old saw out of the canoe, and fetched them to the house.  I
took the bag to where it used to stand, and ripped a hole in the bottom
of it with the saw, for there warn't no knives and forks on the place
--pap done everything with his clasp-knife about the cooking.  Then I
carried the sack about a hundred yards across the grass and through the
willows east of the house, to a shallow lake that was five mile wide and
full of rushes--and ducks too, you might say, in the season.  There was a
slough or a creek leading out of it on the other side that went miles
away, I don't know where, but it didn't go to the river.  The meal sifted
out and made a little track all the way to the lake.  I dropped pap's
whetstone there too, so as to look like it had been done by accident.
Then I tied up the rip in the meal sack with a string, so it wouldn't
leak no more, and took it and my saw to the canoe again.

It was about dark now; so I dropped the canoe down the river under some
willows that hung over the bank, and waited for the moon to rise.  I made
fast to a willow; then I took a bite to eat, and by and by laid down in
the canoe to smoke a pipe and lay out a plan.  I says to myself, they'll
follow the track of that sackful of rocks to the shore and then drag the
river for me.  And they'll follow that meal track to the lake and go
browsing down the creek that leads out of it to find the robbers that
killed me and took the things.  They won't ever hunt the river for
anything but my dead carcass. They'll soon get tired of that, and won't
bother no more about me.  All right; I can stop anywhere I want to.
Jackson's Island is good enough for me; I know that island pretty well,
and nobody ever comes there.  And then I can paddle over to town nights,
and slink around and pick up things I want. Jackson's Island's the place.

I was pretty tired, and the first thing I knowed I was asleep.  When I
woke up I didn't know where I was for a minute.  I set up and looked
around, a little scared.  Then I remembered.  The river looked miles and
miles across.  The moon was so bright I could a counted the drift logs
that went a-slipping along, black and still, hundreds of yards out from
shore. Everything was dead quiet, and it looked late, and SMELT late.
You know what I mean--I don't know the words to put it in.

I took a good gap and a stretch, and was just going to unhitch and start
when I heard a sound away over the water.  I listened.  Pretty soon I
made it out.  It was that dull kind of a regular sound that comes from
oars working in rowlocks when it's a still night.  I peeped out through
the willow branches, and there it was--a skiff, away across the water.  I
couldn't tell how many was in it.  It kept a-coming, and when it was
abreast of me I see there warn't but one man in it.  Think's I, maybe
it's pap, though I warn't expecting him.  He dropped below me with the
current, and by and by he came a-swinging up shore in the easy water, and
he went by so close I could a reached out the gun and touched him.  Well,
it WAS pap, sure enough--and sober, too, by the way he laid his oars.

I didn't lose no time.  The next minute I was a-spinning down stream soft
but quick in the shade of the bank.  I made two mile and a half, and then
struck out a quarter of a mile or more towards the middle of the river,
because pretty soon I would be passing the ferry landing, and people
might see me and hail me.  I got out amongst the driftwood, and then laid
down in the bottom of the canoe and let her float.  I laid there, and had
a good rest and a smoke out of my pipe, looking away into the sky; not a
cloud in it.  The sky looks ever so deep when you lay down on your back
in the moonshine; I never knowed it before.  And how far a body can hear
on the water such nights!  I heard people talking at the ferry landing.
I heard what they said, too--every word of it.  One man said it was
getting towards the long days and the short nights now.  T'other one said
THIS warn't one of the short ones, he reckoned--and then they laughed,
and he said it over again, and they laughed again; then they waked up
another fellow and told him, and laughed, but he didn't laugh; he ripped
out something brisk, and said let him alone.  The first fellow said he
'lowed to tell it to his old woman--she would think it was pretty good;
but he said that warn't nothing to some things he had said in his time.
I heard one man say it was nearly three o'clock, and he hoped daylight
wouldn't wait more than about a week longer.  After that the talk got
further and further away, and I couldn't make out the words any more; but
I could hear the mumble, and now and then a laugh, too, but it seemed a
long ways off.

I was away below the ferry now.  I rose up, and there was Jackson's
Island, about two mile and a half down stream, heavy timbered and
standing up out of the middle of the river, big and dark and solid, like
a steamboat without any lights.  There warn't any signs of the bar at the
head--it was all under water now.

It didn't take me long to get there.  I shot past the head at a ripping
rate, the current was so swift, and then I got into the dead water and
landed on the side towards the Illinois shore.  I run the canoe into a
deep dent in the bank that I knowed about; I had to part the willow
branches to get in; and when I made fast nobody could a seen the canoe
from the outside.

I went up and set down on a log at the head of the island, and looked out
on the big river and the black driftwood and away over to the town, three
mile away, where there was three or four lights twinkling.  A monstrous
big lumber-raft was about a mile up stream, coming along down, with a
lantern in the middle of it.  I watched it come creeping down, and when
it was most abreast of where I stood I heard a man say, "Stern oars,
there! heave her head to stabboard!"  I heard that just as plain as if
the man was by my side.

There was a little gray in the sky now; so I stepped into the woods, and
laid down for a nap before breakfast.




CHAPTER VIII.

THE sun was up so high when I waked that I judged it was after eight
o'clock.  I laid there in the grass and the cool shade thinking about
things, and feeling rested and ruther comfortable and satisfied.  I could
see the sun out at one or two holes, but mostly it was big trees all
about, and gloomy in there amongst them.  There was freckled places on
the ground where the light sifted down through the leaves, and the
freckled places swapped about a little, showing there was a little breeze
up there.  A couple of squirrels set on a limb and jabbered at me very
friendly.

I was powerful lazy and comfortable--didn't want to get up and cook
breakfast.  Well, I was dozing off again when I thinks I hears a deep
sound of "boom!" away up the river.  I rouses up, and rests on my elbow
and listens; pretty soon I hears it again.  I hopped up, and went and
looked out at a hole in the leaves, and I see a bunch of smoke laying on
the water a long ways up--about abreast the ferry.  And there was the
ferryboat full of people floating along down.  I knowed what was the
matter now.  "Boom!" I see the white smoke squirt out of the ferryboat's
side.  You see, they was firing cannon over the water, trying to make my
carcass come to the top.

I was pretty hungry, but it warn't going to do for me to start a fire,
because they might see the smoke.  So I set there and watched the
cannon-smoke and listened to the boom.  The river was a mile wide there,
and it always looks pretty on a summer morning--so I was having a good
enough time seeing them hunt for my remainders if I only had a bite to
eat. Well, then I happened to think how they always put quicksilver in
loaves of bread and float them off, because they always go right to the
drownded carcass and stop there.  So, says I, I'll keep a lookout, and if
any of them's floating around after me I'll give them a show.  I changed
to the Illinois edge of the island to see what luck I could have, and I
warn't disappointed.  A big double loaf come along, and I most got it
with a long stick, but my foot slipped and she floated out further.  Of
course I was where the current set in the closest to the shore--I knowed
enough for that.  But by and by along comes another one, and this time I
won.  I took out the plug and shook out the little dab of quicksilver,
and set my teeth in.  It was "baker's bread"--what the quality eat; none
of your low-down corn-pone.

I got a good place amongst the leaves, and set there on a log, munching
the bread and watching the ferry-boat, and very well satisfied.  And then
something struck me.  I says, now I reckon the widow or the parson or
somebody prayed that this bread would find me, and here it has gone and
done it.  So there ain't no doubt but there is something in that thing
--that is, there's something in it when a body like the widow or the parson
prays, but it don't work for me, and I reckon it don't work for only just
the right kind.

I lit a pipe and had a good long smoke, and went on watching.  The
ferryboat was floating with the current, and I allowed I'd have a chance
to see who was aboard when she come along, because she would come in
close, where the bread did.  When she'd got pretty well along down
towards me, I put out my pipe and went to where I fished out the bread,
and laid down behind a log on the bank in a little open place.  Where the
log forked I could peep through.

By and by she come along, and she drifted in so close that they could a
run out a plank and walked ashore.  Most everybody was on the boat.  Pap,
and Judge Thatcher, and Bessie Thatcher, and Jo Harper, and Tom Sawyer,
and his old Aunt Polly, and Sid and Mary, and plenty more.  Everybody was
talking about the murder, but the captain broke in and says:

"Look sharp, now; the current sets in the closest here, and maybe he's
washed ashore and got tangled amongst the brush at the water's edge.  I
hope so, anyway."

I didn't hope so.  They all crowded up and leaned over the rails, nearly
in my face, and kept still, watching with all their might.  I could see
them first-rate, but they couldn't see me.  Then the captain sung out:

"Stand away!" and the cannon let off such a blast right before me that it
made me deef with the noise and pretty near blind with the smoke, and I
judged I was gone.  If they'd a had some bullets in, I reckon they'd a
got the corpse they was after.  Well, I see I warn't hurt, thanks to
goodness. The boat floated on and went out of sight around the shoulder
of the island.  I could hear the booming now and then, further and
further off, and by and by, after an hour, I didn't hear it no more.  The
island was three mile long.  I judged they had got to the foot, and was
giving it up.  But they didn't yet a while.  They turned around the foot
of the island and started up the channel on the Missouri side, under
steam, and booming once in a while as they went.  I crossed over to that
side and watched them. When they got abreast the head of the island they
quit shooting and dropped over to the Missouri shore and went home to the
town.

I knowed I was all right now.  Nobody else would come a-hunting after me.
I got my traps out of the canoe and made me a nice camp in the thick
woods.  I made a kind of a tent out of my blankets to put my things under
so the rain couldn't get at them.  I catched a catfish and haggled him
open with my saw, and towards sundown I started my camp fire and had
supper.  Then I set out a line to catch some fish for breakfast.

When it was dark I set by my camp fire smoking, and feeling pretty well
satisfied; but by and by it got sort of lonesome, and so I went and set
on the bank and listened to the current swashing along, and counted the
stars and drift logs and rafts that come down, and then went to bed;
there ain't no better way to put in time when you are lonesome; you can't
stay so, you soon get over it.

And so for three days and nights.  No difference--just the same thing.
But the next day I went exploring around down through the island.  I was
boss of it; it all belonged to me, so to say, and I wanted to know all
about it; but mainly I wanted to put in the time.  I found plenty
strawberries, ripe and prime; and green summer grapes, and green
razberries; and the green blackberries was just beginning to show.  They
would all come handy by and by, I judged.

Well, I went fooling along in the deep woods till I judged I warn't far
from the foot of the island.  I had my gun along, but I hadn't shot
nothing; it was for protection; thought I would kill some game nigh home.
About this time I mighty near stepped on a good-sized snake, and it went
sliding off through the grass and flowers, and I after it, trying to get
a shot at it. I clipped along, and all of a sudden I bounded right on to
the ashes of a camp fire that was still smoking.

My heart jumped up amongst my lungs.  I never waited for to look further,
but uncocked my gun and went sneaking back on my tiptoes as fast as ever
I could.  Every now and then I stopped a second amongst the thick leaves
and listened, but my breath come so hard I couldn't hear nothing else.  I
slunk along another piece further, then listened again; and so on, and so
on.  If I see a stump, I took it for a man; if I trod on a stick and
broke it, it made me feel like a person had cut one of my breaths in two
and I only got half, and the short half, too.

When I got to camp I warn't feeling very brash, there warn't much sand in
my craw; but I says, this ain't no time to be fooling around.  So I got
all my traps into my canoe again so as to have them out of sight, and I
put out the fire and scattered the ashes around to look like an old last
year's camp, and then clumb a tree.

I reckon I was up in the tree two hours; but I didn't see nothing, I
didn't hear nothing--I only THOUGHT I heard and seen as much as a
thousand things.  Well, I couldn't stay up there forever; so at last I
got down, but I kept in the thick woods and on the lookout all the time.
All I could get to eat was berries and what was left over from breakfast.

By the time it was night I was pretty hungry.  So when it was good and
dark I slid out from shore before moonrise and paddled over to the
Illinois bank--about a quarter of a mile.  I went out in the woods and
cooked a supper, and I had about made up my mind I would stay there all
night when I hear a PLUNKETY-PLUNK, PLUNKETY-PLUNK, and says to myself,
horses coming; and next I hear people's voices.  I got everything into
the canoe as quick as I could, and then went creeping through the woods
to see what I could find out.  I hadn't got far when I hear a man say:

"We better camp here if we can find a good place; the horses is about
beat out.  Let's look around."

I didn't wait, but shoved out and paddled away easy.  I tied up in the
old place, and reckoned I would sleep in the canoe.

I didn't sleep much.  I couldn't, somehow, for thinking.  And every time
I waked up I thought somebody had me by the neck.  So the sleep didn't do
me no good.  By and by I says to myself, I can't live this way; I'm
a-going to find out who it is that's here on the island with me; I'll
find it out or bust.  Well, I felt better right off.

So I took my paddle and slid out from shore just a step or two, and then
let the canoe drop along down amongst the shadows.  The moon was shining,
and outside of the shadows it made it most as light as day.  I poked
along well on to an hour, everything still as rocks and sound asleep.
Well, by this time I was most down to the foot of the island.  A little
ripply, cool breeze begun to blow, and that was as good as saying the
night was about done.  I give her a turn with the paddle and brung her
nose to shore; then I got my gun and slipped out and into the edge of the
woods.  I sat down there on a log, and looked out through the leaves.  I
see the moon go off watch, and the darkness begin to blanket the river.
But in a little while I see a pale streak over the treetops, and knowed
the day was coming.  So I took my gun and slipped off towards where I had
run across that camp fire, stopping every minute or two to listen.  But I
hadn't no luck somehow; I couldn't seem to find the place.  But by and
by, sure enough, I catched a glimpse of fire away through the trees.  I
went for it, cautious and slow.  By and by I was close enough to have a
look, and there laid a man on the ground.  It most give me the fan-tods.
He had a blanket around his head, and his head was nearly in the fire.  I
set there behind a clump of bushes, in about six foot of him, and kept my
eyes on him steady.  It was getting gray daylight now.  Pretty soon he
gapped and stretched himself and hove off the blanket, and it was Miss
Watson's Jim!  I bet I was glad to see him.  I says:

"Hello, Jim!" and skipped out.

He bounced up and stared at me wild.  Then he drops down on his knees,
and puts his hands together and says:

"Doan' hurt me--don't!  I hain't ever done no harm to a ghos'.  I alwuz
liked dead people, en done all I could for 'em.  You go en git in de
river agin, whah you b'longs, en doan' do nuffn to Ole Jim, 'at 'uz awluz
yo' fren'."

Well, I warn't long making him understand I warn't dead.  I was ever so
glad to see Jim.  I warn't lonesome now.  I told him I warn't afraid of
HIM telling the people where I was.  I talked along, but he only set
there and looked at me; never said nothing.  Then I says:

"It's good daylight.  Le's get breakfast.  Make up your camp fire good."

"What's de use er makin' up de camp fire to cook strawbries en sich
truck? But you got a gun, hain't you?  Den we kin git sumfn better den
strawbries."

"Strawberries and such truck," I says.  "Is that what you live on?"

"I couldn' git nuffn else," he says.

"Why, how long you been on the island, Jim?"

"I come heah de night arter you's killed."

"What, all that time?"

"Yes--indeedy."

"And ain't you had nothing but that kind of rubbage to eat?"

"No, sah--nuffn else."

"Well, you must be most starved, ain't you?"

"I reck'n I could eat a hoss.  I think I could. How long you ben on de
islan'?"

"Since the night I got killed."

"No!  W'y, what has you lived on?  But you got a gun.  Oh, yes, you got a
gun.  Dat's good.  Now you kill sumfn en I'll make up de fire."

So we went over to where the canoe was, and while he built a fire in a
grassy open place amongst the trees, I fetched meal and bacon and coffee,
and coffee-pot and frying-pan, and sugar and tin cups, and the nigger was
set back considerable, because he reckoned it was all done with
witchcraft. I catched a good big catfish, too, and Jim cleaned him with
his knife, and fried him.

When breakfast was ready we lolled on the grass and eat it smoking hot.
Jim laid it in with all his might, for he was most about starved.  Then
when we had got pretty well stuffed, we laid off and lazied.  By and by
Jim says:

"But looky here, Huck, who wuz it dat 'uz killed in dat shanty ef it
warn't you?"

Then I told him the whole thing, and he said it was smart.  He said Tom
Sawyer couldn't get up no better plan than what I had.  Then I says:

"How do you come to be here, Jim, and how'd you get here?"

He looked pretty uneasy, and didn't say nothing for a minute.  Then he
says:

"Maybe I better not tell."

"Why, Jim?"

"Well, dey's reasons.  But you wouldn' tell on me ef I uz to tell you,
would you, Huck?"

"Blamed if I would, Jim."

"Well, I b'lieve you, Huck.  I--I RUN OFF."

"Jim!"

"But mind, you said you wouldn' tell--you know you said you wouldn' tell,
Huck."

"Well, I did.  I said I wouldn't, and I'll stick to it.  Honest INJUN, I
will.  People would call me a low-down Abolitionist and despise me for
keeping mum--but that don't make no difference.  I ain't a-going to tell,
and I ain't a-going back there, anyways.  So, now, le's know all about
it."

"Well, you see, it 'uz dis way.  Ole missus--dat's Miss Watson--she pecks
on me all de time, en treats me pooty rough, but she awluz said she
wouldn' sell me down to Orleans.  But I noticed dey wuz a nigger trader
roun' de place considable lately, en I begin to git oneasy.  Well, one
night I creeps to de do' pooty late, en de do' warn't quite shet, en I
hear old missus tell de widder she gwyne to sell me down to Orleans, but
she didn' want to, but she could git eight hund'd dollars for me, en it
'uz sich a big stack o' money she couldn' resis'.  De widder she try to
git her to say she wouldn' do it, but I never waited to hear de res'.  I
lit out mighty quick, I tell you.

"I tuck out en shin down de hill, en 'spec to steal a skift 'long de sho'
som'ers 'bove de town, but dey wuz people a-stirring yit, so I hid in de
ole tumble-down cooper-shop on de bank to wait for everybody to go 'way.
Well, I wuz dah all night.  Dey wuz somebody roun' all de time.  'Long
'bout six in de mawnin' skifts begin to go by, en 'bout eight er nine
every skift dat went 'long wuz talkin' 'bout how yo' pap come over to de
town en say you's killed.  Dese las' skifts wuz full o' ladies en genlmen
a-goin' over for to see de place.  Sometimes dey'd pull up at de sho' en
take a res' b'fo' dey started acrost, so by de talk I got to know all
'bout de killin'.  I 'uz powerful sorry you's killed, Huck, but I ain't
no mo' now.

"I laid dah under de shavin's all day.  I 'uz hungry, but I warn't
afeard; bekase I knowed ole missus en de widder wuz goin' to start to de
camp-meet'n' right arter breakfas' en be gone all day, en dey knows I
goes off wid de cattle 'bout daylight, so dey wouldn' 'spec to see me
roun' de place, en so dey wouldn' miss me tell arter dark in de evenin'.
De yuther servants wouldn' miss me, kase dey'd shin out en take holiday
soon as de ole folks 'uz out'n de way.

"Well, when it come dark I tuck out up de river road, en went 'bout two
mile er more to whah dey warn't no houses.  I'd made up my mine 'bout
what I's agwyne to do.  You see, ef I kep' on tryin' to git away afoot,
de dogs 'ud track me; ef I stole a skift to cross over, dey'd miss dat
skift, you see, en dey'd know 'bout whah I'd lan' on de yuther side, en
whah to pick up my track.  So I says, a raff is what I's arter; it doan'
MAKE no track.

"I see a light a-comin' roun' de p'int bymeby, so I wade' in en shove' a
log ahead o' me en swum more'n half way acrost de river, en got in
'mongst de drift-wood, en kep' my head down low, en kinder swum agin de
current tell de raff come along.  Den I swum to de stern uv it en tuck
a-holt.  It clouded up en 'uz pooty dark for a little while.  So I clumb
up en laid down on de planks.  De men 'uz all 'way yonder in de middle,
whah de lantern wuz.  De river wuz a-risin', en dey wuz a good current;
so I reck'n'd 'at by fo' in de mawnin' I'd be twenty-five mile down de
river, en den I'd slip in jis b'fo' daylight en swim asho', en take to
de woods on de Illinois side.

"But I didn' have no luck.  When we 'uz mos' down to de head er de islan'
a man begin to come aft wid de lantern, I see it warn't no use fer to
wait, so I slid overboard en struck out fer de islan'.  Well, I had a
notion I could lan' mos' anywhers, but I couldn't--bank too bluff.  I 'uz
mos' to de foot er de islan' b'fo' I found' a good place.  I went into de
woods en jedged I wouldn' fool wid raffs no mo', long as dey move de
lantern roun' so.  I had my pipe en a plug er dog-leg, en some matches in
my cap, en dey warn't wet, so I 'uz all right."

"And so you ain't had no meat nor bread to eat all this time?  Why didn't
you get mud-turkles?"

"How you gwyne to git 'm?  You can't slip up on um en grab um; en how's a
body gwyne to hit um wid a rock?  How could a body do it in de night?  En
I warn't gwyne to show mysef on de bank in de daytime."

"Well, that's so.  You've had to keep in the woods all the time, of
course. Did you hear 'em shooting the cannon?"

"Oh, yes.  I knowed dey was arter you.  I see um go by heah--watched um
thoo de bushes."

Some young birds come along, flying a yard or two at a time and lighting.
Jim said it was a sign it was going to rain.  He said it was a sign when
young chickens flew that way, and so he reckoned it was the same way when
young birds done it.  I was going to catch some of them, but Jim wouldn't
let me.  He said it was death.  He said his father laid mighty sick once,
and some of them catched a bird, and his old granny said his father would
die, and he did.

And Jim said you mustn't count the things you are going to cook for
dinner, because that would bring bad luck.  The same if you shook the
table-cloth after sundown.  And he said if a man owned a beehive and that
man died, the bees must be told about it before sun-up next morning, or
else the bees would all weaken down and quit work and die.  Jim said bees
wouldn't sting idiots; but I didn't believe that, because I had tried
them lots of times myself, and they wouldn't sting me.

I had heard about some of these things before, but not all of them.  Jim
knowed all kinds of signs.  He said he knowed most everything.  I said it
looked to me like all the signs was about bad luck, and so I asked him if
there warn't any good-luck signs.  He says:

"Mighty few--an' DEY ain't no use to a body.  What you want to know when
good luck's a-comin' for?  Want to keep it off?"  And he said:  "Ef you's
got hairy arms en a hairy breas', it's a sign dat you's agwyne to be
rich. Well, dey's some use in a sign like dat, 'kase it's so fur ahead.
You see, maybe you's got to be po' a long time fust, en so you might git
discourage' en kill yo'sef 'f you didn' know by de sign dat you gwyne to
be rich bymeby."

"Have you got hairy arms and a hairy breast, Jim?"

"What's de use to ax dat question?  Don't you see I has?"

"Well, are you rich?"

"No, but I ben rich wunst, and gwyne to be rich agin.  Wunst I had foteen
dollars, but I tuck to specalat'n', en got busted out."

"What did you speculate in, Jim?"

"Well, fust I tackled stock."

"What kind of stock?"

"Why, live stock--cattle, you know.  I put ten dollars in a cow.  But I
ain' gwyne to resk no mo' money in stock.  De cow up 'n' died on my
han's."

"So you lost the ten dollars."

"No, I didn't lose it all.  I on'y los' 'bout nine of it.  I sole de hide
en taller for a dollar en ten cents."

"You had five dollars and ten cents left.  Did you speculate any more?"

"Yes.  You know that one-laigged nigger dat b'longs to old Misto Bradish?
Well, he sot up a bank, en say anybody dat put in a dollar would git fo'
dollars mo' at de en' er de year.  Well, all de niggers went in, but dey
didn't have much.  I wuz de on'y one dat had much.  So I stuck out for
mo' dan fo' dollars, en I said 'f I didn' git it I'd start a bank mysef.
Well, o' course dat nigger want' to keep me out er de business, bekase he
says dey warn't business 'nough for two banks, so he say I could put in
my five dollars en he pay me thirty-five at de en' er de year.

"So I done it.  Den I reck'n'd I'd inves' de thirty-five dollars right
off en keep things a-movin'.  Dey wuz a nigger name' Bob, dat had ketched
a wood-flat, en his marster didn' know it; en I bought it off'n him en
told him to take de thirty-five dollars when de en' er de year come; but
somebody stole de wood-flat dat night, en nex day de one-laigged nigger
say de bank's busted.  So dey didn' none uv us git no money."

"What did you do with the ten cents, Jim?"

"Well, I 'uz gwyne to spen' it, but I had a dream, en de dream tole me to
give it to a nigger name' Balum--Balum's Ass dey call him for short; he's
one er dem chuckleheads, you know.  But he's lucky, dey say, en I see I
warn't lucky.  De dream say let Balum inves' de ten cents en he'd make a
raise for me.  Well, Balum he tuck de money, en when he wuz in church he
hear de preacher say dat whoever give to de po' len' to de Lord, en boun'
to git his money back a hund'd times.  So Balum he tuck en give de ten
cents to de po', en laid low to see what wuz gwyne to come of it."

"Well, what did come of it, Jim?"

"Nuffn never come of it.  I couldn' manage to k'leck dat money no way; en
Balum he couldn'.  I ain' gwyne to len' no mo' money 'dout I see de
security.  Boun' to git yo' money back a hund'd times, de preacher says!
Ef I could git de ten CENTS back, I'd call it squah, en be glad er de
chanst."

"Well, it's all right anyway, Jim, long as you're going to be rich again
some time or other."

"Yes; en I's rich now, come to look at it.  I owns mysef, en I's wuth
eight hund'd dollars.  I wisht I had de money, I wouldn' want no mo'."




CHAPTER IX.

I WANTED to go and look at a place right about the middle of the island
that I'd found when I was exploring; so we started and soon got to it,
because the island was only three miles long and a quarter of a mile
wide.

This place was a tolerable long, steep hill or ridge about forty foot
high. We had a rough time getting to the top, the sides was so steep and
the bushes so thick.  We tramped and clumb around all over it, and by and
by found a good big cavern in the rock, most up to the top on the side
towards Illinois.  The cavern was as big as two or three rooms bunched
together, and Jim could stand up straight in it.  It was cool in there.
Jim was for putting our traps in there right away, but I said we didn't
want to be climbing up and down there all the time.

Jim said if we had the canoe hid in a good place, and had all the traps
in the cavern, we could rush there if anybody was to come to the island,
and they would never find us without dogs.  And, besides, he said them
little birds had said it was going to rain, and did I want the things to
get wet?

So we went back and got the canoe, and paddled up abreast the cavern, and
lugged all the traps up there.  Then we hunted up a place close by to
hide the canoe in, amongst the thick willows.  We took some fish off of
the lines and set them again, and begun to get ready for dinner.

The door of the cavern was big enough to roll a hogshead in, and on one
side of the door the floor stuck out a little bit, and was flat and a
good place to build a fire on.  So we built it there and cooked dinner.

We spread the blankets inside for a carpet, and eat our dinner in there.
We put all the other things handy at the back of the cavern.  Pretty soon
it darkened up, and begun to thunder and lighten; so the birds was right
about it.  Directly it begun to rain, and it rained like all fury, too,
and I never see the wind blow so.  It was one of these regular summer
storms.  It would get so dark that it looked all blue-black outside, and
lovely; and the rain would thrash along by so thick that the trees off a
little ways looked dim and spider-webby; and here would come a blast of
wind that would bend the trees down and turn up the pale underside of the
leaves; and then a perfect ripper of a gust would follow along and set
the branches to tossing their arms as if they was just wild; and next,
when it was just about the bluest and blackest--FST! it was as bright as
glory, and you'd have a little glimpse of tree-tops a-plunging about away
off yonder in the storm, hundreds of yards further than you could see
before; dark as sin again in a second, and now you'd hear the thunder let
go with an awful crash, and then go rumbling, grumbling, tumbling, down
the sky towards the under side of the world, like rolling empty barrels
down stairs--where it's long stairs and they bounce a good deal, you
know.

"Jim, this is nice," I says.  "I wouldn't want to be nowhere else but
here. Pass me along another hunk of fish and some hot corn-bread."

"Well, you wouldn't a ben here 'f it hadn't a ben for Jim.  You'd a ben
down dah in de woods widout any dinner, en gittn' mos' drownded, too; dat
you would, honey.  Chickens knows when it's gwyne to rain, en so do de
birds, chile."

The river went on raising and raising for ten or twelve days, till at
last it was over the banks.  The water was three or four foot deep on the
island in the low places and on the Illinois bottom.  On that side it was
a good many miles wide, but on the Missouri side it was the same old
distance across--a half a mile--because the Missouri shore was just a
wall of high bluffs.

Daytimes we paddled all over the island in the canoe, It was mighty cool
and shady in the deep woods, even if the sun was blazing outside.  We
went winding in and out amongst the trees, and sometimes the vines hung
so thick we had to back away and go some other way.  Well, on every old
broken-down tree you could see rabbits and snakes and such things; and
when the island had been overflowed a day or two they got so tame, on
account of being hungry, that you could paddle right up and put your hand
on them if you wanted to; but not the snakes and turtles--they would
slide off in the water.  The ridge our cavern was in was full of them.
We could a had pets enough if we'd wanted them.

One night we catched a little section of a lumber raft--nice pine planks.
It was twelve foot wide and about fifteen or sixteen foot long, and the
top stood above water six or seven inches--a solid, level floor.  We
could see saw-logs go by in the daylight sometimes, but we let them go;
we didn't show ourselves in daylight.

Another night when we was up at the head of the island, just before
daylight, here comes a frame-house down, on the west side.  She was a
two-story, and tilted over considerable.  We paddled out and got aboard
--clumb in at an upstairs window.  But it was too dark to see yet, so we
made the canoe fast and set in her to wait for daylight.

The light begun to come before we got to the foot of the island.  Then we
looked in at the window.  We could make out a bed, and a table, and two
old chairs, and lots of things around about on the floor, and there was
clothes hanging against the wall.  There was something laying on the
floor in the far corner that looked like a man.  So Jim says:

"Hello, you!"

But it didn't budge.  So I hollered again, and then Jim says:

"De man ain't asleep--he's dead.  You hold still--I'll go en see."

He went, and bent down and looked, and says:

"It's a dead man.  Yes, indeedy; naked, too.  He's ben shot in de back.
I reck'n he's ben dead two er three days.  Come in, Huck, but doan' look
at his face--it's too gashly."

I didn't look at him at all.  Jim throwed some old rags over him, but he
needn't done it; I didn't want to see him.  There was heaps of old greasy
cards scattered around over the floor, and old whisky bottles, and a
couple of masks made out of black cloth; and all over the walls was the
ignorantest kind of words and pictures made with charcoal.  There was two
old dirty calico dresses, and a sun-bonnet, and some women's underclothes
hanging against the wall, and some men's clothing, too.  We put the lot
into the canoe--it might come good.  There was a boy's old speckled straw
hat on the floor; I took that, too.  And there was a bottle that had had
milk in it, and it had a rag stopper for a baby to suck.  We would a took
the bottle, but it was broke.  There was a seedy old chest, and an old
hair trunk with the hinges broke.  They stood open, but there warn't
nothing left in them that was any account.  The way things was scattered
about we reckoned the people left in a hurry, and warn't fixed so as to
carry off most of their stuff.

We got an old tin lantern, and a butcher-knife without any handle, and a
bran-new Barlow knife worth two bits in any store, and a lot of tallow
candles, and a tin candlestick, and a gourd, and a tin cup, and a ratty
old bedquilt off the bed, and a reticule with needles and pins and
beeswax and buttons and thread and all such truck in it, and a hatchet
and some nails, and a fishline as thick as my little finger with some
monstrous hooks on it, and a roll of buckskin, and a leather dog-collar,
and a horseshoe, and some vials of medicine that didn't have no label on
them; and just as we was leaving I found a tolerable good curry-comb, and
Jim he found a ratty old fiddle-bow, and a wooden leg.  The straps was
broke off of it, but, barring that, it was a good enough leg, though it
was too long for me and not long enough for Jim, and we couldn't find the
other one, though we hunted all around.

And so, take it all around, we made a good haul.  When we was ready to
shove off we was a quarter of a mile below the island, and it was pretty
broad day; so I made Jim lay down in the canoe and cover up with the
quilt, because if he set up people could tell he was a nigger a good ways
off.  I paddled over to the Illinois shore, and drifted down most a half
a mile doing it.  I crept up the dead water under the bank, and hadn't no
accidents and didn't see nobody.  We got home all safe.




CHAPTER X.

AFTER breakfast I wanted to talk about the dead man and guess out how he
come to be killed, but Jim didn't want to.  He said it would fetch bad
luck; and besides, he said, he might come and ha'nt us; he said a man
that warn't buried was more likely to go a-ha'nting around than one that
was planted and comfortable.  That sounded pretty reasonable, so I didn't
say no more; but I couldn't keep from studying over it and wishing I
knowed who shot the man, and what they done it for.

We rummaged the clothes we'd got, and found eight dollars in silver sewed
up in the lining of an old blanket overcoat.  Jim said he reckoned the
people in that house stole the coat, because if they'd a knowed the money
was there they wouldn't a left it.  I said I reckoned they killed him,
too; but Jim didn't want to talk about that.  I says:

"Now you think it's bad luck; but what did you say when I fetched in the
snake-skin that I found on the top of the ridge day before yesterday?
You said it was the worst bad luck in the world to touch a snake-skin
with my hands.  Well, here's your bad luck!  We've raked in all this
truck and eight dollars besides.  I wish we could have some bad luck like
this every day, Jim."

"Never you mind, honey, never you mind.  Don't you git too peart.  It's
a-comin'.  Mind I tell you, it's a-comin'."

It did come, too.  It was a Tuesday that we had that talk.  Well, after
dinner Friday we was laying around in the grass at the upper end of the
ridge, and got out of tobacco.  I went to the cavern to get some, and
found a rattlesnake in there.  I killed him, and curled him up on the
foot of Jim's blanket, ever so natural, thinking there'd be some fun when
Jim found him there.  Well, by night I forgot all about the snake, and
when Jim flung himself down on the blanket while I struck a light the
snake's mate was there, and bit him.

He jumped up yelling, and the first thing the light showed was the
varmint curled up and ready for another spring.  I laid him out in a
second with a stick, and Jim grabbed pap's whisky-jug and begun to pour
it down.

He was barefooted, and the snake bit him right on the heel.  That all
comes of my being such a fool as to not remember that wherever you leave
a dead snake its mate always comes there and curls around it.  Jim told
me to chop off the snake's head and throw it away, and then skin the body
and roast a piece of it.  I done it, and he eat it and said it would help
cure him. He made me take off the rattles and tie them around his wrist,
too.  He said that that would help.  Then I slid out quiet and throwed
the snakes clear away amongst the bushes; for I warn't going to let Jim
find out it was all my fault, not if I could help it.

Jim sucked and sucked at the jug, and now and then he got out of his head
and pitched around and yelled; but every time he come to himself he went
to sucking at the jug again.  His foot swelled up pretty big, and so did
his leg; but by and by the drunk begun to come, and so I judged he was
all right; but I'd druther been bit with a snake than pap's whisky.

Jim was laid up for four days and nights.  Then the swelling was all gone
and he was around again.  I made up my mind I wouldn't ever take a-holt
of a snake-skin again with my hands, now that I see what had come of it.
Jim said he reckoned I would believe him next time.  And he said that
handling a snake-skin was such awful bad luck that maybe we hadn't got to
the end of it yet.  He said he druther see the new moon over his left
shoulder as much as a thousand times than take up a snake-skin in his
hand.  Well, I was getting to feel that way myself, though I've always
reckoned that looking at the new moon over your left shoulder is one of
the carelessest and foolishest things a body can do.  Old Hank Bunker
done it once, and bragged about it; and in less than two years he got
drunk and fell off of the shot-tower, and spread himself out so that he
was just a kind of a layer, as you may say; and they slid him edgeways
between two barn doors for a coffin, and buried him so, so they say, but
I didn't see it.  Pap told me.  But anyway it all come of looking at the
moon that way, like a fool.

Well, the days went along, and the river went down between its banks
again; and about the first thing we done was to bait one of the big hooks
with a skinned rabbit and set it and catch a catfish that was as big as a
man, being six foot two inches long, and weighed over two hundred pounds.
We couldn't handle him, of course; he would a flung us into Illinois.  We
just set there and watched him rip and tear around till he drownded.  We
found a brass button in his stomach and a round ball, and lots of
rubbage.  We split the ball open with the hatchet, and there was a spool
in it.  Jim said he'd had it there a long time, to coat it over so and
make a ball of it.  It was as big a fish as was ever catched in the
Mississippi, I reckon.  Jim said he hadn't ever seen a bigger one.  He
would a been worth a good deal over at the village.  They peddle out such
a fish as that by the pound in the market-house there; everybody buys
some of him; his meat's as white as snow and makes a good fry.

Next morning I said it was getting slow and dull, and I wanted to get a
stirring up some way.  I said I reckoned I would slip over the river and
find out what was going on.  Jim liked that notion; but he said I must go
in the dark and look sharp.  Then he studied it over and said, couldn't I
put on some of them old things and dress up like a girl?  That was a good
notion, too.  So we shortened up one of the calico gowns, and I turned up
my trouser-legs to my knees and got into it.  Jim hitched it behind with
the hooks, and it was a fair fit.  I put on the sun-bonnet and tied it
under my chin, and then for a body to look in and see my face was like
looking down a joint of stove-pipe.  Jim said nobody would know me, even
in the daytime, hardly.  I practiced around all day to get the hang of
the things, and by and by I could do pretty well in them, only Jim said I
didn't walk like a girl; and he said I must quit pulling up my gown to
get at my britches-pocket.  I took notice, and done better.

I started up the Illinois shore in the canoe just after dark.

I started across to the town from a little below the ferry-landing, and
the drift of the current fetched me in at the bottom of the town.  I tied
up and started along the bank.  There was a light burning in a little
shanty that hadn't been lived in for a long time, and I wondered who had
took up quarters there.  I slipped up and peeped in at the window.  There
was a woman about forty year old in there knitting by a candle that was
on a pine table.  I didn't know her face; she was a stranger, for you
couldn't start a face in that town that I didn't know.  Now this was
lucky, because I was weakening; I was getting afraid I had come; people
might know my voice and find me out.  But if this woman had been in such
a little town two days she could tell me all I wanted to know; so I
knocked at the door, and made up my mind I wouldn't forget I was a girl.




CHAPTER XI.

"COME in," says the woman, and I did.  She says:  "Take a cheer."

I done it.  She looked me all over with her little shiny eyes, and says:

"What might your name be?"

"Sarah Williams."

"Where 'bouts do you live?  In this neighborhood?'

"No'm.  In Hookerville, seven mile below.  I've walked all the way and
I'm all tired out."

"Hungry, too, I reckon.  I'll find you something."

"No'm, I ain't hungry.  I was so hungry I had to stop two miles below
here at a farm; so I ain't hungry no more.  It's what makes me so late.
My mother's down sick, and out of money and everything, and I come to
tell my uncle Abner Moore.  He lives at the upper end of the town, she
says.  I hain't ever been here before.  Do you know him?"

"No; but I don't know everybody yet.  I haven't lived here quite two
weeks. It's a considerable ways to the upper end of the town.  You better
stay here all night.  Take off your bonnet."

"No," I says; "I'll rest a while, I reckon, and go on.  I ain't afeared
of the dark."

She said she wouldn't let me go by myself, but her husband would be in by
and by, maybe in a hour and a half, and she'd send him along with me.
Then she got to talking about her husband, and about her relations up the
river, and her relations down the river, and about how much better off
they used to was, and how they didn't know but they'd made a mistake
coming to our town, instead of letting well alone--and so on and so on,
till I was afeard I had made a mistake coming to her to find out what was
going on in the town; but by and by she dropped on to pap and the murder,
and then I was pretty willing to let her clatter right along.  She told
about me and Tom Sawyer finding the six thousand dollars (only she got it
ten) and all about pap and what a hard lot he was, and what a hard lot I
was, and at last she got down to where I was murdered.  I says:

"Who done it?  We've heard considerable about these goings on down in
Hookerville, but we don't know who 'twas that killed Huck Finn."

"Well, I reckon there's a right smart chance of people HERE that'd like
to know who killed him.  Some think old Finn done it himself."

"No--is that so?"

"Most everybody thought it at first.  He'll never know how nigh he come
to getting lynched.  But before night they changed around and judged it
was done by a runaway nigger named Jim."

"Why HE--"

I stopped.  I reckoned I better keep still.  She run on, and never
noticed I had put in at all:

"The nigger run off the very night Huck Finn was killed.  So there's a
reward out for him--three hundred dollars.  And there's a reward out for
old Finn, too--two hundred dollars.  You see, he come to town the morning
after the murder, and told about it, and was out with 'em on the
ferryboat hunt, and right away after he up and left.  Before night they
wanted to lynch him, but he was gone, you see.  Well, next day they found
out the nigger was gone; they found out he hadn't ben seen sence ten
o'clock the night the murder was done.  So then they put it on him, you
see; and while they was full of it, next day, back comes old Finn, and
went boo-hooing to Judge Thatcher to get money to hunt for the nigger all
over Illinois with. The judge gave him some, and that evening he got
drunk, and was around till after midnight with a couple of mighty
hard-looking strangers, and then went off with them.  Well, he hain't
come back sence, and they ain't looking for him back till this thing
blows over a little, for people thinks now that he killed his boy and
fixed things so folks would think robbers done it, and then he'd get
Huck's money without having to bother a long time with a lawsuit.  People
do say he warn't any too good to do it.  Oh, he's sly, I reckon.  If he
don't come back for a year he'll be all right.  You can't prove anything
on him, you know; everything will be quieted down then, and he'll walk in
Huck's money as easy as nothing."

"Yes, I reckon so, 'm.  I don't see nothing in the way of it.  Has
everybody quit thinking the nigger done it?"

"Oh, no, not everybody.  A good many thinks he done it.  But they'll get
the nigger pretty soon now, and maybe they can scare it out of him."

"Why, are they after him yet?"

"Well, you're innocent, ain't you!  Does three hundred dollars lay around
every day for people to pick up?  Some folks think the nigger ain't far
from here.  I'm one of them--but I hain't talked it around.  A few days
ago I was talking with an old couple that lives next door in the log
shanty, and they happened to say hardly anybody ever goes to that island
over yonder that they call Jackson's Island.  Don't anybody live there?
says I. No, nobody, says they.  I didn't say any more, but I done some
thinking.  I was pretty near certain I'd seen smoke over there, about the
head of the island, a day or two before that, so I says to myself, like
as not that nigger's hiding over there; anyway, says I, it's worth the
trouble to give the place a hunt.  I hain't seen any smoke sence, so I
reckon maybe he's gone, if it was him; but husband's going over to see
--him and another man.  He was gone up the river; but he got back to-day,
and I told him as soon as he got here two hours ago."

I had got so uneasy I couldn't set still.  I had to do something with my
hands; so I took up a needle off of the table and went to threading it.
My hands shook, and I was making a bad job of it.  When the woman stopped
talking I looked up, and she was looking at me pretty curious and smiling
a little.  I put down the needle and thread, and let on to be interested
--and I was, too--and says:

"Three hundred dollars is a power of money.  I wish my mother could get
it. Is your husband going over there to-night?"

"Oh, yes.  He went up-town with the man I was telling you of, to get a
boat and see if they could borrow another gun.  They'll go over after
midnight."

"Couldn't they see better if they was to wait till daytime?"

"Yes.  And couldn't the nigger see better, too?  After midnight he'll
likely be asleep, and they can slip around through the woods and hunt up
his camp fire all the better for the dark, if he's got one."

"I didn't think of that."

The woman kept looking at me pretty curious, and I didn't feel a bit
comfortable.  Pretty soon she says,

"What did you say your name was, honey?"

"M--Mary Williams."

Somehow it didn't seem to me that I said it was Mary before, so I didn't
look up--seemed to me I said it was Sarah; so I felt sort of cornered,
and was afeared maybe I was looking it, too.  I wished the woman would
say something more; the longer she set still the uneasier I was.  But now
she says:

"Honey, I thought you said it was Sarah when you first come in?"

"Oh, yes'm, I did.  Sarah Mary Williams.  Sarah's my first name.  Some
calls me Sarah, some calls me Mary."

"Oh, that's the way of it?"

"Yes'm."

I was feeling better then, but I wished I was out of there, anyway.  I
couldn't look up yet.

Well, the woman fell to talking about how hard times was, and how poor
they had to live, and how the rats was as free as if they owned the
place, and so forth and so on, and then I got easy again.  She was right
about the rats. You'd see one stick his nose out of a hole in the corner
every little while.  She said she had to have things handy to throw at
them when she was alone, or they wouldn't give her no peace.  She showed
me a bar of lead twisted up into a knot, and said she was a good shot
with it generly, but she'd wrenched her arm a day or two ago, and didn't
know whether she could throw true now.  But she watched for a chance, and
directly banged away at a rat; but she missed him wide, and said "Ouch!"
it hurt her arm so.  Then she told me to try for the next one.  I wanted
to be getting away before the old man got back, but of course I didn't
let on.  I got the thing, and the first rat that showed his nose I let
drive, and if he'd a stayed where he was he'd a been a tolerable sick
rat.  She said that was first-rate, and she reckoned I would hive the
next one.  She went and got the lump of lead and fetched it back, and
brought along a hank of yarn which she wanted me to help her with.  I
held up my two hands and she put the hank over them, and went on talking
about her and her husband's matters.  But she broke off to say:

"Keep your eye on the rats.  You better have the lead in your lap,
handy."

So she dropped the lump into my lap just at that moment, and I clapped my
legs together on it and she went on talking.  But only about a minute.
Then she took off the hank and looked me straight in the face, and very
pleasant, and says:

"Come, now, what's your real name?"

"Wh--what, mum?"

"What's your real name?  Is it Bill, or Tom, or Bob?--or what is it?"

I reckon I shook like a leaf, and I didn't know hardly what to do.  But I
says:

"Please to don't poke fun at a poor girl like me, mum.  If I'm in the way
here, I'll--"

"No, you won't.  Set down and stay where you are.  I ain't going to hurt
you, and I ain't going to tell on you, nuther.  You just tell me your
secret, and trust me.  I'll keep it; and, what's more, I'll help you.
So'll my old man if you want him to.  You see, you're a runaway
'prentice, that's all.  It ain't anything.  There ain't no harm in it.
You've been treated bad, and you made up your mind to cut.  Bless you,
child, I wouldn't tell on you.  Tell me all about it now, that's a good
boy."

So I said it wouldn't be no use to try to play it any longer, and I would
just make a clean breast and tell her everything, but she musn't go back
on her promise.  Then I told her my father and mother was dead, and the
law had bound me out to a mean old farmer in the country thirty mile back
from the river, and he treated me so bad I couldn't stand it no longer;
he went away to be gone a couple of days, and so I took my chance and
stole some of his daughter's old clothes and cleared out, and I had been
three nights coming the thirty miles.  I traveled nights, and hid
daytimes and slept, and the bag of bread and meat I carried from home
lasted me all the way, and I had a-plenty.  I said I believed my uncle
Abner Moore would take care of me, and so that was why I struck out for
this town of Goshen.

"Goshen, child?  This ain't Goshen.  This is St. Petersburg.  Goshen's
ten mile further up the river.  Who told you this was Goshen?"

"Why, a man I met at daybreak this morning, just as I was going to turn
into the woods for my regular sleep.  He told me when the roads forked I
must take the right hand, and five mile would fetch me to Goshen."

"He was drunk, I reckon.  He told you just exactly wrong."

"Well, he did act like he was drunk, but it ain't no matter now.  I got
to be moving along.  I'll fetch Goshen before daylight."

"Hold on a minute.  I'll put you up a snack to eat.  You might want it."

So she put me up a snack, and says:

"Say, when a cow's laying down, which end of her gets up first?  Answer
up prompt now--don't stop to study over it.  Which end gets up first?"

"The hind end, mum."

"Well, then, a horse?"

"The for'rard end, mum."

"Which side of a tree does the moss grow on?"

"North side."

"If fifteen cows is browsing on a hillside, how many of them eats with
their heads pointed the same direction?"

"The whole fifteen, mum."

"Well, I reckon you HAVE lived in the country.  I thought maybe you was
trying to hocus me again.  What's your real name, now?"

"George Peters, mum."

"Well, try to remember it, George.  Don't forget and tell me it's
Elexander before you go, and then get out by saying it's George Elexander
when I catch you.  And don't go about women in that old calico.  You do a
girl tolerable poor, but you might fool men, maybe.  Bless you, child,
when you set out to thread a needle don't hold the thread still and fetch
the needle up to it; hold the needle still and poke the thread at it;
that's the way a woman most always does, but a man always does t'other
way.  And when you throw at a rat or anything, hitch yourself up a tiptoe
and fetch your hand up over your head as awkward as you can, and miss
your rat about six or seven foot. Throw stiff-armed from the shoulder,
like there was a pivot there for it to turn on, like a girl; not from the
wrist and elbow, with your arm out to one side, like a boy.  And, mind
you, when a girl tries to catch anything in her lap she throws her knees
apart; she don't clap them together, the way you did when you catched the
lump of lead.  Why, I spotted you for a boy when you was threading the
needle; and I contrived the other things just to make certain.  Now trot
along to your uncle, Sarah Mary Williams George Elexander Peters, and if
you get into trouble you send word to Mrs. Judith Loftus, which is me,
and I'll do what I can to get you out of it.  Keep the river road all the
way, and next time you tramp take shoes and socks with you. The river
road's a rocky one, and your feet'll be in a condition when you get to
Goshen, I reckon."

I went up the bank about fifty yards, and then I doubled on my tracks and
slipped back to where my canoe was, a good piece below the house.  I
jumped in, and was off in a hurry.  I went up-stream far enough to make
the head of the island, and then started across.  I took off the
sun-bonnet, for I didn't want no blinders on then.  When I was about the
middle I heard the clock begin to strike, so I stops and listens; the
sound come faint over the water but clear--eleven.  When I struck the
head of the island I never waited to blow, though I was most winded, but
I shoved right into the timber where my old camp used to be, and started
a good fire there on a high and dry spot.

Then I jumped in the canoe and dug out for our place, a mile and a half
below, as hard as I could go.  I landed, and slopped through the timber
and up the ridge and into the cavern.  There Jim laid, sound asleep on
the ground.  I roused him out and says:

"Git up and hump yourself, Jim!  There ain't a minute to lose.  They're
after us!"

Jim never asked no questions, he never said a word; but the way he worked
for the next half an hour showed about how he was scared.  By that time
everything we had in the world was on our raft, and she was ready to be
shoved out from the willow cove where she was hid.  We put out the camp
fire at the cavern the first thing, and didn't show a candle outside
after that.

I took the canoe out from the shore a little piece, and took a look; but
if there was a boat around I couldn't see it, for stars and shadows ain't
good to see by.  Then we got out the raft and slipped along down in the
shade, past the foot of the island dead still--never saying a word.




CHAPTER XII.

IT must a been close on to one o'clock when we got below the island at
last, and the raft did seem to go mighty slow.  If a boat was to come
along we was going to take to the canoe and break for the Illinois shore;
and it was well a boat didn't come, for we hadn't ever thought to put the
gun in the canoe, or a fishing-line, or anything to eat.  We was in
ruther too much of a sweat to think of so many things.  It warn't good
judgment to put EVERYTHING on the raft.

If the men went to the island I just expect they found the camp fire I
built, and watched it all night for Jim to come.  Anyways, they stayed
away from us, and if my building the fire never fooled them it warn't no
fault of mine.  I played it as low down on them as I could.

When the first streak of day began to show we tied up to a towhead in a
big bend on the Illinois side, and hacked off cottonwood branches with
the hatchet, and covered up the raft with them so she looked like there
had been a cave-in in the bank there.  A tow-head is a sandbar that has
cottonwoods on it as thick as harrow-teeth.

We had mountains on the Missouri shore and heavy timber on the Illinois
side, and the channel was down the Missouri shore at that place, so we
warn't afraid of anybody running across us.  We laid there all day, and
watched the rafts and steamboats spin down the Missouri shore, and
up-bound steamboats fight the big river in the middle.  I told Jim all
about the time I had jabbering with that woman; and Jim said she was a
smart one, and if she was to start after us herself she wouldn't set down
and watch a camp fire--no, sir, she'd fetch a dog.  Well, then, I said,
why couldn't she tell her husband to fetch a dog?  Jim said he bet she
did think of it by the time the men was ready to start, and he believed
they must a gone up-town to get a dog and so they lost all that time, or
else we wouldn't be here on a towhead sixteen or seventeen mile below the
village--no, indeedy, we would be in that same old town again.  So I said
I didn't care what was the reason they didn't get us as long as they
didn't.

When it was beginning to come on dark we poked our heads out of the
cottonwood thicket, and looked up and down and across; nothing in sight;
so Jim took up some of the top planks of the raft and built a snug wigwam
to get under in blazing weather and rainy, and to keep the things dry.
Jim made a floor for the wigwam, and raised it a foot or more above the
level of the raft, so now the blankets and all the traps was out of reach
of steamboat waves.  Right in the middle of the wigwam we made a layer of
dirt about five or six inches deep with a frame around it for to hold it
to its place; this was to build a fire on in sloppy weather or chilly;
the wigwam would keep it from being seen.  We made an extra steering-oar,
too, because one of the others might get broke on a snag or something.
We fixed up a short forked stick to hang the old lantern on, because we
must always light the lantern whenever we see a steamboat coming
down-stream, to keep from getting run over; but we wouldn't have to light
it for up-stream boats unless we see we was in what they call a
"crossing"; for the river was pretty high yet, very low banks being still
a little under water; so up-bound boats didn't always run the channel,
but hunted easy water.

This second night we run between seven and eight hours, with a current
that was making over four mile an hour.  We catched fish and talked, and
we took a swim now and then to keep off sleepiness.  It was kind of
solemn, drifting down the big, still river, laying on our backs looking
up at the stars, and we didn't ever feel like talking loud, and it warn't
often that we laughed--only a little kind of a low chuckle.  We had
mighty good weather as a general thing, and nothing ever happened to us
at all--that night, nor the next, nor the next.

Every night we passed towns, some of them away up on black hillsides,
nothing but just a shiny bed of lights; not a house could you see.  The
fifth night we passed St. Louis, and it was like the whole world lit up.
In St. Petersburg they used to say there was twenty or thirty thousand
people in St. Louis, but I never believed it till I see that wonderful
spread of lights at two o'clock that still night.  There warn't a sound
there; everybody was asleep.

Every night now I used to slip ashore towards ten o'clock at some little
village, and buy ten or fifteen cents' worth of meal or bacon or other
stuff to eat; and sometimes I lifted a chicken that warn't roosting
comfortable, and took him along.  Pap always said, take a chicken when
you get a chance, because if you don't want him yourself you can easy
find somebody that does, and a good deed ain't ever forgot.  I never see
pap when he didn't want the chicken himself, but that is what he used to
say, anyway.

Mornings before daylight I slipped into cornfields and borrowed a
watermelon, or a mushmelon, or a punkin, or some new corn, or things of
that kind.  Pap always said it warn't no harm to borrow things if you was
meaning to pay them back some time; but the widow said it warn't anything
but a soft name for stealing, and no decent body would do it.  Jim said
he reckoned the widow was partly right and pap was partly right; so the
best way would be for us to pick out two or three things from the list
and say we wouldn't borrow them any more--then he reckoned it wouldn't be
no harm to borrow the others.  So we talked it over all one night,
drifting along down the river, trying to make up our minds whether to
drop the watermelons, or the cantelopes, or the mushmelons, or what.  But
towards daylight we got it all settled satisfactory, and concluded to
drop crabapples and p'simmons.  We warn't feeling just right before that,
but it was all comfortable now.  I was glad the way it come out, too,
because crabapples ain't ever good, and the p'simmons wouldn't be ripe
for two or three months yet.

We shot a water-fowl now and then that got up too early in the morning or
didn't go to bed early enough in the evening.  Take it all round, we
lived pretty high.

The fifth night below St. Louis we had a big storm after midnight, with a
power of thunder and lightning, and the rain poured down in a solid
sheet. We stayed in the wigwam and let the raft take care of itself.
When the lightning glared out we could see a big straight river ahead,
and high, rocky bluffs on both sides.  By and by says I, "Hel-LO, Jim,
looky yonder!" It was a steamboat that had killed herself on a rock.  We
was drifting straight down for her.  The lightning showed her very
distinct.  She was leaning over, with part of her upper deck above water,
and you could see every little chimbly-guy clean and clear, and a chair
by the big bell, with an old slouch hat hanging on the back of it, when
the flashes come.

Well, it being away in the night and stormy, and all so mysterious-like,
I felt just the way any other boy would a felt when I see that wreck
laying there so mournful and lonesome in the middle of the river.  I
wanted to get aboard of her and slink around a little, and see what there
was there.  So I says:

"Le's land on her, Jim."

But Jim was dead against it at first.  He says:

"I doan' want to go fool'n 'long er no wrack.  We's doin' blame' well, en
we better let blame' well alone, as de good book says.  Like as not dey's
a watchman on dat wrack."

"Watchman your grandmother," I says; "there ain't nothing to watch but
the texas and the pilot-house; and do you reckon anybody's going to resk
his life for a texas and a pilot-house such a night as this, when it's
likely to break up and wash off down the river any minute?"  Jim couldn't
say nothing to that, so he didn't try.  "And besides," I says, "we might
borrow something worth having out of the captain's stateroom.  Seegars, I
bet you--and cost five cents apiece, solid cash.  Steamboat captains is
always rich, and get sixty dollars a month, and THEY don't care a cent
what a thing costs, you know, long as they want it.  Stick a candle in
your pocket; I can't rest, Jim, till we give her a rummaging.  Do you
reckon Tom Sawyer would ever go by this thing?  Not for pie, he wouldn't.
He'd call it an adventure--that's what he'd call it; and he'd land on
that wreck if it was his last act.  And wouldn't he throw style into it?
--wouldn't he spread himself, nor nothing?  Why, you'd think it was
Christopher C'lumbus discovering Kingdom-Come.  I wish Tom Sawyer WAS
here."

Jim he grumbled a little, but give in.  He said we mustn't talk any more
than we could help, and then talk mighty low.  The lightning showed us
the wreck again just in time, and we fetched the stabboard derrick, and
made fast there.

The deck was high out here.  We went sneaking down the slope of it to
labboard, in the dark, towards the texas, feeling our way slow with our
feet, and spreading our hands out to fend off the guys, for it was so
dark we couldn't see no sign of them.  Pretty soon we struck the forward
end of the skylight, and clumb on to it; and the next step fetched us in
front of the captain's door, which was open, and by Jimminy, away down
through the texas-hall we see a light! and all in the same second we seem
to hear low voices in yonder!

Jim whispered and said he was feeling powerful sick, and told me to come
along.  I says, all right, and was going to start for the raft; but just
then I heard a voice wail out and say:

"Oh, please don't, boys; I swear I won't ever tell!"

Another voice said, pretty loud:

"It's a lie, Jim Turner.  You've acted this way before.  You always want
more'n your share of the truck, and you've always got it, too, because
you've swore 't if you didn't you'd tell.  But this time you've said it
jest one time too many.  You're the meanest, treacherousest hound in this
country."

By this time Jim was gone for the raft.  I was just a-biling with
curiosity; and I says to myself, Tom Sawyer wouldn't back out now, and so
I won't either; I'm a-going to see what's going on here.  So I dropped on
my hands and knees in the little passage, and crept aft in the dark till
there warn't but one stateroom betwixt me and the cross-hall of the
texas.  Then in there I see a man stretched on the floor and tied hand
and foot, and two men standing over him, and one of them had a dim
lantern in his hand, and the other one had a pistol.  This one kept
pointing the pistol at the man's head on the floor, and saying:

"I'd LIKE to!  And I orter, too--a mean skunk!"

The man on the floor would shrivel up and say, "Oh, please don't, Bill; I
hain't ever goin' to tell."

And every time he said that the man with the lantern would laugh and say:

"'Deed you AIN'T!  You never said no truer thing 'n that, you bet you."
And once he said:  "Hear him beg! and yit if we hadn't got the best of
him and tied him he'd a killed us both.  And what FOR?  Jist for noth'n.
Jist because we stood on our RIGHTS--that's what for.  But I lay you
ain't a-goin' to threaten nobody any more, Jim Turner.  Put UP that
pistol, Bill."

Bill says:

"I don't want to, Jake Packard.  I'm for killin' him--and didn't he kill
old Hatfield jist the same way--and don't he deserve it?"

"But I don't WANT him killed, and I've got my reasons for it."

"Bless yo' heart for them words, Jake Packard!  I'll never forgit you
long's I live!" says the man on the floor, sort of blubbering.

Packard didn't take no notice of that, but hung up his lantern on a nail
and started towards where I was there in the dark, and motioned Bill to
come.  I crawfished as fast as I could about two yards, but the boat
slanted so that I couldn't make very good time; so to keep from getting
run over and catched I crawled into a stateroom on the upper side.  The
man came a-pawing along in the dark, and when Packard got to my
stateroom, he says:

"Here--come in here."

And in he come, and Bill after him.  But before they got in I was up in
the upper berth, cornered, and sorry I come.  Then they stood there, with
their hands on the ledge of the berth, and talked.  I couldn't see them,
but I could tell where they was by the whisky they'd been having.  I was
glad I didn't drink whisky; but it wouldn't made much difference anyway,
because most of the time they couldn't a treed me because I didn't
breathe.  I was too scared.  And, besides, a body COULDN'T breathe and
hear such talk.  They talked low and earnest.  Bill wanted to kill
Turner.  He says:

"He's said he'll tell, and he will.  If we was to give both our shares to
him NOW it wouldn't make no difference after the row and the way we've
served him.  Shore's you're born, he'll turn State's evidence; now you
hear ME.  I'm for putting him out of his troubles."

"So'm I," says Packard, very quiet.

"Blame it, I'd sorter begun to think you wasn't.  Well, then, that's all
right.  Le's go and do it."

"Hold on a minute; I hain't had my say yit.  You listen to me.
Shooting's good, but there's quieter ways if the thing's GOT to be done.
But what I say is this:  it ain't good sense to go court'n around after a
halter if you can git at what you're up to in some way that's jist as
good and at the same time don't bring you into no resks.  Ain't that so?"

"You bet it is.  But how you goin' to manage it this time?"

"Well, my idea is this:  we'll rustle around and gather up whatever
pickins we've overlooked in the staterooms, and shove for shore and hide
the truck. Then we'll wait.  Now I say it ain't a-goin' to be more'n two
hours befo' this wrack breaks up and washes off down the river.  See?
He'll be drownded, and won't have nobody to blame for it but his own
self.  I reckon that's a considerble sight better 'n killin' of him.  I'm
unfavorable to killin' a man as long as you can git aroun' it; it ain't
good sense, it ain't good morals.  Ain't I right?"

"Yes, I reck'n you are.  But s'pose she DON'T break up and wash off?"

"Well, we can wait the two hours anyway and see, can't we?"

"All right, then; come along."

So they started, and I lit out, all in a cold sweat, and scrambled
forward. It was dark as pitch there; but I said, in a kind of a coarse
whisper, "Jim!" and he answered up, right at my elbow, with a sort of a
moan, and I says:

"Quick, Jim, it ain't no time for fooling around and moaning; there's a
gang of murderers in yonder, and if we don't hunt up their boat and set
her drifting down the river so these fellows can't get away from the
wreck there's one of 'em going to be in a bad fix.  But if we find their
boat we can put ALL of 'em in a bad fix--for the sheriff 'll get 'em.
Quick--hurry!  I'll hunt the labboard side, you hunt the stabboard.
You start at the raft, and--"

"Oh, my lordy, lordy!  RAF'?  Dey ain' no raf' no mo'; she done broke
loose en gone I--en here we is!"




CHAPTER XIII.

WELL, I catched my breath and most fainted.  Shut up on a wreck with such
a gang as that!  But it warn't no time to be sentimentering.  We'd GOT to
find that boat now--had to have it for ourselves.  So we went a-quaking
and shaking down the stabboard side, and slow work it was, too--seemed a
week before we got to the stern.  No sign of a boat.  Jim said he didn't
believe he could go any further--so scared he hadn't hardly any strength
left, he said.  But I said, come on, if we get left on this wreck we are
in a fix, sure.  So on we prowled again.  We struck for the stern of the
texas, and found it, and then scrabbled along forwards on the skylight,
hanging on from shutter to shutter, for the edge of the skylight was in
the water.  When we got pretty close to the cross-hall door there was the
skiff, sure enough!  I could just barely see her.  I felt ever so
thankful.  In another second I would a been aboard of her, but just then
the door opened.  One of the men stuck his head out only about a couple
of foot from me, and I thought I was gone; but he jerked it in again, and
says:

"Heave that blame lantern out o' sight, Bill!"

He flung a bag of something into the boat, and then got in himself and
set down.  It was Packard.  Then Bill HE come out and got in.  Packard
says, in a low voice:

"All ready--shove off!"

I couldn't hardly hang on to the shutters, I was so weak.  But Bill says:

"Hold on--'d you go through him?"

"No.  Didn't you?"

"No.  So he's got his share o' the cash yet."

"Well, then, come along; no use to take truck and leave money."

"Say, won't he suspicion what we're up to?"

"Maybe he won't.  But we got to have it anyway. Come along."

So they got out and went in.

The door slammed to because it was on the careened side; and in a half
second I was in the boat, and Jim come tumbling after me.  I out with my
knife and cut the rope, and away we went!

We didn't touch an oar, and we didn't speak nor whisper, nor hardly even
breathe.  We went gliding swift along, dead silent, past the tip of the
paddle-box, and past the stern; then in a second or two more we was a
hundred yards below the wreck, and the darkness soaked her up, every last
sign of her, and we was safe, and knowed it.

When we was three or four hundred yards down-stream we see the lantern
show like a little spark at the texas door for a second, and we knowed by
that that the rascals had missed their boat, and was beginning to
understand that they was in just as much trouble now as Jim Turner was.

Then Jim manned the oars, and we took out after our raft.  Now was the
first time that I begun to worry about the men--I reckon I hadn't had
time to before.  I begun to think how dreadful it was, even for
murderers, to be in such a fix.  I says to myself, there ain't no telling
but I might come to be a murderer myself yet, and then how would I like
it?  So says I to Jim:

"The first light we see we'll land a hundred yards below it or above it,
in a place where it's a good hiding-place for you and the skiff, and then
I'll go and fix up some kind of a yarn, and get somebody to go for that
gang and get them out of their scrape, so they can be hung when their
time comes."

But that idea was a failure; for pretty soon it begun to storm again, and
this time worse than ever.  The rain poured down, and never a light
showed; everybody in bed, I reckon.  We boomed along down the river,
watching for lights and watching for our raft.  After a long time the
rain let up, but the clouds stayed, and the lightning kept whimpering,
and by and by a flash showed us a black thing ahead, floating, and we
made for it.

It was the raft, and mighty glad was we to get aboard of it again.  We
seen a light now away down to the right, on shore.  So I said I would go
for it. The skiff was half full of plunder which that gang had stole
there on the wreck.  We hustled it on to the raft in a pile, and I told
Jim to float along down, and show a light when he judged he had gone
about two mile, and keep it burning till I come; then I manned my oars
and shoved for the light.  As I got down towards it three or four more
showed--up on a hillside.  It was a village.  I closed in above the shore
light, and laid on my oars and floated.  As I went by I see it was a
lantern hanging on the jackstaff of a double-hull ferryboat.  I skimmed
around for the watchman, a-wondering whereabouts he slept; and by and by
I found him roosting on the bitts forward, with his head down between his
knees.  I gave his shoulder two or three little shoves, and begun to cry.

He stirred up in a kind of a startlish way; but when he see it was only
me he took a good gap and stretch, and then he says:

"Hello, what's up?  Don't cry, bub.  What's the trouble?"

I says:

"Pap, and mam, and sis, and--"

Then I broke down.  He says:

"Oh, dang it now, DON'T take on so; we all has to have our troubles, and
this 'n 'll come out all right.  What's the matter with 'em?"

"They're--they're--are you the watchman of the boat?"

"Yes," he says, kind of pretty-well-satisfied like.  "I'm the captain and
the owner and the mate and the pilot and watchman and head deck-hand; and
sometimes I'm the freight and passengers.  I ain't as rich as old Jim
Hornback, and I can't be so blame' generous and good to Tom, Dick, and
Harry as what he is, and slam around money the way he does; but I've told
him a many a time 't I wouldn't trade places with him; for, says I, a
sailor's life's the life for me, and I'm derned if I'D live two mile out
o' town, where there ain't nothing ever goin' on, not for all his
spondulicks and as much more on top of it.  Says I--"

I broke in and says:

"They're in an awful peck of trouble, and--"

"WHO is?"

"Why, pap and mam and sis and Miss Hooker; and if you'd take your
ferryboat and go up there--"

"Up where?  Where are they?"

"On the wreck."

"What wreck?"

"Why, there ain't but one."

"What, you don't mean the Walter Scott?"

"Yes."

"Good land! what are they doin' THERE, for gracious sakes?"

"Well, they didn't go there a-purpose."

"I bet they didn't!  Why, great goodness, there ain't no chance for 'em
if they don't git off mighty quick!  Why, how in the nation did they ever
git into such a scrape?"

"Easy enough.  Miss Hooker was a-visiting up there to the town--"

"Yes, Booth's Landing--go on."

"She was a-visiting there at Booth's Landing, and just in the edge of the
evening she started over with her nigger woman in the horse-ferry to stay
all night at her friend's house, Miss What-you-may-call-her I disremember
her name--and they lost their steering-oar, and swung around and went
a-floating down, stern first, about two mile, and saddle-baggsed on the
wreck, and the ferryman and the nigger woman and the horses was all lost,
but Miss Hooker she made a grab and got aboard the wreck.  Well, about an
hour after dark we come along down in our trading-scow, and it was so
dark we didn't notice the wreck till we was right on it; and so WE
saddle-baggsed; but all of us was saved but Bill Whipple--and oh, he WAS
the best cretur!--I most wish 't it had been me, I do."

"My George!  It's the beatenest thing I ever struck.  And THEN what did
you all do?"

"Well, we hollered and took on, but it's so wide there we couldn't make
nobody hear.  So pap said somebody got to get ashore and get help
somehow. I was the only one that could swim, so I made a dash for it, and
Miss Hooker she said if I didn't strike help sooner, come here and hunt
up her uncle, and he'd fix the thing.  I made the land about a mile
below, and been fooling along ever since, trying to get people to do
something, but they said, 'What, in such a night and such a current?
There ain't no sense in it; go for the steam ferry.'  Now if you'll go
and--"

"By Jackson, I'd LIKE to, and, blame it, I don't know but I will; but who
in the dingnation's a-going' to PAY for it?  Do you reckon your pap--"

"Why THAT'S all right.  Miss Hooker she tole me, PARTICULAR, that her
uncle Hornback--"

"Great guns! is HE her uncle?  Looky here, you break for that light over
yonder-way, and turn out west when you git there, and about a quarter of
a mile out you'll come to the tavern; tell 'em to dart you out to Jim
Hornback's, and he'll foot the bill.  And don't you fool around any,
because he'll want to know the news.  Tell him I'll have his niece all
safe before he can get to town.  Hump yourself, now; I'm a-going up
around the corner here to roust out my engineer."

I struck for the light, but as soon as he turned the corner I went back
and got into my skiff and bailed her out, and then pulled up shore in the
easy water about six hundred yards, and tucked myself in among some
woodboats; for I couldn't rest easy till I could see the ferryboat start.
But take it all around, I was feeling ruther comfortable on accounts of
taking all this trouble for that gang, for not many would a done it.  I
wished the widow knowed about it.  I judged she would be proud of me for
helping these rapscallions, because rapscallions and dead beats is the
kind the widow and good people takes the most interest in.

Well, before long here comes the wreck, dim and dusky, sliding along
down! A kind of cold shiver went through me, and then I struck out for
her.  She was very deep, and I see in a minute there warn't much chance
for anybody being alive in her.  I pulled all around her and hollered a
little, but there wasn't any answer; all dead still.  I felt a little bit
heavy-hearted about the gang, but not much, for I reckoned if they could
stand it I could.

Then here comes the ferryboat; so I shoved for the middle of the river on
a long down-stream slant; and when I judged I was out of eye-reach I laid
on my oars, and looked back and see her go and smell around the wreck for
Miss Hooker's remainders, because the captain would know her uncle
Hornback would want them; and then pretty soon the ferryboat give it up
and went for the shore, and I laid into my work and went a-booming down
the river.

It did seem a powerful long time before Jim's light showed up; and when
it did show it looked like it was a thousand mile off.  By the time I got
there the sky was beginning to get a little gray in the east; so we
struck for an island, and hid the raft, and sunk the skiff, and turned in
and slept like dead people.




CHAPTER XIV.

BY and by, when we got up, we turned over the truck the gang had stole
off of the wreck, and found boots, and blankets, and clothes, and all
sorts of other things, and a lot of books, and a spyglass, and three
boxes of seegars.  We hadn't ever been this rich before in neither of our
lives.  The seegars was prime.  We laid off all the afternoon in the
woods talking, and me reading the books, and having a general good time.
I told Jim all about what happened inside the wreck and at the ferryboat,
and I said these kinds of things was adventures; but he said he didn't
want no more adventures.  He said that when I went in the texas and he
crawled back to get on the raft and found her gone he nearly died,
because he judged it was all up with HIM anyway it could be fixed; for if
he didn't get saved he would get drownded; and if he did get saved,
whoever saved him would send him back home so as to get the reward, and
then Miss Watson would sell him South, sure.  Well, he was right; he was
most always right; he had an uncommon level head for a nigger.

I read considerable to Jim about kings and dukes and earls and such, and
how gaudy they dressed, and how much style they put on, and called each
other your majesty, and your grace, and your lordship, and so on, 'stead
of mister; and Jim's eyes bugged out, and he was interested.  He says:

"I didn' know dey was so many un um.  I hain't hearn 'bout none un um,
skasely, but ole King Sollermun, onless you counts dem kings dat's in a
pack er k'yards.  How much do a king git?"

"Get?"  I says; "why, they get a thousand dollars a month if they want
it; they can have just as much as they want; everything belongs to them."

"AIN' dat gay?  En what dey got to do, Huck?"

"THEY don't do nothing!  Why, how you talk! They just set around."

"No; is dat so?"

"Of course it is.  They just set around--except, maybe, when there's a
war; then they go to the war.  But other times they just lazy around; or
go hawking--just hawking and sp--Sh!--d' you hear a noise?"

We skipped out and looked; but it warn't nothing but the flutter of a
steamboat's wheel away down, coming around the point; so we come back.

"Yes," says I, "and other times, when things is dull, they fuss with the
parlyment; and if everybody don't go just so he whacks their heads off.
But mostly they hang round the harem."

"Roun' de which?"

"Harem."

"What's de harem?"

"The place where he keeps his wives.  Don't you know about the harem?
Solomon had one; he had about a million wives."

"Why, yes, dat's so; I--I'd done forgot it.  A harem's a bo'd'n-house, I
reck'n.  Mos' likely dey has rackety times in de nussery.  En I reck'n de
wives quarrels considable; en dat 'crease de racket.  Yit dey say
Sollermun de wises' man dat ever live'.  I doan' take no stock in dat.
Bekase why: would a wise man want to live in de mids' er sich a
blim-blammin' all de time?  No--'deed he wouldn't.  A wise man 'ud take
en buil' a biler-factry; en den he could shet DOWN de biler-factry when
he want to res'."

"Well, but he WAS the wisest man, anyway; because the widow she told me
so, her own self."

"I doan k'yer what de widder say, he WARN'T no wise man nuther.  He had
some er de dad-fetchedes' ways I ever see.  Does you know 'bout dat chile
dat he 'uz gwyne to chop in two?"

"Yes, the widow told me all about it."

"WELL, den!  Warn' dat de beatenes' notion in de worl'?  You jes' take en
look at it a minute.  Dah's de stump, dah--dat's one er de women; heah's
you--dat's de yuther one; I's Sollermun; en dish yer dollar bill's de
chile.  Bofe un you claims it.  What does I do?  Does I shin aroun'
mongs' de neighbors en fine out which un you de bill DO b'long to, en
han' it over to de right one, all safe en soun', de way dat anybody dat
had any gumption would?  No; I take en whack de bill in TWO, en give half
un it to you, en de yuther half to de yuther woman.  Dat's de way
Sollermun was gwyne to do wid de chile.  Now I want to ast you:  what's
de use er dat half a bill?--can't buy noth'n wid it.  En what use is a
half a chile?  I wouldn' give a dern for a million un um."

"But hang it, Jim, you've clean missed the point--blame it, you've missed
it a thousand mile."

"Who?  Me?  Go 'long.  Doan' talk to me 'bout yo' pints.  I reck'n I
knows sense when I sees it; en dey ain' no sense in sich doin's as dat.
De 'spute warn't 'bout a half a chile, de 'spute was 'bout a whole chile;
en de man dat think he kin settle a 'spute 'bout a whole chile wid a half
a chile doan' know enough to come in out'n de rain.  Doan' talk to me
'bout Sollermun, Huck, I knows him by de back."

"But I tell you you don't get the point."

"Blame de point!  I reck'n I knows what I knows.  En mine you, de REAL
pint is down furder--it's down deeper.  It lays in de way Sollermun was
raised.  You take a man dat's got on'y one or two chillen; is dat man
gwyne to be waseful o' chillen?  No, he ain't; he can't 'ford it.  HE
know how to value 'em.  But you take a man dat's got 'bout five million
chillen runnin' roun' de house, en it's diffunt.  HE as soon chop a chile
in two as a cat. Dey's plenty mo'.  A chile er two, mo' er less, warn't
no consekens to Sollermun, dad fatch him!"

I never see such a nigger.  If he got a notion in his head once, there
warn't no getting it out again.  He was the most down on Solomon of any
nigger I ever see.  So I went to talking about other kings, and let
Solomon slide.  I told about Louis Sixteenth that got his head cut off in
France long time ago; and about his little boy the dolphin, that would a
been a king, but they took and shut him up in jail, and some say he died
there.

"Po' little chap."

"But some says he got out and got away, and come to America."

"Dat's good!  But he'll be pooty lonesome--dey ain' no kings here, is
dey, Huck?"

"No."

"Den he cain't git no situation.  What he gwyne to do?"

"Well, I don't know.  Some of them gets on the police, and some of them
learns people how to talk French."

"Why, Huck, doan' de French people talk de same way we does?"

"NO, Jim; you couldn't understand a word they said--not a single word."

"Well, now, I be ding-busted!  How do dat come?"

"I don't know; but it's so.  I got some of their jabber out of a book.
S'pose a man was to come to you and say Polly-voo-franzy--what would you
think?"

"I wouldn' think nuff'n; I'd take en bust him over de head--dat is, if he
warn't white.  I wouldn't 'low no nigger to call me dat."

"Shucks, it ain't calling you anything.  It's only saying, do you know
how to talk French?"

"Well, den, why couldn't he SAY it?"

"Why, he IS a-saying it.  That's a Frenchman's WAY of saying it."

"Well, it's a blame ridicklous way, en I doan' want to hear no mo' 'bout
it.  Dey ain' no sense in it."

"Looky here, Jim; does a cat talk like we do?"

"No, a cat don't."

"Well, does a cow?"

"No, a cow don't, nuther."

"Does a cat talk like a cow, or a cow talk like a cat?"

"No, dey don't."

"It's natural and right for 'em to talk different from each other, ain't
it?"

"Course."

"And ain't it natural and right for a cat and a cow to talk different
from US?"

"Why, mos' sholy it is."

"Well, then, why ain't it natural and right for a FRENCHMAN to talk
different from us?  You answer me that."

"Is a cat a man, Huck?"

"No."

"Well, den, dey ain't no sense in a cat talkin' like a man.  Is a cow a
man?--er is a cow a cat?"

"No, she ain't either of them."

"Well, den, she ain't got no business to talk like either one er the
yuther of 'em.  Is a Frenchman a man?"

"Yes."

"WELL, den!  Dad blame it, why doan' he TALK like a man?  You answer me
DAT!"

I see it warn't no use wasting words--you can't learn a nigger to argue.
So I quit.




CHAPTER XV.

WE judged that three nights more would fetch us to Cairo, at the bottom
of Illinois, where the Ohio River comes in, and that was what we was
after.  We would sell the raft and get on a steamboat and go way up the
Ohio amongst the free States, and then be out of trouble.

Well, the second night a fog begun to come on, and we made for a towhead
to tie to, for it wouldn't do to try to run in a fog; but when I paddled
ahead in the canoe, with the line to make fast, there warn't anything but
little saplings to tie to.  I passed the line around one of them right on
the edge of the cut bank, but there was a stiff current, and the raft
come booming down so lively she tore it out by the roots and away she
went.  I see the fog closing down, and it made me so sick and scared I
couldn't budge for most a half a minute it seemed to me--and then there
warn't no raft in sight; you couldn't see twenty yards.  I jumped into
the canoe and run back to the stern, and grabbed the paddle and set her
back a stroke.  But she didn't come.  I was in such a hurry I hadn't
untied her.  I got up and tried to untie her, but I was so excited my
hands shook so I couldn't hardly do anything with them.

As soon as I got started I took out after the raft, hot and heavy, right
down the towhead.  That was all right as far as it went, but the towhead
warn't sixty yards long, and the minute I flew by the foot of it I shot
out into the solid white fog, and hadn't no more idea which way I was
going than a dead man.

Thinks I, it won't do to paddle; first I know I'll run into the bank or a
towhead or something; I got to set still and float, and yet it's mighty
fidgety business to have to hold your hands still at such a time.  I
whooped and listened.  Away down there somewheres I hears a small whoop,
and up comes my spirits.  I went tearing after it, listening sharp to
hear it again.  The next time it come I see I warn't heading for it, but
heading away to the right of it.  And the next time I was heading away to
the left of it--and not gaining on it much either, for I was flying
around, this way and that and t'other, but it was going straight ahead
all the time.

I did wish the fool would think to beat a tin pan, and beat it all the
time, but he never did, and it was the still places between the whoops
that was making the trouble for me.  Well, I fought along, and directly I
hears the whoop BEHIND me.  I was tangled good now.  That was somebody
else's whoop, or else I was turned around.

I throwed the paddle down.  I heard the whoop again; it was behind me
yet, but in a different place; it kept coming, and kept changing its
place, and I kept answering, till by and by it was in front of me again,
and I knowed the current had swung the canoe's head down-stream, and I
was all right if that was Jim and not some other raftsman hollering.  I
couldn't tell nothing about voices in a fog, for nothing don't look
natural nor sound natural in a fog.

The whooping went on, and in about a minute I come a-booming down on a
cut bank with smoky ghosts of big trees on it, and the current throwed me
off to the left and shot by, amongst a lot of snags that fairly roared,
the currrent was tearing by them so swift.

In another second or two it was solid white and still again.  I set
perfectly still then, listening to my heart thump, and I reckon I didn't
draw a breath while it thumped a hundred.

I just give up then.  I knowed what the matter was.  That cut bank was an
island, and Jim had gone down t'other side of it.  It warn't no towhead
that you could float by in ten minutes.  It had the big timber of a
regular island; it might be five or six miles long and more than half a
mile wide.

I kept quiet, with my ears cocked, about fifteen minutes, I reckon.  I
was floating along, of course, four or five miles an hour; but you don't
ever think of that.  No, you FEEL like you are laying dead still on the
water; and if a little glimpse of a snag slips by you don't think to
yourself how fast YOU'RE going, but you catch your breath and think, my!
how that snag's tearing along.  If you think it ain't dismal and lonesome
out in a fog that way by yourself in the night, you try it once--you'll
see.

Next, for about a half an hour, I whoops now and then; at last I hears
the answer a long ways off, and tries to follow it, but I couldn't do it,
and directly I judged I'd got into a nest of towheads, for I had little
dim glimpses of them on both sides of me--sometimes just a narrow channel
between, and some that I couldn't see I knowed was there because I'd hear
the wash of the current against the old dead brush and trash that hung
over the banks.  Well, I warn't long loosing the whoops down amongst the
towheads; and I only tried to chase them a little while, anyway, because
it was worse than chasing a Jack-o'-lantern.  You never knowed a sound
dodge around so, and swap places so quick and so much.

I had to claw away from the bank pretty lively four or five times, to
keep from knocking the islands out of the river; and so I judged the raft
must be butting into the bank every now and then, or else it would get
further ahead and clear out of hearing--it was floating a little faster
than what I was.

Well, I seemed to be in the open river again by and by, but I couldn't
hear no sign of a whoop nowheres.  I reckoned Jim had fetched up on a
snag, maybe, and it was all up with him.  I was good and tired, so I laid
down in the canoe and said I wouldn't bother no more.  I didn't want to
go to sleep, of course; but I was so sleepy I couldn't help it; so I
thought I would take jest one little cat-nap.

But I reckon it was more than a cat-nap, for when I waked up the stars
was shining bright, the fog was all gone, and I was spinning down a big
bend stern first.  First I didn't know where I was; I thought I was
dreaming; and when things began to come back to me they seemed to come up
dim out of last week.

It was a monstrous big river here, with the tallest and the thickest kind
of timber on both banks; just a solid wall, as well as I could see by the
stars.  I looked away down-stream, and seen a black speck on the water.
I took after it; but when I got to it it warn't nothing but a couple of
sawlogs made fast together.  Then I see another speck, and chased that;
then another, and this time I was right.  It was the raft.

When I got to it Jim was setting there with his head down between his
knees, asleep, with his right arm hanging over the steering-oar.  The
other oar was smashed off, and the raft was littered up with leaves and
branches and dirt.  So she'd had a rough time.

I made fast and laid down under Jim's nose on the raft, and began to gap,
and stretch my fists out against Jim, and says:

"Hello, Jim, have I been asleep?  Why didn't you stir me up?"

"Goodness gracious, is dat you, Huck?  En you ain' dead--you ain'
drownded--you's back agin?  It's too good for true, honey, it's too
good for true. Lemme look at you chile, lemme feel o' you.  No, you ain'
dead! you's back agin, 'live en soun', jis de same ole Huck--de same ole
Huck, thanks to goodness!"

"What's the matter with you, Jim?  You been a-drinking?"

"Drinkin'?  Has I ben a-drinkin'?  Has I had a chance to be a-drinkin'?"

"Well, then, what makes you talk so wild?"

"How does I talk wild?"

"HOW?  Why, hain't you been talking about my coming back, and all that
stuff, as if I'd been gone away?"

"Huck--Huck Finn, you look me in de eye; look me in de eye.  HAIN'T you
ben gone away?"

"Gone away?  Why, what in the nation do you mean?  I hain't been gone
anywheres.  Where would I go to?"

"Well, looky here, boss, dey's sumf'n wrong, dey is.  Is I ME, or who IS
I? Is I heah, or whah IS I?  Now dat's what I wants to know."

"Well, I think you're here, plain enough, but I think you're a
tangle-headed old fool, Jim."

"I is, is I?  Well, you answer me dis:  Didn't you tote out de line in de
canoe fer to make fas' to de tow-head?"

"No, I didn't.  What tow-head?  I hain't see no tow-head."

"You hain't seen no towhead?  Looky here, didn't de line pull loose en de
raf' go a-hummin' down de river, en leave you en de canoe behine in de
fog?"

"What fog?"

"Why, de fog!--de fog dat's been aroun' all night.  En didn't you whoop,
en didn't I whoop, tell we got mix' up in de islands en one un us got
los' en t'other one was jis' as good as los', 'kase he didn' know whah he
wuz? En didn't I bust up agin a lot er dem islands en have a turrible
time en mos' git drownded?  Now ain' dat so, boss--ain't it so?  You
answer me dat."

"Well, this is too many for me, Jim.  I hain't seen no fog, nor no
islands, nor no troubles, nor nothing.  I been setting here talking with
you all night till you went to sleep about ten minutes ago, and I reckon
I done the same.  You couldn't a got drunk in that time, so of course
you've been dreaming."

"Dad fetch it, how is I gwyne to dream all dat in ten minutes?"

"Well, hang it all, you did dream it, because there didn't any of it
happen."

"But, Huck, it's all jis' as plain to me as--"

"It don't make no difference how plain it is; there ain't nothing in it.
I know, because I've been here all the time."

Jim didn't say nothing for about five minutes, but set there studying
over it.  Then he says:

"Well, den, I reck'n I did dream it, Huck; but dog my cats ef it ain't de
powerfullest dream I ever see.  En I hain't ever had no dream b'fo' dat's
tired me like dis one."

"Oh, well, that's all right, because a dream does tire a body like
everything sometimes.  But this one was a staving dream; tell me all
about it, Jim."

So Jim went to work and told me the whole thing right through, just as it
happened, only he painted it up considerable.  Then he said he must start
in and "'terpret" it, because it was sent for a warning.  He said the
first towhead stood for a man that would try to do us some good, but the
current was another man that would get us away from him.  The whoops was
warnings that would come to us every now and then, and if we didn't try
hard to make out to understand them they'd just take us into bad luck,
'stead of keeping us out of it.  The lot of towheads was troubles we was
going to get into with quarrelsome people and all kinds of mean folks,
but if we minded our business and didn't talk back and aggravate them, we
would pull through and get out of the fog and into the big clear river,
which was the free States, and wouldn't have no more trouble.

It had clouded up pretty dark just after I got on to the raft, but it was
clearing up again now.

"Oh, well, that's all interpreted well enough as far as it goes, Jim," I
says; "but what does THESE things stand for?"

It was the leaves and rubbish on the raft and the smashed oar.  You could
see them first-rate now.

Jim looked at the trash, and then looked at me, and back at the trash
again.  He had got the dream fixed so strong in his head that he couldn't
seem to shake it loose and get the facts back into its place again right
away.  But when he did get the thing straightened around he looked at me
steady without ever smiling, and says:

"What do dey stan' for?  I'se gwyne to tell you.  When I got all wore out
wid work, en wid de callin' for you, en went to sleep, my heart wuz mos'
broke bekase you wuz los', en I didn' k'yer no' mo' what become er me en
de raf'.  En when I wake up en fine you back agin, all safe en soun', de
tears come, en I could a got down on my knees en kiss yo' foot, I's so
thankful. En all you wuz thinkin' 'bout wuz how you could make a fool uv
ole Jim wid a lie.  Dat truck dah is TRASH; en trash is what people is
dat puts dirt on de head er dey fren's en makes 'em ashamed."

Then he got up slow and walked to the wigwam, and went in there without
saying anything but that.  But that was enough.  It made me feel so mean
I could almost kissed HIS foot to get him to take it back.

It was fifteen minutes before I could work myself up to go and humble
myself to a nigger; but I done it, and I warn't ever sorry for it
afterwards, neither.  I didn't do him no more mean tricks, and I wouldn't
done that one if I'd a knowed it would make him feel that way.




CHAPTER XVI.

WE slept most all day, and started out at night, a little ways behind a
monstrous long raft that was as long going by as a procession.  She had
four long sweeps at each end, so we judged she carried as many as thirty
men, likely.  She had five big wigwams aboard, wide apart, and an open
camp fire in the middle, and a tall flag-pole at each end.  There was a
power of style about her.  It AMOUNTED to something being a raftsman on
such a craft as that.

We went drifting down into a big bend, and the night clouded up and got
hot.  The river was very wide, and was walled with solid timber on both
sides; you couldn't see a break in it hardly ever, or a light.  We talked
about Cairo, and wondered whether we would know it when we got to it.  I
said likely we wouldn't, because I had heard say there warn't but about a
dozen houses there, and if they didn't happen to have them lit up, how
was we going to know we was passing a town?  Jim said if the two big
rivers joined together there, that would show.  But I said maybe we might
think we was passing the foot of an island and coming into the same old
river again. That disturbed Jim--and me too.  So the question was, what
to do?  I said, paddle ashore the first time a light showed, and tell
them pap was behind, coming along with a trading-scow, and was a green
hand at the business, and wanted to know how far it was to Cairo.  Jim
thought it was a good idea, so we took a smoke on it and waited.

There warn't nothing to do now but to look out sharp for the town, and
not pass it without seeing it.  He said he'd be mighty sure to see it,
because he'd be a free man the minute he seen it, but if he missed it
he'd be in a slave country again and no more show for freedom.  Every
little while he jumps up and says:

"Dah she is?"

But it warn't.  It was Jack-o'-lanterns, or lightning bugs; so he set
down again, and went to watching, same as before.  Jim said it made him
all over trembly and feverish to be so close to freedom.  Well, I can
tell you it made me all over trembly and feverish, too, to hear him,
because I begun to get it through my head that he WAS most free--and who
was to blame for it?  Why, ME.  I couldn't get that out of my conscience,
no how nor no way. It got to troubling me so I couldn't rest; I couldn't
stay still in one place.  It hadn't ever come home to me before, what
this thing was that I was doing.  But now it did; and it stayed with me,
and scorched me more and more.  I tried to make out to myself that I
warn't to blame, because I didn't run Jim off from his rightful owner;
but it warn't no use, conscience up and says, every time, "But you knowed
he was running for his freedom, and you could a paddled ashore and told
somebody."  That was so--I couldn't get around that noway.  That was
where it pinched.  Conscience says to me, "What had poor Miss Watson done
to you that you could see her nigger go off right under your eyes and
never say one single word?  What did that poor old woman do to you that
you could treat her so mean?  Why, she tried to learn you your book, she
tried to learn you your manners, she tried to be good to you every way
she knowed how.  THAT'S what she done."

I got to feeling so mean and so miserable I most wished I was dead.  I
fidgeted up and down the raft, abusing myself to myself, and Jim was
fidgeting up and down past me.  We neither of us could keep still.  Every
time he danced around and says, "Dah's Cairo!" it went through me like a
shot, and I thought if it WAS Cairo I reckoned I would die of
miserableness.

Jim talked out loud all the time while I was talking to myself.  He was
saying how the first thing he would do when he got to a free State he
would go to saving up money and never spend a single cent, and when he
got enough he would buy his wife, which was owned on a farm close to
where Miss Watson lived; and then they would both work to buy the two
children, and if their master wouldn't sell them, they'd get an
Ab'litionist to go and steal them.

It most froze me to hear such talk.  He wouldn't ever dared to talk such
talk in his life before.  Just see what a difference it made in him the
minute he judged he was about free.  It was according to the old saying,
"Give a nigger an inch and he'll take an ell."  Thinks I, this is what
comes of my not thinking.  Here was this nigger, which I had as good as
helped to run away, coming right out flat-footed and saying he would
steal his children--children that belonged to a man I didn't even know; a
man that hadn't ever done me no harm.

I was sorry to hear Jim say that, it was such a lowering of him.  My
conscience got to stirring me up hotter than ever, until at last I says
to it, "Let up on me--it ain't too late yet--I'll paddle ashore at the
first light and tell."  I felt easy and happy and light as a feather
right off.  All my troubles was gone.  I went to looking out sharp for a
light, and sort of singing to myself.  By and by one showed.  Jim sings
out:

"We's safe, Huck, we's safe!  Jump up and crack yo' heels!  Dat's de good
ole Cairo at las', I jis knows it!"

I says:

"I'll take the canoe and go and see, Jim.  It mightn't be, you know."

He jumped and got the canoe ready, and put his old coat in the bottom for
me to set on, and give me the paddle; and as I shoved off, he says:

"Pooty soon I'll be a-shout'n' for joy, en I'll say, it's all on accounts
o' Huck; I's a free man, en I couldn't ever ben free ef it hadn' ben for
Huck; Huck done it.  Jim won't ever forgit you, Huck; you's de bes' fren'
Jim's ever had; en you's de ONLY fren' ole Jim's got now."

I was paddling off, all in a sweat to tell on him; but when he says this,
it seemed to kind of take the tuck all out of me.  I went along slow
then, and I warn't right down certain whether I was glad I started or
whether I warn't.  When I was fifty yards off, Jim says:

"Dah you goes, de ole true Huck; de on'y white genlman dat ever kep' his
promise to ole Jim."

Well, I just felt sick.  But I says, I GOT to do it--I can't get OUT of
it.  Right then along comes a skiff with two men in it with guns, and
they stopped and I stopped.  One of them says:

"What's that yonder?"

"A piece of a raft," I says.

"Do you belong on it?"

"Yes, sir."

"Any men on it?"

"Only one, sir."

"Well, there's five niggers run off to-night up yonder, above the head of
the bend.  Is your man white or black?"

I didn't answer up prompt.  I tried to, but the words wouldn't come. I
tried for a second or two to brace up and out with it, but I warn't man
enough--hadn't the spunk of a rabbit.  I see I was weakening; so I just
give up trying, and up and says:

"He's white."

"I reckon we'll go and see for ourselves."

"I wish you would," says I, "because it's pap that's there, and maybe
you'd help me tow the raft ashore where the light is.  He's sick--and so
is mam and Mary Ann."

"Oh, the devil! we're in a hurry, boy.  But I s'pose we've got to.  Come,
buckle to your paddle, and let's get along."

I buckled to my paddle and they laid to their oars.  When we had made a
stroke or two, I says:

"Pap'll be mighty much obleeged to you, I can tell you.  Everybody goes
away when I want them to help me tow the raft ashore, and I can't do it
by myself."

"Well, that's infernal mean.  Odd, too.  Say, boy, what's the matter with
your father?"

"It's the--a--the--well, it ain't anything much."

They stopped pulling.  It warn't but a mighty little ways to the raft
now. One says:

"Boy, that's a lie.  What IS the matter with your pap?  Answer up square
now, and it'll be the better for you."

"I will, sir, I will, honest--but don't leave us, please.  It's the--the
--Gentlemen, if you'll only pull ahead, and let me heave you the
headline, you won't have to come a-near the raft--please do."

"Set her back, John, set her back!" says one.  They backed water.  "Keep
away, boy--keep to looard.  Confound it, I just expect the wind has
blowed it to us.  Your pap's got the small-pox, and you know it precious
well.  Why didn't you come out and say so?  Do you want to spread it all
over?"

"Well," says I, a-blubbering, "I've told everybody before, and they just
went away and left us."

"Poor devil, there's something in that.  We are right down sorry for you,
but we--well, hang it, we don't want the small-pox, you see.  Look here,
I'll tell you what to do.  Don't you try to land by yourself, or you'll
smash everything to pieces.  You float along down about twenty miles, and
you'll come to a town on the left-hand side of the river.  It will be
long after sun-up then, and when you ask for help you tell them your
folks are all down with chills and fever.  Don't be a fool again, and let
people guess what is the matter.  Now we're trying to do you a kindness;
so you just put twenty miles between us, that's a good boy.  It wouldn't
do any good to land yonder where the light is--it's only a wood-yard.
Say, I reckon your father's poor, and I'm bound to say he's in pretty
hard luck.  Here, I'll put a twenty-dollar gold piece on this board, and
you get it when it floats by.  I feel mighty mean to leave you; but my
kingdom! it won't do to fool with small-pox, don't you see?"

"Hold on, Parker," says the other man, "here's a twenty to put on the
board for me.  Good-bye, boy; you do as Mr. Parker told you, and you'll
be all right."

"That's so, my boy--good-bye, good-bye.  If you see any runaway niggers
you get help and nab them, and you can make some money by it."

"Good-bye, sir," says I; "I won't let no runaway niggers get by me if I
can help it."

They went off and I got aboard the raft, feeling bad and low, because I
knowed very well I had done wrong, and I see it warn't no use for me to
try to learn to do right; a body that don't get STARTED right when he's
little ain't got no show--when the pinch comes there ain't nothing to
back him up and keep him to his work, and so he gets beat.  Then I
thought a minute, and says to myself, hold on; s'pose you'd a done right
and give Jim up, would you felt better than what you do now?  No, says I,
I'd feel bad--I'd feel just the same way I do now.  Well, then, says I,
what's the use you learning to do right when it's troublesome to do right
and ain't no trouble to do wrong, and the wages is just the same?  I was
stuck.  I couldn't answer that.  So I reckoned I wouldn't bother no more
about it, but after this always do whichever come handiest at the time.

I went into the wigwam; Jim warn't there.  I looked all around; he warn't
anywhere.  I says:

"Jim!"

"Here I is, Huck.  Is dey out o' sight yit?  Don't talk loud."

He was in the river under the stern oar, with just his nose out.  I told
him they were out of sight, so he come aboard.  He says:

"I was a-listenin' to all de talk, en I slips into de river en was gwyne
to shove for sho' if dey come aboard.  Den I was gwyne to swim to de raf'
agin when dey was gone.  But lawsy, how you did fool 'em, Huck!  Dat WUZ
de smartes' dodge!  I tell you, chile, I'spec it save' ole Jim--ole Jim
ain't going to forgit you for dat, honey."

Then we talked about the money.  It was a pretty good raise--twenty
dollars apiece.  Jim said we could take deck passage on a steamboat now,
and the money would last us as far as we wanted to go in the free States.
He said twenty mile more warn't far for the raft to go, but he wished we
was already there.

Towards daybreak we tied up, and Jim was mighty particular about hiding
the raft good.  Then he worked all day fixing things in bundles, and
getting all ready to quit rafting.

That night about ten we hove in sight of the lights of a town away down
in a left-hand bend.

I went off in the canoe to ask about it.  Pretty soon I found a man out
in the river with a skiff, setting a trot-line.  I ranged up and says:

"Mister, is that town Cairo?"

"Cairo? no.  You must be a blame' fool."

"What town is it, mister?"

"If you want to know, go and find out.  If you stay here botherin' around
me for about a half a minute longer you'll get something you won't want."

I paddled to the raft.  Jim was awful disappointed, but I said never
mind, Cairo would be the next place, I reckoned.

We passed another town before daylight, and I was going out again; but it
was high ground, so I didn't go.  No high ground about Cairo, Jim said.
I had forgot it.  We laid up for the day on a towhead tolerable close to
the left-hand bank.  I begun to suspicion something.  So did Jim.  I
says:

"Maybe we went by Cairo in the fog that night."

He says:

"Doan' le's talk about it, Huck.  Po' niggers can't have no luck.  I
awluz 'spected dat rattlesnake-skin warn't done wid its work."

"I wish I'd never seen that snake-skin, Jim--I do wish I'd never laid
eyes on it."

"It ain't yo' fault, Huck; you didn' know.  Don't you blame yo'self 'bout
it."

When it was daylight, here was the clear Ohio water inshore, sure enough,
and outside was the old regular Muddy!  So it was all up with Cairo.

We talked it all over.  It wouldn't do to take to the shore; we couldn't
take the raft up the stream, of course.  There warn't no way but to wait
for dark, and start back in the canoe and take the chances.  So we slept
all day amongst the cottonwood thicket, so as to be fresh for the work,
and when we went back to the raft about dark the canoe was gone!

We didn't say a word for a good while.  There warn't anything to say.  We
both knowed well enough it was some more work of the rattlesnake-skin; so
what was the use to talk about it?  It would only look like we was
finding fault, and that would be bound to fetch more bad luck--and keep
on fetching it, too, till we knowed enough to keep still.

By and by we talked about what we better do, and found there warn't no
way but just to go along down with the raft till we got a chance to buy a
canoe to go back in.  We warn't going to borrow it when there warn't
anybody around, the way pap would do, for that might set people after us.

So we shoved out after dark on the raft.

Anybody that don't believe yet that it's foolishness to handle a
snake-skin, after all that that snake-skin done for us, will believe
it now if they read on and see what more it done for us.

The place to buy canoes is off of rafts laying up at shore.  But we
didn't see no rafts laying up; so we went along during three hours and
more.  Well, the night got gray and ruther thick, which is the next
meanest thing to fog.  You can't tell the shape of the river, and you
can't see no distance. It got to be very late and still, and then along
comes a steamboat up the river.  We lit the lantern, and judged she would
see it.  Up-stream boats didn't generly come close to us; they go out and
follow the bars and hunt for easy water under the reefs; but nights like
this they bull right up the channel against the whole river.

We could hear her pounding along, but we didn't see her good till she was
close.  She aimed right for us.  Often they do that and try to see how
close they can come without touching; sometimes the wheel bites off a
sweep, and then the pilot sticks his head out and laughs, and thinks he's
mighty smart.  Well, here she comes, and we said she was going to try and
shave us; but she didn't seem to be sheering off a bit.  She was a big
one, and she was coming in a hurry, too, looking like a black cloud with
rows of glow-worms around it; but all of a sudden she bulged out, big and
scary, with a long row of wide-open furnace doors shining like red-hot
teeth, and her monstrous bows and guards hanging right over us.  There
was a yell at us, and a jingling of bells to stop the engines, a powwow
of cussing, and whistling of steam--and as Jim went overboard on one side
and I on the other, she come smashing straight through the raft.

I dived--and I aimed to find the bottom, too, for a thirty-foot wheel had
got to go over me, and I wanted it to have plenty of room.  I could
always stay under water a minute; this time I reckon I stayed under a
minute and a half.  Then I bounced for the top in a hurry, for I was
nearly busting.  I popped out to my armpits and blowed the water out of
my nose, and puffed a bit.  Of course there was a booming current; and of
course that boat started her engines again ten seconds after she stopped
them, for they never cared much for raftsmen; so now she was churning
along up the river, out of sight in the thick weather, though I could
hear her.

I sung out for Jim about a dozen times, but I didn't get any answer; so I
grabbed a plank that touched me while I was "treading water," and struck
out for shore, shoving it ahead of me.  But I made out to see that the
drift of the current was towards the left-hand shore, which meant that I
was in a crossing; so I changed off and went that way.

It was one of these long, slanting, two-mile crossings; so I was a good
long time in getting over.  I made a safe landing, and clumb up the bank.
I couldn't see but a little ways, but I went poking along over rough
ground for a quarter of a mile or more, and then I run across a big
old-fashioned double log-house before I noticed it.  I was going to rush
by and get away, but a lot of dogs jumped out and went to howling and
barking at me, and I knowed better than to move another peg.




CHAPTER XVII.

IN about a minute somebody spoke out of a window without putting his head
out, and says:

"Be done, boys!  Who's there?"

I says:

"It's me."

"Who's me?"

"George Jackson, sir."

"What do you want?"

"I don't want nothing, sir.  I only want to go along by, but the dogs
won't let me."

"What are you prowling around here this time of night for--hey?"

"I warn't prowling around, sir, I fell overboard off of the steamboat."

"Oh, you did, did you?  Strike a light there, somebody.  What did you say
your name was?"

"George Jackson, sir.  I'm only a boy."

"Look here, if you're telling the truth you needn't be afraid--nobody'll
hurt you.  But don't try to budge; stand right where you are.  Rouse out
Bob and Tom, some of you, and fetch the guns.  George Jackson, is there
anybody with you?"

"No, sir, nobody."

I heard the people stirring around in the house now, and see a light.
The man sung out:

"Snatch that light away, Betsy, you old fool--ain't you got any sense?
Put it on the floor behind the front door.  Bob, if you and Tom are
ready, take your places."

"All ready."

"Now, George Jackson, do you know the Shepherdsons?"

"No, sir; I never heard of them."

"Well, that may be so, and it mayn't.  Now, all ready.  Step forward,
George Jackson.  And mind, don't you hurry--come mighty slow.  If there's
anybody with you, let him keep back--if he shows himself he'll be shot.
Come along now.  Come slow; push the door open yourself--just enough to
squeeze in, d' you hear?"

I didn't hurry; I couldn't if I'd a wanted to.  I took one slow step at a
time and there warn't a sound, only I thought I could hear my heart.  The
dogs were as still as the humans, but they followed a little behind me.
When I got to the three log doorsteps I heard them unlocking and
unbarring and unbolting.  I put my hand on the door and pushed it a
little and a little more till somebody said, "There, that's enough--put
your head in." I done it, but I judged they would take it off.

The candle was on the floor, and there they all was, looking at me, and
me at them, for about a quarter of a minute:  Three big men with guns
pointed at me, which made me wince, I tell you; the oldest, gray and
about sixty, the other two thirty or more--all of them fine and handsome
--and the sweetest old gray-headed lady, and back of her two young women
which I couldn't see right well.  The old gentleman says:

"There; I reckon it's all right.  Come in."

As soon as I was in the old gentleman he locked the door and barred it
and bolted it, and told the young men to come in with their guns, and
they all went in a big parlor that had a new rag carpet on the floor, and
got together in a corner that was out of the range of the front windows
--there warn't none on the side.  They held the candle, and took a good
look at me, and all said, "Why, HE ain't a Shepherdson--no, there ain't
any Shepherdson about him."  Then the old man said he hoped I wouldn't
mind being searched for arms, because he didn't mean no harm by it--it
was only to make sure.  So he didn't pry into my pockets, but only felt
outside with his hands, and said it was all right.  He told me to make
myself easy and at home, and tell all about myself; but the old lady
says:

"Why, bless you, Saul, the poor thing's as wet as he can be; and don't
you reckon it may be he's hungry?"

"True for you, Rachel--I forgot."

So the old lady says:

"Betsy" (this was a nigger woman), "you fly around and get him something
to eat as quick as you can, poor thing; and one of you girls go and wake
up Buck and tell him--oh, here he is himself.  Buck, take this little
stranger and get the wet clothes off from him and dress him up in some of
yours that's dry."

Buck looked about as old as me--thirteen or fourteen or along there,
though he was a little bigger than me.  He hadn't on anything but a
shirt, and he was very frowzy-headed.  He came in gaping and digging one
fist into his eyes, and he was dragging a gun along with the other one.
He says:

"Ain't they no Shepherdsons around?"

They said, no, 'twas a false alarm.

"Well," he says, "if they'd a ben some, I reckon I'd a got one."

They all laughed, and Bob says:

"Why, Buck, they might have scalped us all, you've been so slow in
coming."

"Well, nobody come after me, and it ain't right I'm always kept down; I
don't get no show."

"Never mind, Buck, my boy," says the old man, "you'll have show enough,
all in good time, don't you fret about that.  Go 'long with you now, and
do as your mother told you."

When we got up-stairs to his room he got me a coarse shirt and a
roundabout and pants of his, and I put them on.  While I was at it he
asked me what my name was, but before I could tell him he started to tell
me about a bluejay and a young rabbit he had catched in the woods day
before yesterday, and he asked me where Moses was when the candle went
out.  I said I didn't know; I hadn't heard about it before, no way.

"Well, guess," he says.

"How'm I going to guess," says I, "when I never heard tell of it before?"

"But you can guess, can't you?  It's just as easy."

"WHICH candle?"  I says.

"Why, any candle," he says.

"I don't know where he was," says I; "where was he?"

"Why, he was in the DARK!  That's where he was!"

"Well, if you knowed where he was, what did you ask me for?"

"Why, blame it, it's a riddle, don't you see?  Say, how long are you
going to stay here?  You got to stay always.  We can just have booming
times--they don't have no school now.  Do you own a dog?  I've got a
dog--and he'll go in the river and bring out chips that you throw in.  Do
you like to comb up Sundays, and all that kind of foolishness?  You bet I
don't, but ma she makes me.  Confound these ole britches!  I reckon I'd
better put 'em on, but I'd ruther not, it's so warm.  Are you all ready?
All right.  Come along, old hoss."

Cold corn-pone, cold corn-beef, butter and buttermilk--that is what they
had for me down there, and there ain't nothing better that ever I've come
across yet.  Buck and his ma and all of them smoked cob pipes, except the
nigger woman, which was gone, and the two young women.  They all smoked
and talked, and I eat and talked.  The young women had quilts around
them, and their hair down their backs.  They all asked me questions, and
I told them how pap and me and all the family was living on a little farm
down at the bottom of Arkansaw, and my sister Mary Ann run off and got
married and never was heard of no more, and Bill went to hunt them and he
warn't heard of no more, and Tom and Mort died, and then there warn't
nobody but just me and pap left, and he was just trimmed down to nothing,
on account of his troubles; so when he died I took what there was left,
because the farm didn't belong to us, and started up the river, deck
passage, and fell overboard; and that was how I come to be here.  So they
said I could have a home there as long as I wanted it.  Then it was most
daylight and everybody went to bed, and I went to bed with Buck, and when
I waked up in the morning, drat it all, I had forgot what my name was.
So I laid there about an hour trying to think, and when Buck waked up I
says:

"Can you spell, Buck?"

"Yes," he says.

"I bet you can't spell my name," says I.

"I bet you what you dare I can," says he.

"All right," says I, "go ahead."

"G-e-o-r-g-e J-a-x-o-n--there now," he says.

"Well," says I, "you done it, but I didn't think you could.  It ain't no
slouch of a name to spell--right off without studying."

I set it down, private, because somebody might want ME to spell it next,
and so I wanted to be handy with it and rattle it off like I was used to
it.

It was a mighty nice family, and a mighty nice house, too.  I hadn't seen
no house out in the country before that was so nice and had so much
style.  It didn't have an iron latch on the front door, nor a wooden one
with a buckskin string, but a brass knob to turn, the same as houses in
town. There warn't no bed in the parlor, nor a sign of a bed; but heaps
of parlors in towns has beds in them.  There was a big fireplace that was
bricked on the bottom, and the bricks was kept clean and red by pouring
water on them and scrubbing them with another brick; sometimes they wash
them over with red water-paint that they call Spanish-brown, same as they
do in town.  They had big brass dog-irons that could hold up a saw-log.
There was a clock on the middle of the mantelpiece, with a picture of a
town painted on the bottom half of the glass front, and a round place in
the middle of it for the sun, and you could see the pendulum swinging
behind it.  It was beautiful to hear that clock tick; and sometimes when
one of these peddlers had been along and scoured her up and got her in
good shape, she would start in and strike a hundred and fifty before she
got tuckered out.  They wouldn't took any money for her.

Well, there was a big outlandish parrot on each side of the clock, made
out of something like chalk, and painted up gaudy.  By one of the parrots
was a cat made of crockery, and a crockery dog by the other; and when you
pressed down on them they squeaked, but didn't open their mouths nor look
different nor interested.  They squeaked through underneath.  There was a
couple of big wild-turkey-wing fans spread out behind those things.  On
the table in the middle of the room was a kind of a lovely crockery
basket that had apples and oranges and peaches and grapes piled up in it,
which was much redder and yellower and prettier than real ones is, but
they warn't real because you could see where pieces had got chipped off
and showed the white chalk, or whatever it was, underneath.

This table had a cover made out of beautiful oilcloth, with a red and
blue spread-eagle painted on it, and a painted border all around.  It
come all the way from Philadelphia, they said.  There was some books,
too, piled up perfectly exact, on each corner of the table.  One was a
big family Bible full of pictures.  One was Pilgrim's Progress, about a
man that left his family, it didn't say why.  I read considerable in it
now and then.  The statements was interesting, but tough.  Another was
Friendship's Offering, full of beautiful stuff and poetry; but I didn't
read the poetry.  Another was Henry Clay's Speeches, and another was Dr.
Gunn's Family Medicine, which told you all about what to do if a body was
sick or dead.  There was a hymn book, and a lot of other books.  And
there was nice split-bottom chairs, and perfectly sound, too--not bagged
down in the middle and busted, like an old basket.

They had pictures hung on the walls--mainly Washingtons and Lafayettes,
and battles, and Highland Marys, and one called "Signing the
Declaration." There was some that they called crayons, which one of the
daughters which was dead made her own self when she was only fifteen
years old.  They was different from any pictures I ever see before
--blacker, mostly, than is common.  One was a woman in a slim black dress,
belted small under the armpits, with bulges like a cabbage in the middle
of the sleeves, and a large black scoop-shovel bonnet with a black veil,
and white slim ankles crossed about with black tape, and very wee black
slippers, like a chisel, and she was leaning pensive on a tombstone on
her right elbow, under a weeping willow, and her other hand hanging down
her side holding a white handkerchief and a reticule, and underneath the
picture it said "Shall I Never See Thee More Alas."  Another one was a
young lady with her hair all combed up straight to the top of her head,
and knotted there in front of a comb like a chair-back, and she was
crying into a handkerchief and had a dead bird laying on its back in her
other hand with its heels up, and underneath the picture it said "I Shall
Never Hear Thy Sweet Chirrup More Alas."  There was one where a young
lady was at a window looking up at the moon, and tears running down her
cheeks; and she had an open letter in one hand with black sealing wax
showing on one edge of it, and she was mashing a locket with a chain to
it against her mouth, and underneath the picture it said "And Art Thou
Gone Yes Thou Art Gone Alas."  These was all nice pictures, I reckon, but
I didn't somehow seem to take to them, because if ever I was down a
little they always give me the fan-tods.  Everybody was sorry she died,
because she had laid out a lot more of these pictures to do, and a body
could see by what she had done what they had lost.  But I reckoned that
with her disposition she was having a better time in the graveyard.  She
was at work on what they said was her greatest picture when she took
sick, and every day and every night it was her prayer to be allowed to
live till she got it done, but she never got the chance.  It was a
picture of a young woman in a long white gown, standing on the rail of a
bridge all ready to jump off, with her hair all down her back, and
looking up to the moon, with the tears running down her face, and she had
two arms folded across her breast, and two arms stretched out in front,
and two more reaching up towards the moon--and the idea was to see which
pair would look best, and then scratch out all the other arms; but, as I
was saying, she died before she got her mind made up, and now they kept
this picture over the head of the bed in her room, and every time her
birthday come they hung flowers on it.  Other times it was hid with a
little curtain.  The young woman in the picture had a kind of a nice
sweet face, but there was so many arms it made her look too spidery,
seemed to me.

This young girl kept a scrap-book when she was alive, and used to paste
obituaries and accidents and cases of patient suffering in it out of the
Presbyterian Observer, and write poetry after them out of her own head.
It was very good poetry.  This is what she wrote about a boy by the name
of Stephen Dowling Bots that fell down a well and was drownded:

ODE TO STEPHEN DOWLING BOTS, DEC'D

And did young Stephen sicken, And did young Stephen die? And did the sad
hearts thicken, And did the mourners cry?

No; such was not the fate of Young Stephen Dowling Bots; Though sad
hearts round him thickened, 'Twas not from sickness' shots.

No whooping-cough did rack his frame, Nor measles drear with spots; Not
these impaired the sacred name Of Stephen Dowling Bots.

Despised love struck not with woe That head of curly knots, Nor stomach
troubles laid him low, Young Stephen Dowling Bots.

O no.  Then list with tearful eye, Whilst I his fate do tell. His soul
did from this cold world fly By falling down a well.

They got him out and emptied him; Alas it was too late; His spirit was
gone for to sport aloft In the realms of the good and great.

If Emmeline Grangerford could make poetry like that before she was
fourteen, there ain't no telling what she could a done by and by.  Buck
said she could rattle off poetry like nothing.  She didn't ever have to
stop to think.  He said she would slap down a line, and if she couldn't
find anything to rhyme with it would just scratch it out and slap down
another one, and go ahead. She warn't particular; she could write about
anything you choose to give her to write about just so it was sadful.
Every time a man died, or a woman died, or a child died, she would be on
hand with her "tribute" before he was cold.  She called them tributes.
The neighbors said it was the doctor first, then Emmeline, then the
undertaker--the undertaker never got in ahead of Emmeline but once, and
then she hung fire on a rhyme for the dead person's name, which was
Whistler.  She warn't ever the same after that; she never complained, but
she kinder pined away and did not live long.  Poor thing, many's the time
I made myself go up to the little room that used to be hers and get out
her poor old scrap-book and read in it when her pictures had been
aggravating me and I had soured on her a little.  I liked all that
family, dead ones and all, and warn't going to let anything come between
us.  Poor Emmeline made poetry about all the dead people when she was
alive, and it didn't seem right that there warn't nobody to make some
about her now she was gone; so I tried to sweat out a verse or two
myself, but I couldn't seem to make it go somehow.  They kept Emmeline's
room trim and nice, and all the things fixed in it just the way she liked
to have them when she was alive, and nobody ever slept there.  The old
lady took care of the room herself, though there was plenty of niggers,
and she sewed there a good deal and read her Bible there mostly.

Well, as I was saying about the parlor, there was beautiful curtains on
the windows:  white, with pictures painted on them of castles with vines
all down the walls, and cattle coming down to drink.  There was a little
old piano, too, that had tin pans in it, I reckon, and nothing was ever
so lovely as to hear the young ladies sing "The Last Link is Broken" and
play "The Battle of Prague" on it.  The walls of all the rooms was
plastered, and most had carpets on the floors, and the whole house was
whitewashed on the outside.

It was a double house, and the big open place betwixt them was roofed and
floored, and sometimes the table was set there in the middle of the day,
and it was a cool, comfortable place.  Nothing couldn't be better.  And
warn't the cooking good, and just bushels of it too!




CHAPTER XVIII.

COL.  GRANGERFORD was a gentleman, you see.  He was a gentleman all over;
and so was his family.  He was well born, as the saying is, and that's
worth as much in a man as it is in a horse, so the Widow Douglas said,
and nobody ever denied that she was of the first aristocracy in our town;
and pap he always said it, too, though he warn't no more quality than a
mudcat himself.  Col.  Grangerford was very tall and very slim, and had a
darkish-paly complexion, not a sign of red in it anywheres; he was clean
shaved every morning all over his thin face, and he had the thinnest kind
of lips, and the thinnest kind of nostrils, and a high nose, and heavy
eyebrows, and the blackest kind of eyes, sunk so deep back that they
seemed like they was looking out of caverns at you, as you may say.  His
forehead was high, and his hair was black and straight and hung to his
shoulders. His hands was long and thin, and every day of his life he put
on a clean shirt and a full suit from head to foot made out of linen so
white it hurt your eyes to look at it; and on Sundays he wore a blue
tail-coat with brass buttons on it.  He carried a mahogany cane with a
silver head to it.  There warn't no frivolishness about him, not a bit,
and he warn't ever loud.  He was as kind as he could be--you could feel
that, you know, and so you had confidence.  Sometimes he smiled, and it
was good to see; but when he straightened himself up like a liberty-pole,
and the lightning begun to flicker out from under his eyebrows, you
wanted to climb a tree first, and find out what the matter was
afterwards.  He didn't ever have to tell anybody to mind their manners
--everybody was always good-mannered where he was.  Everybody loved to have
him around, too; he was sunshine most always--I mean he made it seem
like good weather.  When he turned into a cloudbank it was awful dark for
half a minute, and that was enough; there wouldn't nothing go wrong again
for a week.

When him and the old lady come down in the morning all the family got up
out of their chairs and give them good-day, and didn't set down again
till they had set down.  Then Tom and Bob went to the sideboard where the
decanter was, and mixed a glass of bitters and handed it to him, and he
held it in his hand and waited till Tom's and Bob's was mixed, and then
they bowed and said, "Our duty to you, sir, and madam;" and THEY bowed
the least bit in the world and said thank you, and so they drank, all
three, and Bob and Tom poured a spoonful of water on the sugar and the
mite of whisky or apple brandy in the bottom of their tumblers, and give
it to me and Buck, and we drank to the old people too.

Bob was the oldest and Tom next--tall, beautiful men with very broad
shoulders and brown faces, and long black hair and black eyes.  They
dressed in white linen from head to foot, like the old gentleman, and
wore broad Panama hats.

Then there was Miss Charlotte; she was twenty-five, and tall and proud
and grand, but as good as she could be when she warn't stirred up; but
when she was she had a look that would make you wilt in your tracks, like
her father.  She was beautiful.

So was her sister, Miss Sophia, but it was a different kind.  She was
gentle and sweet like a dove, and she was only twenty.

Each person had their own nigger to wait on them--Buck too.  My nigger
had a monstrous easy time, because I warn't used to having anybody do
anything for me, but Buck's was on the jump most of the time.

This was all there was of the family now, but there used to be more
--three sons; they got killed; and Emmeline that died.

The old gentleman owned a lot of farms and over a hundred niggers.
Sometimes a stack of people would come there, horseback, from ten or
fifteen mile around, and stay five or six days, and have such junketings
round about and on the river, and dances and picnics in the woods
daytimes, and balls at the house nights.  These people was mostly
kinfolks of the family.  The men brought their guns with them.  It was a
handsome lot of quality, I tell you.

There was another clan of aristocracy around there--five or six families
--mostly of the name of Shepherdson.  They was as high-toned and well
born and rich and grand as the tribe of Grangerfords.  The Shepherdsons
and Grangerfords used the same steamboat landing, which was about two
mile above our house; so sometimes when I went up there with a lot of our
folks I used to see a lot of the Shepherdsons there on their fine horses.

One day Buck and me was away out in the woods hunting, and heard a horse
coming.  We was crossing the road.  Buck says:

"Quick!  Jump for the woods!"

We done it, and then peeped down the woods through the leaves.  Pretty
soon a splendid young man come galloping down the road, setting his horse
easy and looking like a soldier.  He had his gun across his pommel.  I
had seen him before.  It was young Harney Shepherdson.  I heard Buck's
gun go off at my ear, and Harney's hat tumbled off from his head.  He
grabbed his gun and rode straight to the place where we was hid.  But we
didn't wait.  We started through the woods on a run.  The woods warn't
thick, so I looked over my shoulder to dodge the bullet, and twice I seen
Harney cover Buck with his gun; and then he rode away the way he come--to
get his hat, I reckon, but I couldn't see.  We never stopped running till
we got home.  The old gentleman's eyes blazed a minute--'twas pleasure,
mainly, I judged--then his face sort of smoothed down, and he says,
kind of gentle:

"I don't like that shooting from behind a bush.  Why didn't you step into
the road, my boy?"

"The Shepherdsons don't, father.  They always take advantage."

Miss Charlotte she held her head up like a queen while Buck was telling
his tale, and her nostrils spread and her eyes snapped.  The two young
men looked dark, but never said nothing.  Miss Sophia she turned pale,
but the color come back when she found the man warn't hurt.

Soon as I could get Buck down by the corn-cribs under the trees by
ourselves, I says:

"Did you want to kill him, Buck?"

"Well, I bet I did."

"What did he do to you?"

"Him?  He never done nothing to me."

"Well, then, what did you want to kill him for?"

"Why, nothing--only it's on account of the feud."

"What's a feud?"

"Why, where was you raised?  Don't you know what a feud is?"

"Never heard of it before--tell me about it."

"Well," says Buck, "a feud is this way:  A man has a quarrel with another
man, and kills him; then that other man's brother kills HIM; then the
other brothers, on both sides, goes for one another; then the COUSINS
chip in--and by and by everybody's killed off, and there ain't no more
feud.  But it's kind of slow, and takes a long time."

"Has this one been going on long, Buck?"

"Well, I should RECKON!  It started thirty year ago, or som'ers along
there.  There was trouble 'bout something, and then a lawsuit to settle
it; and the suit went agin one of the men, and so he up and shot the man
that won the suit--which he would naturally do, of course.  Anybody
would."

"What was the trouble about, Buck?--land?"

"I reckon maybe--I don't know."

"Well, who done the shooting?  Was it a Grangerford or a Shepherdson?"

"Laws, how do I know?  It was so long ago."

"Don't anybody know?"

"Oh, yes, pa knows, I reckon, and some of the other old people; but they
don't know now what the row was about in the first place."

"Has there been many killed, Buck?"

"Yes; right smart chance of funerals.  But they don't always kill.  Pa's
got a few buckshot in him; but he don't mind it 'cuz he don't weigh much,
anyway.  Bob's been carved up some with a bowie, and Tom's been hurt once
or twice."

"Has anybody been killed this year, Buck?"

"Yes; we got one and they got one.  'Bout three months ago my cousin Bud,
fourteen year old, was riding through the woods on t'other side of the
river, and didn't have no weapon with him, which was blame' foolishness,
and in a lonesome place he hears a horse a-coming behind him, and sees
old Baldy Shepherdson a-linkin' after him with his gun in his hand and
his white hair a-flying in the wind; and 'stead of jumping off and taking
to the brush, Bud 'lowed he could out-run him; so they had it, nip and
tuck, for five mile or more, the old man a-gaining all the time; so at
last Bud seen it warn't any use, so he stopped and faced around so as to
have the bullet holes in front, you know, and the old man he rode up and
shot him down.  But he didn't git much chance to enjoy his luck, for
inside of a week our folks laid HIM out."

"I reckon that old man was a coward, Buck."

"I reckon he WARN'T a coward.  Not by a blame' sight.  There ain't a
coward amongst them Shepherdsons--not a one.  And there ain't no cowards
amongst the Grangerfords either.  Why, that old man kep' up his end in a
fight one day for half an hour against three Grangerfords, and come out
winner.  They was all a-horseback; he lit off of his horse and got behind
a little woodpile, and kep' his horse before him to stop the bullets; but
the Grangerfords stayed on their horses and capered around the old man,
and peppered away at him, and he peppered away at them.  Him and his
horse both went home pretty leaky and crippled, but the Grangerfords had
to be FETCHED home--and one of 'em was dead, and another died the next
day.  No, sir; if a body's out hunting for cowards he don't want to fool
away any time amongst them Shepherdsons, becuz they don't breed any of
that KIND."

Next Sunday we all went to church, about three mile, everybody
a-horseback. The men took their guns along, so did Buck, and kept them
between their knees or stood them handy against the wall.  The
Shepherdsons done the same.  It was pretty ornery preaching--all about
brotherly love, and such-like tiresomeness; but everybody said it was a
good sermon, and they all talked it over going home, and had such a
powerful lot to say about faith and good works and free grace and
preforeordestination, and I don't know what all, that it did seem to me
to be one of the roughest Sundays I had run across yet.

About an hour after dinner everybody was dozing around, some in their
chairs and some in their rooms, and it got to be pretty dull.  Buck and a
dog was stretched out on the grass in the sun sound asleep.  I went up to
our room, and judged I would take a nap myself.  I found that sweet Miss
Sophia standing in her door, which was next to ours, and she took me in
her room and shut the door very soft, and asked me if I liked her, and I
said I did; and she asked me if I would do something for her and not tell
anybody, and I said I would.  Then she said she'd forgot her Testament,
and left it in the seat at church between two other books, and would I
slip out quiet and go there and fetch it to her, and not say nothing to
nobody.  I said I would. So I slid out and slipped off up the road, and
there warn't anybody at the church, except maybe a hog or two, for there
warn't any lock on the door, and hogs likes a puncheon floor in
summer-time because it's cool.  If you notice, most folks don't go
to church only when they've got to; but a hog is different.

Says I to myself, something's up; it ain't natural for a girl to be in
such a sweat about a Testament.  So I give it a shake, and out drops a
little piece of paper with "HALF-PAST TWO" wrote on it with a pencil.  I
ransacked it, but couldn't find anything else.  I couldn't make anything
out of that, so I put the paper in the book again, and when I got home
and upstairs there was Miss Sophia in her door waiting for me.  She
pulled me in and shut the door; then she looked in the Testament till she
found the paper, and as soon as she read it she looked glad; and before a
body could think she grabbed me and give me a squeeze, and said I was the
best boy in the world, and not to tell anybody.  She was mighty red in
the face for a minute, and her eyes lighted up, and it made her powerful
pretty.  I was a good deal astonished, but when I got my breath I asked
her what the paper was about, and she asked me if I had read it, and I
said no, and she asked me if I could read writing, and I told her "no,
only coarse-hand," and then she said the paper warn't anything but a
book-mark to keep her place, and I might go and play now.

I went off down to the river, studying over this thing, and pretty soon I
noticed that my nigger was following along behind.  When we was out of
sight of the house he looked back and around a second, and then comes
a-running, and says:

"Mars Jawge, if you'll come down into de swamp I'll show you a whole
stack o' water-moccasins."

Thinks I, that's mighty curious; he said that yesterday.  He oughter know
a body don't love water-moccasins enough to go around hunting for them.
What is he up to, anyway?  So I says:

"All right; trot ahead."

I followed a half a mile; then he struck out over the swamp, and waded
ankle deep as much as another half-mile.  We come to a little flat piece
of land which was dry and very thick with trees and bushes and vines, and
he says:

"You shove right in dah jist a few steps, Mars Jawge; dah's whah dey is.
I's seed 'm befo'; I don't k'yer to see 'em no mo'."

Then he slopped right along and went away, and pretty soon the trees hid
him.  I poked into the place a-ways and come to a little open patch as
big as a bedroom all hung around with vines, and found a man laying there
asleep--and, by jings, it was my old Jim!

I waked him up, and I reckoned it was going to be a grand surprise to him
to see me again, but it warn't.  He nearly cried he was so glad, but he
warn't surprised.  Said he swum along behind me that night, and heard me
yell every time, but dasn't answer, because he didn't want nobody to pick
HIM up and take him into slavery again.  Says he:

"I got hurt a little, en couldn't swim fas', so I wuz a considable ways
behine you towards de las'; when you landed I reck'ned I could ketch up
wid you on de lan' 'dout havin' to shout at you, but when I see dat house
I begin to go slow.  I 'uz off too fur to hear what dey say to you--I wuz
'fraid o' de dogs; but when it 'uz all quiet agin I knowed you's in de
house, so I struck out for de woods to wait for day.  Early in de mawnin'
some er de niggers come along, gwyne to de fields, en dey tuk me en
showed me dis place, whah de dogs can't track me on accounts o' de water,
en dey brings me truck to eat every night, en tells me how you's a-gitt'n
along."

"Why didn't you tell my Jack to fetch me here sooner, Jim?"

"Well, 'twarn't no use to 'sturb you, Huck, tell we could do sumfn--but
we's all right now.  I ben a-buyin' pots en pans en vittles, as I got a
chanst, en a-patchin' up de raf' nights when--"

"WHAT raft, Jim?"

"Our ole raf'."

"You mean to say our old raft warn't smashed all to flinders?"

"No, she warn't.  She was tore up a good deal--one en' of her was; but
dey warn't no great harm done, on'y our traps was mos' all los'.  Ef we
hadn' dive' so deep en swum so fur under water, en de night hadn' ben so
dark, en we warn't so sk'yerd, en ben sich punkin-heads, as de sayin' is,
we'd a seed de raf'.  But it's jis' as well we didn't, 'kase now she's
all fixed up agin mos' as good as new, en we's got a new lot o' stuff, in
de place o' what 'uz los'."

"Why, how did you get hold of the raft again, Jim--did you catch her?"

"How I gwyne to ketch her en I out in de woods?  No; some er de niggers
foun' her ketched on a snag along heah in de ben', en dey hid her in a
crick 'mongst de willows, en dey wuz so much jawin' 'bout which un 'um
she b'long to de mos' dat I come to heah 'bout it pooty soon, so I ups en
settles de trouble by tellin' 'um she don't b'long to none uv um, but to
you en me; en I ast 'm if dey gwyne to grab a young white genlman's
propaty, en git a hid'n for it?  Den I gin 'm ten cents apiece, en dey
'uz mighty well satisfied, en wisht some mo' raf's 'ud come along en make
'm rich agin. Dey's mighty good to me, dese niggers is, en whatever I
wants 'm to do fur me I doan' have to ast 'm twice, honey.  Dat Jack's a
good nigger, en pooty smart."

"Yes, he is.  He ain't ever told me you was here; told me to come, and
he'd show me a lot of water-moccasins.  If anything happens HE ain't
mixed up in it.  He can say he never seen us together, and it 'll be the
truth."

I don't want to talk much about the next day.  I reckon I'll cut it
pretty short.  I waked up about dawn, and was a-going to turn over and go
to sleep again when I noticed how still it was--didn't seem to be anybody
stirring.  That warn't usual.  Next I noticed that Buck was up and gone.
Well, I gets up, a-wondering, and goes down stairs--nobody around;
everything as still as a mouse.  Just the same outside.  Thinks I, what
does it mean?  Down by the wood-pile I comes across my Jack, and says:

"What's it all about?"

Says he:

"Don't you know, Mars Jawge?"

"No," says I, "I don't."

"Well, den, Miss Sophia's run off! 'deed she has.  She run off in de
night some time--nobody don't know jis' when; run off to get married to
dat young Harney Shepherdson, you know--leastways, so dey 'spec.  De
fambly foun' it out 'bout half an hour ago--maybe a little mo'--en' I
TELL you dey warn't no time los'.  Sich another hurryin' up guns en
hosses YOU never see!  De women folks has gone for to stir up de
relations, en ole Mars Saul en de boys tuck dey guns en rode up de river
road for to try to ketch dat young man en kill him 'fo' he kin git acrost
de river wid Miss Sophia.  I reck'n dey's gwyne to be mighty rough
times."

"Buck went off 'thout waking me up."

"Well, I reck'n he DID!  Dey warn't gwyne to mix you up in it.  Mars Buck
he loaded up his gun en 'lowed he's gwyne to fetch home a Shepherdson or
bust. Well, dey'll be plenty un 'm dah, I reck'n, en you bet you he'll
fetch one ef he gits a chanst."

I took up the river road as hard as I could put.  By and by I begin to
hear guns a good ways off.  When I cOme in sight of the log store and the
woodpile where the steamboats lands I worked along under the trees and
brush till I got to a good place, and then I clumb up into the forks of a
cottonwood that was out of reach, and watched.  There was a wood-rank
four foot high a little ways in front of the tree, and first I was going
to hide behind that; but maybe it was luckier I didn't.

There was four or five men cavorting around on their horses in the open
place before the log store, cussing and yelling, and trying to get at a
couple of young chaps that was behind the wood-rank alongside of the
steamboat landing; but they couldn't come it.  Every time one of them
showed himself on the river side of the woodpile he got shot at.  The two
boys was squatting back to back behind the pile, so they could watch both
ways.

By and by the men stopped cavorting around and yelling.  They started
riding towards the store; then up gets one of the boys, draws a steady
bead over the wood-rank, and drops one of them out of his saddle.  All
the men jumped off of their horses and grabbed the hurt one and started
to carry him to the store; and that minute the two boys started on the
run.  They got half way to the tree I was in before the men noticed.
Then the men see them, and jumped on their horses and took out after
them.  They gained on the boys, but it didn't do no good, the boys had
too good a start; they got to the woodpile that was in front of my tree,
and slipped in behind it, and so they had the bulge on the men again.
One of the boys was Buck, and the other was a slim young chap about
nineteen years old.

The men ripped around awhile, and then rode away.  As soon as they was
out of sight I sung out to Buck and told him.  He didn't know what to
make of my voice coming out of the tree at first.  He was awful
surprised.  He told me to watch out sharp and let him know when the men
come in sight again; said they was up to some devilment or other
--wouldn't be gone long.  I wished I was out of that tree, but I dasn't
come down.  Buck begun to cry and rip, and 'lowed that him and his cousin
Joe (that was the other young chap) would make up for this day yet.  He
said his father and his two brothers was killed, and two or three of the
enemy.  Said the Shepherdsons laid for them in ambush.  Buck said his
father and brothers ought to waited for their relations--the Shepherdsons
was too strong for them.  I asked him what was become of young Harney and
Miss Sophia.  He said they'd got across the river and was safe.  I was
glad of that; but the way Buck did take on because he didn't manage to
kill Harney that day he shot at him--I hain't ever heard anything like
it.

All of a sudden, bang! bang! bang! goes three or four guns--the men had
slipped around through the woods and come in from behind without their
horses!  The boys jumped for the river--both of them hurt--and as they
swum down the current the men run along the bank shooting at them and
singing out, "Kill them, kill them!"  It made me so sick I most fell out
of the tree.  I ain't a-going to tell ALL that happened--it would make me
sick again if I was to do that.  I wished I hadn't ever come ashore that
night to see such things.  I ain't ever going to get shut of them--lots
of times I dream about them.

I stayed in the tree till it begun to get dark, afraid to come down.
Sometimes I heard guns away off in the woods; and twice I seen little
gangs of men gallop past the log store with guns; so I reckoned the
trouble was still a-going on.  I was mighty downhearted; so I made up my
mind I wouldn't ever go anear that house again, because I reckoned I was
to blame, somehow. I judged that that piece of paper meant that Miss
Sophia was to meet Harney somewheres at half-past two and run off; and I
judged I ought to told her father about that paper and the curious way
she acted, and then maybe he would a locked her up, and this awful mess
wouldn't ever happened.

When I got down out of the tree I crept along down the river bank a
piece, and found the two bodies laying in the edge of the water, and
tugged at them till I got them ashore; then I covered up their faces, and
got away as quick as I could.  I cried a little when I was covering up
Buck's face, for he was mighty good to me.

It was just dark now.  I never went near the house, but struck through
the woods and made for the swamp.  Jim warn't on his island, so I tramped
off in a hurry for the crick, and crowded through the willows, red-hot to
jump aboard and get out of that awful country.  The raft was gone!  My
souls, but I was scared!  I couldn't get my breath for most a minute.
Then I raised a yell.  A voice not twenty-five foot from me says:

"Good lan'! is dat you, honey?  Doan' make no noise."

It was Jim's voice--nothing ever sounded so good before.  I run along the
bank a piece and got aboard, and Jim he grabbed me and hugged me, he was
so glad to see me.  He says:

"Laws bless you, chile, I 'uz right down sho' you's dead agin.  Jack's
been heah; he say he reck'n you's ben shot, kase you didn' come home no
mo'; so I's jes' dis minute a startin' de raf' down towards de mouf er de
crick, so's to be all ready for to shove out en leave soon as Jack comes
agin en tells me for certain you IS dead.  Lawsy, I's mighty glad to git
you back again, honey."

I says:

"All right--that's mighty good; they won't find me, and they'll think
I've been killed, and floated down the river--there's something up there
that 'll help them think so--so don't you lose no time, Jim, but just
shove off for the big water as fast as ever you can."

I never felt easy till the raft was two mile below there and out in the
middle of the Mississippi.  Then we hung up our signal lantern, and
judged that we was free and safe once more.  I hadn't had a bite to eat
since yesterday, so Jim he got out some corn-dodgers and buttermilk, and
pork and cabbage and greens--there ain't nothing in the world so good
when it's cooked right--and whilst I eat my supper we talked and had a
good time.  I was powerful glad to get away from the feuds, and so was
Jim to get away from the swamp.  We said there warn't no home like a
raft, after all.  Other places do seem so cramped up and smothery, but a
raft don't.  You feel mighty free and easy and comfortable on a raft.




CHAPTER XIX.

TWO or three days and nights went by; I reckon I might say they swum by,
they slid along so quiet and smooth and lovely.  Here is the way we put
in the time.  It was a monstrous big river down there--sometimes a mile
and a half wide; we run nights, and laid up and hid daytimes; soon as
night was most gone we stopped navigating and tied up--nearly always in
the dead water under a towhead; and then cut young cottonwoods and
willows, and hid the raft with them.  Then we set out the lines.  Next we
slid into the river and had a swim, so as to freshen up and cool off;
then we set down on the sandy bottom where the water was about knee deep,
and watched the daylight come.  Not a sound anywheres--perfectly still
--just like the whole world was asleep, only sometimes the bullfrogs
a-cluttering, maybe.  The first thing to see, looking away over the water,
was a kind of dull line--that was the woods on t'other side; you couldn't
make nothing else out; then a pale place in the sky; then more paleness
spreading around; then the river softened up away off, and warn't black
any more, but gray; you could see little dark spots drifting along ever
so far away--trading scows, and such things; and long black streaks
--rafts; sometimes you could hear a sweep screaking; or jumbled up voices,
it was so still, and sounds come so far; and by and by you could see a
streak on the water which you know by the look of the streak that there's
a snag there in a swift current which breaks on it and makes that streak
look that way; and you see the mist curl up off of the water, and the
east reddens up, and the river, and you make out a log-cabin in the edge
of the woods, away on the bank on t'other side of the river, being a
woodyard, likely, and piled by them cheats so you can throw a dog through
it anywheres; then the nice breeze springs up, and comes fanning you from
over there, so cool and fresh and sweet to smell on account of the woods
and the flowers; but sometimes not that way, because they've left dead
fish laying around, gars and such, and they do get pretty rank; and next
you've got the full day, and everything smiling in the sun, and the
song-birds just going it!

A little smoke couldn't be noticed now, so we would take some fish off of
the lines and cook up a hot breakfast.  And afterwards we would watch the
lonesomeness of the river, and kind of lazy along, and by and by lazy off
to sleep.  Wake up by and by, and look to see what done it, and maybe see
a steamboat coughing along up-stream, so far off towards the other side
you couldn't tell nothing about her only whether she was a stern-wheel or
side-wheel; then for about an hour there wouldn't be nothing to hear nor
nothing to see--just solid lonesomeness.  Next you'd see a raft sliding
by, away off yonder, and maybe a galoot on it chopping, because they're
most always doing it on a raft; you'd see the axe flash and come down
--you don't hear nothing; you see that axe go up again, and by the time
it's above the man's head then you hear the K'CHUNK!--it had took all
that time to come over the water.  So we would put in the day, lazying
around, listening to the stillness.  Once there was a thick fog, and the
rafts and things that went by was beating tin pans so the steamboats
wouldn't run over them.  A scow or a raft went by so close we could hear
them talking and cussing and laughing--heard them plain; but we couldn't
see no sign of them; it made you feel crawly; it was like spirits
carrying on that way in the air.  Jim said he believed it was spirits;
but I says:

"No; spirits wouldn't say, 'Dern the dern fog.'"

Soon as it was night out we shoved; when we got her out to about the
middle we let her alone, and let her float wherever the current wanted
her to; then we lit the pipes, and dangled our legs in the water, and
talked about all kinds of things--we was always naked, day and night,
whenever the mosquitoes would let us--the new clothes Buck's folks made
for me was too good to be comfortable, and besides I didn't go much on
clothes, nohow.

Sometimes we'd have that whole river all to ourselves for the longest
time. Yonder was the banks and the islands, across the water; and maybe a
spark--which was a candle in a cabin window; and sometimes on the water
you could see a spark or two--on a raft or a scow, you know; and maybe
you could hear a fiddle or a song coming over from one of them crafts.
It's lovely to live on a raft.  We had the sky up there, all speckled
with stars, and we used to lay on our backs and look up at them, and
discuss about whether they was made or only just happened.  Jim he
allowed they was made, but I allowed they happened; I judged it would
have took too long to MAKE so many.  Jim said the moon could a LAID them;
well, that looked kind of reasonable, so I didn't say nothing against it,
because I've seen a frog lay most as many, so of course it could be done.
We used to watch the stars that fell, too, and see them streak down.  Jim
allowed they'd got spoiled and was hove out of the nest.

Once or twice of a night we would see a steamboat slipping along in the
dark, and now and then she would belch a whole world of sparks up out of
her chimbleys, and they would rain down in the river and look awful
pretty; then she would turn a corner and her lights would wink out and
her powwow shut off and leave the river still again; and by and by her
waves would get to us, a long time after she was gone, and joggle the
raft a bit, and after that you wouldn't hear nothing for you couldn't
tell how long, except maybe frogs or something.

After midnight the people on shore went to bed, and then for two or three
hours the shores was black--no more sparks in the cabin windows.  These
sparks was our clock--the first one that showed again meant morning was
coming, so we hunted a place to hide and tie up right away.

One morning about daybreak I found a canoe and crossed over a chute to
the main shore--it was only two hundred yards--and paddled about a mile
up a crick amongst the cypress woods, to see if I couldn't get some
berries. Just as I was passing a place where a kind of a cowpath crossed
the crick, here comes a couple of men tearing up the path as tight as
they could foot it.  I thought I was a goner, for whenever anybody was
after anybody I judged it was ME--or maybe Jim.  I was about to dig out
from there in a hurry, but they was pretty close to me then, and sung out
and begged me to save their lives--said they hadn't been doing nothing,
and was being chased for it--said there was men and dogs a-coming.  They
wanted to jump right in, but I says:

"Don't you do it.  I don't hear the dogs and horses yet; you've got time
to crowd through the brush and get up the crick a little ways; then you
take to the water and wade down to me and get in--that'll throw the dogs
off the scent."

They done it, and soon as they was aboard I lit out for our towhead, and
in about five or ten minutes we heard the dogs and the men away off,
shouting. We heard them come along towards the crick, but couldn't see
them; they seemed to stop and fool around a while; then, as we got
further and further away all the time, we couldn't hardly hear them at
all; by the time we had left a mile of woods behind us and struck the
river, everything was quiet, and we paddled over to the towhead and hid
in the cottonwoods and was safe.

One of these fellows was about seventy or upwards, and had a bald head
and very gray whiskers.  He had an old battered-up slouch hat on, and a
greasy blue woollen shirt, and ragged old blue jeans britches stuffed
into his boot-tops, and home-knit galluses--no, he only had one.  He had
an old long-tailed blue jeans coat with slick brass buttons flung over
his arm, and both of them had big, fat, ratty-looking carpet-bags.

The other fellow was about thirty, and dressed about as ornery.  After
breakfast we all laid off and talked, and the first thing that come out
was that these chaps didn't know one another.

"What got you into trouble?" says the baldhead to t'other chap.

"Well, I'd been selling an article to take the tartar off the teeth--and
it does take it off, too, and generly the enamel along with it--but I
stayed about one night longer than I ought to, and was just in the act of
sliding out when I ran across you on the trail this side of town, and you
told me they were coming, and begged me to help you to get off.  So I
told you I was expecting trouble myself, and would scatter out WITH you.
That's the whole yarn--what's yourn?

"Well, I'd ben a-running' a little temperance revival thar 'bout a week,
and was the pet of the women folks, big and little, for I was makin' it
mighty warm for the rummies, I TELL you, and takin' as much as five or
six dollars a night--ten cents a head, children and niggers free--and
business a-growin' all the time, when somehow or another a little report
got around last night that I had a way of puttin' in my time with a
private jug on the sly.  A nigger rousted me out this mornin', and told
me the people was getherin' on the quiet with their dogs and horses, and
they'd be along pretty soon and give me 'bout half an hour's start, and
then run me down if they could; and if they got me they'd tar and feather
me and ride me on a rail, sure.  I didn't wait for no breakfast--I warn't
hungry."

"Old man," said the young one, "I reckon we might double-team it
together; what do you think?"

"I ain't undisposed.  What's your line--mainly?"

"Jour printer by trade; do a little in patent medicines; theater-actor
--tragedy, you know; take a turn to mesmerism and phrenology when there's a
chance; teach singing-geography school for a change; sling a lecture
sometimes--oh, I do lots of things--most anything that comes handy, so it
ain't work.  What's your lay?"

"I've done considerble in the doctoring way in my time.  Layin' on o'
hands is my best holt--for cancer and paralysis, and sich things; and I
k'n tell a fortune pretty good when I've got somebody along to find out
the facts for me.  Preachin's my line, too, and workin' camp-meetin's,
and missionaryin' around."

Nobody never said anything for a while; then the young man hove a sigh
and says:

"Alas!"

"What 're you alassin' about?" says the bald-head.

"To think I should have lived to be leading such a life, and be degraded
down into such company."  And he begun to wipe the corner of his eye with
a rag.

"Dern your skin, ain't the company good enough for you?" says the
baldhead, pretty pert and uppish.

"Yes, it IS good enough for me; it's as good as I deserve; for who
fetched me so low when I was so high?  I did myself.  I don't blame YOU,
gentlemen--far from it; I don't blame anybody.  I deserve it all.  Let
the cold world do its worst; one thing I know--there's a grave somewhere
for me. The world may go on just as it's always done, and take everything
from me--loved ones, property, everything; but it can't take that.
Some day I'll lie down in it and forget it all, and my poor broken heart
will be at rest."  He went on a-wiping.

"Drot your pore broken heart," says the baldhead; "what are you heaving
your pore broken heart at US f'r?  WE hain't done nothing."

"No, I know you haven't.  I ain't blaming you, gentlemen.  I brought
myself down--yes, I did it myself.  It's right I should suffer--perfectly
right--I don't make any moan."

"Brought you down from whar?  Whar was you brought down from?"

"Ah, you would not believe me; the world never believes--let it pass
--'tis no matter.  The secret of my birth--"

"The secret of your birth!  Do you mean to say--"

"Gentlemen," says the young man, very solemn, "I will reveal it to you,
for I feel I may have confidence in you.  By rights I am a duke!"

Jim's eyes bugged out when he heard that; and I reckon mine did, too.
Then the baldhead says:  "No! you can't mean it?"

"Yes.  My great-grandfather, eldest son of the Duke of Bridgewater, fled
to this country about the end of the last century, to breathe the pure
air of freedom; married here, and died, leaving a son, his own father
dying about the same time.  The second son of the late duke seized the
titles and estates--the infant real duke was ignored.  I am the lineal
descendant of that infant--I am the rightful Duke of Bridgewater; and
here am I, forlorn, torn from my high estate, hunted of men, despised by
the cold world, ragged, worn, heart-broken, and degraded to the
companionship of felons on a raft!"

Jim pitied him ever so much, and so did I. We tried to comfort him, but
he said it warn't much use, he couldn't be much comforted; said if we was
a mind to acknowledge him, that would do him more good than most anything
else; so we said we would, if he would tell us how.  He said we ought to
bow when we spoke to him, and say "Your Grace," or "My Lord," or "Your
Lordship"--and he wouldn't mind it if we called him plain
"Bridgewater," which, he said, was a title anyway, and not a name; and
one of us ought to wait on him at dinner, and do any little thing for him
he wanted done.

Well, that was all easy, so we done it.  All through dinner Jim stood
around and waited on him, and says, "Will yo' Grace have some o' dis or
some o' dat?" and so on, and a body could see it was mighty pleasing to
him.

But the old man got pretty silent by and by--didn't have much to say, and
didn't look pretty comfortable over all that petting that was going on
around that duke.  He seemed to have something on his mind.  So, along in
the afternoon, he says:

"Looky here, Bilgewater," he says, "I'm nation sorry for you, but you
ain't the only person that's had troubles like that."

"No?"

"No you ain't.  You ain't the only person that's ben snaked down
wrongfully out'n a high place."

"Alas!"

"No, you ain't the only person that's had a secret of his birth."  And,
by jings, HE begins to cry.

"Hold!  What do you mean?"

"Bilgewater, kin I trust you?" says the old man, still sort of sobbing.

"To the bitter death!"  He took the old man by the hand and squeezed it,
and says, "That secret of your being:  speak!"

"Bilgewater, I am the late Dauphin!"

You bet you, Jim and me stared this time.  Then the duke says:

"You are what?"

"Yes, my friend, it is too true--your eyes is lookin' at this very moment
on the pore disappeared Dauphin, Looy the Seventeen, son of Looy the
Sixteen and Marry Antonette."

"You!  At your age!  No!  You mean you're the late Charlemagne; you must
be six or seven hundred years old, at the very least."

"Trouble has done it, Bilgewater, trouble has done it; trouble has brung
these gray hairs and this premature balditude.  Yes, gentlemen, you see
before you, in blue jeans and misery, the wanderin', exiled, trampled-on,
and sufferin' rightful King of France."

Well, he cried and took on so that me and Jim didn't know hardly what to
do, we was so sorry--and so glad and proud we'd got him with us, too.  So
we set in, like we done before with the duke, and tried to comfort HIM.
But he said it warn't no use, nothing but to be dead and done with it all
could do him any good; though he said it often made him feel easier and
better for a while if people treated him according to his rights, and got
down on one knee to speak to him, and always called him "Your Majesty,"
and waited on him first at meals, and didn't set down in his presence
till he asked them. So Jim and me set to majestying him, and doing this
and that and t'other for him, and standing up till he told us we might
set down.  This done him heaps of good, and so he got cheerful and
comfortable.  But the duke kind of soured on him, and didn't look a bit
satisfied with the way things was going; still, the king acted real
friendly towards him, and said the duke's great-grandfather and all the
other Dukes of Bilgewater was a good deal thought of by HIS father, and
was allowed to come to the palace considerable; but the duke stayed huffy
a good while, till by and by the king says:

"Like as not we got to be together a blamed long time on this h-yer raft,
Bilgewater, and so what's the use o' your bein' sour?  It 'll only make
things oncomfortable.  It ain't my fault I warn't born a duke, it ain't
your fault you warn't born a king--so what's the use to worry?  Make the
best o' things the way you find 'em, says I--that's my motto.  This ain't
no bad thing that we've struck here--plenty grub and an easy life--come,
give us your hand, duke, and le's all be friends."

The duke done it, and Jim and me was pretty glad to see it.  It took away
all the uncomfortableness and we felt mighty good over it, because it
would a been a miserable business to have any unfriendliness on the raft;
for what you want, above all things, on a raft, is for everybody to be
satisfied, and feel right and kind towards the others.

It didn't take me long to make up my mind that these liars warn't no
kings nor dukes at all, but just low-down humbugs and frauds.  But I
never said nothing, never let on; kept it to myself; it's the best way;
then you don't have no quarrels, and don't get into no trouble.  If they
wanted us to call them kings and dukes, I hadn't no objections, 'long as
it would keep peace in the family; and it warn't no use to tell Jim, so I
didn't tell him.  If I never learnt nothing else out of pap, I learnt
that the best way to get along with his kind of people is to let them
have their own way.




CHAPTER XX.

THEY asked us considerable many questions; wanted to know what we covered
up the raft that way for, and laid by in the daytime instead of running
--was Jim a runaway nigger?  Says I:

"Goodness sakes! would a runaway nigger run SOUTH?"

No, they allowed he wouldn't.  I had to account for things some way, so I
says:

"My folks was living in Pike County, in Missouri, where I was born, and
they all died off but me and pa and my brother Ike.  Pa, he 'lowed he'd
break up and go down and live with Uncle Ben, who's got a little
one-horse place on the river, forty-four mile below Orleans.  Pa was
pretty poor, and had some debts; so when he'd squared up there warn't
nothing left but sixteen dollars and our nigger, Jim.  That warn't enough
to take us fourteen hundred mile, deck passage nor no other way.  Well,
when the river rose pa had a streak of luck one day; he ketched this
piece of a raft; so we reckoned we'd go down to Orleans on it.  Pa's luck
didn't hold out; a steamboat run over the forrard corner of the raft one
night, and we all went overboard and dove under the wheel; Jim and me
come up all right, but pa was drunk, and Ike was only four years old, so
they never come up no more.  Well, for the next day or two we had
considerable trouble, because people was always coming out in skiffs and
trying to take Jim away from me, saying they believed he was a runaway
nigger.  We don't run daytimes no more now; nights they don't bother us."

The duke says:

"Leave me alone to cipher out a way so we can run in the daytime if we
want to.  I'll think the thing over--I'll invent a plan that'll fix it.
We'll let it alone for to-day, because of course we don't want to go by
that town yonder in daylight--it mightn't be healthy."

Towards night it begun to darken up and look like rain; the heat
lightning was squirting around low down in the sky, and the leaves was
beginning to shiver--it was going to be pretty ugly, it was easy to see
that.  So the duke and the king went to overhauling our wigwam, to see
what the beds was like.  My bed was a straw tick better than Jim's, which
was a corn-shuck tick; there's always cobs around about in a shuck tick,
and they poke into you and hurt; and when you roll over the dry shucks
sound like you was rolling over in a pile of dead leaves; it makes such a
rustling that you wake up.  Well, the duke allowed he would take my bed;
but the king allowed he wouldn't.  He says:

"I should a reckoned the difference in rank would a sejested to you that
a corn-shuck bed warn't just fitten for me to sleep on.  Your Grace 'll
take the shuck bed yourself."

Jim and me was in a sweat again for a minute, being afraid there was
going to be some more trouble amongst them; so we was pretty glad when
the duke says:

"'Tis my fate to be always ground into the mire under the iron heel of
oppression.  Misfortune has broken my once haughty spirit; I yield, I
submit; 'tis my fate.  I am alone in the world--let me suffer; can bear
it."

We got away as soon as it was good and dark.  The king told us to stand
well out towards the middle of the river, and not show a light till we
got a long ways below the town.  We come in sight of the little bunch of
lights by and by--that was the town, you know--and slid by, about a half
a mile out, all right.  When we was three-quarters of a mile below we
hoisted up our signal lantern; and about ten o'clock it come on to rain
and blow and thunder and lighten like everything; so the king told us to
both stay on watch till the weather got better; then him and the duke
crawled into the wigwam and turned in for the night.  It was my watch
below till twelve, but I wouldn't a turned in anyway if I'd had a bed,
because a body don't see such a storm as that every day in the week, not
by a long sight.  My souls, how the wind did scream along!  And every
second or two there'd come a glare that lit up the white-caps for a half
a mile around, and you'd see the islands looking dusty through the rain,
and the trees thrashing around in the wind; then comes a H-WHACK!--bum!
bum! bumble-umble-um-bum-bum-bum-bum--and the thunder would go rumbling
and grumbling away, and quit--and then RIP comes another flash and
another sockdolager.  The waves most washed me off the raft sometimes,
but I hadn't any clothes on, and didn't mind.  We didn't have no trouble
about snags; the lightning was glaring and flittering around so constant
that we could see them plenty soon enough to throw her head this way or
that and miss them.

I had the middle watch, you know, but I was pretty sleepy by that time,
so Jim he said he would stand the first half of it for me; he was always
mighty good that way, Jim was.  I crawled into the wigwam, but the king
and the duke had their legs sprawled around so there warn't no show for
me; so I laid outside--I didn't mind the rain, because it was warm, and
the waves warn't running so high now.  About two they come up again,
though, and Jim was going to call me; but he changed his mind, because he
reckoned they warn't high enough yet to do any harm; but he was mistaken
about that, for pretty soon all of a sudden along comes a regular ripper
and washed me overboard.  It most killed Jim a-laughing.  He was the
easiest nigger to laugh that ever was, anyway.

I took the watch, and Jim he laid down and snored away; and by and by the
storm let up for good and all; and the first cabin-light that showed I
rousted him out, and we slid the raft into hiding quarters for the day.

The king got out an old ratty deck of cards after breakfast, and him and
the duke played seven-up a while, five cents a game.  Then they got tired
of it, and allowed they would "lay out a campaign," as they called it.
The duke went down into his carpet-bag, and fetched up a lot of little
printed bills and read them out loud.  One bill said, "The celebrated Dr.
Armand de Montalban, of Paris," would "lecture on the Science of
Phrenology" at such and such a place, on the blank day of blank, at ten
cents admission, and "furnish charts of character at twenty-five cents
apiece."  The duke said that was HIM.  In another bill he was the
"world-renowned Shakespearian tragedian, Garrick the Younger, of Drury
Lane, London."  In other bills he had a lot of other names and done other
wonderful things, like finding water and gold with a "divining-rod,"
"dissipating witch spells," and so on.  By and by he says:

"But the histrionic muse is the darling.  Have you ever trod the boards,
Royalty?"

"No," says the king.

"You shall, then, before you're three days older, Fallen Grandeur," says
the duke.  "The first good town we come to we'll hire a hall and do the
sword fight in Richard III. and the balcony scene in Romeo and Juliet.
How does that strike you?"

"I'm in, up to the hub, for anything that will pay, Bilgewater; but, you
see, I don't know nothing about play-actin', and hain't ever seen much of
it.  I was too small when pap used to have 'em at the palace.  Do you
reckon you can learn me?"

"Easy!"

"All right.  I'm jist a-freezn' for something fresh, anyway.  Le's
commence right away."

So the duke he told him all about who Romeo was and who Juliet was, and
said he was used to being Romeo, so the king could be Juliet.

"But if Juliet's such a young gal, duke, my peeled head and my white
whiskers is goin' to look oncommon odd on her, maybe."

"No, don't you worry; these country jakes won't ever think of that.
Besides, you know, you'll be in costume, and that makes all the
difference in the world; Juliet's in a balcony, enjoying the moonlight
before she goes to bed, and she's got on her night-gown and her ruffled
nightcap.  Here are the costumes for the parts."

He got out two or three curtain-calico suits, which he said was meedyevil
armor for Richard III. and t'other chap, and a long white cotton
nightshirt and a ruffled nightcap to match.  The king was satisfied; so
the duke got out his book and read the parts over in the most splendid
spread-eagle way, prancing around and acting at the same time, to show
how it had got to be done; then he give the book to the king and told him
to get his part by heart.

There was a little one-horse town about three mile down the bend, and
after dinner the duke said he had ciphered out his idea about how to run
in daylight without it being dangersome for Jim; so he allowed he would
go down to the town and fix that thing.  The king allowed he would go,
too, and see if he couldn't strike something.  We was out of coffee, so
Jim said I better go along with them in the canoe and get some.

When we got there there warn't nobody stirring; streets empty, and
perfectly dead and still, like Sunday.  We found a sick nigger sunning
himself in a back yard, and he said everybody that warn't too young or
too sick or too old was gone to camp-meeting, about two mile back in the
woods.  The king got the directions, and allowed he'd go and work that
camp-meeting for all it was worth, and I might go, too.

The duke said what he was after was a printing-office.  We found it; a
little bit of a concern, up over a carpenter shop--carpenters and
printers all gone to the meeting, and no doors locked.  It was a dirty,
littered-up place, and had ink marks, and handbills with pictures of
horses and runaway niggers on them, all over the walls.  The duke shed
his coat and said he was all right now.  So me and the king lit out for
the camp-meeting.

We got there in about a half an hour fairly dripping, for it was a most
awful hot day.  There was as much as a thousand people there from twenty
mile around.  The woods was full of teams and wagons, hitched
everywheres, feeding out of the wagon-troughs and stomping to keep off
the flies.  There was sheds made out of poles and roofed over with
branches, where they had lemonade and gingerbread to sell, and piles of
watermelons and green corn and such-like truck.

The preaching was going on under the same kinds of sheds, only they was
bigger and held crowds of people.  The benches was made out of outside
slabs of logs, with holes bored in the round side to drive sticks into
for legs. They didn't have no backs.  The preachers had high platforms to
stand on at one end of the sheds.  The women had on sun-bonnets; and some
had linsey-woolsey frocks, some gingham ones, and a few of the young ones
had on calico.  Some of the young men was barefooted, and some of the
children didn't have on any clothes but just a tow-linen shirt.  Some of
the old women was knitting, and some of the young folks was courting on
the sly.

The first shed we come to the preacher was lining out a hymn.  He lined
out two lines, everybody sung it, and it was kind of grand to hear it,
there was so many of them and they done it in such a rousing way; then he
lined out two more for them to sing--and so on.  The people woke up more
and more, and sung louder and louder; and towards the end some begun to
groan, and some begun to shout.  Then the preacher begun to preach, and
begun in earnest, too; and went weaving first to one side of the platform
and then the other, and then a-leaning down over the front of it, with
his arms and his body going all the time, and shouting his words out with
all his might; and every now and then he would hold up his Bible and
spread it open, and kind of pass it around this way and that, shouting,
"It's the brazen serpent in the wilderness!  Look upon it and live!"  And
people would shout out, "Glory!--A-a-MEN!"  And so he went on, and the
people groaning and crying and saying amen:

"Oh, come to the mourners' bench! come, black with sin! (AMEN!) come,
sick and sore! (AMEN!) come, lame and halt and blind! (AMEN!) come, pore
and needy, sunk in shame! (A-A-MEN!) come, all that's worn and soiled and
suffering!--come with a broken spirit! come with a contrite heart! come
in your rags and sin and dirt! the waters that cleanse is free, the door
of heaven stands open--oh, enter in and be at rest!" (A-A-MEN!  GLORY,
GLORY HALLELUJAH!)

And so on.  You couldn't make out what the preacher said any more, on
account of the shouting and crying.  Folks got up everywheres in the
crowd, and worked their way just by main strength to the mourners' bench,
with the tears running down their faces; and when all the mourners had
got up there to the front benches in a crowd, they sung and shouted and
flung themselves down on the straw, just crazy and wild.

Well, the first I knowed the king got a-going, and you could hear him
over everybody; and next he went a-charging up on to the platform, and
the preacher he begged him to speak to the people, and he done it.  He
told them he was a pirate--been a pirate for thirty years out in the
Indian Ocean--and his crew was thinned out considerable last spring in
a fight, and he was home now to take out some fresh men, and thanks to
goodness he'd been robbed last night and put ashore off of a steamboat
without a cent, and he was glad of it; it was the blessedest thing that
ever happened to him, because he was a changed man now, and happy for the
first time in his life; and, poor as he was, he was going to start right
off and work his way back to the Indian Ocean, and put in the rest of his
life trying to turn the pirates into the true path; for he could do it
better than anybody else, being acquainted with all pirate crews in that
ocean; and though it would take him a long time to get there without
money, he would get there anyway, and every time he convinced a pirate he
would say to him, "Don't you thank me, don't you give me no credit; it
all belongs to them dear people in Pokeville camp-meeting, natural
brothers and benefactors of the race, and that dear preacher there, the
truest friend a pirate ever had!"

And then he busted into tears, and so did everybody.  Then somebody sings
out, "Take up a collection for him, take up a collection!"  Well, a half
a dozen made a jump to do it, but somebody sings out, "Let HIM pass the
hat around!"  Then everybody said it, the preacher too.

So the king went all through the crowd with his hat swabbing his eyes,
and blessing the people and praising them and thanking them for being so
good to the poor pirates away off there; and every little while the
prettiest kind of girls, with the tears running down their cheeks, would
up and ask him would he let them kiss him for to remember him by; and he
always done it; and some of them he hugged and kissed as many as five or
six times--and he was invited to stay a week; and everybody wanted him to
live in their houses, and said they'd think it was an honor; but he said
as this was the last day of the camp-meeting he couldn't do no good, and
besides he was in a sweat to get to the Indian Ocean right off and go to
work on the pirates.

When we got back to the raft and he come to count up he found he had
collected eighty-seven dollars and seventy-five cents.  And then he had
fetched away a three-gallon jug of whisky, too, that he found under a
wagon when he was starting home through the woods.  The king said, take
it all around, it laid over any day he'd ever put in in the missionarying
line.  He said it warn't no use talking, heathens don't amount to shucks
alongside of pirates to work a camp-meeting with.

The duke was thinking HE'D been doing pretty well till the king come to
show up, but after that he didn't think so so much.  He had set up and
printed off two little jobs for farmers in that printing-office--horse
bills--and took the money, four dollars.  And he had got in ten
dollars' worth of advertisements for the paper, which he said he would
put in for four dollars if they would pay in advance--so they done it.
The price of the paper was two dollars a year, but he took in three
subscriptions for half a dollar apiece on condition of them paying him in
advance; they were going to pay in cordwood and onions as usual, but he
said he had just bought the concern and knocked down the price as low as
he could afford it, and was going to run it for cash.  He set up a little
piece of poetry, which he made, himself, out of his own head--three
verses--kind of sweet and saddish--the name of it was, "Yes, crush, cold
world, this breaking heart"--and he left that all set up and ready to
print in the paper, and didn't charge nothing for it.  Well, he took in
nine dollars and a half, and said he'd done a pretty square day's work
for it.

Then he showed us another little job he'd printed and hadn't charged for,
because it was for us.  It had a picture of a runaway nigger with a
bundle on a stick over his shoulder, and "$200 reward" under it.  The
reading was all about Jim, and just described him to a dot.  It said he
run away from St. Jacques' plantation, forty mile below New Orleans, last
winter, and likely went north, and whoever would catch him and send him
back he could have the reward and expenses.

"Now," says the duke, "after to-night we can run in the daytime if we
want to.  Whenever we see anybody coming we can tie Jim hand and foot
with a rope, and lay him in the wigwam and show this handbill and say we
captured him up the river, and were too poor to travel on a steamboat, so
we got this little raft on credit from our friends and are going down to
get the reward.  Handcuffs and chains would look still better on Jim, but
it wouldn't go well with the story of us being so poor.  Too much like
jewelry.  Ropes are the correct thing--we must preserve the unities, as
we say on the boards."

We all said the duke was pretty smart, and there couldn't be no trouble
about running daytimes.  We judged we could make miles enough that night
to get out of the reach of the powwow we reckoned the duke's work in the
printing office was going to make in that little town; then we could boom
right along if we wanted to.

We laid low and kept still, and never shoved out till nearly ten o'clock;
then we slid by, pretty wide away from the town, and didn't hoist our
lantern till we was clear out of sight of it.

When Jim called me to take the watch at four in the morning, he says:

"Huck, does you reck'n we gwyne to run acrost any mo' kings on dis trip?"

"No," I says, "I reckon not."

"Well," says he, "dat's all right, den.  I doan' mine one er two kings,
but dat's enough.  Dis one's powerful drunk, en de duke ain' much
better."

I found Jim had been trying to get him to talk French, so he could hear
what it was like; but he said he had been in this country so long, and
had so much trouble, he'd forgot it.




CHAPTER XXI.

IT was after sun-up now, but we went right on and didn't tie up.  The
king and the duke turned out by and by looking pretty rusty; but after
they'd jumped overboard and took a swim it chippered them up a good deal.
After breakfast the king he took a seat on the corner of the raft, and
pulled off his boots and rolled up his britches, and let his legs dangle
in the water, so as to be comfortable, and lit his pipe, and went to
getting his Romeo and Juliet by heart.  When he had got it pretty good
him and the duke begun to practice it together.  The duke had to learn
him over and over again how to say every speech; and he made him sigh,
and put his hand on his heart, and after a while he said he done it
pretty well; "only," he says, "you mustn't bellow out ROMEO! that way,
like a bull--you must say it soft and sick and languishy, so--R-o-o-meo!
that is the idea; for Juliet's a dear sweet mere child of a girl, you
know, and she doesn't bray like a jackass."

Well, next they got out a couple of long swords that the duke made out of
oak laths, and begun to practice the sword fight--the duke called himself
Richard III.; and the way they laid on and pranced around the raft was
grand to see.  But by and by the king tripped and fell overboard, and
after that they took a rest, and had a talk about all kinds of adventures
they'd had in other times along the river.

After dinner the duke says:

"Well, Capet, we'll want to make this a first-class show, you know, so I
guess we'll add a little more to it.  We want a little something to
answer encores with, anyway."

"What's onkores, Bilgewater?"

The duke told him, and then says:

"I'll answer by doing the Highland fling or the sailor's hornpipe; and
you--well, let me see--oh, I've got it--you can do Hamlet's soliloquy."

"Hamlet's which?"

"Hamlet's soliloquy, you know; the most celebrated thing in Shakespeare.
Ah, it's sublime, sublime!  Always fetches the house.  I haven't got it
in the book--I've only got one volume--but I reckon I can piece it out
from memory.  I'll just walk up and down a minute, and see if I can call
it back from recollection's vaults."

So he went to marching up and down, thinking, and frowning horrible every
now and then; then he would hoist up his eyebrows; next he would squeeze
his hand on his forehead and stagger back and kind of moan; next he would
sigh, and next he'd let on to drop a tear.  It was beautiful to see him.
By and by he got it.  He told us to give attention.  Then he strikes a
most noble attitude, with one leg shoved forwards, and his arms stretched
away up, and his head tilted back, looking up at the sky; and then he
begins to rip and rave and grit his teeth; and after that, all through
his speech, he howled, and spread around, and swelled up his chest, and
just knocked the spots out of any acting ever I see before.  This is the
speech--I learned it, easy enough, while he was learning it to the king:

To be, or not to be; that is the bare bodkin That makes calamity of so
long life; For who would fardels bear, till Birnam Wood do come to
Dunsinane, But that the fear of something after death Murders the
innocent sleep, Great nature's second course, And makes us rather sling
the arrows of outrageous fortune Than fly to others that we know not of.
There's the respect must give us pause: Wake Duncan with thy knocking!  I
would thou couldst; For who would bear the whips and scorns of time, The
oppressor's wrong, the proud man's contumely, The law's delay, and the
quietus which his pangs might take, In the dead waste and middle of the
night, when churchyards yawn In customary suits of solemn black, But that
the undiscovered country from whose bourne no traveler returns, Breathes
forth contagion on the world, And thus the native hue of resolution, like
the poor cat i' the adage, Is sicklied o'er with care, And all the clouds
that lowered o'er our housetops, With this regard their currents turn
awry, And lose the name of action. 'Tis a consummation devoutly to be
wished.  But soft you, the fair Ophelia: Ope not thy ponderous and marble
jaws, But get thee to a nunnery--go!

Well, the old man he liked that speech, and he mighty soon got it so he
could do it first-rate.  It seemed like he was just born for it; and when
he had his hand in and was excited, it was perfectly lovely the way he
would rip and tear and rair up behind when he was getting it off.

The first chance we got the duke he had some showbills printed; and after
that, for two or three days as we floated along, the raft was a most
uncommon lively place, for there warn't nothing but sword fighting and
rehearsing--as the duke called it--going on all the time.  One morning,
when we was pretty well down the State of Arkansaw, we come in sight of a
little one-horse town in a big bend; so we tied up about three-quarters
of a mile above it, in the mouth of a crick which was shut in like a
tunnel by the cypress trees, and all of us but Jim took the canoe and
went down there to see if there was any chance in that place for our
show.

We struck it mighty lucky; there was going to be a circus there that
afternoon, and the country people was already beginning to come in, in
all kinds of old shackly wagons, and on horses.  The circus would leave
before night, so our show would have a pretty good chance.  The duke he
hired the courthouse, and we went around and stuck up our bills.  They
read like this:

Shaksperean Revival ! ! !
Wonderful Attraction!
For One Night Only!

The world renowned tragedians, David Garrick the Younger, of Drury Lane
Theatre London, and Edmund Kean the elder, of the Royal Haymarket
Theatre, Whitechapel, Pudding Lane, Piccadilly, London, and the Royal
Continental Theatres, in their sublime Shaksperean Spectacle entitled

The Balcony Scene in Romeo and Juliet ! ! !

Romeo...................Mr. Garrick
Juliet..................Mr. Kean

Assisted by the whole strength of the company!
New costumes, new scenery, new appointments!
Also: The thrilling, masterly, and blood-curdling
Broad-sword conflict In Richard III. ! ! !

Richard III.............Mr. Garrick
Richmond................Mr. Kean

Also: (by special request) Hamlet's Immortal Soliloquy ! !
By The Illustrious Kean! Done by him 300 consecutive nights in Paris!
For One Night Only, On account of imperative European engagements!
Admission 25 cents; children and servants, 10 cents.

Then we went loafing around town.  The stores and houses was most all
old, shackly, dried up frame concerns that hadn't ever been painted; they
was set up three or four foot above ground on stilts, so as to be out of
reach of the water when the river was over-flowed.  The houses had little
gardens around them, but they didn't seem to raise hardly anything in
them but jimpson-weeds, and sunflowers, and ash piles, and old curled-up
boots and shoes, and pieces of bottles, and rags, and played-out tinware.
The fences was made of different kinds of boards, nailed on at different
times; and they leaned every which way, and had gates that didn't generly
have but one hinge--a leather one.  Some of the fences had been
white-washed some time or another, but the duke said it was in Clumbus'
time, like enough.  There was generly hogs in the garden, and people
driving them out.

All the stores was along one street.  They had white domestic awnings in
front, and the country people hitched their horses to the awning-posts.
There was empty drygoods boxes under the awnings, and loafers roosting on
them all day long, whittling them with their Barlow knives; and chawing
tobacco, and gaping and yawning and stretching--a mighty ornery lot.
They generly had on yellow straw hats most as wide as an umbrella, but
didn't wear no coats nor waistcoats, they called one another Bill, and
Buck, and Hank, and Joe, and Andy, and talked lazy and drawly, and used
considerable many cuss words.  There was as many as one loafer leaning up
against every awning-post, and he most always had his hands in his
britches-pockets, except when he fetched them out to lend a chaw of
tobacco or scratch.  What a body was hearing amongst them all the time
was:

"Gimme a chaw 'v tobacker, Hank."

"Cain't; I hain't got but one chaw left.  Ask Bill."

Maybe Bill he gives him a chaw; maybe he lies and says he ain't got none.
Some of them kinds of loafers never has a cent in the world, nor a chaw
of tobacco of their own.  They get all their chawing by borrowing; they
say to a fellow, "I wisht you'd len' me a chaw, Jack, I jist this minute
give Ben Thompson the last chaw I had"--which is a lie pretty much
everytime; it don't fool nobody but a stranger; but Jack ain't no
stranger, so he says:

"YOU give him a chaw, did you?  So did your sister's cat's grandmother.
You pay me back the chaws you've awready borry'd off'n me, Lafe Buckner,
then I'll loan you one or two ton of it, and won't charge you no back
intrust, nuther."

"Well, I DID pay you back some of it wunst."

"Yes, you did--'bout six chaws.  You borry'd store tobacker and paid back
nigger-head."

Store tobacco is flat black plug, but these fellows mostly chaws the
natural leaf twisted.  When they borrow a chaw they don't generly cut it
off with a knife, but set the plug in between their teeth, and gnaw with
their teeth and tug at the plug with their hands till they get it in two;
then sometimes the one that owns the tobacco looks mournful at it when
it's handed back, and says, sarcastic:

"Here, gimme the CHAW, and you take the PLUG."

All the streets and lanes was just mud; they warn't nothing else BUT mud
--mud as black as tar and nigh about a foot deep in some places, and two
or three inches deep in ALL the places.  The hogs loafed and grunted
around everywheres.  You'd see a muddy sow and a litter of pigs come
lazying along the street and whollop herself right down in the way, where
folks had to walk around her, and she'd stretch out and shut her eyes and
wave her ears whilst the pigs was milking her, and look as happy as if
she was on salary. And pretty soon you'd hear a loafer sing out, "Hi!  SO
boy! sick him, Tige!" and away the sow would go, squealing most horrible,
with a dog or two swinging to each ear, and three or four dozen more
a-coming; and then you would see all the loafers get up and watch the thing
out of sight, and laugh at the fun and look grateful for the noise.  Then
they'd settle back again till there was a dog fight.  There couldn't
anything wake them up all over, and make them happy all over, like a dog
fight--unless it might be putting turpentine on a stray dog and setting
fire to him, or tying a tin pan to his tail and see him run himself to
death.

On the river front some of the houses was sticking out over the bank, and
they was bowed and bent, and about ready to tumble in, The people had
moved out of them.  The bank was caved away under one corner of some
others, and that corner was hanging over.  People lived in them yet, but
it was dangersome, because sometimes a strip of land as wide as a house
caves in at a time.  Sometimes a belt of land a quarter of a mile deep
will start in and cave along and cave along till it all caves into the
river in one summer. Such a town as that has to be always moving back,
and back, and back, because the river's always gnawing at it.

The nearer it got to noon that day the thicker and thicker was the wagons
and horses in the streets, and more coming all the time.  Families
fetched their dinners with them from the country, and eat them in the
wagons.  There was considerable whisky drinking going on, and I seen
three fights.  By and by somebody sings out:

"Here comes old Boggs!--in from the country for his little old monthly
drunk; here he comes, boys!"

All the loafers looked glad; I reckoned they was used to having fun out
of Boggs.  One of them says:

"Wonder who he's a-gwyne to chaw up this time.  If he'd a-chawed up all
the men he's ben a-gwyne to chaw up in the last twenty year he'd have
considerable ruputation now."

Another one says, "I wisht old Boggs 'd threaten me, 'cuz then I'd know I
warn't gwyne to die for a thousan' year."

Boggs comes a-tearing along on his horse, whooping and yelling like an
Injun, and singing out:

"Cler the track, thar.  I'm on the waw-path, and the price uv coffins is
a-gwyne to raise."

He was drunk, and weaving about in his saddle; he was over fifty year
old, and had a very red face.  Everybody yelled at him and laughed at him
and sassed him, and he sassed back, and said he'd attend to them and lay
them out in their regular turns, but he couldn't wait now because he'd
come to town to kill old Colonel Sherburn, and his motto was, "Meat
first, and spoon vittles to top off on."

He see me, and rode up and says:

"Whar'd you come f'm, boy?  You prepared to die?"

Then he rode on.  I was scared, but a man says:

"He don't mean nothing; he's always a-carryin' on like that when he's
drunk.  He's the best naturedest old fool in Arkansaw--never hurt nobody,
drunk nor sober."

Boggs rode up before the biggest store in town, and bent his head down so
he could see under the curtain of the awning and yells:

"Come out here, Sherburn! Come out and meet the man you've swindled.
You're the houn' I'm after, and I'm a-gwyne to have you, too!"

And so he went on, calling Sherburn everything he could lay his tongue
to, and the whole street packed with people listening and laughing and
going on.  By and by a proud-looking man about fifty-five--and he was a
heap the best dressed man in that town, too--steps out of the store, and
the crowd drops back on each side to let him come.  He says to Boggs,
mighty ca'm and slow--he says:

"I'm tired of this, but I'll endure it till one o'clock.  Till one
o'clock, mind--no longer.  If you open your mouth against me only once
after that time you can't travel so far but I will find you."

Then he turns and goes in.  The crowd looked mighty sober; nobody
stirred, and there warn't no more laughing.  Boggs rode off blackguarding
Sherburn as loud as he could yell, all down the street; and pretty soon
back he comes and stops before the store, still keeping it up.  Some men
crowded around him and tried to get him to shut up, but he wouldn't; they
told him it would be one o'clock in about fifteen minutes, and so he MUST
go home--he must go right away.  But it didn't do no good.  He cussed
away with all his might, and throwed his hat down in the mud and rode
over it, and pretty soon away he went a-raging down the street again,
with his gray hair a-flying. Everybody that could get a chance at him
tried their best to coax him off of his horse so they could lock him up
and get him sober; but it warn't no use--up the street he would tear
again, and give Sherburn another cussing.  By and by somebody says:

"Go for his daughter!--quick, go for his daughter; sometimes he'll listen
to her.  If anybody can persuade him, she can."

So somebody started on a run.  I walked down street a ways and stopped.
In about five or ten minutes here comes Boggs again, but not on his
horse.  He was a-reeling across the street towards me, bare-headed, with
a friend on both sides of him a-holt of his arms and hurrying him along.
He was quiet, and looked uneasy; and he warn't hanging back any, but was
doing some of the hurrying himself.  Somebody sings out:

"Boggs!"

I looked over there to see who said it, and it was that Colonel Sherburn.
He was standing perfectly still in the street, and had a pistol raised in
his right hand--not aiming it, but holding it out with the barrel tilted
up towards the sky.  The same second I see a young girl coming on the
run, and two men with her.  Boggs and the men turned round to see who
called him, and when they see the pistol the men jumped to one side, and
the pistol-barrel come down slow and steady to a level--both barrels
cocked. Boggs throws up both of his hands and says, "O Lord, don't
shoot!"  Bang! goes the first shot, and he staggers back, clawing at the
air--bang! goes the second one, and he tumbles backwards on to the
ground, heavy and solid, with his arms spread out.  That young girl
screamed out and comes rushing, and down she throws herself on her
father, crying, and saying, "Oh, he's killed him, he's killed him!"  The
crowd closed up around them, and shouldered and jammed one another, with
their necks stretched, trying to see, and people on the inside trying to
shove them back and shouting, "Back, back! give him air, give him air!"

Colonel Sherburn he tossed his pistol on to the ground, and turned around
on his heels and walked off.

They took Boggs to a little drug store, the crowd pressing around just
the same, and the whole town following, and I rushed and got a good place
at the window, where I was close to him and could see in.  They laid him
on the floor and put one large Bible under his head, and opened another
one and spread it on his breast; but they tore open his shirt first, and
I seen where one of the bullets went in.  He made about a dozen long
gasps, his breast lifting the Bible up when he drawed in his breath, and
letting it down again when he breathed it out--and after that he laid
still; he was dead.  Then they pulled his daughter away from him,
screaming and crying, and took her off.  She was about sixteen, and very
sweet and gentle looking, but awful pale and scared.

Well, pretty soon the whole town was there, squirming and scrouging and
pushing and shoving to get at the window and have a look, but people that
had the places wouldn't give them up, and folks behind them was saying
all the time, "Say, now, you've looked enough, you fellows; 'tain't right
and 'tain't fair for you to stay thar all the time, and never give nobody
a chance; other folks has their rights as well as you."

There was considerable jawing back, so I slid out, thinking maybe there
was going to be trouble.  The streets was full, and everybody was
excited. Everybody that seen the shooting was telling how it happened,
and there was a big crowd packed around each one of these fellows,
stretching their necks and listening.  One long, lanky man, with long
hair and a big white fur stovepipe hat on the back of his head, and a
crooked-handled cane, marked out the places on the ground where Boggs
stood and where Sherburn stood, and the people following him around from
one place to t'other and watching everything he done, and bobbing their
heads to show they understood, and stooping a little and resting their
hands on their thighs to watch him mark the places on the ground with his
cane; and then he stood up straight and stiff where Sherburn had stood,
frowning and having his hat-brim down over his eyes, and sung out,
"Boggs!" and then fetched his cane down slow to a level, and says "Bang!"
staggered backwards, says "Bang!" again, and fell down flat on his back.
The people that had seen the thing said he done it perfect; said it was
just exactly the way it all happened.  Then as much as a dozen people got
out their bottles and treated him.

Well, by and by somebody said Sherburn ought to be lynched.  In about a
minute everybody was saying it; so away they went, mad and yelling, and
snatching down every clothes-line they come to to do the hanging with.




CHAPTER XXII.

THEY swarmed up towards Sherburn's house, a-whooping and raging like
Injuns, and everything had to clear the way or get run over and tromped
to mush, and it was awful to see.  Children was heeling it ahead of the
mob, screaming and trying to get out of the way; and every window along
the road was full of women's heads, and there was nigger boys in every
tree, and bucks and wenches looking over every fence; and as soon as the
mob would get nearly to them they would break and skaddle back out of
reach.  Lots of the women and girls was crying and taking on, scared most
to death.

They swarmed up in front of Sherburn's palings as thick as they could jam
together, and you couldn't hear yourself think for the noise.  It was a
little twenty-foot yard.  Some sung out "Tear down the fence! tear down
the fence!"  Then there was a racket of ripping and tearing and smashing,
and down she goes, and the front wall of the crowd begins to roll in like
a wave.

Just then Sherburn steps out on to the roof of his little front porch,
with a double-barrel gun in his hand, and takes his stand, perfectly ca'm
and deliberate, not saying a word.  The racket stopped, and the wave
sucked back.

Sherburn never said a word--just stood there, looking down.  The
stillness was awful creepy and uncomfortable.  Sherburn run his eye slow
along the crowd; and wherever it struck the people tried a little to
out-gaze him, but they couldn't; they dropped their eyes and looked sneaky.
Then pretty soon Sherburn sort of laughed; not the pleasant kind, but the
kind that makes you feel like when you are eating bread that's got sand
in it.

Then he says, slow and scornful:

"The idea of YOU lynching anybody!  It's amusing.  The idea of you
thinking you had pluck enough to lynch a MAN!  Because you're brave
enough to tar and feather poor friendless cast-out women that come along
here, did that make you think you had grit enough to lay your hands on a
MAN?  Why, a MAN'S safe in the hands of ten thousand of your kind--as
long as it's daytime and you're not behind him.

"Do I know you?  I know you clear through was born and raised in the
South, and I've lived in the North; so I know the average all around.
The average man's a coward.  In the North he lets anybody walk over him
that wants to, and goes home and prays for a humble spirit to bear it.
In the South one man all by himself, has stopped a stage full of men in
the daytime, and robbed the lot.  Your newspapers call you a brave people
so much that you think you are braver than any other people--whereas
you're just AS brave, and no braver.  Why don't your juries hang
murderers?  Because they're afraid the man's friends will shoot them in
the back, in the dark--and it's just what they WOULD do.

"So they always acquit; and then a MAN goes in the night, with a hundred
masked cowards at his back and lynches the rascal.  Your mistake is, that
you didn't bring a man with you; that's one mistake, and the other is
that you didn't come in the dark and fetch your masks.  You brought PART
of a man--Buck Harkness, there--and if you hadn't had him to start you,
you'd a taken it out in blowing.

"You didn't want to come.  The average man don't like trouble and danger.
YOU don't like trouble and danger.  But if only HALF a man--like Buck
Harkness, there--shouts 'Lynch him! lynch him!' you're afraid to back
down--afraid you'll be found out to be what you are--COWARDS--and so
you raise a yell, and hang yourselves on to that half-a-man's coat-tail,
and come raging up here, swearing what big things you're going to do.
The pitifulest thing out is a mob; that's what an army is--a mob; they
don't fight with courage that's born in them, but with courage that's
borrowed from their mass, and from their officers.  But a mob without any
MAN at the head of it is BENEATH pitifulness.  Now the thing for YOU to
do is to droop your tails and go home and crawl in a hole.  If any real
lynching's going to be done it will be done in the dark, Southern
fashion; and when they come they'll bring their masks, and fetch a MAN
along.  Now LEAVE--and take your half-a-man with you"--tossing his gun up
across his left arm and cocking it when he says this.

The crowd washed back sudden, and then broke all apart, and went tearing
off every which way, and Buck Harkness he heeled it after them, looking
tolerable cheap.  I could a stayed if I wanted to, but I didn't want to.

I went to the circus and loafed around the back side till the watchman
went by, and then dived in under the tent.  I had my twenty-dollar gold
piece and some other money, but I reckoned I better save it, because
there ain't no telling how soon you are going to need it, away from home
and amongst strangers that way.  You can't be too careful.  I ain't
opposed to spending money on circuses when there ain't no other way, but
there ain't no use in WASTING it on them.

It was a real bully circus.  It was the splendidest sight that ever was
when they all come riding in, two and two, a gentleman and lady, side by
side, the men just in their drawers and undershirts, and no shoes nor
stirrups, and resting their hands on their thighs easy and comfortable
--there must a been twenty of them--and every lady with a lovely
complexion, and perfectly beautiful, and looking just like a gang of real
sure-enough queens, and dressed in clothes that cost millions of dollars,
and just littered with diamonds.  It was a powerful fine sight; I never
see anything so lovely.  And then one by one they got up and stood, and
went a-weaving around the ring so gentle and wavy and graceful, the men
looking ever so tall and airy and straight, with their heads bobbing and
skimming along, away up there under the tent-roof, and every lady's
rose-leafy dress flapping soft and silky around her hips, and she looking
like the most loveliest parasol.

And then faster and faster they went, all of them dancing, first one foot
out in the air and then the other, the horses leaning more and more, and
the ringmaster going round and round the center-pole, cracking his whip
and shouting "Hi!--hi!" and the clown cracking jokes behind him; and by
and by all hands dropped the reins, and every lady put her knuckles on
her hips and every gentleman folded his arms, and then how the horses did
lean over and hump themselves!  And so one after the other they all
skipped off into the ring, and made the sweetest bow I ever see, and then
scampered out, and everybody clapped their hands and went just about
wild.

Well, all through the circus they done the most astonishing things; and
all the time that clown carried on so it most killed the people.  The
ringmaster couldn't ever say a word to him but he was back at him quick
as a wink with the funniest things a body ever said; and how he ever
COULD think of so many of them, and so sudden and so pat, was what I
couldn't noway understand. Why, I couldn't a thought of them in a year.
And by and by a drunk man tried to get into the ring--said he wanted to
ride; said he could ride as well as anybody that ever was.  They argued
and tried to keep him out, but he wouldn't listen, and the whole show
come to a standstill.  Then the people begun to holler at him and make
fun of him, and that made him mad, and he begun to rip and tear; so that
stirred up the people, and a lot of men begun to pile down off of the
benches and swarm towards the ring, saying, "Knock him down! throw him
out!" and one or two women begun to scream.  So, then, the ringmaster he
made a little speech, and said he hoped there wouldn't be no disturbance,
and if the man would promise he wouldn't make no more trouble he would
let him ride if he thought he could stay on the horse.  So everybody
laughed and said all right, and the man got on. The minute he was on, the
horse begun to rip and tear and jump and cavort around, with two circus
men hanging on to his bridle trying to hold him, and the drunk man
hanging on to his neck, and his heels flying in the air every jump, and
the whole crowd of people standing up shouting and laughing till tears
rolled down.  And at last, sure enough, all the circus men could do, the
horse broke loose, and away he went like the very nation, round and round
the ring, with that sot laying down on him and hanging to his neck, with
first one leg hanging most to the ground on one side, and then t'other
one on t'other side, and the people just crazy.  It warn't funny to me,
though; I was all of a tremble to see his danger.  But pretty soon he
struggled up astraddle and grabbed the bridle, a-reeling this way and
that; and the next minute he sprung up and dropped the bridle and stood!
and the horse a-going like a house afire too.  He just stood up there,
a-sailing around as easy and comfortable as if he warn't ever drunk in his
life--and then he begun to pull off his clothes and sling them.  He shed
them so thick they kind of clogged up the air, and altogether he shed
seventeen suits. And, then, there he was, slim and handsome, and dressed
the gaudiest and prettiest you ever saw, and he lit into that horse with
his whip and made him fairly hum--and finally skipped off, and made his
bow and danced off to the dressing-room, and everybody just a-howling
with pleasure and astonishment.

Then the ringmaster he see how he had been fooled, and he WAS the sickest
ringmaster you ever see, I reckon.  Why, it was one of his own men!  He
had got up that joke all out of his own head, and never let on to nobody.
Well, I felt sheepish enough to be took in so, but I wouldn't a been in
that ringmaster's place, not for a thousand dollars.  I don't know; there
may be bullier circuses than what that one was, but I never struck them
yet. Anyways, it was plenty good enough for ME; and wherever I run across
it, it can have all of MY custom every time.

Well, that night we had OUR show; but there warn't only about twelve
people there--just enough to pay expenses.  And they laughed all the
time, and that made the duke mad; and everybody left, anyway, before the
show was over, but one boy which was asleep.  So the duke said these
Arkansaw lunkheads couldn't come up to Shakespeare; what they wanted was
low comedy--and maybe something ruther worse than low comedy, he
reckoned.  He said he could size their style.  So next morning he got
some big sheets of wrapping paper and some black paint, and drawed off
some handbills, and stuck them up all over the village.  The bills said:

AT THE COURT HOUSE! FOR 3 NIGHTS ONLY!
The World-Renowned Tragedians
DAVID GARRICK THE YOUNGER!
AND EDMUND KEAN THE ELDER!
Of the London and
Continental Theatres,
In their Thrilling Tragedy of
THE KING'S CAMELEOPARD,
OR THE ROYAL NONESUCH ! ! !
Admission 50 cents.

Then at the bottom was the biggest line of all, which said:

LADIES AND CHILDREN NOT ADMITTED.

"There," says he, "if that line don't fetch them, I don't know Arkansaw!"




CHAPTER XXIII.

WELL, all day him and the king was hard at it, rigging up a stage and a
curtain and a row of candles for footlights; and that night the house was
jam full of men in no time.  When the place couldn't hold no more, the
duke he quit tending door and went around the back way and come on to the
stage and stood up before the curtain and made a little speech, and
praised up this tragedy, and said it was the most thrillingest one that
ever was; and so he went on a-bragging about the tragedy, and about
Edmund Kean the Elder, which was to play the main principal part in it;
and at last when he'd got everybody's expectations up high enough, he
rolled up the curtain, and the next minute the king come a-prancing out
on all fours, naked; and he was painted all over, ring-streaked-and-
striped, all sorts of colors, as splendid as a rainbow.  And--but never
mind the rest of his outfit; it was just wild, but it was awful funny.
The people most killed themselves laughing; and when the king got done
capering and capered off behind the scenes, they roared and clapped and
stormed and haw-hawed till he come back and done it over again, and after
that they made him do it another time. Well, it would make a cow laugh to
see the shines that old idiot cut.

Then the duke he lets the curtain down, and bows to the people, and says
the great tragedy will be performed only two nights more, on accounts of
pressing London engagements, where the seats is all sold already for it
in Drury Lane; and then he makes them another bow, and says if he has
succeeded in pleasing them and instructing them, he will be deeply
obleeged if they will mention it to their friends and get them to come
and see it.

Twenty people sings out:

"What, is it over?  Is that ALL?"

The duke says yes.  Then there was a fine time.  Everybody sings out,
"Sold!" and rose up mad, and was a-going for that stage and them
tragedians.  But a big, fine looking man jumps up on a bench and shouts:

"Hold on!  Just a word, gentlemen."  They stopped to listen.  "We are
sold--mighty badly sold.  But we don't want to be the laughing stock of
this whole town, I reckon, and never hear the last of this thing as long
as we live.  NO.  What we want is to go out of here quiet, and talk this
show up, and sell the REST of the town!  Then we'll all be in the same
boat.  Ain't that sensible?" ("You bet it is!--the jedge is right!"
everybody sings out.) "All right, then--not a word about any sell.  Go
along home, and advise everybody to come and see the tragedy."

Next day you couldn't hear nothing around that town but how splendid that
show was.  House was jammed again that night, and we sold this crowd the
same way.  When me and the king and the duke got home to the raft we all
had a supper; and by and by, about midnight, they made Jim and me back
her out and float her down the middle of the river, and fetch her in and
hide her about two mile below town.

The third night the house was crammed again--and they warn't new-comers
this time, but people that was at the show the other two nights.  I stood
by the duke at the door, and I see that every man that went in had his
pockets bulging, or something muffled up under his coat--and I see it
warn't no perfumery, neither, not by a long sight.  I smelt sickly eggs
by the barrel, and rotten cabbages, and such things; and if I know the
signs of a dead cat being around, and I bet I do, there was sixty-four of
them went in.  I shoved in there for a minute, but it was too various for
me; I couldn't stand it.  Well, when the place couldn't hold no more
people the duke he give a fellow a quarter and told him to tend door for
him a minute, and then he started around for the stage door, I after him;
but the minute we turned the corner and was in the dark he says:

"Walk fast now till you get away from the houses, and then shin for the
raft like the dickens was after you!"

I done it, and he done the same.  We struck the raft at the same time,
and in less than two seconds we was gliding down stream, all dark and
still, and edging towards the middle of the river, nobody saying a word.
I reckoned the poor king was in for a gaudy time of it with the audience,
but nothing of the sort; pretty soon he crawls out from under the wigwam,
and says:

"Well, how'd the old thing pan out this time, duke?"  He hadn't been
up-town at all.

We never showed a light till we was about ten mile below the village.
Then we lit up and had a supper, and the king and the duke fairly laughed
their bones loose over the way they'd served them people.  The duke says:

"Greenhorns, flatheads!  I knew the first house would keep mum and let
the rest of the town get roped in; and I knew they'd lay for us the third
night, and consider it was THEIR turn now.  Well, it IS their turn, and
I'd give something to know how much they'd take for it.  I WOULD just
like to know how they're putting in their opportunity.  They can turn it
into a picnic if they want to--they brought plenty provisions."

Them rapscallions took in four hundred and sixty-five dollars in that
three nights.  I never see money hauled in by the wagon-load like that
before.  By and by, when they was asleep and snoring, Jim says:

"Don't it s'prise you de way dem kings carries on, Huck?"

"No," I says, "it don't."

"Why don't it, Huck?"

"Well, it don't, because it's in the breed.  I reckon they're all alike,"

"But, Huck, dese kings o' ourn is reglar rapscallions; dat's jist what
dey is; dey's reglar rapscallions."

"Well, that's what I'm a-saying; all kings is mostly rapscallions, as fur
as I can make out."

"Is dat so?"

"You read about them once--you'll see.  Look at Henry the Eight; this 'n
's a Sunday-school Superintendent to HIM.  And look at Charles Second,
and Louis Fourteen, and Louis Fifteen, and James Second, and Edward
Second, and Richard Third, and forty more; besides all them Saxon
heptarchies that used to rip around so in old times and raise Cain.  My,
you ought to seen old Henry the Eight when he was in bloom.  He WAS a
blossom.  He used to marry a new wife every day, and chop off her head
next morning.  And he would do it just as indifferent as if he was
ordering up eggs.  'Fetch up Nell Gwynn,' he says.  They fetch her up.
Next morning, 'Chop off her head!'  And they chop it off.  'Fetch up Jane
Shore,' he says; and up she comes, Next morning, 'Chop off her head'--and
they chop it off.  'Ring up Fair Rosamun.'  Fair Rosamun answers the
bell.  Next morning, 'Chop off her head.'  And he made every one of them
tell him a tale every night; and he kept that up till he had hogged a
thousand and one tales that way, and then he put them all in a book, and
called it Domesday Book--which was a good name and stated the case.  You
don't know kings, Jim, but I know them; and this old rip of ourn is one
of the cleanest I've struck in history.  Well, Henry he takes a notion he
wants to get up some trouble with this country. How does he go at it
--give notice?--give the country a show?  No.  All of a sudden he heaves
all the tea in Boston Harbor overboard, and whacks out a declaration of
independence, and dares them to come on.  That was HIS style--he never
give anybody a chance.  He had suspicions of his father, the Duke of
Wellington.  Well, what did he do?  Ask him to show up?  No--drownded
him in a butt of mamsey, like a cat.  S'pose people left money laying
around where he was--what did he do?  He collared it.  S'pose he
contracted to do a thing, and you paid him, and didn't set down there and
see that he done it--what did he do?  He always done the other thing.
S'pose he opened his mouth--what then?  If he didn't shut it up powerful
quick he'd lose a lie every time.  That's the kind of a bug Henry was;
and if we'd a had him along 'stead of our kings he'd a fooled that town a
heap worse than ourn done.  I don't say that ourn is lambs, because they
ain't, when you come right down to the cold facts; but they ain't nothing
to THAT old ram, anyway.  All I say is, kings is kings, and you got to
make allowances.  Take them all around, they're a mighty ornery lot.
It's the way they're raised."

"But dis one do SMELL so like de nation, Huck."

"Well, they all do, Jim.  We can't help the way a king smells; history
don't tell no way."

"Now de duke, he's a tolerble likely man in some ways."

"Yes, a duke's different.  But not very different.  This one's a middling
hard lot for a duke.  When he's drunk there ain't no near-sighted man
could tell him from a king."

"Well, anyways, I doan' hanker for no mo' un um, Huck.  Dese is all I kin
stan'."

"It's the way I feel, too, Jim.  But we've got them on our hands, and we
got to remember what they are, and make allowances.  Sometimes I wish we
could hear of a country that's out of kings."

What was the use to tell Jim these warn't real kings and dukes?  It
wouldn't a done no good; and, besides, it was just as I said:  you
couldn't tell them from the real kind.

I went to sleep, and Jim didn't call me when it was my turn.  He often
done that.  When I waked up just at daybreak he was sitting there with
his head down betwixt his knees, moaning and mourning to himself.  I
didn't take notice nor let on.  I knowed what it was about.  He was
thinking about his wife and his children, away up yonder, and he was low
and homesick; because he hadn't ever been away from home before in his
life; and I do believe he cared just as much for his people as white
folks does for their'n.  It don't seem natural, but I reckon it's so.  He
was often moaning and mourning that way nights, when he judged I was
asleep, and saying, "Po' little 'Lizabeth! po' little Johnny! it's mighty
hard; I spec' I ain't ever gwyne to see you no mo', no mo'!"  He was a
mighty good nigger, Jim was.

But this time I somehow got to talking to him about his wife and young
ones; and by and by he says:

"What makes me feel so bad dis time 'uz bekase I hear sumpn over yonder
on de bank like a whack, er a slam, while ago, en it mine me er de time I
treat my little 'Lizabeth so ornery.  She warn't on'y 'bout fo' year ole,
en she tuck de sk'yarlet fever, en had a powful rough spell; but she got
well, en one day she was a-stannin' aroun', en I says to her, I says:

"'Shet de do'.'

"She never done it; jis' stood dah, kiner smilin' up at me.  It make me
mad; en I says agin, mighty loud, I says:

"'Doan' you hear me?  Shet de do'!'

"She jis stood de same way, kiner smilin' up.  I was a-bilin'!  I says:

"'I lay I MAKE you mine!'

"En wid dat I fetch' her a slap side de head dat sont her a-sprawlin'.
Den I went into de yuther room, en 'uz gone 'bout ten minutes; en when I
come back dah was dat do' a-stannin' open YIT, en dat chile stannin' mos'
right in it, a-lookin' down and mournin', en de tears runnin' down.  My,
but I WUZ mad!  I was a-gwyne for de chile, but jis' den--it was a do'
dat open innerds--jis' den, 'long come de wind en slam it to, behine de
chile, ker-BLAM!--en my lan', de chile never move'!  My breff mos' hop
outer me; en I feel so--so--I doan' know HOW I feel.  I crope out, all
a-tremblin', en crope aroun' en open de do' easy en slow, en poke my head
in behine de chile, sof' en still, en all uv a sudden I says POW! jis' as
loud as I could yell.  SHE NEVER BUDGE!  Oh, Huck, I bust out a-cryin' en
grab her up in my arms, en say, 'Oh, de po' little thing!  De Lord God
Amighty fogive po' ole Jim, kaze he never gwyne to fogive hisself as
long's he live!'  Oh, she was plumb deef en dumb, Huck, plumb deef en
dumb--en I'd ben a-treat'n her so!"




CHAPTER XXIV.

NEXT day, towards night, we laid up under a little willow towhead out in
the middle, where there was a village on each side of the river, and the
duke and the king begun to lay out a plan for working them towns.  Jim he
spoke to the duke, and said he hoped it wouldn't take but a few hours,
because it got mighty heavy and tiresome to him when he had to lay all
day in the wigwam tied with the rope.  You see, when we left him all
alone we had to tie him, because if anybody happened on to him all by
himself and not tied it wouldn't look much like he was a runaway nigger,
you know. So the duke said it WAS kind of hard to have to lay roped all
day, and he'd cipher out some way to get around it.

He was uncommon bright, the duke was, and he soon struck it.  He dressed
Jim up in King Lear's outfit--it was a long curtain-calico gown, and a
white horse-hair wig and whiskers; and then he took his theater paint and
painted Jim's face and hands and ears and neck all over a dead, dull,
solid blue, like a man that's been drownded nine days.  Blamed if he
warn't the horriblest looking outrage I ever see.  Then the duke took and
wrote out a sign on a shingle so:

Sick Arab--but harmless when not out of his head.

And he nailed that shingle to a lath, and stood the lath up four or five
foot in front of the wigwam.  Jim was satisfied.  He said it was a sight
better than lying tied a couple of years every day, and trembling all
over every time there was a sound.  The duke told him to make himself
free and easy, and if anybody ever come meddling around, he must hop out
of the wigwam, and carry on a little, and fetch a howl or two like a wild
beast, and he reckoned they would light out and leave him alone.  Which
was sound enough judgment; but you take the average man, and he wouldn't
wait for him to howl.  Why, he didn't only look like he was dead, he
looked considerable more than that.

These rapscallions wanted to try the Nonesuch again, because there was so
much money in it, but they judged it wouldn't be safe, because maybe the
news might a worked along down by this time.  They couldn't hit no
project that suited exactly; so at last the duke said he reckoned he'd
lay off and work his brains an hour or two and see if he couldn't put up
something on the Arkansaw village; and the king he allowed he would drop
over to t'other village without any plan, but just trust in Providence to
lead him the profitable way--meaning the devil, I reckon.  We had all
bought store clothes where we stopped last; and now the king put his'n
on, and he told me to put mine on.  I done it, of course.  The king's
duds was all black, and he did look real swell and starchy.  I never
knowed how clothes could change a body before.  Why, before, he looked
like the orneriest old rip that ever was; but now, when he'd take off his
new white beaver and make a bow and do a smile, he looked that grand and
good and pious that you'd say he had walked right out of the ark, and
maybe was old Leviticus himself.  Jim cleaned up the canoe, and I got my
paddle ready.  There was a big steamboat laying at the shore away up
under the point, about three mile above the town--been there a couple
of hours, taking on freight.  Says the king:

"Seein' how I'm dressed, I reckon maybe I better arrive down from St.
Louis or Cincinnati, or some other big place.  Go for the steamboat,
Huckleberry; we'll come down to the village on her."

I didn't have to be ordered twice to go and take a steamboat ride.  I
fetched the shore a half a mile above the village, and then went scooting
along the bluff bank in the easy water.  Pretty soon we come to a nice
innocent-looking young country jake setting on a log swabbing the sweat
off of his face, for it was powerful warm weather; and he had a couple of
big carpet-bags by him.

"Run her nose in shore," says the king.  I done it.  "Wher' you bound
for, young man?"

"For the steamboat; going to Orleans."

"Git aboard," says the king.  "Hold on a minute, my servant 'll he'p you
with them bags.  Jump out and he'p the gentleman, Adolphus"--meaning me,
I see.

I done so, and then we all three started on again.  The young chap was
mighty thankful; said it was tough work toting his baggage such weather.
He asked the king where he was going, and the king told him he'd come
down the river and landed at the other village this morning, and now he
was going up a few mile to see an old friend on a farm up there.  The
young fellow says:

"When I first see you I says to myself, 'It's Mr. Wilks, sure, and he
come mighty near getting here in time.'  But then I says again, 'No, I
reckon it ain't him, or else he wouldn't be paddling up the river.'  You
AIN'T him, are you?"

"No, my name's Blodgett--Elexander Blodgett--REVEREND Elexander Blodgett,
I s'pose I must say, as I'm one o' the Lord's poor servants.  But still
I'm jist as able to be sorry for Mr. Wilks for not arriving in time, all
the same, if he's missed anything by it--which I hope he hasn't."

"Well, he don't miss any property by it, because he'll get that all
right; but he's missed seeing his brother Peter die--which he mayn't
mind, nobody can tell as to that--but his brother would a give anything
in this world to see HIM before he died; never talked about nothing else
all these three weeks; hadn't seen him since they was boys together--and
hadn't ever seen his brother William at all--that's the deef and dumb
one--William ain't more than thirty or thirty-five.  Peter and George
were the only ones that come out here; George was the married brother;
him and his wife both died last year.  Harvey and William's the only ones
that's left now; and, as I was saying, they haven't got here in time."

"Did anybody send 'em word?"

"Oh, yes; a month or two ago, when Peter was first took; because Peter
said then that he sorter felt like he warn't going to get well this time.
You see, he was pretty old, and George's g'yirls was too young to be much
company for him, except Mary Jane, the red-headed one; and so he was
kinder lonesome after George and his wife died, and didn't seem to care
much to live.  He most desperately wanted to see Harvey--and William,
too, for that matter--because he was one of them kind that can't bear to
make a will.  He left a letter behind for Harvey, and said he'd told in
it where his money was hid, and how he wanted the rest of the property
divided up so George's g'yirls would be all right--for George didn't
leave nothing.  And that letter was all they could get him to put a pen
to."

"Why do you reckon Harvey don't come?  Wher' does he live?"

"Oh, he lives in England--Sheffield--preaches there--hasn't ever been in
this country.  He hasn't had any too much time--and besides he mightn't a
got the letter at all, you know."

"Too bad, too bad he couldn't a lived to see his brothers, poor soul.
You going to Orleans, you say?"

"Yes, but that ain't only a part of it.  I'm going in a ship, next
Wednesday, for Ryo Janeero, where my uncle lives."

"It's a pretty long journey.  But it'll be lovely; wisht I was a-going.
Is Mary Jane the oldest?  How old is the others?"

"Mary Jane's nineteen, Susan's fifteen, and Joanna's about fourteen
--that's the one that gives herself to good works and has a hare-lip."


"Poor things! to be left alone in the cold world so."

"Well, they could be worse off.  Old Peter had friends, and they ain't
going to let them come to no harm.  There's Hobson, the Babtis' preacher;
and Deacon Lot Hovey, and Ben Rucker, and Abner Shackleford, and Levi
Bell, the lawyer; and Dr. Robinson, and their wives, and the widow
Bartley, and--well, there's a lot of them; but these are the ones that
Peter was thickest with, and used to write about sometimes, when he wrote
home; so Harvey 'll know where to look for friends when he gets here."

Well, the old man went on asking questions till he just fairly emptied
that young fellow.  Blamed if he didn't inquire about everybody and
everything in that blessed town, and all about the Wilkses; and about
Peter's business--which was a tanner; and about George's--which was a
carpenter; and about Harvey's--which was a dissentering minister; and so
on, and so on.  Then he says:

"What did you want to walk all the way up to the steamboat for?"

"Because she's a big Orleans boat, and I was afeard she mightn't stop
there.  When they're deep they won't stop for a hail.  A Cincinnati boat
will, but this is a St. Louis one."

"Was Peter Wilks well off?"

"Oh, yes, pretty well off.  He had houses and land, and it's reckoned he
left three or four thousand in cash hid up som'ers."

"When did you say he died?"

"I didn't say, but it was last night."

"Funeral to-morrow, likely?"

"Yes, 'bout the middle of the day."

"Well, it's all terrible sad; but we've all got to go, one time or
another. So what we want to do is to be prepared; then we're all right."

"Yes, sir, it's the best way.  Ma used to always say that."

When we struck the boat she was about done loading, and pretty soon she
got off.  The king never said nothing about going aboard, so I lost my
ride, after all.  When the boat was gone the king made me paddle up
another mile to a lonesome place, and then he got ashore and says:

"Now hustle back, right off, and fetch the duke up here, and the new
carpet-bags.  And if he's gone over to t'other side, go over there and
git him.  And tell him to git himself up regardless.  Shove along, now."

I see what HE was up to; but I never said nothing, of course.  When I got
back with the duke we hid the canoe, and then they set down on a log, and
the king told him everything, just like the young fellow had said it
--every last word of it.  And all the time he was a-doing it he tried to
talk like an Englishman; and he done it pretty well, too, for a slouch.
I can't imitate him, and so I ain't a-going to try to; but he really done
it pretty good.  Then he says:

"How are you on the deef and dumb, Bilgewater?"

The duke said, leave him alone for that; said he had played a deef and
dumb person on the histronic boards.  So then they waited for a
steamboat.

About the middle of the afternoon a couple of little boats come along,
but they didn't come from high enough up the river; but at last there was
a big one, and they hailed her.  She sent out her yawl, and we went
aboard, and she was from Cincinnati; and when they found we only wanted
to go four or five mile they was booming mad, and gave us a cussing, and
said they wouldn't land us.  But the king was ca'm.  He says:

"If gentlemen kin afford to pay a dollar a mile apiece to be took on and
put off in a yawl, a steamboat kin afford to carry 'em, can't it?"

So they softened down and said it was all right; and when we got to the
village they yawled us ashore.  About two dozen men flocked down when
they see the yawl a-coming, and when the king says:

"Kin any of you gentlemen tell me wher' Mr. Peter Wilks lives?" they give
a glance at one another, and nodded their heads, as much as to say, "What
d' I tell you?"  Then one of them says, kind of soft and gentle:

"I'm sorry sir, but the best we can do is to tell you where he DID live
yesterday evening."

Sudden as winking the ornery old cretur went an to smash, and fell up
against the man, and put his chin on his shoulder, and cried down his
back, and says:

"Alas, alas, our poor brother--gone, and we never got to see him; oh,
it's too, too hard!"

Then he turns around, blubbering, and makes a lot of idiotic signs to the
duke on his hands, and blamed if he didn't drop a carpet-bag and bust out
a-crying.  If they warn't the beatenest lot, them two frauds, that ever I
struck.

Well, the men gathered around and sympathized with them, and said all
sorts of kind things to them, and carried their carpet-bags up the hill
for them, and let them lean on them and cry, and told the king all about
his brother's last moments, and the king he told it all over again on his
hands to the duke, and both of them took on about that dead tanner like
they'd lost the twelve disciples.  Well, if ever I struck anything like
it, I'm a nigger. It was enough to make a body ashamed of the human race.




CHAPTER XXV.

THE news was all over town in two minutes, and you could see the people
tearing down on the run from every which way, some of them putting on
their coats as they come.  Pretty soon we was in the middle of a crowd,
and the noise of the tramping was like a soldier march.  The windows and
dooryards was full; and every minute somebody would say, over a fence:

"Is it THEM?"

And somebody trotting along with the gang would answer back and say:

"You bet it is."

When we got to the house the street in front of it was packed, and the
three girls was standing in the door.  Mary Jane WAS red-headed, but that
don't make no difference, she was most awful beautiful, and her face and
her eyes was all lit up like glory, she was so glad her uncles was come.
The king he spread his arms, and Mary Jane she jumped for them, and the
hare-lip jumped for the duke, and there they HAD it!  Everybody most,
leastways women, cried for joy to see them meet again at last and have
such good times.

Then the king he hunched the duke private--I see him do it--and then he
looked around and see the coffin, over in the corner on two chairs; so
then him and the duke, with a hand across each other's shoulder, and
t'other hand to their eyes, walked slow and solemn over there, everybody
dropping back to give them room, and all the talk and noise stopping,
people saying "Sh!" and all the men taking their hats off and drooping
their heads, so you could a heard a pin fall.  And when they got there
they bent over and looked in the coffin, and took one sight, and then
they bust out a-crying so you could a heard them to Orleans, most; and
then they put their arms around each other's necks, and hung their chins
over each other's shoulders; and then for three minutes, or maybe four, I
never see two men leak the way they done.  And, mind you, everybody was
doing the same; and the place was that damp I never see anything like it.
Then one of them got on one side of the coffin, and t'other on t'other
side, and they kneeled down and rested their foreheads on the coffin, and
let on to pray all to themselves.  Well, when it come to that it worked
the crowd like you never see anything like it, and everybody broke down
and went to sobbing right out loud--the poor girls, too; and every woman,
nearly, went up to the girls, without saying a word, and kissed them,
solemn, on the forehead, and then put their hand on their head, and
looked up towards the sky, with the tears running down, and then busted
out and went off sobbing and swabbing, and give the next woman a show.  I
never see anything so disgusting.

Well, by and by the king he gets up and comes forward a little, and works
himself up and slobbers out a speech, all full of tears and flapdoodle
about its being a sore trial for him and his poor brother to lose the
diseased, and to miss seeing diseased alive after the long journey of
four thousand mile, but it's a trial that's sweetened and sanctified to
us by this dear sympathy and these holy tears, and so he thanks them out
of his heart and out of his brother's heart, because out of their mouths
they can't, words being too weak and cold, and all that kind of rot and
slush, till it was just sickening; and then he blubbers out a pious
goody-goody Amen, and turns himself loose and goes to crying fit to bust.

And the minute the words were out of his mouth somebody over in the crowd
struck up the doxolojer, and everybody joined in with all their might,
and it just warmed you up and made you feel as good as church letting
out. Music is a good thing; and after all that soul-butter and hogwash I
never see it freshen up things so, and sound so honest and bully.

Then the king begins to work his jaw again, and says how him and his
nieces would be glad if a few of the main principal friends of the family
would take supper here with them this evening, and help set up with the
ashes of the diseased; and says if his poor brother laying yonder could
speak he knows who he would name, for they was names that was very dear
to him, and mentioned often in his letters; and so he will name the same,
to wit, as follows, vizz.:--Rev. Mr. Hobson, and Deacon Lot Hovey, and
Mr. Ben Rucker, and Abner Shackleford, and Levi Bell, and Dr. Robinson,
and their wives, and the widow Bartley.

Rev. Hobson and Dr. Robinson was down to the end of the town a-hunting
together--that is, I mean the doctor was shipping a sick man to t'other
world, and the preacher was pinting him right.  Lawyer Bell was away up
to Louisville on business.  But the rest was on hand, and so they all
come and shook hands with the king and thanked him and talked to him; and
then they shook hands with the duke and didn't say nothing, but just kept
a-smiling and bobbing their heads like a passel of sapheads whilst he
made all sorts of signs with his hands and said "Goo-goo--goo-goo-goo"
all the time, like a baby that can't talk.

So the king he blattered along, and managed to inquire about pretty much
everybody and dog in town, by his name, and mentioned all sorts of little
things that happened one time or another in the town, or to George's
family, or to Peter.  And he always let on that Peter wrote him the
things; but that was a lie:  he got every blessed one of them out of that
young flathead that we canoed up to the steamboat.

Then Mary Jane she fetched the letter her father left behind, and the
king he read it out loud and cried over it.  It give the dwelling-house
and three thousand dollars, gold, to the girls; and it give the tanyard
(which was doing a good business), along with some other houses and land
(worth about seven thousand), and three thousand dollars in gold to
Harvey and William, and told where the six thousand cash was hid down
cellar.  So these two frauds said they'd go and fetch it up, and have
everything square and above-board; and told me to come with a candle.  We
shut the cellar door behind us, and when they found the bag they spilt it
out on the floor, and it was a lovely sight, all them yaller-boys.  My,
the way the king's eyes did shine!  He slaps the duke on the shoulder and
says:

"Oh, THIS ain't bully nor noth'n!  Oh, no, I reckon not!  Why, Biljy, it
beats the Nonesuch, DON'T it?"

The duke allowed it did.  They pawed the yaller-boys, and sifted them
through their fingers and let them jingle down on the floor; and the king
says:

"It ain't no use talkin'; bein' brothers to a rich dead man and
representatives of furrin heirs that's got left is the line for you and
me, Bilge.  Thish yer comes of trust'n to Providence.  It's the best way,
in the long run.  I've tried 'em all, and ther' ain't no better way."

Most everybody would a been satisfied with the pile, and took it on
trust; but no, they must count it.  So they counts it, and it comes out
four hundred and fifteen dollars short.  Says the king:

"Dern him, I wonder what he done with that four hundred and fifteen
dollars?"

They worried over that awhile, and ransacked all around for it.  Then the
duke says:

"Well, he was a pretty sick man, and likely he made a mistake--I reckon
that's the way of it.  The best way's to let it go, and keep still about
it.  We can spare it."

"Oh, shucks, yes, we can SPARE it.  I don't k'yer noth'n 'bout that--it's
the COUNT I'm thinkin' about.  We want to be awful square and open and
above-board here, you know.  We want to lug this h-yer money up stairs
and count it before everybody--then ther' ain't noth'n suspicious.  But
when the dead man says ther's six thous'n dollars, you know, we don't
want to--"

"Hold on," says the duke.  "Le's make up the deffisit," and he begun to
haul out yaller-boys out of his pocket.

"It's a most amaz'n' good idea, duke--you HAVE got a rattlin' clever head
on you," says the king.  "Blest if the old Nonesuch ain't a heppin' us
out agin," and HE begun to haul out yaller-jackets and stack them up.

It most busted them, but they made up the six thousand clean and clear.

"Say," says the duke, "I got another idea.  Le's go up stairs and count
this money, and then take and GIVE IT TO THE GIRLS."

"Good land, duke, lemme hug you!  It's the most dazzling idea 'at ever a
man struck.  You have cert'nly got the most astonishin' head I ever see.
Oh, this is the boss dodge, ther' ain't no mistake 'bout it.  Let 'em
fetch along their suspicions now if they want to--this 'll lay 'em out."

When we got up-stairs everybody gethered around the table, and the king
he counted it and stacked it up, three hundred dollars in a pile--twenty
elegant little piles.  Everybody looked hungry at it, and licked their
chops.  Then they raked it into the bag again, and I see the king begin
to swell himself up for another speech.  He says:

"Friends all, my poor brother that lays yonder has done generous by them
that's left behind in the vale of sorrers.  He has done generous by these
yer poor little lambs that he loved and sheltered, and that's left
fatherless and motherless.  Yes, and we that knowed him knows that he
would a done MORE generous by 'em if he hadn't ben afeard o' woundin' his
dear William and me.  Now, WOULDN'T he?  Ther' ain't no question 'bout it
in MY mind.  Well, then, what kind o' brothers would it be that 'd stand
in his way at sech a time?  And what kind o' uncles would it be that 'd
rob--yes, ROB--sech poor sweet lambs as these 'at he loved so at sech a
time?  If I know William--and I THINK I do--he--well, I'll jest ask him."
He turns around and begins to make a lot of signs to the duke with his
hands, and the duke he looks at him stupid and leather-headed a while;
then all of a sudden he seems to catch his meaning, and jumps for the
king, goo-gooing with all his might for joy, and hugs him about fifteen
times before he lets up.  Then the king says, "I knowed it; I reckon THAT
'll convince anybody the way HE feels about it.  Here, Mary Jane, Susan,
Joanner, take the money--take it ALL.  It's the gift of him that lays
yonder, cold but joyful."

Mary Jane she went for him, Susan and the hare-lip went for the duke, and
then such another hugging and kissing I never see yet.  And everybody
crowded up with the tears in their eyes, and most shook the hands off of
them frauds, saying all the time:

"You DEAR good souls!--how LOVELY!--how COULD you!"

Well, then, pretty soon all hands got to talking about the diseased
again, and how good he was, and what a loss he was, and all that; and
before long a big iron-jawed man worked himself in there from outside,
and stood a-listening and looking, and not saying anything; and nobody
saying anything to him either, because the king was talking and they was
all busy listening.  The king was saying--in the middle of something he'd
started in on--

"--they bein' partickler friends o' the diseased.  That's why they're
invited here this evenin'; but tomorrow we want ALL to come--everybody;
for he respected everybody, he liked everybody, and so it's fitten that
his funeral orgies sh'd be public."

And so he went a-mooning on and on, liking to hear himself talk, and
every little while he fetched in his funeral orgies again, till the duke
he couldn't stand it no more; so he writes on a little scrap of paper,
"OBSEQUIES, you old fool," and folds it up, and goes to goo-gooing and
reaching it over people's heads to him.  The king he reads it and puts it
in his pocket, and says:

"Poor William, afflicted as he is, his HEART'S aluz right.  Asks me to
invite everybody to come to the funeral--wants me to make 'em all
welcome.  But he needn't a worried--it was jest what I was at."

Then he weaves along again, perfectly ca'm, and goes to dropping in his
funeral orgies again every now and then, just like he done before.  And
when he done it the third time he says:

"I say orgies, not because it's the common term, because it ain't
--obsequies bein' the common term--but because orgies is the right term.
Obsequies ain't used in England no more now--it's gone out.  We say
orgies now in England.  Orgies is better, because it means the thing
you're after more exact.  It's a word that's made up out'n the Greek
ORGO, outside, open, abroad; and the Hebrew JEESUM, to plant, cover up;
hence inTER.  So, you see, funeral orgies is an open er public funeral."

He was the WORST I ever struck.  Well, the iron-jawed man he laughed
right in his face.  Everybody was shocked.  Everybody says, "Why,
DOCTOR!" and Abner Shackleford says:

"Why, Robinson, hain't you heard the news?  This is Harvey Wilks."

The king he smiled eager, and shoved out his flapper, and says:

"Is it my poor brother's dear good friend and physician?  I--"

"Keep your hands off of me!" says the doctor.  "YOU talk like an
Englishman, DON'T you?  It's the worst imitation I ever heard.  YOU Peter
Wilks's brother!  You're a fraud, that's what you are!"

Well, how they all took on!  They crowded around the doctor and tried to
quiet him down, and tried to explain to him and tell him how Harvey 'd
showed in forty ways that he WAS Harvey, and knowed everybody by name,
and the names of the very dogs, and begged and BEGGED him not to hurt
Harvey's feelings and the poor girl's feelings, and all that.  But it
warn't no use; he stormed right along, and said any man that pretended to
be an Englishman and couldn't imitate the lingo no better than what he
did was a fraud and a liar.  The poor girls was hanging to the king and
crying; and all of a sudden the doctor ups and turns on THEM.  He says:

"I was your father's friend, and I'm your friend; and I warn you as a
friend, and an honest one that wants to protect you and keep you out of
harm and trouble, to turn your backs on that scoundrel and have nothing
to do with him, the ignorant tramp, with his idiotic Greek and Hebrew, as
he calls it.  He is the thinnest kind of an impostor--has come here with
a lot of empty names and facts which he picked up somewheres, and you
take them for PROOFS, and are helped to fool yourselves by these foolish
friends here, who ought to know better.  Mary Jane Wilks, you know me for
your friend, and for your unselfish friend, too.  Now listen to me; turn
this pitiful rascal out--I BEG you to do it.  Will you?"

Mary Jane straightened herself up, and my, but she was handsome!  She
says:

"HERE is my answer."  She hove up the bag of money and put it in the
king's hands, and says, "Take this six thousand dollars, and invest for
me and my sisters any way you want to, and don't give us no receipt for
it."

Then she put her arm around the king on one side, and Susan and the
hare-lip done the same on the other.  Everybody clapped their hands and
stomped on the floor like a perfect storm, whilst the king held up his
head and smiled proud.  The doctor says:

"All right; I wash MY hands of the matter.  But I warn you all that a
time 's coming when you're going to feel sick whenever you think of this
day." And away he went.

"All right, doctor," says the king, kinder mocking him; "we'll try and
get 'em to send for you;" which made them all laugh, and they said it was
a prime good hit.




CHAPTER XXVI.

WELL, when they was all gone the king he asks Mary Jane how they was off
for spare rooms, and she said she had one spare room, which would do for
Uncle William, and she'd give her own room to Uncle Harvey, which was a
little bigger, and she would turn into the room with her sisters and
sleep on a cot; and up garret was a little cubby, with a pallet in it.
The king said the cubby would do for his valley--meaning me.

So Mary Jane took us up, and she showed them their rooms, which was plain
but nice.  She said she'd have her frocks and a lot of other traps took
out of her room if they was in Uncle Harvey's way, but he said they
warn't.  The frocks was hung along the wall, and before them was a
curtain made out of calico that hung down to the floor.  There was an old
hair trunk in one corner, and a guitar-box in another, and all sorts of
little knickknacks and jimcracks around, like girls brisken up a room
with.  The king said it was all the more homely and more pleasanter for
these fixings, and so don't disturb them.  The duke's room was pretty
small, but plenty good enough, and so was my cubby.

That night they had a big supper, and all them men and women was there,
and I stood behind the king and the duke's chairs and waited on them, and
the niggers waited on the rest.  Mary Jane she set at the head of the
table, with Susan alongside of her, and said how bad the biscuits was,
and how mean the preserves was, and how ornery and tough the fried
chickens was--and all that kind of rot, the way women always do for to
force out compliments; and the people all knowed everything was tiptop,
and said so--said "How DO you get biscuits to brown so nice?" and "Where,
for the land's sake, DID you get these amaz'n pickles?" and all that kind
of humbug talky-talk, just the way people always does at a supper, you
know.

And when it was all done me and the hare-lip had supper in the kitchen
off of the leavings, whilst the others was helping the niggers clean up
the things.  The hare-lip she got to pumping me about England, and blest
if I didn't think the ice was getting mighty thin sometimes.  She says:

"Did you ever see the king?"

"Who?  William Fourth?  Well, I bet I have--he goes to our church."  I
knowed he was dead years ago, but I never let on.  So when I says he goes
to our church, she says:

"What--regular?"

"Yes--regular.  His pew's right over opposite ourn--on t'other side the
pulpit."

"I thought he lived in London?"

"Well, he does.  Where WOULD he live?"

"But I thought YOU lived in Sheffield?"

I see I was up a stump.  I had to let on to get choked with a chicken
bone, so as to get time to think how to get down again.  Then I says:

"I mean he goes to our church regular when he's in Sheffield.  That's
only in the summer time, when he comes there to take the sea baths."

"Why, how you talk--Sheffield ain't on the sea."

"Well, who said it was?"

"Why, you did."

"I DIDN'T nuther."

"You did!"

"I didn't."

"You did."

"I never said nothing of the kind."

"Well, what DID you say, then?"

"Said he come to take the sea BATHS--that's what I said."

"Well, then, how's he going to take the sea baths if it ain't on the
sea?"

"Looky here," I says; "did you ever see any Congress-water?"

"Yes."

"Well, did you have to go to Congress to get it?"

"Why, no."

"Well, neither does William Fourth have to go to the sea to get a sea
bath."

"How does he get it, then?"

"Gets it the way people down here gets Congress-water--in barrels.  There
in the palace at Sheffield they've got furnaces, and he wants his water
hot.  They can't bile that amount of water away off there at the sea.
They haven't got no conveniences for it."

"Oh, I see, now.  You might a said that in the first place and saved
time."

When she said that I see I was out of the woods again, and so I was
comfortable and glad.  Next, she says:

"Do you go to church, too?"

"Yes--regular."

"Where do you set?"

"Why, in our pew."

"WHOSE pew?"

"Why, OURN--your Uncle Harvey's."

"His'n?  What does HE want with a pew?"

"Wants it to set in.  What did you RECKON he wanted with it?"

"Why, I thought he'd be in the pulpit."

Rot him, I forgot he was a preacher.  I see I was up a stump again, so I
played another chicken bone and got another think.  Then I says:

"Blame it, do you suppose there ain't but one preacher to a church?"

"Why, what do they want with more?"

"What!--to preach before a king?  I never did see such a girl as you.
They don't have no less than seventeen."

"Seventeen!  My land!  Why, I wouldn't set out such a string as that, not
if I NEVER got to glory.  It must take 'em a week."

"Shucks, they don't ALL of 'em preach the same day--only ONE of 'em."

"Well, then, what does the rest of 'em do?"

"Oh, nothing much.  Loll around, pass the plate--and one thing or
another.  But mainly they don't do nothing."

"Well, then, what are they FOR?"

"Why, they're for STYLE.  Don't you know nothing?"

"Well, I don't WANT to know no such foolishness as that.  How is servants
treated in England?  Do they treat 'em better 'n we treat our niggers?"

"NO!  A servant ain't nobody there.  They treat them worse than dogs."

"Don't they give 'em holidays, the way we do, Christmas and New Year's
week, and Fourth of July?"

"Oh, just listen!  A body could tell YOU hain't ever been to England by
that.  Why, Hare-l--why, Joanna, they never see a holiday from year's end
to year's end; never go to the circus, nor theater, nor nigger shows, nor
nowheres."

"Nor church?"

"Nor church."

"But YOU always went to church."

Well, I was gone up again.  I forgot I was the old man's servant.  But
next minute I whirled in on a kind of an explanation how a valley was
different from a common servant and HAD to go to church whether he wanted
to or not, and set with the family, on account of its being the law.  But
I didn't do it pretty good, and when I got done I see she warn't
satisfied.  She says:

"Honest injun, now, hain't you been telling me a lot of lies?"

"Honest injun," says I.

"None of it at all?"

"None of it at all.  Not a lie in it," says I.

"Lay your hand on this book and say it."

I see it warn't nothing but a dictionary, so I laid my hand on it and
said it.  So then she looked a little better satisfied, and says:

"Well, then, I'll believe some of it; but I hope to gracious if I'll
believe the rest."

"What is it you won't believe, Joe?" says Mary Jane, stepping in with
Susan behind her.  "It ain't right nor kind for you to talk so to him,
and him a stranger and so far from his people.  How would you like to be
treated so?"

"That's always your way, Maim--always sailing in to help somebody before
they're hurt.  I hain't done nothing to him.  He's told some stretchers,
I reckon, and I said I wouldn't swallow it all; and that's every bit and
grain I DID say.  I reckon he can stand a little thing like that, can't
he?"

"I don't care whether 'twas little or whether 'twas big; he's here in our
house and a stranger, and it wasn't good of you to say it.  If you was in
his place it would make you feel ashamed; and so you oughtn't to say a
thing to another person that will make THEM feel ashamed."

"Why, Maim, he said--"

"It don't make no difference what he SAID--that ain't the thing.  The
thing is for you to treat him KIND, and not be saying things to make him
remember he ain't in his own country and amongst his own folks."

I says to myself, THIS is a girl that I'm letting that old reptle rob her
of her money!

Then Susan SHE waltzed in; and if you'll believe me, she did give
Hare-lip hark from the tomb!

Says I to myself, and this is ANOTHER one that I'm letting him rob her of
her money!

Then Mary Jane she took another inning, and went in sweet and lovely
again--which was her way; but when she got done there warn't hardly
anything left o' poor Hare-lip.  So she hollered.

"All right, then," says the other girls; "you just ask his pardon."

She done it, too; and she done it beautiful.  She done it so beautiful it
was good to hear; and I wished I could tell her a thousand lies, so she
could do it again.

I says to myself, this is ANOTHER one that I'm letting him rob her of her
money.  And when she got through they all jest laid theirselves out to
make me feel at home and know I was amongst friends.  I felt so ornery
and low down and mean that I says to myself, my mind's made up; I'll hive
that money for them or bust.

So then I lit out--for bed, I said, meaning some time or another.  When I
got by myself I went to thinking the thing over.  I says to myself, shall
I go to that doctor, private, and blow on these frauds?  No--that won't
do. He might tell who told him; then the king and the duke would make it
warm for me.  Shall I go, private, and tell Mary Jane?  No--I dasn't do
it. Her face would give them a hint, sure; they've got the money, and
they'd slide right out and get away with it.  If she was to fetch in help
I'd get mixed up in the business before it was done with, I judge.  No;
there ain't no good way but one.  I got to steal that money, somehow; and
I got to steal it some way that they won't suspicion that I done it.
They've got a good thing here, and they ain't a-going to leave till
they've played this family and this town for all they're worth, so I'll
find a chance time enough. I'll steal it and hide it; and by and by, when
I'm away down the river, I'll write a letter and tell Mary Jane where
it's hid.  But I better hive it tonight if I can, because the doctor
maybe hasn't let up as much as he lets on he has; he might scare them out
of here yet.

So, thinks I, I'll go and search them rooms.  Upstairs the hall was dark,
but I found the duke's room, and started to paw around it with my hands;
but I recollected it wouldn't be much like the king to let anybody else
take care of that money but his own self; so then I went to his room and
begun to paw around there.  But I see I couldn't do nothing without a
candle, and I dasn't light one, of course.  So I judged I'd got to do the
other thing--lay for them and eavesdrop.  About that time I hears their
footsteps coming, and was going to skip under the bed; I reached for it,
but it wasn't where I thought it would be; but I touched the curtain that
hid Mary Jane's frocks, so I jumped in behind that and snuggled in
amongst the gowns, and stood there perfectly still.

They come in and shut the door; and the first thing the duke done was to
get down and look under the bed.  Then I was glad I hadn't found the bed
when I wanted it.  And yet, you know, it's kind of natural to hide under
the bed when you are up to anything private.  They sets down then, and
the king says:

"Well, what is it?  And cut it middlin' short, because it's better for us
to be down there a-whoopin' up the mournin' than up here givin' 'em a
chance to talk us over."

"Well, this is it, Capet.  I ain't easy; I ain't comfortable.  That
doctor lays on my mind.  I wanted to know your plans.  I've got a notion,
and I think it's a sound one."

"What is it, duke?"

"That we better glide out of this before three in the morning, and clip
it down the river with what we've got.  Specially, seeing we got it so
easy--GIVEN back to us, flung at our heads, as you may say, when of
course we allowed to have to steal it back.  I'm for knocking off and
lighting out."

That made me feel pretty bad.  About an hour or two ago it would a been a
little different, but now it made me feel bad and disappointed, The king
rips out and says:

"What!  And not sell out the rest o' the property?  March off like a
passel of fools and leave eight or nine thous'n' dollars' worth o'
property layin' around jest sufferin' to be scooped in?--and all good,
salable stuff, too."

The duke he grumbled; said the bag of gold was enough, and he didn't want
to go no deeper--didn't want to rob a lot of orphans of EVERYTHING they
had.

"Why, how you talk!" says the king.  "We sha'n't rob 'em of nothing at
all but jest this money.  The people that BUYS the property is the
suff'rers; because as soon 's it's found out 'at we didn't own it--which
won't be long after we've slid--the sale won't be valid, and it 'll all
go back to the estate.  These yer orphans 'll git their house back agin,
and that's enough for THEM; they're young and spry, and k'n easy earn a
livin'.  THEY ain't a-goin to suffer.  Why, jest think--there's thous'n's
and thous'n's that ain't nigh so well off.  Bless you, THEY ain't got
noth'n' to complain of."

Well, the king he talked him blind; so at last he give in, and said all
right, but said he believed it was blamed foolishness to stay, and that
doctor hanging over them.  But the king says:

"Cuss the doctor!  What do we k'yer for HIM?  Hain't we got all the fools
in town on our side?  And ain't that a big enough majority in any town?"

So they got ready to go down stairs again.  The duke says:

"I don't think we put that money in a good place."

That cheered me up.  I'd begun to think I warn't going to get a hint of
no kind to help me.  The king says:

"Why?"

"Because Mary Jane 'll be in mourning from this out; and first you know
the nigger that does up the rooms will get an order to box these duds up
and put 'em away; and do you reckon a nigger can run across money and not
borrow some of it?"

"Your head's level agin, duke," says the king; and he comes a-fumbling
under the curtain two or three foot from where I was.  I stuck tight to
the wall and kept mighty still, though quivery; and I wondered what them
fellows would say to me if they catched me; and I tried to think what I'd
better do if they did catch me.  But the king he got the bag before I
could think more than about a half a thought, and he never suspicioned I
was around.  They took and shoved the bag through a rip in the straw tick
that was under the feather-bed, and crammed it in a foot or two amongst
the straw and said it was all right now, because a nigger only makes up
the feather-bed, and don't turn over the straw tick only about twice a
year, and so it warn't in no danger of getting stole now.

But I knowed better.  I had it out of there before they was half-way down
stairs.  I groped along up to my cubby, and hid it there till I could get
a chance to do better.  I judged I better hide it outside of the house
somewheres, because if they missed it they would give the house a good
ransacking:  I knowed that very well.  Then I turned in, with my clothes
all on; but I couldn't a gone to sleep if I'd a wanted to, I was in such
a sweat to get through with the business.  By and by I heard the king and
the duke come up; so I rolled off my pallet and laid with my chin at the
top of my ladder, and waited to see if anything was going to happen.  But
nothing did.

So I held on till all the late sounds had quit and the early ones hadn't
begun yet; and then I slipped down the ladder.




CHAPTER XXVII.

I CREPT to their doors and listened; they was snoring.  So I tiptoed
along, and got down stairs all right.  There warn't a sound anywheres.  I
peeped through a crack of the dining-room door, and see the men that was
watching the corpse all sound asleep on their chairs.  The door was open
into the parlor, where the corpse was laying, and there was a candle in
both rooms. I passed along, and the parlor door was open; but I see there
warn't nobody in there but the remainders of Peter; so I shoved on by;
but the front door was locked, and the key wasn't there.  Just then I
heard somebody coming down the stairs, back behind me.  I run in the
parlor and took a swift look around, and the only place I see to hide the
bag was in the coffin.  The lid was shoved along about a foot, showing
the dead man's face down in there, with a wet cloth over it, and his
shroud on.  I tucked the money-bag in under the lid, just down beyond
where his hands was crossed, which made me creep, they was so cold, and
then I run back across the room and in behind the door.

The person coming was Mary Jane.  She went to the coffin, very soft, and
kneeled down and looked in; then she put up her handkerchief, and I see
she begun to cry, though I couldn't hear her, and her back was to me.  I
slid out, and as I passed the dining-room I thought I'd make sure them
watchers hadn't seen me; so I looked through the crack, and everything
was all right.  They hadn't stirred.

I slipped up to bed, feeling ruther blue, on accounts of the thing
playing out that way after I had took so much trouble and run so much
resk about it.  Says I, if it could stay where it is, all right; because
when we get down the river a hundred mile or two I could write back to
Mary Jane, and she could dig him up again and get it; but that ain't the
thing that's going to happen; the thing that's going to happen is, the
money 'll be found when they come to screw on the lid.  Then the king 'll
get it again, and it 'll be a long day before he gives anybody another
chance to smouch it from him. Of course I WANTED to slide down and get it
out of there, but I dasn't try it.  Every minute it was getting earlier
now, and pretty soon some of them watchers would begin to stir, and I
might get catched--catched with six thousand dollars in my hands that
nobody hadn't hired me to take care of.  I don't wish to be mixed up in
no such business as that, I says to myself.

When I got down stairs in the morning the parlor was shut up, and the
watchers was gone.  There warn't nobody around but the family and the
widow Bartley and our tribe.  I watched their faces to see if anything
had been happening, but I couldn't tell.

Towards the middle of the day the undertaker come with his man, and they
set the coffin in the middle of the room on a couple of chairs, and then
set all our chairs in rows, and borrowed more from the neighbors till the
hall and the parlor and the dining-room was full.  I see the coffin lid
was the way it was before, but I dasn't go to look in under it, with
folks around.

Then the people begun to flock in, and the beats and the girls took seats
in the front row at the head of the coffin, and for a half an hour the
people filed around slow, in single rank, and looked down at the dead
man's face a minute, and some dropped in a tear, and it was all very
still and solemn, only the girls and the beats holding handkerchiefs to
their eyes and keeping their heads bent, and sobbing a little.  There
warn't no other sound but the scraping of the feet on the floor and
blowing noses--because people always blows them more at a funeral than
they do at other places except church.

When the place was packed full the undertaker he slid around in his black
gloves with his softy soothering ways, putting on the last touches, and
getting people and things all ship-shape and comfortable, and making no
more sound than a cat.  He never spoke; he moved people around, he
squeezed in late ones, he opened up passageways, and done it with nods,
and signs with his hands.  Then he took his place over against the wall.
He was the softest, glidingest, stealthiest man I ever see; and there
warn't no more smile to him than there is to a ham.

They had borrowed a melodeum--a sick one; and when everything was ready a
young woman set down and worked it, and it was pretty skreeky and
colicky, and everybody joined in and sung, and Peter was the only one
that had a good thing, according to my notion.  Then the Reverend Hobson
opened up, slow and solemn, and begun to talk; and straight off the most
outrageous row busted out in the cellar a body ever heard; it was only
one dog, but he made a most powerful racket, and he kept it up right
along; the parson he had to stand there, over the coffin, and wait--you
couldn't hear yourself think.  It was right down awkward, and nobody
didn't seem to know what to do.  But pretty soon they see that
long-legged undertaker make a sign to the preacher as much as to say,
"Don't you worry--just depend on me."  Then he stooped down and begun to
glide along the wall, just his shoulders showing over the people's heads.
So he glided along, and the powwow and racket getting more and more
outrageous all the time; and at last, when he had gone around two sides
of the room, he disappears down cellar.  Then in about two seconds we
heard a whack, and the dog he finished up with a most amazing howl or
two, and then everything was dead still, and the parson begun his solemn
talk where he left off.  In a minute or two here comes this undertaker's
back and shoulders gliding along the wall again; and so he glided and
glided around three sides of the room, and then rose up, and shaded his
mouth with his hands, and stretched his neck out towards the preacher,
over the people's heads, and says, in a kind of a coarse whisper, "HE HAD
A RAT!"  Then he drooped down and glided along the wall again to his
place.  You could see it was a great satisfaction to the people, because
naturally they wanted to know.  A little thing like that don't cost
nothing, and it's just the little things that makes a man to be looked up
to and liked.  There warn't no more popular man in town than what that
undertaker was.

Well, the funeral sermon was very good, but pison long and tiresome; and
then the king he shoved in and got off some of his usual rubbage, and at
last the job was through, and the undertaker begun to sneak up on the
coffin with his screw-driver.  I was in a sweat then, and watched him
pretty keen. But he never meddled at all; just slid the lid along as soft
as mush, and screwed it down tight and fast.  So there I was!  I didn't
know whether the money was in there or not.  So, says I, s'pose somebody
has hogged that bag on the sly?--now how do I know whether to write to
Mary Jane or not? S'pose she dug him up and didn't find nothing, what
would she think of me? Blame it, I says, I might get hunted up and
jailed; I'd better lay low and keep dark, and not write at all; the
thing's awful mixed now; trying to better it, I've worsened it a hundred
times, and I wish to goodness I'd just let it alone, dad fetch the whole
business!

They buried him, and we come back home, and I went to watching faces
again--I couldn't help it, and I couldn't rest easy.  But nothing come
of it; the faces didn't tell me nothing.

The king he visited around in the evening, and sweetened everybody up,
and made himself ever so friendly; and he give out the idea that his
congregation over in England would be in a sweat about him, so he must
hurry and settle up the estate right away and leave for home.  He was
very sorry he was so pushed, and so was everybody; they wished he could
stay longer, but they said they could see it couldn't be done.  And he
said of course him and William would take the girls home with them; and
that pleased everybody too, because then the girls would be well fixed
and amongst their own relations; and it pleased the girls, too--tickled
them so they clean forgot they ever had a trouble in the world; and told
him to sell out as quick as he wanted to, they would be ready.  Them poor
things was that glad and happy it made my heart ache to see them getting
fooled and lied to so, but I didn't see no safe way for me to chip in and
change the general tune.

Well, blamed if the king didn't bill the house and the niggers and all
the property for auction straight off--sale two days after the funeral;
but anybody could buy private beforehand if they wanted to.

So the next day after the funeral, along about noon-time, the girls' joy
got the first jolt.  A couple of nigger traders come along, and the king
sold them the niggers reasonable, for three-day drafts as they called it,
and away they went, the two sons up the river to Memphis, and their
mother down the river to Orleans.  I thought them poor girls and them
niggers would break their hearts for grief; they cried around each other,
and took on so it most made me down sick to see it.  The girls said they
hadn't ever dreamed of seeing the family separated or sold away from the
town.  I can't ever get it out of my memory, the sight of them poor
miserable girls and niggers hanging around each other's necks and crying;
and I reckon I couldn't a stood it all, but would a had to bust out and
tell on our gang if I hadn't knowed the sale warn't no account and the
niggers would be back home in a week or two.

The thing made a big stir in the town, too, and a good many come out
flatfooted and said it was scandalous to separate the mother and the
children that way.  It injured the frauds some; but the old fool he
bulled right along, spite of all the duke could say or do, and I tell you
the duke was powerful uneasy.

Next day was auction day.  About broad day in the morning the king and
the duke come up in the garret and woke me up, and I see by their look
that there was trouble.  The king says:

"Was you in my room night before last?"

"No, your majesty"--which was the way I always called him when nobody but
our gang warn't around.

"Was you in there yisterday er last night?"

"No, your majesty."

"Honor bright, now--no lies."

"Honor bright, your majesty, I'm telling you the truth.  I hain't been
a-near your room since Miss Mary Jane took you and the duke and showed it
to you."

The duke says:

"Have you seen anybody else go in there?"

"No, your grace, not as I remember, I believe."

"Stop and think."

I studied awhile and see my chance; then I says:

"Well, I see the niggers go in there several times."

Both of them gave a little jump, and looked like they hadn't ever
expected it, and then like they HAD.  Then the duke says:

"What, all of them?"

"No--leastways, not all at once--that is, I don't think I ever see them
all come OUT at once but just one time."

"Hello!  When was that?"

"It was the day we had the funeral.  In the morning.  It warn't early,
because I overslept.  I was just starting down the ladder, and I see
them."

"Well, go on, GO on!  What did they do?  How'd they act?"

"They didn't do nothing.  And they didn't act anyway much, as fur as I
see. They tiptoed away; so I seen, easy enough, that they'd shoved in
there to do up your majesty's room, or something, s'posing you was up;
and found you WARN'T up, and so they was hoping to slide out of the way
of trouble without waking you up, if they hadn't already waked you up."

"Great guns, THIS is a go!" says the king; and both of them looked pretty
sick and tolerable silly.  They stood there a-thinking and scratching
their heads a minute, and the duke he bust into a kind of a little raspy
chuckle, and says:

"It does beat all how neat the niggers played their hand.  They let on to
be SORRY they was going out of this region!  And I believed they WAS
sorry, and so did you, and so did everybody.  Don't ever tell ME any more
that a nigger ain't got any histrionic talent.  Why, the way they played
that thing it would fool ANYBODY.  In my opinion, there's a fortune in
'em.  If I had capital and a theater, I wouldn't want a better lay-out
than that--and here we've gone and sold 'em for a song.  Yes, and ain't
privileged to sing the song yet.  Say, where IS that song--that draft?"

"In the bank for to be collected.  Where WOULD it be?"

"Well, THAT'S all right then, thank goodness."

Says I, kind of timid-like:

"Is something gone wrong?"

The king whirls on me and rips out:

"None o' your business!  You keep your head shet, and mind y'r own
affairs--if you got any.  Long as you're in this town don't you forgit
THAT--you hear?"  Then he says to the duke, "We got to jest swaller it
and say noth'n':  mum's the word for US."

As they was starting down the ladder the duke he chuckles again, and
says:

"Quick sales AND small profits!  It's a good business--yes."

The king snarls around on him and says:

"I was trying to do for the best in sellin' 'em out so quick.  If the
profits has turned out to be none, lackin' considable, and none to carry,
is it my fault any more'n it's yourn?"

"Well, THEY'D be in this house yet and we WOULDN'T if I could a got my
advice listened to."

The king sassed back as much as was safe for him, and then swapped around
and lit into ME again.  He give me down the banks for not coming and
TELLING him I see the niggers come out of his room acting that way--said
any fool would a KNOWED something was up.  And then waltzed in and cussed
HIMSELF awhile, and said it all come of him not laying late and taking
his natural rest that morning, and he'd be blamed if he'd ever do it
again.  So they went off a-jawing; and I felt dreadful glad I'd worked it
all off on to the niggers, and yet hadn't done the niggers no harm by it.




CHAPTER XXVIII.

BY and by it was getting-up time.  So I come down the ladder and started
for down-stairs; but as I come to the girls' room the door was open, and
I see Mary Jane setting by her old hair trunk, which was open and she'd
been packing things in it--getting ready to go to England.  But she had
stopped now with a folded gown in her lap, and had her face in her hands,
crying.  I felt awful bad to see it; of course anybody would.  I went in
there and says:

"Miss Mary Jane, you can't a-bear to see people in trouble, and I can't
--most always.  Tell me about it."

So she done it.  And it was the niggers--I just expected it.  She said
the beautiful trip to England was most about spoiled for her; she didn't
know HOW she was ever going to be happy there, knowing the mother and the
children warn't ever going to see each other no more--and then busted out
bitterer than ever, and flung up her hands, and says:

"Oh, dear, dear, to think they ain't EVER going to see each other any
more!"

"But they WILL--and inside of two weeks--and I KNOW it!" says I.

Laws, it was out before I could think!  And before I could budge she
throws her arms around my neck and told me to say it AGAIN, say it AGAIN,
say it AGAIN!

I see I had spoke too sudden and said too much, and was in a close place.
I asked her to let me think a minute; and she set there, very impatient
and excited and handsome, but looking kind of happy and eased-up, like a
person that's had a tooth pulled out.  So I went to studying it out.  I
says to myself, I reckon a body that ups and tells the truth when he is
in a tight place is taking considerable many resks, though I ain't had no
experience, and can't say for certain; but it looks so to me, anyway; and
yet here's a case where I'm blest if it don't look to me like the truth
is better and actuly SAFER than a lie.  I must lay it by in my mind, and
think it over some time or other, it's so kind of strange and unregular.
I never see nothing like it.  Well, I says to myself at last, I'm a-going
to chance it; I'll up and tell the truth this time, though it does seem
most like setting down on a kag of powder and touching it off just to see
where you'll go to. Then I says:

"Miss Mary Jane, is there any place out of town a little ways where you
could go and stay three or four days?"

"Yes; Mr. Lothrop's.  Why?"

"Never mind why yet.  If I'll tell you how I know the niggers will see
each other again inside of two weeks--here in this house--and PROVE how I
know it--will you go to Mr. Lothrop's and stay four days?"

"Four days!" she says; "I'll stay a year!"

"All right," I says, "I don't want nothing more out of YOU than just your
word--I druther have it than another man's kiss-the-Bible."  She smiled
and reddened up very sweet, and I says, "If you don't mind it, I'll shut
the door--and bolt it."

Then I come back and set down again, and says:

"Don't you holler.  Just set still and take it like a man.  I got to tell
the truth, and you want to brace up, Miss Mary, because it's a bad kind,
and going to be hard to take, but there ain't no help for it.  These
uncles of yourn ain't no uncles at all; they're a couple of frauds
--regular dead-beats.  There, now we're over the worst of it, you can stand
the rest middling easy."

It jolted her up like everything, of course; but I was over the shoal
water now, so I went right along, her eyes a-blazing higher and higher
all the time, and told her every blame thing, from where we first struck
that young fool going up to the steamboat, clear through to where she
flung herself on to the king's breast at the front door and he kissed her
sixteen or seventeen times--and then up she jumps, with her face afire
like sunset, and says:

"The brute!  Come, don't waste a minute--not a SECOND--we'll have them
tarred and feathered, and flung in the river!"

Says I:

"Cert'nly.  But do you mean BEFORE you go to Mr. Lothrop's, or--"

"Oh," she says, "what am I THINKING about!" she says, and set right down
again.  "Don't mind what I said--please don't--you WON'T, now, WILL you?"
Laying her silky hand on mine in that kind of a way that I said I would
die first.  "I never thought, I was so stirred up," she says; "now go on,
and I won't do so any more.  You tell me what to do, and whatever you say
I'll do it."

"Well," I says, "it's a rough gang, them two frauds, and I'm fixed so I
got to travel with them a while longer, whether I want to or not--I
druther not tell you why; and if you was to blow on them this town would
get me out of their claws, and I'd be all right; but there'd be another
person that you don't know about who'd be in big trouble.  Well, we got
to save HIM, hain't we?  Of course.  Well, then, we won't blow on them."

Saying them words put a good idea in my head.  I see how maybe I could
get me and Jim rid of the frauds; get them jailed here, and then leave.
But I didn't want to run the raft in the daytime without anybody aboard
to answer questions but me; so I didn't want the plan to begin working
till pretty late to-night.  I says:

"Miss Mary Jane, I'll tell you what we'll do, and you won't have to stay
at Mr. Lothrop's so long, nuther.  How fur is it?"

"A little short of four miles--right out in the country, back here."

"Well, that 'll answer.  Now you go along out there, and lay low till
nine or half-past to-night, and then get them to fetch you home again
--tell them you've thought of something.  If you get here before eleven put
a candle in this window, and if I don't turn up wait TILL eleven, and
THEN if I don't turn up it means I'm gone, and out of the way, and safe.
Then you come out and spread the news around, and get these beats
jailed."

"Good," she says, "I'll do it."

"And if it just happens so that I don't get away, but get took up along
with them, you must up and say I told you the whole thing beforehand, and
you must stand by me all you can."

"Stand by you! indeed I will.  They sha'n't touch a hair of your head!"
she says, and I see her nostrils spread and her eyes snap when she said
it, too.

"If I get away I sha'n't be here," I says, "to prove these rapscallions
ain't your uncles, and I couldn't do it if I WAS here.  I could swear
they was beats and bummers, that's all, though that's worth something.
Well, there's others can do that better than what I can, and they're
people that ain't going to be doubted as quick as I'd be.  I'll tell you
how to find them.  Gimme a pencil and a piece of paper.  There--'Royal
Nonesuch, Bricksville.'  Put it away, and don't lose it.  When the court
wants to find out something about these two, let them send up to
Bricksville and say they've got the men that played the Royal Nonesuch,
and ask for some witnesses--why, you'll have that entire town down here
before you can hardly wink, Miss Mary.  And they'll come a-biling, too."

I judged we had got everything fixed about right now.  So I says:

"Just let the auction go right along, and don't worry.  Nobody don't have
to pay for the things they buy till a whole day after the auction on
accounts of the short notice, and they ain't going out of this till they
get that money; and the way we've fixed it the sale ain't going to count,
and they ain't going to get no money.  It's just like the way it was with
the niggers--it warn't no sale, and the niggers will be back before
long.  Why, they can't collect the money for the NIGGERS yet--they're in
the worst kind of a fix, Miss Mary."

"Well," she says, "I'll run down to breakfast now, and then I'll start
straight for Mr. Lothrop's."

"'Deed, THAT ain't the ticket, Miss Mary Jane," I says, "by no manner of
means; go BEFORE breakfast."

"Why?"

"What did you reckon I wanted you to go at all for, Miss Mary?"

"Well, I never thought--and come to think, I don't know.  What was it?"

"Why, it's because you ain't one of these leather-face people.  I don't
want no better book than what your face is.  A body can set down and read
it off like coarse print.  Do you reckon you can go and face your uncles
when they come to kiss you good-morning, and never--"

"There, there, don't!  Yes, I'll go before breakfast--I'll be glad to.
And leave my sisters with them?"

"Yes; never mind about them.  They've got to stand it yet a while.  They
might suspicion something if all of you was to go.  I don't want you to
see them, nor your sisters, nor nobody in this town; if a neighbor was to
ask how is your uncles this morning your face would tell something.  No,
you go right along, Miss Mary Jane, and I'll fix it with all of them.
I'll tell Miss Susan to give your love to your uncles and say you've went
away for a few hours for to get a little rest and change, or to see a
friend, and you'll be back to-night or early in the morning."

"Gone to see a friend is all right, but I won't have my love given to
them."

"Well, then, it sha'n't be."  It was well enough to tell HER so--no harm
in it.  It was only a little thing to do, and no trouble; and it's the
little things that smooths people's roads the most, down here below; it
would make Mary Jane comfortable, and it wouldn't cost nothing.  Then I
says:  "There's one more thing--that bag of money."

"Well, they've got that; and it makes me feel pretty silly to think HOW
they got it."

"No, you're out, there.  They hain't got it."

"Why, who's got it?"

"I wish I knowed, but I don't.  I HAD it, because I stole it from them;
and I stole it to give to you; and I know where I hid it, but I'm afraid
it ain't there no more.  I'm awful sorry, Miss Mary Jane, I'm just as
sorry as I can be; but I done the best I could; I did honest.  I come
nigh getting caught, and I had to shove it into the first place I come
to, and run--and it warn't a good place."

"Oh, stop blaming yourself--it's too bad to do it, and I won't allow it
--you couldn't help it; it wasn't your fault.  Where did you hide it?"

I didn't want to set her to thinking about her troubles again; and I
couldn't seem to get my mouth to tell her what would make her see that
corpse laying in the coffin with that bag of money on his stomach.  So
for a minute I didn't say nothing; then I says:

"I'd ruther not TELL you where I put it, Miss Mary Jane, if you don't
mind letting me off; but I'll write it for you on a piece of paper, and
you can read it along the road to Mr. Lothrop's, if you want to.  Do you
reckon that 'll do?"

"Oh, yes."

So I wrote:  "I put it in the coffin.  It was in there when you was
crying there, away in the night.  I was behind the door, and I was mighty
sorry for you, Miss Mary Jane."

It made my eyes water a little to remember her crying there all by
herself in the night, and them devils laying there right under her own
roof, shaming her and robbing her; and when I folded it up and give it to
her I see the water come into her eyes, too; and she shook me by the
hand, hard, and says:

"GOOD-bye.  I'm going to do everything just as you've told me; and if I
don't ever see you again, I sha'n't ever forget you and I'll think of
you a many and a many a time, and I'll PRAY for you, too!"--and she was
gone.

Pray for me!  I reckoned if she knowed me she'd take a job that was more
nearer her size.  But I bet she done it, just the same--she was just that
kind.  She had the grit to pray for Judus if she took the notion--there
warn't no back-down to her, I judge.  You may say what you want to, but
in my opinion she had more sand in her than any girl I ever see; in my
opinion she was just full of sand.  It sounds like flattery, but it ain't
no flattery.  And when it comes to beauty--and goodness, too--she lays
over them all.  I hain't ever seen her since that time that I see her go
out of that door; no, I hain't ever seen her since, but I reckon I've
thought of her a many and a many a million times, and of her saying she
would pray for me; and if ever I'd a thought it would do any good for me
to pray for HER, blamed if I wouldn't a done it or bust.

Well, Mary Jane she lit out the back way, I reckon; because nobody see
her go.  When I struck Susan and the hare-lip, I says:

"What's the name of them people over on t'other side of the river that
you all goes to see sometimes?"

They says:

"There's several; but it's the Proctors, mainly."

"That's the name," I says; "I most forgot it.  Well, Miss Mary Jane she
told me to tell you she's gone over there in a dreadful hurry--one of
them's sick."

"Which one?"

"I don't know; leastways, I kinder forget; but I thinks it's--"

"Sakes alive, I hope it ain't HANNER?"

"I'm sorry to say it," I says, "but Hanner's the very one."

"My goodness, and she so well only last week!  Is she took bad?"

"It ain't no name for it.  They set up with her all night, Miss Mary Jane
said, and they don't think she'll last many hours."

"Only think of that, now!  What's the matter with her?"

I couldn't think of anything reasonable, right off that way, so I says:

"Mumps."

"Mumps your granny!  They don't set up with people that's got the mumps."

"They don't, don't they?  You better bet they do with THESE mumps.  These
mumps is different.  It's a new kind, Miss Mary Jane said."

"How's it a new kind?"

"Because it's mixed up with other things."

"What other things?"

"Well, measles, and whooping-cough, and erysiplas, and consumption, and
yaller janders, and brain-fever, and I don't know what all."

"My land!  And they call it the MUMPS?"

"That's what Miss Mary Jane said."

"Well, what in the nation do they call it the MUMPS for?"

"Why, because it IS the mumps.  That's what it starts with."

"Well, ther' ain't no sense in it.  A body might stump his toe, and take
pison, and fall down the well, and break his neck, and bust his brains
out, and somebody come along and ask what killed him, and some numskull
up and say, 'Why, he stumped his TOE.'  Would ther' be any sense in that?
NO.  And ther' ain't no sense in THIS, nuther.  Is it ketching?"

"Is it KETCHING?  Why, how you talk.  Is a HARROW catching--in the dark?
If you don't hitch on to one tooth, you're bound to on another, ain't
you? And you can't get away with that tooth without fetching the whole
harrow along, can you?  Well, these kind of mumps is a kind of a harrow,
as you may say--and it ain't no slouch of a harrow, nuther, you come to
get it hitched on good."

"Well, it's awful, I think," says the hare-lip.  "I'll go to Uncle Harvey
and--"

"Oh, yes," I says, "I WOULD.  Of COURSE I would.  I wouldn't lose no
time."

"Well, why wouldn't you?"

"Just look at it a minute, and maybe you can see.  Hain't your uncles
obleegd to get along home to England as fast as they can?  And do you
reckon they'd be mean enough to go off and leave you to go all that
journey by yourselves?  YOU know they'll wait for you.  So fur, so good.
Your uncle Harvey's a preacher, ain't he?  Very well, then; is a PREACHER
going to deceive a steamboat clerk? is he going to deceive a SHIP CLERK?
--so as to get them to let Miss Mary Jane go aboard?  Now YOU know he
ain't.  What WILL he do, then?  Why, he'll say, 'It's a great pity, but
my church matters has got to get along the best way they can; for my
niece has been exposed to the dreadful pluribus-unum mumps, and so it's
my bounden duty to set down here and wait the three months it takes to
show on her if she's got it.'  But never mind, if you think it's best to
tell your uncle Harvey--"

"Shucks, and stay fooling around here when we could all be having good
times in England whilst we was waiting to find out whether Mary Jane's
got it or not?  Why, you talk like a muggins."

"Well, anyway, maybe you'd better tell some of the neighbors."

"Listen at that, now.  You do beat all for natural stupidness.  Can't you
SEE that THEY'D go and tell?  Ther' ain't no way but just to not tell
anybody at ALL."

"Well, maybe you're right--yes, I judge you ARE right."

"But I reckon we ought to tell Uncle Harvey she's gone out a while,
anyway, so he won't be uneasy about her?"

"Yes, Miss Mary Jane she wanted you to do that.  She says, 'Tell them to
give Uncle Harvey and William my love and a kiss, and say I've run over
the river to see Mr.'--Mr.--what IS the name of that rich family your
uncle Peter used to think so much of?--I mean the one that--"

"Why, you must mean the Apthorps, ain't it?"

"Of course; bother them kind of names, a body can't ever seem to remember
them, half the time, somehow.  Yes, she said, say she has run over for to
ask the Apthorps to be sure and come to the auction and buy this house,
because she allowed her uncle Peter would ruther they had it than anybody
else; and she's going to stick to them till they say they'll come, and
then, if she ain't too tired, she's coming home; and if she is, she'll be
home in the morning anyway.  She said, don't say nothing about the
Proctors, but only about the Apthorps--which 'll be perfectly true,
because she is going there to speak about their buying the house; I know
it, because she told me so herself."

"All right," they said, and cleared out to lay for their uncles, and give
them the love and the kisses, and tell them the message.

Everything was all right now.  The girls wouldn't say nothing because
they wanted to go to England; and the king and the duke would ruther Mary
Jane was off working for the auction than around in reach of Doctor
Robinson.  I felt very good; I judged I had done it pretty neat--I
reckoned Tom Sawyer couldn't a done it no neater himself.  Of course he
would a throwed more style into it, but I can't do that very handy, not
being brung up to it.

Well, they held the auction in the public square, along towards the end
of the afternoon, and it strung along, and strung along, and the old man
he was on hand and looking his level pisonest, up there longside of the
auctioneer, and chipping in a little Scripture now and then, or a little
goody-goody saying of some kind, and the duke he was around goo-gooing
for sympathy all he knowed how, and just spreading himself generly.

But by and by the thing dragged through, and everything was sold
--everything but a little old trifling lot in the graveyard.  So they'd got
to work that off--I never see such a girafft as the king was for wanting
to swallow EVERYTHING.  Well, whilst they was at it a steamboat landed,
and in about two minutes up comes a crowd a-whooping and yelling and
laughing and carrying on, and singing out:

"HERE'S your opposition line! here's your two sets o' heirs to old Peter
Wilks--and you pays your money and you takes your choice!"




CHAPTER XXIX.


THEY was fetching a very nice-looking old gentleman along, and a
nice-looking younger one, with his right arm in a sling.  And, my souls,
how the people yelled and laughed, and kept it up.  But I didn't see no
joke about it, and I judged it would strain the duke and the king some to
see any.  I reckoned they'd turn pale.  But no, nary a pale did THEY
turn. The duke he never let on he suspicioned what was up, but just went
a goo-gooing around, happy and satisfied, like a jug that's googling out
buttermilk; and as for the king, he just gazed and gazed down sorrowful
on them new-comers like it give him the stomach-ache in his very heart to
think there could be such frauds and rascals in the world.  Oh, he done
it admirable.  Lots of the principal people gethered around the king, to
let him see they was on his side.  That old gentleman that had just come
looked all puzzled to death.  Pretty soon he begun to speak, and I see
straight off he pronounced LIKE an Englishman--not the king's way, though
the king's WAS pretty good for an imitation.  I can't give the old gent's
words, nor I can't imitate him; but he turned around to the crowd, and
says, about like this:

"This is a surprise to me which I wasn't looking for; and I'll
acknowledge, candid and frank, I ain't very well fixed to meet it and
answer it; for my brother and me has had misfortunes; he's broke his arm,
and our baggage got put off at a town above here last night in the night
by a mistake.  I am Peter Wilks' brother Harvey, and this is his brother
William, which can't hear nor speak--and can't even make signs to amount
to much, now't he's only got one hand to work them with.  We are who we
say we are; and in a day or two, when I get the baggage, I can prove it.
But up till then I won't say nothing more, but go to the hotel and wait."

So him and the new dummy started off; and the king he laughs, and
blethers out:

"Broke his arm--VERY likely, AIN'T it?--and very convenient, too, for a
fraud that's got to make signs, and ain't learnt how.  Lost their
baggage! That's MIGHTY good!--and mighty ingenious--under the
CIRCUMSTANCES!"

So he laughed again; and so did everybody else, except three or four, or
maybe half a dozen.  One of these was that doctor; another one was a
sharp-looking gentleman, with a carpet-bag of the old-fashioned kind made
out of carpet-stuff, that had just come off of the steamboat and was
talking to him in a low voice, and glancing towards the king now and then
and nodding their heads--it was Levi Bell, the lawyer that was gone up to
Louisville; and another one was a big rough husky that come along and
listened to all the old gentleman said, and was listening to the king
now. And when the king got done this husky up and says:

"Say, looky here; if you are Harvey Wilks, when'd you come to this town?"

"The day before the funeral, friend," says the king.

"But what time o' day?"

"In the evenin'--'bout an hour er two before sundown."

"HOW'D you come?"

"I come down on the Susan Powell from Cincinnati."

"Well, then, how'd you come to be up at the Pint in the MORNIN'--in a
canoe?"

"I warn't up at the Pint in the mornin'."

"It's a lie."

Several of them jumped for him and begged him not to talk that way to an
old man and a preacher.

"Preacher be hanged, he's a fraud and a liar.  He was up at the Pint that
mornin'.  I live up there, don't I?  Well, I was up there, and he was up
there.  I see him there.  He come in a canoe, along with Tim Collins and
a boy."

The doctor he up and says:

"Would you know the boy again if you was to see him, Hines?"

"I reckon I would, but I don't know.  Why, yonder he is, now.  I know him
perfectly easy."

It was me he pointed at.  The doctor says:

"Neighbors, I don't know whether the new couple is frauds or not; but if
THESE two ain't frauds, I am an idiot, that's all.  I think it's our duty
to see that they don't get away from here till we've looked into this
thing. Come along, Hines; come along, the rest of you.  We'll take these
fellows to the tavern and affront them with t'other couple, and I reckon
we'll find out SOMETHING before we get through."

It was nuts for the crowd, though maybe not for the king's friends; so we
all started.  It was about sundown.  The doctor he led me along by the
hand, and was plenty kind enough, but he never let go my hand.

We all got in a big room in the hotel, and lit up some candles, and
fetched in the new couple.  First, the doctor says:

"I don't wish to be too hard on these two men, but I think they're
frauds, and they may have complices that we don't know nothing about.  If
they have, won't the complices get away with that bag of gold Peter Wilks
left?  It ain't unlikely.  If these men ain't frauds, they won't object
to sending for that money and letting us keep it till they prove they're
all right--ain't that so?"

Everybody agreed to that.  So I judged they had our gang in a pretty
tight place right at the outstart.  But the king he only looked
sorrowful, and says:

"Gentlemen, I wish the money was there, for I ain't got no disposition to
throw anything in the way of a fair, open, out-and-out investigation o'
this misable business; but, alas, the money ain't there; you k'n send and
see, if you want to."

"Where is it, then?"

"Well, when my niece give it to me to keep for her I took and hid it
inside o' the straw tick o' my bed, not wishin' to bank it for the few
days we'd be here, and considerin' the bed a safe place, we not bein'
used to niggers, and suppos'n' 'em honest, like servants in England.  The
niggers stole it the very next mornin' after I had went down stairs; and
when I sold 'em I hadn't missed the money yit, so they got clean away
with it.  My servant here k'n tell you 'bout it, gentlemen."

The doctor and several said "Shucks!" and I see nobody didn't altogether
believe him.  One man asked me if I see the niggers steal it.  I said no,
but I see them sneaking out of the room and hustling away, and I never
thought nothing, only I reckoned they was afraid they had waked up my
master and was trying to get away before he made trouble with them.  That
was all they asked me.  Then the doctor whirls on me and says:

"Are YOU English, too?"

I says yes; and him and some others laughed, and said, "Stuff!"

Well, then they sailed in on the general investigation, and there we had
it, up and down, hour in, hour out, and nobody never said a word about
supper, nor ever seemed to think about it--and so they kept it up, and
kept it up; and it WAS the worst mixed-up thing you ever see.  They made
the king tell his yarn, and they made the old gentleman tell his'n; and
anybody but a lot of prejudiced chuckleheads would a SEEN that the old
gentleman was spinning truth and t'other one lies.  And by and by they
had me up to tell what I knowed.  The king he give me a left-handed look
out of the corner of his eye, and so I knowed enough to talk on the right
side.  I begun to tell about Sheffield, and how we lived there, and all
about the English Wilkses, and so on; but I didn't get pretty fur till
the doctor begun to laugh; and Levi Bell, the lawyer, says:

"Set down, my boy; I wouldn't strain myself if I was you.  I reckon you
ain't used to lying, it don't seem to come handy; what you want is
practice.  You do it pretty awkward."

I didn't care nothing for the compliment, but I was glad to be let off,
anyway.

The doctor he started to say something, and turns and says:

"If you'd been in town at first, Levi Bell--" The king broke in and
reached out his hand, and says:

"Why, is this my poor dead brother's old friend that he's wrote so often
about?"

The lawyer and him shook hands, and the lawyer smiled and looked pleased,
and they talked right along awhile, and then got to one side and talked
low; and at last the lawyer speaks up and says:

"That 'll fix it.  I'll take the order and send it, along with your
brother's, and then they'll know it's all right."

So they got some paper and a pen, and the king he set down and twisted
his head to one side, and chawed his tongue, and scrawled off something;
and then they give the pen to the duke--and then for the first time the
duke looked sick.  But he took the pen and wrote.  So then the lawyer
turns to the new old gentleman and says:

"You and your brother please write a line or two and sign your names."

The old gentleman wrote, but nobody couldn't read it.  The lawyer looked
powerful astonished, and says:

"Well, it beats ME"--and snaked a lot of old letters out of his pocket,
and examined them, and then examined the old man's writing, and then THEM
again; and then says:  "These old letters is from Harvey Wilks; and
here's THESE two handwritings, and anybody can see they didn't write
them" (the king and the duke looked sold and foolish, I tell you, to see
how the lawyer had took them in), "and here's THIS old gentleman's hand
writing, and anybody can tell, easy enough, HE didn't write them--fact
is, the scratches he makes ain't properly WRITING at all.  Now, here's
some letters from--"

The new old gentleman says:

"If you please, let me explain.  Nobody can read my hand but my brother
there--so he copies for me.  It's HIS hand you've got there, not mine."

"WELL!" says the lawyer, "this IS a state of things.  I've got some of
William's letters, too; so if you'll get him to write a line or so we can
com--"

"He CAN'T write with his left hand," says the old gentleman.  "If he
could use his right hand, you would see that he wrote his own letters and
mine too.  Look at both, please--they're by the same hand."

The lawyer done it, and says:

"I believe it's so--and if it ain't so, there's a heap stronger
resemblance than I'd noticed before, anyway.  Well, well, well!  I
thought we was right on the track of a solution, but it's gone to grass,
partly.  But anyway, one thing is proved--THESE two ain't either of 'em
Wilkses"--and he wagged his head towards the king and the duke.

Well, what do you think?  That muleheaded old fool wouldn't give in THEN!
Indeed he wouldn't.  Said it warn't no fair test.  Said his brother
William was the cussedest joker in the world, and hadn't tried to write
--HE see William was going to play one of his jokes the minute he put the
pen to paper.  And so he warmed up and went warbling and warbling right
along till he was actuly beginning to believe what he was saying
HIMSELF; but pretty soon the new gentleman broke in, and says:

"I've thought of something.  Is there anybody here that helped to lay out
my br--helped to lay out the late Peter Wilks for burying?"

"Yes," says somebody, "me and Ab Turner done it.  We're both here."

Then the old man turns towards the king, and says:

"Perhaps this gentleman can tell me what was tattooed on his breast?"

Blamed if the king didn't have to brace up mighty quick, or he'd a
squshed down like a bluff bank that the river has cut under, it took him
so sudden; and, mind you, it was a thing that was calculated to make most
ANYBODY sqush to get fetched such a solid one as that without any notice,
because how was HE going to know what was tattooed on the man?  He
whitened a little; he couldn't help it; and it was mighty still in there,
and everybody bending a little forwards and gazing at him.  Says I to
myself, NOW he'll throw up the sponge--there ain't no more use.  Well,
did he?  A body can't hardly believe it, but he didn't.  I reckon he
thought he'd keep the thing up till he tired them people out, so they'd
thin out, and him and the duke could break loose and get away.  Anyway,
he set there, and pretty soon he begun to smile, and says:

"Mf!  It's a VERY tough question, AIN'T it!  YES, sir, I k'n tell you
what's tattooed on his breast.  It's jest a small, thin, blue arrow
--that's what it is; and if you don't look clost, you can't see it.  NOW
what do you say--hey?"

Well, I never see anything like that old blister for clean out-and-out
cheek.

The new old gentleman turns brisk towards Ab Turner and his pard, and his
eye lights up like he judged he'd got the king THIS time, and says:

"There--you've heard what he said!  Was there any such mark on Peter
Wilks' breast?"

Both of them spoke up and says:

"We didn't see no such mark."

"Good!" says the old gentleman.  "Now, what you DID see on his breast was
a small dim P, and a B (which is an initial he dropped when he was
young), and a W, with dashes between them, so:  P--B--W"--and he marked
them that way on a piece of paper.  "Come, ain't that what you saw?"

Both of them spoke up again, and says:

"No, we DIDN'T.  We never seen any marks at all."

Well, everybody WAS in a state of mind now, and they sings out:

"The whole BILIN' of 'm 's frauds!  Le's duck 'em! le's drown 'em! le's
ride 'em on a rail!" and everybody was whooping at once, and there was a
rattling powwow.  But the lawyer he jumps on the table and yells, and
says:

"Gentlemen--gentleMEN!  Hear me just a word--just a SINGLE word--if you
PLEASE!  There's one way yet--let's go and dig up the corpse and look."

That took them.

"Hooray!" they all shouted, and was starting right off; but the lawyer
and the doctor sung out:

"Hold on, hold on!  Collar all these four men and the boy, and fetch THEM
along, too!"

"We'll do it!" they all shouted; "and if we don't find them marks we'll
lynch the whole gang!"

I WAS scared, now, I tell you.  But there warn't no getting away, you
know. They gripped us all, and marched us right along, straight for the
graveyard, which was a mile and a half down the river, and the whole town
at our heels, for we made noise enough, and it was only nine in the
evening.

As we went by our house I wished I hadn't sent Mary Jane out of town;
because now if I could tip her the wink she'd light out and save me, and
blow on our dead-beats.

Well, we swarmed along down the river road, just carrying on like
wildcats; and to make it more scary the sky was darking up, and the
lightning beginning to wink and flitter, and the wind to shiver amongst
the leaves. This was the most awful trouble and most dangersome I ever
was in; and I was kinder stunned; everything was going so different from
what I had allowed for; stead of being fixed so I could take my own time
if I wanted to, and see all the fun, and have Mary Jane at my back to
save me and set me free when the close-fit come, here was nothing in the
world betwixt me and sudden death but just them tattoo-marks.  If they
didn't find them--

I couldn't bear to think about it; and yet, somehow, I couldn't think
about nothing else.  It got darker and darker, and it was a beautiful
time to give the crowd the slip; but that big husky had me by the wrist
--Hines--and a body might as well try to give Goliar the slip.  He dragged
me right along, he was so excited, and I had to run to keep up.

When they got there they swarmed into the graveyard and washed over it
like an overflow.  And when they got to the grave they found they had
about a hundred times as many shovels as they wanted, but nobody hadn't
thought to fetch a lantern.  But they sailed into digging anyway by the
flicker of the lightning, and sent a man to the nearest house, a half a
mile off, to borrow one.

So they dug and dug like everything; and it got awful dark, and the rain
started, and the wind swished and swushed along, and the lightning come
brisker and brisker, and the thunder boomed; but them people never took
no notice of it, they was so full of this business; and one minute you
could see everything and every face in that big crowd, and the shovelfuls
of dirt sailing up out of the grave, and the next second the dark wiped
it all out, and you couldn't see nothing at all.

At last they got out the coffin and begun to unscrew the lid, and then
such another crowding and shouldering and shoving as there was, to
scrouge in and get a sight, you never see; and in the dark, that way, it
was awful.  Hines he hurt my wrist dreadful pulling and tugging so, and I
reckon he clean forgot I was in the world, he was so excited and panting.

All of a sudden the lightning let go a perfect sluice of white glare, and
somebody sings out:

"By the living jingo, here's the bag of gold on his breast!"

Hines let out a whoop, like everybody else, and dropped my wrist and give
a big surge to bust his way in and get a look, and the way I lit out and
shinned for the road in the dark there ain't nobody can tell.

I had the road all to myself, and I fairly flew--leastways, I had it all
to myself except the solid dark, and the now-and-then glares, and the
buzzing of the rain, and the thrashing of the wind, and the splitting of
the thunder; and sure as you are born I did clip it along!

When I struck the town I see there warn't nobody out in the storm, so I
never hunted for no back streets, but humped it straight through the main
one; and when I begun to get towards our house I aimed my eye and set it.
No light there; the house all dark--which made me feel sorry and
disappointed, I didn't know why.  But at last, just as I was sailing by,
FLASH comes the light in Mary Jane's window! and my heart swelled up
sudden, like to bust; and the same second the house and all was behind me
in the dark, and wasn't ever going to be before me no more in this world.
She WAS the best girl I ever see, and had the most sand.

The minute I was far enough above the town to see I could make the
towhead, I begun to look sharp for a boat to borrow, and the first time
the lightning showed me one that wasn't chained I snatched it and shoved.
It was a canoe, and warn't fastened with nothing but a rope.  The towhead
was a rattling big distance off, away out there in the middle of the
river, but I didn't lose no time; and when I struck the raft at last I
was so fagged I would a just laid down to blow and gasp if I could
afforded it.  But I didn't.  As I sprung aboard I sung out:

"Out with you, Jim, and set her loose!  Glory be to goodness, we're shut
of them!"

Jim lit out, and was a-coming for me with both arms spread, he was so
full of joy; but when I glimpsed him in the lightning my heart shot up in
my mouth and I went overboard backwards; for I forgot he was old King
Lear and a drownded A-rab all in one, and it most scared the livers and
lights out of me.  But Jim fished me out, and was going to hug me and
bless me, and so on, he was so glad I was back and we was shut of the
king and the duke, but I says:

"Not now; have it for breakfast, have it for breakfast!  Cut loose and
let her slide!"

So in two seconds away we went a-sliding down the river, and it DID seem
so good to be free again and all by ourselves on the big river, and
nobody to bother us.  I had to skip around a bit, and jump up and crack
my heels a few times--I couldn't help it; but about the third crack I
noticed a sound that I knowed mighty well, and held my breath and
listened and waited; and sure enough, when the next flash busted out over
the water, here they come!--and just a-laying to their oars and making
their skiff hum!  It was the king and the duke.

So I wilted right down on to the planks then, and give up; and it was all
I could do to keep from crying.




CHAPTER XXX.

WHEN they got aboard the king went for me, and shook me by the collar,
and says:

"Tryin' to give us the slip, was ye, you pup!  Tired of our company,
hey?"

I says:

"No, your majesty, we warn't--PLEASE don't, your majesty!"

"Quick, then, and tell us what WAS your idea, or I'll shake the insides
out o' you!"

"Honest, I'll tell you everything just as it happened, your majesty.  The
man that had a-holt of me was very good to me, and kept saying he had a
boy about as big as me that died last year, and he was sorry to see a boy
in such a dangerous fix; and when they was all took by surprise by
finding the gold, and made a rush for the coffin, he lets go of me and
whispers, 'Heel it now, or they'll hang ye, sure!' and I lit out.  It
didn't seem no good for ME to stay--I couldn't do nothing, and I didn't
want to be hung if I could get away.  So I never stopped running till I
found the canoe; and when I got here I told Jim to hurry, or they'd catch
me and hang me yet, and said I was afeard you and the duke wasn't alive
now, and I was awful sorry, and so was Jim, and was awful glad when we
see you coming; you may ask Jim if I didn't."

Jim said it was so; and the king told him to shut up, and said, "Oh, yes,
it's MIGHTY likely!" and shook me up again, and said he reckoned he'd
drownd me.  But the duke says:

"Leggo the boy, you old idiot!  Would YOU a done any different?  Did you
inquire around for HIM when you got loose?  I don't remember it."

So the king let go of me, and begun to cuss that town and everybody in
it. But the duke says:

"You better a blame' sight give YOURSELF a good cussing, for you're the
one that's entitled to it most.  You hain't done a thing from the start
that had any sense in it, except coming out so cool and cheeky with that
imaginary blue-arrow mark.  That WAS bright--it was right down bully; and
it was the thing that saved us.  For if it hadn't been for that they'd a
jailed us till them Englishmen's baggage come--and then--the
penitentiary, you bet! But that trick took 'em to the graveyard, and the
gold done us a still bigger kindness; for if the excited fools hadn't let
go all holts and made that rush to get a look we'd a slept in our cravats
to-night--cravats warranted to WEAR, too--longer than WE'D need 'em."

They was still a minute--thinking; then the king says, kind of
absent-minded like:

"Mf!  And we reckoned the NIGGERS stole it!"

That made me squirm!

"Yes," says the duke, kinder slow and deliberate and sarcastic, "WE did."

After about a half a minute the king drawls out:

"Leastways, I did."

The duke says, the same way:

"On the contrary, I did."

The king kind of ruffles up, and says:

"Looky here, Bilgewater, what'r you referrin' to?"

The duke says, pretty brisk:

"When it comes to that, maybe you'll let me ask, what was YOU referring
to?"

"Shucks!" says the king, very sarcastic; "but I don't know--maybe you was
asleep, and didn't know what you was about."

The duke bristles up now, and says:

"Oh, let UP on this cussed nonsense; do you take me for a blame' fool?
Don't you reckon I know who hid that money in that coffin?"

"YES, sir!  I know you DO know, because you done it yourself!"

"It's a lie!"--and the duke went for him.  The king sings out:

"Take y'r hands off!--leggo my throat!--I take it all back!"

The duke says:

"Well, you just own up, first, that you DID hide that money there,
intending to give me the slip one of these days, and come back and dig it
up, and have it all to yourself."

"Wait jest a minute, duke--answer me this one question, honest and fair;
if you didn't put the money there, say it, and I'll b'lieve you, and take
back everything I said."

"You old scoundrel, I didn't, and you know I didn't.  There, now!"

"Well, then, I b'lieve you.  But answer me only jest this one more--now
DON'T git mad; didn't you have it in your mind to hook the money and hide
it?"

The duke never said nothing for a little bit; then he says:

"Well, I don't care if I DID, I didn't DO it, anyway.  But you not only
had it in mind to do it, but you DONE it."

"I wisht I never die if I done it, duke, and that's honest.  I won't say
I warn't goin' to do it, because I WAS; but you--I mean somebody--got in
ahead o' me."

"It's a lie!  You done it, and you got to SAY you done it, or--"

The king began to gurgle, and then he gasps out:

"'Nough!--I OWN UP!"

I was very glad to hear him say that; it made me feel much more easier
than what I was feeling before.  So the duke took his hands off and says:

"If you ever deny it again I'll drown you.  It's WELL for you to set
there and blubber like a baby--it's fitten for you, after the way you've
acted. I never see such an old ostrich for wanting to gobble everything
--and I a-trusting you all the time, like you was my own father.  You ought
to been ashamed of yourself to stand by and hear it saddled on to a lot
of poor niggers, and you never say a word for 'em.  It makes me feel
ridiculous to think I was soft enough to BELIEVE that rubbage.  Cuss you,
I can see now why you was so anxious to make up the deffisit--you wanted
to get what money I'd got out of the Nonesuch and one thing or another,
and scoop it ALL!"

The king says, timid, and still a-snuffling:

"Why, duke, it was you that said make up the deffisit; it warn't me."

"Dry up!  I don't want to hear no more out of you!" says the duke.  "And
NOW you see what you GOT by it.  They've got all their own money back,
and all of OURN but a shekel or two BESIDES.  G'long to bed, and don't
you deffersit ME no more deffersits, long 's YOU live!"

So the king sneaked into the wigwam and took to his bottle for comfort,
and before long the duke tackled HIS bottle; and so in about a half an
hour they was as thick as thieves again, and the tighter they got the
lovinger they got, and went off a-snoring in each other's arms.  They
both got powerful mellow, but I noticed the king didn't get mellow enough
to forget to remember to not deny about hiding the money-bag again.  That
made me feel easy and satisfied.  Of course when they got to snoring we
had a long gabble, and I told Jim everything.




CHAPTER XXXI.

WE dasn't stop again at any town for days and days; kept right along down
the river.  We was down south in the warm weather now, and a mighty long
ways from home.  We begun to come to trees with Spanish moss on them,
hanging down from the limbs like long, gray beards.  It was the first I
ever see it growing, and it made the woods look solemn and dismal.  So
now the frauds reckoned they was out of danger, and they begun to work
the villages again.

First they done a lecture on temperance; but they didn't make enough for
them both to get drunk on.  Then in another village they started a
dancing-school; but they didn't know no more how to dance than a kangaroo
does; so the first prance they made the general public jumped in and
pranced them out of town.  Another time they tried to go at yellocution;
but they didn't yellocute long till the audience got up and give them a
solid good cussing, and made them skip out.  They tackled missionarying,
and mesmerizing, and doctoring, and telling fortunes, and a little of
everything; but they couldn't seem to have no luck.  So at last they got
just about dead broke, and laid around the raft as she floated along,
thinking and thinking, and never saying nothing, by the half a day at a
time, and dreadful blue and desperate.

And at last they took a change and begun to lay their heads together in
the wigwam and talk low and confidential two or three hours at a time.
Jim and me got uneasy.  We didn't like the look of it.  We judged they
was studying up some kind of worse deviltry than ever.  We turned it over
and over, and at last we made up our minds they was going to break into
somebody's house or store, or was going into the counterfeit-money
business, or something. So then we was pretty scared, and made up an
agreement that we wouldn't have nothing in the world to do with such
actions, and if we ever got the least show we would give them the cold
shake and clear out and leave them behind. Well, early one morning we hid
the raft in a good, safe place about two mile below a little bit of a
shabby village named Pikesville, and the king he went ashore and told us
all to stay hid whilst he went up to town and smelt around to see if
anybody had got any wind of the Royal Nonesuch there yet. ("House to rob,
you MEAN," says I to myself; "and when you get through robbing it you'll
come back here and wonder what has become of me and Jim and the raft--and
you'll have to take it out in wondering.") And he said if he warn't back
by midday the duke and me would know it was all right, and we was to come
along.

So we stayed where we was.  The duke he fretted and sweated around, and
was in a mighty sour way.  He scolded us for everything, and we couldn't
seem to do nothing right; he found fault with every little thing.
Something was a-brewing, sure.  I was good and glad when midday come and
no king; we could have a change, anyway--and maybe a chance for THE
change on top of it.  So me and the duke went up to the village, and
hunted around there for the king, and by and by we found him in the back
room of a little low doggery, very tight, and a lot of loafers
bullyragging him for sport, and he a-cussing and a-threatening with all
his might, and so tight he couldn't walk, and couldn't do nothing to
them.  The duke he begun to abuse him for an old fool, and the king begun
to sass back, and the minute they was fairly at it I lit out and shook
the reefs out of my hind legs, and spun down the river road like a deer,
for I see our chance; and I made up my mind that it would be a long day
before they ever see me and Jim again.  I got down there all out of
breath but loaded up with joy, and sung out:

"Set her loose, Jim! we're all right now!"

But there warn't no answer, and nobody come out of the wigwam.  Jim was
gone!  I set up a shout--and then another--and then another one; and run
this way and that in the woods, whooping and screeching; but it warn't no
use--old Jim was gone.  Then I set down and cried; I couldn't help it.
But I couldn't set still long.  Pretty soon I went out on the road,
trying to think what I better do, and I run across a boy walking, and
asked him if he'd seen a strange nigger dressed so and so, and he says:

"Yes."

"Whereabouts?" says I.

"Down to Silas Phelps' place, two mile below here.  He's a runaway
nigger, and they've got him.  Was you looking for him?"

"You bet I ain't!  I run across him in the woods about an hour or two
ago, and he said if I hollered he'd cut my livers out--and told me to lay
down and stay where I was; and I done it.  Been there ever since; afeard
to come out."

"Well," he says, "you needn't be afeard no more, becuz they've got him.
He run off f'm down South, som'ers."

"It's a good job they got him."

"Well, I RECKON!  There's two hunderd dollars reward on him.  It's like
picking up money out'n the road."

"Yes, it is--and I could a had it if I'd been big enough; I see him
FIRST. Who nailed him?"

"It was an old fellow--a stranger--and he sold out his chance in him for
forty dollars, becuz he's got to go up the river and can't wait.  Think
o' that, now!  You bet I'D wait, if it was seven year."

"That's me, every time," says I.  "But maybe his chance ain't worth no
more than that, if he'll sell it so cheap.  Maybe there's something ain't
straight about it."

"But it IS, though--straight as a string.  I see the handbill myself.  It
tells all about him, to a dot--paints him like a picture, and tells the
plantation he's frum, below NewrLEANS.  No-sirree-BOB, they ain't no
trouble 'bout THAT speculation, you bet you.  Say, gimme a chaw tobacker,
won't ye?"

I didn't have none, so he left.  I went to the raft, and set down in the
wigwam to think.  But I couldn't come to nothing.  I thought till I wore
my head sore, but I couldn't see no way out of the trouble.  After all
this long journey, and after all we'd done for them scoundrels, here it
was all come to nothing, everything all busted up and ruined, because
they could have the heart to serve Jim such a trick as that, and make him
a slave again all his life, and amongst strangers, too, for forty dirty
dollars.

Once I said to myself it would be a thousand times better for Jim to be a
slave at home where his family was, as long as he'd GOT to be a slave,
and so I'd better write a letter to Tom Sawyer and tell him to tell Miss
Watson where he was.  But I soon give up that notion for two things:
she'd be mad and disgusted at his rascality and ungratefulness for
leaving her, and so she'd sell him straight down the river again; and if
she didn't, everybody naturally despises an ungrateful nigger, and they'd
make Jim feel it all the time, and so he'd feel ornery and disgraced.
And then think of ME!  It would get all around that Huck Finn helped a
nigger to get his freedom; and if I was ever to see anybody from that
town again I'd be ready to get down and lick his boots for shame.  That's
just the way:  a person does a low-down thing, and then he don't want to
take no consequences of it. Thinks as long as he can hide it, it ain't no
disgrace.  That was my fix exactly.  The more I studied about this the
more my conscience went to grinding me, and the more wicked and low-down
and ornery I got to feeling. And at last, when it hit me all of a sudden
that here was the plain hand of Providence slapping me in the face and
letting me know my wickedness was being watched all the time from up
there in heaven, whilst I was stealing a poor old woman's nigger that
hadn't ever done me no harm, and now was showing me there's One that's
always on the lookout, and ain't a-going to allow no such miserable
doings to go only just so fur and no further, I most dropped in my tracks
I was so scared.  Well, I tried the best I could to kinder soften it up
somehow for myself by saying I was brung up wicked, and so I warn't so
much to blame; but something inside of me kept saying, "There was the
Sunday-school, you could a gone to it; and if you'd a done it they'd a
learnt you there that people that acts as I'd been acting about that
nigger goes to everlasting fire."

It made me shiver.  And I about made up my mind to pray, and see if I
couldn't try to quit being the kind of a boy I was and be better.  So I
kneeled down.  But the words wouldn't come.  Why wouldn't they?  It
warn't no use to try and hide it from Him.  Nor from ME, neither.  I
knowed very well why they wouldn't come.  It was because my heart warn't
right; it was because I warn't square; it was because I was playing
double.  I was letting ON to give up sin, but away inside of me I was
holding on to the biggest one of all.  I was trying to make my mouth SAY
I would do the right thing and the clean thing, and go and write to that
nigger's owner and tell where he was; but deep down in me I knowed it was
a lie, and He knowed it.  You can't pray a lie--I found that out.

So I was full of trouble, full as I could be; and didn't know what to do.
At last I had an idea; and I says, I'll go and write the letter--and then
see if I can pray.  Why, it was astonishing, the way I felt as light as a
feather right straight off, and my troubles all gone.  So I got a piece
of paper and a pencil, all glad and excited, and set down and wrote:

Miss Watson, your runaway nigger Jim is down here two mile below
Pikesville, and Mr. Phelps has got him and he will give him up for the
reward if you send.

HUCK FINN.

I felt good and all washed clean of sin for the first time I had ever
felt so in my life, and I knowed I could pray now.  But I didn't do it
straight off, but laid the paper down and set there thinking--thinking
how good it was all this happened so, and how near I come to being lost
and going to hell.  And went on thinking.  And got to thinking over our
trip down the river; and I see Jim before me all the time:  in the day
and in the night-time, sometimes moonlight, sometimes storms, and we
a-floating along, talking and singing and laughing.  But somehow I
couldn't seem to strike no places to harden me against him, but only the
other kind.  I'd see him standing my watch on top of his'n, 'stead of
calling me, so I could go on sleeping; and see him how glad he was when I
come back out of the fog; and when I come to him again in the swamp, up
there where the feud was; and such-like times; and would always call me
honey, and pet me and do everything he could think of for me, and how
good he always was; and at last I struck the time I saved him by telling
the men we had small-pox aboard, and he was so grateful, and said I was
the best friend old Jim ever had in the world, and the ONLY one he's got
now; and then I happened to look around and see that paper.

It was a close place.  I took it up, and held it in my hand.  I was
a-trembling, because I'd got to decide, forever, betwixt two things, and
I knowed it.  I studied a minute, sort of holding my breath, and then
says to myself:

"All right, then, I'll GO to hell"--and tore it up.

It was awful thoughts and awful words, but they was said.  And I let them
stay said; and never thought no more about reforming.  I shoved the whole
thing out of my head, and said I would take up wickedness again, which
was in my line, being brung up to it, and the other warn't.  And for a
starter I would go to work and steal Jim out of slavery again; and if I
could think up anything worse, I would do that, too; because as long as I
was in, and in for good, I might as well go the whole hog.

Then I set to thinking over how to get at it, and turned over some
considerable many ways in my mind; and at last fixed up a plan that
suited me.  So then I took the bearings of a woody island that was down
the river a piece, and as soon as it was fairly dark I crept out with my
raft and went for it, and hid it there, and then turned in.  I slept the
night through, and got up before it was light, and had my breakfast, and
put on my store clothes, and tied up some others and one thing or another
in a bundle, and took the canoe and cleared for shore.  I landed below
where I judged was Phelps's place, and hid my bundle in the woods, and
then filled up the canoe with water, and loaded rocks into her and sunk
her where I could find her again when I wanted her, about a quarter of a
mile below a little steam sawmill that was on the bank.

Then I struck up the road, and when I passed the mill I see a sign on it,
"Phelps's Sawmill," and when I come to the farm-houses, two or three
hundred yards further along, I kept my eyes peeled, but didn't see nobody
around, though it was good daylight now.  But I didn't mind, because I
didn't want to see nobody just yet--I only wanted to get the lay of the
land. According to my plan, I was going to turn up there from the
village, not from below.  So I just took a look, and shoved along,
straight for town. Well, the very first man I see when I got there was
the duke.  He was sticking up a bill for the Royal Nonesuch--three-night
performance--like that other time.  They had the cheek, them frauds!  I
was right on him before I could shirk.  He looked astonished, and says:

"Hel-LO!  Where'd YOU come from?"  Then he says, kind of glad and eager,
"Where's the raft?--got her in a good place?"

I says:

"Why, that's just what I was going to ask your grace."

Then he didn't look so joyful, and says:

"What was your idea for asking ME?" he says.

"Well," I says, "when I see the king in that doggery yesterday I says to
myself, we can't get him home for hours, till he's soberer; so I went
a-loafing around town to put in the time and wait.  A man up and offered
me ten cents to help him pull a skiff over the river and back to fetch a
sheep, and so I went along; but when we was dragging him to the boat, and
the man left me a-holt of the rope and went behind him to shove him
along, he was too strong for me and jerked loose and run, and we after
him.  We didn't have no dog, and so we had to chase him all over the
country till we tired him out.  We never got him till dark; then we
fetched him over, and I started down for the raft.  When I got there and
see it was gone, I says to myself, 'They've got into trouble and had to
leave; and they've took my nigger, which is the only nigger I've got in
the world, and now I'm in a strange country, and ain't got no property no
more, nor nothing, and no way to make my living;' so I set down and
cried.  I slept in the woods all night.  But what DID become of the raft,
then?--and Jim--poor Jim!"

"Blamed if I know--that is, what's become of the raft.  That old fool had
made a trade and got forty dollars, and when we found him in the doggery
the loafers had matched half-dollars with him and got every cent but what
he'd spent for whisky; and when I got him home late last night and found
the raft gone, we said, 'That little rascal has stole our raft and shook
us, and run off down the river.'"

"I wouldn't shake my NIGGER, would I?--the only nigger I had in the
world, and the only property."

"We never thought of that.  Fact is, I reckon we'd come to consider him
OUR nigger; yes, we did consider him so--goodness knows we had trouble
enough for him.  So when we see the raft was gone and we flat broke,
there warn't anything for it but to try the Royal Nonesuch another shake.
And I've pegged along ever since, dry as a powder-horn.  Where's that ten
cents? Give it here."

I had considerable money, so I give him ten cents, but begged him to
spend it for something to eat, and give me some, because it was all the
money I had, and I hadn't had nothing to eat since yesterday.  He never
said nothing.  The next minute he whirls on me and says:

"Do you reckon that nigger would blow on us?  We'd skin him if he done
that!"

"How can he blow?  Hain't he run off?"

"No!  That old fool sold him, and never divided with me, and the money's
gone."

"SOLD him?"  I says, and begun to cry; "why, he was MY nigger, and that
was my money.  Where is he?--I want my nigger."

"Well, you can't GET your nigger, that's all--so dry up your blubbering.
Looky here--do you think YOU'D venture to blow on us?  Blamed if I think
I'd trust you.  Why, if you WAS to blow on us--"

He stopped, but I never see the duke look so ugly out of his eyes before.
I went on a-whimpering, and says:

"I don't want to blow on nobody; and I ain't got no time to blow, nohow.
I got to turn out and find my nigger."

He looked kinder bothered, and stood there with his bills fluttering on
his arm, thinking, and wrinkling up his forehead.  At last he says:

"I'll tell you something.  We got to be here three days.  If you'll
promise you won't blow, and won't let the nigger blow, I'll tell you
where to find him."

So I promised, and he says:

"A farmer by the name of Silas Ph--" and then he stopped.  You see, he
started to tell me the truth; but when he stopped that way, and begun to
study and think again, I reckoned he was changing his mind.  And so he
was. He wouldn't trust me; he wanted to make sure of having me out of the
way the whole three days.  So pretty soon he says:

"The man that bought him is named Abram Foster--Abram G. Foster--and he
lives forty mile back here in the country, on the road to Lafayette."

"All right," I says, "I can walk it in three days.  And I'll start this
very afternoon."

"No you wont, you'll start NOW; and don't you lose any time about it,
neither, nor do any gabbling by the way.  Just keep a tight tongue in
your head and move right along, and then you won't get into trouble with
US, d'ye hear?"

That was the order I wanted, and that was the one I played for.  I wanted
to be left free to work my plans.

"So clear out," he says; "and you can tell Mr. Foster whatever you want
to. Maybe you can get him to believe that Jim IS your nigger--some idiots
don't require documents--leastways I've heard there's such down South
here.  And when you tell him the handbill and the reward's bogus, maybe
he'll believe you when you explain to him what the idea was for getting
'em out.  Go 'long now, and tell him anything you want to; but mind you
don't work your jaw any BETWEEN here and there."

So I left, and struck for the back country.  I didn't look around, but I
kinder felt like he was watching me.  But I knowed I could tire him out
at that.  I went straight out in the country as much as a mile before I
stopped; then I doubled back through the woods towards Phelps'.  I
reckoned I better start in on my plan straight off without fooling
around, because I wanted to stop Jim's mouth till these fellows could get
away.  I didn't want no trouble with their kind.  I'd seen all I wanted
to of them, and wanted to get entirely shut of them.




CHAPTER XXXII.

WHEN I got there it was all still and Sunday-like, and hot and sunshiny;
the hands was gone to the fields; and there was them kind of faint
dronings of bugs and flies in the air that makes it seem so lonesome and
like everybody's dead and gone; and if a breeze fans along and quivers
the leaves it makes you feel mournful, because you feel like it's spirits
whispering--spirits that's been dead ever so many years--and you always
think they're talking about YOU.  As a general thing it makes a body wish
HE was dead, too, and done with it all.

Phelps' was one of these little one-horse cotton plantations, and they
all look alike.  A rail fence round a two-acre yard; a stile made out of
logs sawed off and up-ended in steps, like barrels of a different length,
to climb over the fence with, and for the women to stand on when they are
going to jump on to a horse; some sickly grass-patches in the big yard,
but mostly it was bare and smooth, like an old hat with the nap rubbed
off; big double log-house for the white folks--hewed logs, with the
chinks stopped up with mud or mortar, and these mud-stripes been
whitewashed some time or another; round-log kitchen, with a big broad,
open but roofed passage joining it to the house; log smoke-house back of
the kitchen; three little log nigger-cabins in a row t'other side the
smoke-house; one little hut all by itself away down against the back
fence, and some outbuildings down a piece the other side; ash-hopper and
big kettle to bile soap in by the little hut; bench by the kitchen door,
with bucket of water and a gourd; hound asleep there in the sun; more
hounds asleep round about; about three shade trees away off in a corner;
some currant bushes and gooseberry bushes in one place by the fence;
outside of the fence a garden and a watermelon patch; then the cotton
fields begins, and after the fields the woods.

I went around and clumb over the back stile by the ash-hopper, and
started for the kitchen.  When I got a little ways I heard the dim hum of
a spinning-wheel wailing along up and sinking along down again; and then
I knowed for certain I wished I was dead--for that IS the lonesomest
sound in the whole world.

I went right along, not fixing up any particular plan, but just trusting
to Providence to put the right words in my mouth when the time come; for
I'd noticed that Providence always did put the right words in my mouth if
I left it alone.

When I got half-way, first one hound and then another got up and went for
me, and of course I stopped and faced them, and kept still.  And such
another powwow as they made!  In a quarter of a minute I was a kind of a
hub of a wheel, as you may say--spokes made out of dogs--circle of
fifteen of them packed together around me, with their necks and noses
stretched up towards me, a-barking and howling; and more a-coming; you
could see them sailing over fences and around corners from everywheres.

A nigger woman come tearing out of the kitchen with a rolling-pin in her
hand, singing out, "Begone YOU Tige! you Spot! begone sah!" and she
fetched first one and then another of them a clip and sent them howling,
and then the rest followed; and the next second half of them come back,
wagging their tails around me, and making friends with me.  There ain't
no harm in a hound, nohow.

And behind the woman comes a little nigger girl and two little nigger
boys without anything on but tow-linen shirts, and they hung on to their
mother's gown, and peeped out from behind her at me, bashful, the way
they always do.  And here comes the white woman running from the house,
about forty-five or fifty year old, bareheaded, and her spinning-stick in
her hand; and behind her comes her little white children, acting the same
way the little niggers was doing.  She was smiling all over so she could
hardly stand--and says:

"It's YOU, at last!--AIN'T it?"

I out with a "Yes'm" before I thought.

She grabbed me and hugged me tight; and then gripped me by both hands and
shook and shook; and the tears come in her eyes, and run down over; and
she couldn't seem to hug and shake enough, and kept saying, "You don't
look as much like your mother as I reckoned you would; but law sakes, I
don't care for that, I'm so glad to see you!  Dear, dear, it does seem
like I could eat you up!  Children, it's your cousin Tom!--tell him
howdy."

But they ducked their heads, and put their fingers in their mouths, and
hid behind her.  So she run on:

"Lize, hurry up and get him a hot breakfast right away--or did you get
your breakfast on the boat?"

I said I had got it on the boat.  So then she started for the house,
leading me by the hand, and the children tagging after.  When we got
there she set me down in a split-bottomed chair, and set herself down on
a little low stool in front of me, holding both of my hands, and says:

"Now I can have a GOOD look at you; and, laws-a-me, I've been hungry for
it a many and a many a time, all these long years, and it's come at last!
We been expecting you a couple of days and more.  What kep' you?--boat
get aground?"

"Yes'm--she--"

"Don't say yes'm--say Aunt Sally.  Where'd she get aground?"

I didn't rightly know what to say, because I didn't know whether the boat
would be coming up the river or down.  But I go a good deal on instinct;
and my instinct said she would be coming up--from down towards Orleans.
That didn't help me much, though; for I didn't know the names of bars
down that way.  I see I'd got to invent a bar, or forget the name of the
one we got aground on--or--Now I struck an idea, and fetched it out:

"It warn't the grounding--that didn't keep us back but a little.  We
blowed out a cylinder-head."

"Good gracious! anybody hurt?"

"No'm.  Killed a nigger."

"Well, it's lucky; because sometimes people do get hurt.  Two years ago
last Christmas your uncle Silas was coming up from Newrleans on the old
Lally Rook, and she blowed out a cylinder-head and crippled a man.  And I
think he died afterwards.  He was a Baptist.  Your uncle Silas knowed a
family in Baton Rouge that knowed his people very well.  Yes, I remember
now, he DID die.  Mortification set in, and they had to amputate him.
But it didn't save him.  Yes, it was mortification--that was it.  He
turned blue all over, and died in the hope of a glorious resurrection.
They say he was a sight to look at.  Your uncle's been up to the town
every day to fetch you. And he's gone again, not more'n an hour ago;
he'll be back any minute now. You must a met him on the road, didn't
you?--oldish man, with a--"

"No, I didn't see nobody, Aunt Sally.  The boat landed just at daylight,
and I left my baggage on the wharf-boat and went looking around the town
and out a piece in the country, to put in the time and not get here too
soon; and so I come down the back way."

"Who'd you give the baggage to?"

"Nobody."

"Why, child, it 'll be stole!"

"Not where I hid it I reckon it won't," I says.

"How'd you get your breakfast so early on the boat?"

It was kinder thin ice, but I says:

"The captain see me standing around, and told me I better have something
to eat before I went ashore; so he took me in the texas to the officers'
lunch, and give me all I wanted."

I was getting so uneasy I couldn't listen good.  I had my mind on the
children all the time; I wanted to get them out to one side and pump them
a little, and find out who I was.  But I couldn't get no show, Mrs.
Phelps kept it up and run on so.  Pretty soon she made the cold chills
streak all down my back, because she says:

"But here we're a-running on this way, and you hain't told me a word
about Sis, nor any of them.  Now I'll rest my works a little, and you
start up yourn; just tell me EVERYTHING--tell me all about 'm all every
one of 'm; and how they are, and what they're doing, and what they told
you to tell me; and every last thing you can think of."

Well, I see I was up a stump--and up it good.  Providence had stood by me
this fur all right, but I was hard and tight aground now.  I see it
warn't a bit of use to try to go ahead--I'd got to throw up my hand.  So
I says to myself, here's another place where I got to resk the truth.  I
opened my mouth to begin; but she grabbed me and hustled me in behind the
bed, and says:

"Here he comes!  Stick your head down lower--there, that'll do; you can't
be seen now.  Don't you let on you're here.  I'll play a joke on him.
Children, don't you say a word."

I see I was in a fix now.  But it warn't no use to worry; there warn't
nothing to do but just hold still, and try and be ready to stand from
under when the lightning struck.

I had just one little glimpse of the old gentleman when he come in; then
the bed hid him.  Mrs. Phelps she jumps for him, and says:

"Has he come?"

"No," says her husband.

"Good-NESS gracious!" she says, "what in the warld can have become of
him?"

"I can't imagine," says the old gentleman; "and I must say it makes me
dreadful uneasy."

"Uneasy!" she says; "I'm ready to go distracted!  He MUST a come; and
you've missed him along the road.  I KNOW it's so--something tells me
so."

"Why, Sally, I COULDN'T miss him along the road--YOU know that."

"But oh, dear, dear, what WILL Sis say!  He must a come!  You must a
missed him.  He--"

"Oh, don't distress me any more'n I'm already distressed.  I don't know
what in the world to make of it.  I'm at my wit's end, and I don't mind
acknowledging 't I'm right down scared.  But there's no hope that he's
come; for he COULDN'T come and me miss him.  Sally, it's terrible--just
terrible--something's happened to the boat, sure!"

"Why, Silas!  Look yonder!--up the road!--ain't that somebody coming?"

He sprung to the window at the head of the bed, and that give Mrs. Phelps
the chance she wanted.  She stooped down quick at the foot of the bed and
give me a pull, and out I come; and when he turned back from the window
there she stood, a-beaming and a-smiling like a house afire, and I
standing pretty meek and sweaty alongside.  The old gentleman stared, and
says:

"Why, who's that?"

"Who do you reckon 't is?"

"I hain't no idea.  Who IS it?"

"It's TOM SAWYER!"

By jings, I most slumped through the floor!  But there warn't no time to
swap knives; the old man grabbed me by the hand and shook, and kept on
shaking; and all the time how the woman did dance around and laugh and
cry; and then how they both did fire off questions about Sid, and Mary,
and the rest of the tribe.

But if they was joyful, it warn't nothing to what I was; for it was like
being born again, I was so glad to find out who I was.  Well, they froze
to me for two hours; and at last, when my chin was so tired it couldn't
hardly go any more, I had told them more about my family--I mean the
Sawyer family--than ever happened to any six Sawyer families.  And I
explained all about how we blowed out a cylinder-head at the mouth of
White River, and it took us three days to fix it.  Which was all right,
and worked first-rate; because THEY didn't know but what it would take
three days to fix it.  If I'd a called it a bolthead it would a done just
as well.

Now I was feeling pretty comfortable all down one side, and pretty
uncomfortable all up the other.  Being Tom Sawyer was easy and
comfortable, and it stayed easy and comfortable till by and by I hear a
steamboat coughing along down the river.  Then I says to myself, s'pose
Tom Sawyer comes down on that boat?  And s'pose he steps in here any
minute, and sings out my name before I can throw him a wink to keep
quiet?

Well, I couldn't HAVE it that way; it wouldn't do at all.  I must go up
the road and waylay him.  So I told the folks I reckoned I would go up to
the town and fetch down my baggage.  The old gentleman was for going
along with me, but I said no, I could drive the horse myself, and I
druther he wouldn't take no trouble about me.




CHAPTER XXXIII.

SO I started for town in the wagon, and when I was half-way I see a wagon
coming, and sure enough it was Tom Sawyer, and I stopped and waited till
he come along.  I says "Hold on!" and it stopped alongside, and his mouth
opened up like a trunk, and stayed so; and he swallowed two or three
times like a person that's got a dry throat, and then says:

"I hain't ever done you no harm.  You know that.  So, then, what you want
to come back and ha'nt ME for?"

I says:

"I hain't come back--I hain't been GONE."

When he heard my voice it righted him up some, but he warn't quite
satisfied yet.  He says:

"Don't you play nothing on me, because I wouldn't on you.  Honest injun
now, you ain't a ghost?"

"Honest injun, I ain't," I says.

"Well--I--I--well, that ought to settle it, of course; but I can't
somehow seem to understand it no way.  Looky here, warn't you ever
murdered AT ALL?"

"No.  I warn't ever murdered at all--I played it on them.  You come in
here and feel of me if you don't believe me."

So he done it; and it satisfied him; and he was that glad to see me again
he didn't know what to do.  And he wanted to know all about it right off,
because it was a grand adventure, and mysterious, and so it hit him where
he lived.  But I said, leave it alone till by and by; and told his driver
to wait, and we drove off a little piece, and I told him the kind of a
fix I was in, and what did he reckon we better do?  He said, let him
alone a minute, and don't disturb him.  So he thought and thought, and
pretty soon he says:

"It's all right; I've got it.  Take my trunk in your wagon, and let on
it's your'n; and you turn back and fool along slow, so as to get to the
house about the time you ought to; and I'll go towards town a piece, and
take a fresh start, and get there a quarter or a half an hour after you;
and you needn't let on to know me at first."

I says:

"All right; but wait a minute.  There's one more thing--a thing that
NOBODY don't know but me.  And that is, there's a nigger here that I'm
a-trying to steal out of slavery, and his name is JIM--old Miss Watson's
Jim."

He says:

"What!  Why, Jim is--"

He stopped and went to studying.  I says:

"I know what you'll say.  You'll say it's dirty, low-down business; but
what if it is?  I'm low down; and I'm a-going to steal him, and I want
you keep mum and not let on.  Will you?"

His eye lit up, and he says:

"I'll HELP you steal him!"

Well, I let go all holts then, like I was shot.  It was the most
astonishing speech I ever heard--and I'm bound to say Tom Sawyer fell
considerable in my estimation.  Only I couldn't believe it.  Tom Sawyer a
NIGGER-STEALER!

"Oh, shucks!"  I says; "you're joking."

"I ain't joking, either."

"Well, then," I says, "joking or no joking, if you hear anything said
about a runaway nigger, don't forget to remember that YOU don't know
nothing about him, and I don't know nothing about him."

Then we took the trunk and put it in my wagon, and he drove off his way
and I drove mine.  But of course I forgot all about driving slow on
accounts of being glad and full of thinking; so I got home a heap too
quick for that length of a trip.  The old gentleman was at the door, and
he says:

"Why, this is wonderful!  Whoever would a thought it was in that mare to
do it?  I wish we'd a timed her.  And she hain't sweated a hair--not a
hair. It's wonderful.  Why, I wouldn't take a hundred dollars for that
horse now--I wouldn't, honest; and yet I'd a sold her for fifteen
before, and thought 'twas all she was worth."

That's all he said.  He was the innocentest, best old soul I ever see.
But it warn't surprising; because he warn't only just a farmer, he was a
preacher, too, and had a little one-horse log church down back of the
plantation, which he built it himself at his own expense, for a church
and schoolhouse, and never charged nothing for his preaching, and it was
worth it, too.  There was plenty other farmer-preachers like that, and
done the same way, down South.

In about half an hour Tom's wagon drove up to the front stile, and Aunt
Sally she see it through the window, because it was only about fifty
yards, and says:

"Why, there's somebody come!  I wonder who 'tis?  Why, I do believe it's
a stranger.  Jimmy" (that's one of the children) "run and tell Lize to
put on another plate for dinner."

Everybody made a rush for the front door, because, of course, a stranger
don't come EVERY year, and so he lays over the yaller-fever, for
interest, when he does come.  Tom was over the stile and starting for the
house; the wagon was spinning up the road for the village, and we was all
bunched in the front door.  Tom had his store clothes on, and an
audience--and that was always nuts for Tom Sawyer.  In them circumstances
it warn't no trouble to him to throw in an amount of style that was
suitable.  He warn't a boy to meeky along up that yard like a sheep; no,
he come ca'm and important, like the ram.  When he got a-front of us he
lifts his hat ever so gracious and dainty, like it was the lid of a box
that had butterflies asleep in it and he didn't want to disturb them, and
says:

"Mr. Archibald Nichols, I presume?"

"No, my boy," says the old gentleman, "I'm sorry to say 't your driver
has deceived you; Nichols's place is down a matter of three mile more.
Come in, come in."

Tom he took a look back over his shoulder, and says, "Too late--he's out
of sight."

"Yes, he's gone, my son, and you must come in and eat your dinner with
us; and then we'll hitch up and take you down to Nichols's."

"Oh, I CAN'T make you so much trouble; I couldn't think of it.  I'll walk
--I don't mind the distance."

"But we won't LET you walk--it wouldn't be Southern hospitality to do it.
Come right in."

"Oh, DO," says Aunt Sally; "it ain't a bit of trouble to us, not a bit in
the world.  You must stay.  It's a long, dusty three mile, and we can't
let you walk.  And, besides, I've already told 'em to put on another
plate when I see you coming; so you mustn't disappoint us.  Come right in
and make yourself at home."

So Tom he thanked them very hearty and handsome, and let himself be
persuaded, and come in; and when he was in he said he was a stranger from
Hicksville, Ohio, and his name was William Thompson--and he made another
bow.

Well, he run on, and on, and on, making up stuff about Hicksville and
everybody in it he could invent, and I getting a little nervious, and
wondering how this was going to help me out of my scrape; and at last,
still talking along, he reached over and kissed Aunt Sally right on the
mouth, and then settled back again in his chair comfortable, and was
going on talking; but she jumped up and wiped it off with the back of her
hand, and says:

"You owdacious puppy!"

He looked kind of hurt, and says:

"I'm surprised at you, m'am."

"You're s'rp--Why, what do you reckon I am?  I've a good notion to take
and--Say, what do you mean by kissing me?"

He looked kind of humble, and says:

"I didn't mean nothing, m'am.  I didn't mean no harm.  I--I--thought
you'd like it."

"Why, you born fool!"  She took up the spinning stick, and it looked like
it was all she could do to keep from giving him a crack with it.  "What
made you think I'd like it?"

"Well, I don't know.  Only, they--they--told me you would."

"THEY told you I would.  Whoever told you's ANOTHER lunatic.  I never
heard the beat of it.  Who's THEY?"

"Why, everybody.  They all said so, m'am."

It was all she could do to hold in; and her eyes snapped, and her fingers
worked like she wanted to scratch him; and she says:

"Who's 'everybody'?  Out with their names, or ther'll be an idiot short."

He got up and looked distressed, and fumbled his hat, and says:

"I'm sorry, and I warn't expecting it.  They told me to.  They all told
me to.  They all said, kiss her; and said she'd like it.  They all said
it--every one of them.  But I'm sorry, m'am, and I won't do it no more
--I won't, honest."

"You won't, won't you?  Well, I sh'd RECKON you won't!"

"No'm, I'm honest about it; I won't ever do it again--till you ask me."

"Till I ASK you!  Well, I never see the beat of it in my born days!  I
lay you'll be the Methusalem-numskull of creation before ever I ask you
--or the likes of you."

"Well," he says, "it does surprise me so.  I can't make it out, somehow.
They said you would, and I thought you would.  But--" He stopped and
looked around slow, like he wished he could run across a friendly eye
somewheres, and fetched up on the old gentleman's, and says, "Didn't YOU
think she'd like me to kiss her, sir?"

"Why, no; I--I--well, no, I b'lieve I didn't."

Then he looks on around the same way to me, and says:

"Tom, didn't YOU think Aunt Sally 'd open out her arms and say, 'Sid
Sawyer--'"

"My land!" she says, breaking in and jumping for him, "you impudent young
rascal, to fool a body so--" and was going to hug him, but he fended her
off, and says:

"No, not till you've asked me first."

So she didn't lose no time, but asked him; and hugged him and kissed him
over and over again, and then turned him over to the old man, and he took
what was left.  And after they got a little quiet again she says:

"Why, dear me, I never see such a surprise.  We warn't looking for YOU at
all, but only Tom.  Sis never wrote to me about anybody coming but him."

"It's because it warn't INTENDED for any of us to come but Tom," he says;
"but I begged and begged, and at the last minute she let me come, too;
so, coming down the river, me and Tom thought it would be a first-rate
surprise for him to come here to the house first, and for me to by and by
tag along and drop in, and let on to be a stranger.  But it was a
mistake, Aunt Sally.  This ain't no healthy place for a stranger to
come."

"No--not impudent whelps, Sid.  You ought to had your jaws boxed; I
hain't been so put out since I don't know when.  But I don't care, I
don't mind the terms--I'd be willing to stand a thousand such jokes to
have you here. Well, to think of that performance!  I don't deny it, I
was most putrified with astonishment when you give me that smack."

We had dinner out in that broad open passage betwixt the house and the
kitchen; and there was things enough on that table for seven families
--and all hot, too; none of your flabby, tough meat that's laid in a
cupboard in a damp cellar all night and tastes like a hunk of old cold
cannibal in the morning.  Uncle Silas he asked a pretty long blessing
over it, but it was worth it; and it didn't cool it a bit, neither, the
way I've seen them kind of interruptions do lots of times.  There was a
considerable good deal of talk all the afternoon, and me and Tom was on
the lookout all the time; but it warn't no use, they didn't happen to say
nothing about any runaway nigger, and we was afraid to try to work up to
it.  But at supper, at night, one of the little boys says:

"Pa, mayn't Tom and Sid and me go to the show?"

"No," says the old man, "I reckon there ain't going to be any; and you
couldn't go if there was; because the runaway nigger told Burton and me
all about that scandalous show, and Burton said he would tell the people;
so I reckon they've drove the owdacious loafers out of town before this
time."

So there it was!--but I couldn't help it.  Tom and me was to sleep in the
same room and bed; so, being tired, we bid good-night and went up to bed
right after supper, and clumb out of the window and down the
lightning-rod, and shoved for the town; for I didn't believe anybody was
going to give the king and the duke a hint, and so if I didn't hurry up
and give them one they'd get into trouble sure.

On the road Tom he told me all about how it was reckoned I was murdered,
and how pap disappeared pretty soon, and didn't come back no more, and
what a stir there was when Jim run away; and I told Tom all about our
Royal Nonesuch rapscallions, and as much of the raft voyage as I had
time to; and as we struck into the town and up through the middle of
it--it was as much as half-after eight, then--here comes a raging rush of
people with torches, and an awful whooping and yelling, and banging tin
pans and blowing horns; and we jumped to one side to let them go by; and
as they went by I see they had the king and the duke astraddle of a
rail--that is, I knowed it WAS the king and the duke, though they was
all over tar and feathers, and didn't look like nothing in the world
that was human--just looked like a couple of monstrous big soldier-
plumes.  Well, it made me sick to see it; and I was sorry for them poor
pitiful rascals, it seemed like I couldn't ever feel any hardness
against them any more in the world.  It was a dreadful thing to see.
Human beings CAN be awful cruel to one another.

We see we was too late--couldn't do no good.  We asked some stragglers
about it, and they said everybody went to the show looking very innocent;
and laid low and kept dark till the poor old king was in the middle of
his cavortings on the stage; then somebody give a signal, and the house
rose up and went for them.

So we poked along back home, and I warn't feeling so brash as I was
before, but kind of ornery, and humble, and to blame, somehow--though I
hadn't done nothing.  But that's always the way; it don't make no
difference whether you do right or wrong, a person's conscience ain't got
no sense, and just goes for him anyway.  If I had a yaller dog that
didn't know no more than a person's conscience does I would pison him.
It takes up more room than all the rest of a person's insides, and yet
ain't no good, nohow.  Tom Sawyer he says the same.




CHAPTER XXXIV.

WE stopped talking, and got to thinking.  By and by Tom says:

"Looky here, Huck, what fools we are to not think of it before!  I bet I
know where Jim is."

"No!  Where?"

"In that hut down by the ash-hopper.  Why, looky here.  When we was at
dinner, didn't you see a nigger man go in there with some vittles?"

"Yes."

"What did you think the vittles was for?"

"For a dog."

"So 'd I. Well, it wasn't for a dog."

"Why?"

"Because part of it was watermelon."

"So it was--I noticed it.  Well, it does beat all that I never thought
about a dog not eating watermelon.  It shows how a body can see and don't
see at the same time."

"Well, the nigger unlocked the padlock when he went in, and he locked it
again when he came out.  He fetched uncle a key about the time we got up
from table--same key, I bet.  Watermelon shows man, lock shows prisoner;
and it ain't likely there's two prisoners on such a little plantation,
and where the people's all so kind and good.  Jim's the prisoner.  All
right--I'm glad we found it out detective fashion; I wouldn't give
shucks for any other way.  Now you work your mind, and study out a plan
to steal Jim, and I will study out one, too; and we'll take the one we
like the best."

What a head for just a boy to have!  If I had Tom Sawyer's head I
wouldn't trade it off to be a duke, nor mate of a steamboat, nor clown in
a circus, nor nothing I can think of.  I went to thinking out a plan, but
only just to be doing something; I knowed very well where the right plan
was going to come from.  Pretty soon Tom says:

"Ready?"

"Yes," I says.

"All right--bring it out."

"My plan is this," I says.  "We can easy find out if it's Jim in there.
Then get up my canoe to-morrow night, and fetch my raft over from the
island.  Then the first dark night that comes steal the key out of the
old man's britches after he goes to bed, and shove off down the river on
the raft with Jim, hiding daytimes and running nights, the way me and Jim
used to do before.  Wouldn't that plan work?"

"WORK?  Why, cert'nly it would work, like rats a-fighting.  But it's too
blame' simple; there ain't nothing TO it.  What's the good of a plan that
ain't no more trouble than that?  It's as mild as goose-milk.  Why, Huck,
it wouldn't make no more talk than breaking into a soap factory."

I never said nothing, because I warn't expecting nothing different; but I
knowed mighty well that whenever he got HIS plan ready it wouldn't have
none of them objections to it.

And it didn't.  He told me what it was, and I see in a minute it was
worth fifteen of mine for style, and would make Jim just as free a man as
mine would, and maybe get us all killed besides.  So I was satisfied, and
said we would waltz in on it.  I needn't tell what it was here, because I
knowed it wouldn't stay the way, it was.  I knowed he would be changing
it around every which way as we went along, and heaving in new
bullinesses wherever he got a chance.  And that is what he done.

Well, one thing was dead sure, and that was that Tom Sawyer was in
earnest, and was actuly going to help steal that nigger out of slavery.
That was the thing that was too many for me.  Here was a boy that was
respectable and well brung up; and had a character to lose; and folks at
home that had characters; and he was bright and not leather-headed; and
knowing and not ignorant; and not mean, but kind; and yet here he was,
without any more pride, or rightness, or feeling, than to stoop to this
business, and make himself a shame, and his family a shame, before
everybody.  I COULDN'T understand it no way at all.  It was outrageous,
and I knowed I ought to just up and tell him so; and so be his true
friend, and let him quit the thing right where he was and save himself.
And I DID start to tell him; but he shut me up, and says:

"Don't you reckon I know what I'm about?  Don't I generly know what I'm
about?"

"Yes."

"Didn't I SAY I was going to help steal the nigger?"

"Yes."

"WELL, then."

That's all he said, and that's all I said.  It warn't no use to say any
more; because when he said he'd do a thing, he always done it.  But I
couldn't make out how he was willing to go into this thing; so I just let
it go, and never bothered no more about it.  If he was bound to have it
so, I couldn't help it.

When we got home the house was all dark and still; so we went on down to
the hut by the ash-hopper for to examine it.  We went through the yard so
as to see what the hounds would do.  They knowed us, and didn't make no
more noise than country dogs is always doing when anything comes by in
the night.  When we got to the cabin we took a look at the front and the
two sides; and on the side I warn't acquainted with--which was the north
side--we found a square window-hole, up tolerable high, with just one
stout board nailed across it.  I says:

"Here's the ticket.  This hole's big enough for Jim to get through if we
wrench off the board."

Tom says:

"It's as simple as tit-tat-toe, three-in-a-row, and as easy as playing
hooky.  I should HOPE we can find a way that's a little more complicated
than THAT, Huck Finn."

"Well, then," I says, "how 'll it do to saw him out, the way I done
before I was murdered that time?"

"That's more LIKE," he says.  "It's real mysterious, and troublesome, and
good," he says; "but I bet we can find a way that's twice as long.  There
ain't no hurry; le's keep on looking around."

Betwixt the hut and the fence, on the back side, was a lean-to that
joined the hut at the eaves, and was made out of plank.  It was as long
as the hut, but narrow--only about six foot wide.  The door to it was at
the south end, and was padlocked.  Tom he went to the soap-kettle and
searched around, and fetched back the iron thing they lift the lid with;
so he took it and prized out one of the staples.  The chain fell down,
and we opened the door and went in, and shut it, and struck a match, and
see the shed was only built against a cabin and hadn't no connection with
it; and there warn't no floor to the shed, nor nothing in it but some old
rusty played-out hoes and spades and picks and a crippled plow.  The
match went out, and so did we, and shoved in the staple again, and the
door was locked as good as ever. Tom was joyful.  He says;

"Now we're all right.  We'll DIG him out.  It 'll take about a week!"

Then we started for the house, and I went in the back door--you only have
to pull a buckskin latch-string, they don't fasten the doors--but that
warn't romantical enough for Tom Sawyer; no way would do him but he must
climb up the lightning-rod.  But after he got up half way about three
times, and missed fire and fell every time, and the last time most busted
his brains out, he thought he'd got to give it up; but after he was
rested he allowed he would give her one more turn for luck, and this time
he made the trip.

In the morning we was up at break of day, and down to the nigger cabins
to pet the dogs and make friends with the nigger that fed Jim--if it WAS
Jim that was being fed.  The niggers was just getting through breakfast
and starting for the fields; and Jim's nigger was piling up a tin pan
with bread and meat and things; and whilst the others was leaving, the
key come from the house.

This nigger had a good-natured, chuckle-headed face, and his wool was all
tied up in little bunches with thread.  That was to keep witches off.  He
said the witches was pestering him awful these nights, and making him see
all kinds of strange things, and hear all kinds of strange words and
noises, and he didn't believe he was ever witched so long before in his
life.  He got so worked up, and got to running on so about his troubles,
he forgot all about what he'd been a-going to do.  So Tom says:

"What's the vittles for?  Going to feed the dogs?"

The nigger kind of smiled around gradually over his face, like when you
heave a brickbat in a mud-puddle, and he says:

"Yes, Mars Sid, A dog.  Cur'us dog, too.  Does you want to go en look at
'im?"

"Yes."

I hunched Tom, and whispers:

"You going, right here in the daybreak?  THAT warn't the plan."

"No, it warn't; but it's the plan NOW."

So, drat him, we went along, but I didn't like it much.  When we got in
we couldn't hardly see anything, it was so dark; but Jim was there, sure
enough, and could see us; and he sings out:

"Why, HUCK!  En good LAN'! ain' dat Misto Tom?"

I just knowed how it would be; I just expected it.  I didn't know nothing
to do; and if I had I couldn't a done it, because that nigger busted in
and says:

"Why, de gracious sakes! do he know you genlmen?"

We could see pretty well now.  Tom he looked at the nigger, steady and
kind of wondering, and says:

"Does WHO know us?"

"Why, dis-yer runaway nigger."

"I don't reckon he does; but what put that into your head?"

"What PUT it dar?  Didn' he jis' dis minute sing out like he knowed you?"

Tom says, in a puzzled-up kind of way:

"Well, that's mighty curious.  WHO sung out? WHEN did he sing out?  WHAT
did he sing out?" And turns to me, perfectly ca'm, and says, "Did YOU
hear anybody sing out?"

Of course there warn't nothing to be said but the one thing; so I says:

"No; I ain't heard nobody say nothing."

Then he turns to Jim, and looks him over like he never see him before,
and says:

"Did you sing out?"

"No, sah," says Jim; "I hain't said nothing, sah."

"Not a word?"

"No, sah, I hain't said a word."

"Did you ever see us before?"

"No, sah; not as I knows on."

So Tom turns to the nigger, which was looking wild and distressed, and
says, kind of severe:

"What do you reckon's the matter with you, anyway?  What made you think
somebody sung out?"

"Oh, it's de dad-blame' witches, sah, en I wisht I was dead, I do.  Dey's
awluz at it, sah, en dey do mos' kill me, dey sk'yers me so.  Please to
don't tell nobody 'bout it sah, er ole Mars Silas he'll scole me; 'kase
he say dey AIN'T no witches.  I jis' wish to goodness he was heah now
--DEN what would he say!  I jis' bet he couldn' fine no way to git aroun'
it DIS time.  But it's awluz jis' so; people dat's SOT, stays sot; dey
won't look into noth'n'en fine it out f'r deyselves, en when YOU fine it
out en tell um 'bout it, dey doan' b'lieve you."

Tom give him a dime, and said we wouldn't tell nobody; and told him to
buy some more thread to tie up his wool with; and then looks at Jim, and
says:

"I wonder if Uncle Silas is going to hang this nigger.  If I was to catch
a nigger that was ungrateful enough to run away, I wouldn't give him up,
I'd hang him."  And whilst the nigger stepped to the door to look at the
dime and bite it to see if it was good, he whispers to Jim and says:

"Don't ever let on to know us.  And if you hear any digging going on
nights, it's us; we're going to set you free."

Jim only had time to grab us by the hand and squeeze it; then the nigger
come back, and we said we'd come again some time if the nigger wanted us
to; and he said he would, more particular if it was dark, because the
witches went for him mostly in the dark, and it was good to have folks
around then.




CHAPTER XXXV.

IT would be most an hour yet till breakfast, so we left and struck down
into the woods; because Tom said we got to have SOME light to see how to
dig by, and a lantern makes too much, and might get us into trouble; what
we must have was a lot of them rotten chunks that's called fox-fire, and
just makes a soft kind of a glow when you lay them in a dark place.  We
fetched an armful and hid it in the weeds, and set down to rest, and Tom
says, kind of dissatisfied:

"Blame it, this whole thing is just as easy and awkward as it can be.
And so it makes it so rotten difficult to get up a difficult plan.  There
ain't no watchman to be drugged--now there OUGHT to be a watchman.  There
ain't even a dog to give a sleeping-mixture to.  And there's Jim chained
by one leg, with a ten-foot chain, to the leg of his bed:  why, all you
got to do is to lift up the bedstead and slip off the chain.  And Uncle
Silas he trusts everybody; sends the key to the punkin-headed nigger, and
don't send nobody to watch the nigger.  Jim could a got out of that
window-hole before this, only there wouldn't be no use trying to travel
with a ten-foot chain on his leg.  Why, drat it, Huck, it's the stupidest
arrangement I ever see. You got to invent ALL the difficulties.  Well, we
can't help it; we got to do the best we can with the materials we've got.
Anyhow, there's one thing--there's more honor in getting him out
through a lot of difficulties and dangers, where there warn't one of them
furnished to you by the people who it was their duty to furnish them, and
you had to contrive them all out of your own head.  Now look at just that
one thing of the lantern.  When you come down to the cold facts, we
simply got to LET ON that a lantern's resky.  Why, we could work with a
torchlight procession if we wanted to, I believe.  Now, whilst I think of
it, we got to hunt up something to make a saw out of the first chance we
get."

"What do we want of a saw?"

"What do we WANT of it?  Hain't we got to saw the leg of Jim's bed
off, so as to get the chain loose?"

"Why, you just said a body could lift up the bedstead and slip the chain
off."

"Well, if that ain't just like you, Huck Finn.  You CAN get up the
infant-schooliest ways of going at a thing.  Why, hain't you ever read
any books at all?--Baron Trenck, nor Casanova, nor Benvenuto Chelleeny,
nor Henri IV., nor none of them heroes?  Who ever heard of getting a
prisoner loose in such an old-maidy way as that?  No; the way all the
best authorities does is to saw the bed-leg in two, and leave it just so,
and swallow the sawdust, so it can't be found, and put some dirt and
grease around the sawed place so the very keenest seneskal can't see no
sign of it's being sawed, and thinks the bed-leg is perfectly sound.
Then, the night you're ready, fetch the leg a kick, down she goes; slip
off your chain, and there you are.  Nothing to do but hitch your rope
ladder to the battlements, shin down it, break your leg in the moat
--because a rope ladder is nineteen foot too short, you know--and there's
your horses and your trusty vassles, and they scoop you up and fling you
across a saddle, and away you go to your native Langudoc, or Navarre, or
wherever it is. It's gaudy, Huck.  I wish there was a moat to this cabin.
If we get time, the night of the escape, we'll dig one."

I says:

"What do we want of a moat when we're going to snake him out from under
the cabin?"

But he never heard me.  He had forgot me and everything else.  He had his
chin in his hand, thinking.  Pretty soon he sighs and shakes his head;
then sighs again, and says:

"No, it wouldn't do--there ain't necessity enough for it."

"For what?"  I says.

"Why, to saw Jim's leg off," he says.

"Good land!"  I says; "why, there ain't NO necessity for it.  And what
would you want to saw his leg off for, anyway?"

"Well, some of the best authorities has done it.  They couldn't get the
chain off, so they just cut their hand off and shoved.  And a leg would
be better still.  But we got to let that go.  There ain't necessity
enough in this case; and, besides, Jim's a nigger, and wouldn't
understand the reasons for it, and how it's the custom in Europe; so
we'll let it go.  But there's one thing--he can have a rope ladder; we
can tear up our sheets and make him a rope ladder easy enough.  And we
can send it to him in a pie; it's mostly done that way.  And I've et
worse pies."

"Why, Tom Sawyer, how you talk," I says; "Jim ain't got no use for a rope
ladder."

"He HAS got use for it.  How YOU talk, you better say; you don't know
nothing about it.  He's GOT to have a rope ladder; they all do."

"What in the nation can he DO with it?"

"DO with it?  He can hide it in his bed, can't he?  That's what they all
do; and HE'S got to, too.  Huck, you don't ever seem to want to do
anything that's regular; you want to be starting something fresh all the
time. S'pose he DON'T do nothing with it? ain't it there in his bed, for
a clew, after he's gone? and don't you reckon they'll want clews?  Of
course they will.  And you wouldn't leave them any?  That would be a
PRETTY howdy-do, WOULDN'T it!  I never heard of such a thing."

"Well," I says, "if it's in the regulations, and he's got to have it, all
right, let him have it; because I don't wish to go back on no
regulations; but there's one thing, Tom Sawyer--if we go to tearing up
our sheets to make Jim a rope ladder, we're going to get into trouble
with Aunt Sally, just as sure as you're born.  Now, the way I look at it,
a hickry-bark ladder don't cost nothing, and don't waste nothing, and is
just as good to load up a pie with, and hide in a straw tick, as any rag
ladder you can start; and as for Jim, he ain't had no experience, and so
he don't care what kind of a--"

"Oh, shucks, Huck Finn, if I was as ignorant as you I'd keep still
--that's what I'D do.  Who ever heard of a state prisoner escaping by a
hickry-bark ladder?  Why, it's perfectly ridiculous."

"Well, all right, Tom, fix it your own way; but if you'll take my advice,
you'll let me borrow a sheet off of the clothesline."

He said that would do.  And that gave him another idea, and he says:

"Borrow a shirt, too."

"What do we want of a shirt, Tom?"

"Want it for Jim to keep a journal on."

"Journal your granny--JIM can't write."

"S'pose he CAN'T write--he can make marks on the shirt, can't he, if we
make him a pen out of an old pewter spoon or a piece of an old iron
barrel-hoop?"

"Why, Tom, we can pull a feather out of a goose and make him a better
one; and quicker, too."

"PRISONERS don't have geese running around the donjon-keep to pull pens
out of, you muggins.  They ALWAYS make their pens out of the hardest,
toughest, troublesomest piece of old brass candlestick or something like
that they can get their hands on; and it takes them weeks and weeks and
months and months to file it out, too, because they've got to do it by
rubbing it on the wall.  THEY wouldn't use a goose-quill if they had it.
It ain't regular."

"Well, then, what'll we make him the ink out of?"

"Many makes it out of iron-rust and tears; but that's the common sort and
women; the best authorities uses their own blood.  Jim can do that; and
when he wants to send any little common ordinary mysterious message to
let the world know where he's captivated, he can write it on the bottom
of a tin plate with a fork and throw it out of the window.  The Iron Mask
always done that, and it's a blame' good way, too."

"Jim ain't got no tin plates.  They feed him in a pan."

"That ain't nothing; we can get him some."

"Can't nobody READ his plates."

"That ain't got anything to DO with it, Huck Finn.  All HE'S got to do is
to write on the plate and throw it out.  You don't HAVE to be able to
read it. Why, half the time you can't read anything a prisoner writes on
a tin plate, or anywhere else."

"Well, then, what's the sense in wasting the plates?"

"Why, blame it all, it ain't the PRISONER'S plates."

"But it's SOMEBODY'S plates, ain't it?"

"Well, spos'n it is?  What does the PRISONER care whose--"

He broke off there, because we heard the breakfast-horn blowing.  So we
cleared out for the house.

Along during the morning I borrowed a sheet and a white shirt off of the
clothes-line; and I found an old sack and put them in it, and we went
down and got the fox-fire, and put that in too.  I called it borrowing,
because that was what pap always called it; but Tom said it warn't
borrowing, it was stealing.  He said we was representing prisoners; and
prisoners don't care how they get a thing so they get it, and nobody
don't blame them for it, either.  It ain't no crime in a prisoner to
steal the thing he needs to get away with, Tom said; it's his right; and
so, as long as we was representing a prisoner, we had a perfect right to
steal anything on this place we had the least use for to get ourselves
out of prison with.  He said if we warn't prisoners it would be a very
different thing, and nobody but a mean, ornery person would steal when he
warn't a prisoner.  So we allowed we would steal everything there was
that come handy.  And yet he made a mighty fuss, one day, after that,
when I stole a watermelon out of the nigger-patch and eat it; and he made
me go and give the niggers a dime without telling them what it was for.
Tom said that what he meant was, we could steal anything we NEEDED. Well,
I says, I needed the watermelon.  But he said I didn't need it to get out
of prison with; there's where the difference was.  He said if I'd a
wanted it to hide a knife in, and smuggle it to Jim to kill the seneskal
with, it would a been all right.  So I let it go at that, though I
couldn't see no advantage in my representing a prisoner if I got to set
down and chaw over a lot of gold-leaf distinctions like that every time I
see a chance to hog a watermelon.

Well, as I was saying, we waited that morning till everybody was settled
down to business, and nobody in sight around the yard; then Tom he
carried the sack into the lean-to whilst I stood off a piece to keep
watch.  By and by he come out, and we went and set down on the woodpile
to talk.  He says:

"Everything's all right now except tools; and that's easy fixed."

"Tools?"  I says.

"Yes."

"Tools for what?"

"Why, to dig with.  We ain't a-going to GNAW him out, are we?"

"Ain't them old crippled picks and things in there good enough to dig a
nigger out with?"  I says.

He turns on me, looking pitying enough to make a body cry, and says:

"Huck Finn, did you EVER hear of a prisoner having picks and shovels, and
all the modern conveniences in his wardrobe to dig himself out with?  Now
I want to ask you--if you got any reasonableness in you at all--what kind
of a show would THAT give him to be a hero?  Why, they might as well lend
him the key and done with it.  Picks and shovels--why, they wouldn't
furnish 'em to a king."

"Well, then," I says, "if we don't want the picks and shovels, what do we
want?"

"A couple of case-knives."

"To dig the foundations out from under that cabin with?"

"Yes."

"Confound it, it's foolish, Tom."

"It don't make no difference how foolish it is, it's the RIGHT way--and
it's the regular way.  And there ain't no OTHER way, that ever I heard
of, and I've read all the books that gives any information about these
things. They always dig out with a case-knife--and not through dirt, mind
you; generly it's through solid rock.  And it takes them weeks and weeks
and weeks, and for ever and ever.  Why, look at one of them prisoners in
the bottom dungeon of the Castle Deef, in the harbor of Marseilles, that
dug himself out that way; how long was HE at it, you reckon?"

"I don't know."

"Well, guess."

"I don't know.  A month and a half."

"THIRTY-SEVEN YEAR--and he come out in China.  THAT'S the kind.  I wish
the bottom of THIS fortress was solid rock."

"JIM don't know nobody in China."

"What's THAT got to do with it?  Neither did that other fellow.  But
you're always a-wandering off on a side issue.  Why can't you stick to
the main point?"

"All right--I don't care where he comes out, so he COMES out; and Jim
don't, either, I reckon.  But there's one thing, anyway--Jim's too old to
be dug out with a case-knife.  He won't last."

"Yes he will LAST, too.  You don't reckon it's going to take thirty-seven
years to dig out through a DIRT foundation, do you?"

"How long will it take, Tom?"

"Well, we can't resk being as long as we ought to, because it mayn't take
very long for Uncle Silas to hear from down there by New Orleans.  He'll
hear Jim ain't from there.  Then his next move will be to advertise Jim,
or something like that.  So we can't resk being as long digging him out
as we ought to.  By rights I reckon we ought to be a couple of years; but
we can't.  Things being so uncertain, what I recommend is this:  that we
really dig right in, as quick as we can; and after that, we can LET ON,
to ourselves, that we was at it thirty-seven years.  Then we can snatch
him out and rush him away the first time there's an alarm.  Yes, I reckon
that 'll be the best way."

"Now, there's SENSE in that," I says.  "Letting on don't cost nothing;
letting on ain't no trouble; and if it's any object, I don't mind letting
on we was at it a hundred and fifty year.  It wouldn't strain me none,
after I got my hand in.  So I'll mosey along now, and smouch a couple of
case-knives."

"Smouch three," he says; "we want one to make a saw out of."

"Tom, if it ain't unregular and irreligious to sejest it," I says,
"there's an old rusty saw-blade around yonder sticking under the
weather-boarding behind the smoke-house."

He looked kind of weary and discouraged-like, and says:

"It ain't no use to try to learn you nothing, Huck.  Run along and smouch
the knives--three of them."  So I done it.




CHAPTER XXXVI.

AS soon as we reckoned everybody was asleep that night we went down the
lightning-rod, and shut ourselves up in the lean-to, and got out our pile
of fox-fire, and went to work.  We cleared everything out of the way,
about four or five foot along the middle of the bottom log.  Tom said he
was right behind Jim's bed now, and we'd dig in under it, and when we got
through there couldn't nobody in the cabin ever know there was any hole
there, because Jim's counter-pin hung down most to the ground, and you'd
have to raise it up and look under to see the hole.  So we dug and dug
with the case-knives till most midnight; and then we was dog-tired, and
our hands was blistered, and yet you couldn't see we'd done anything
hardly.  At last I says:

"This ain't no thirty-seven year job; this is a thirty-eight year job,
Tom Sawyer."

He never said nothing.  But he sighed, and pretty soon he stopped
digging, and then for a good little while I knowed that he was thinking.
Then he says:

"It ain't no use, Huck, it ain't a-going to work.  If we was prisoners it
would, because then we'd have as many years as we wanted, and no hurry;
and we wouldn't get but a few minutes to dig, every day, while they was
changing watches, and so our hands wouldn't get blistered, and we could
keep it up right along, year in and year out, and do it right, and the
way it ought to be done.  But WE can't fool along; we got to rush; we
ain't got no time to spare.  If we was to put in another night this way
we'd have to knock off for a week to let our hands get well--couldn't
touch a case-knife with them sooner."

"Well, then, what we going to do, Tom?"

"I'll tell you.  It ain't right, and it ain't moral, and I wouldn't like
it to get out; but there ain't only just the one way:  we got to dig him
out with the picks, and LET ON it's case-knives."

"NOW you're TALKING!"  I says; "your head gets leveler and leveler all
the time, Tom Sawyer," I says.  "Picks is the thing, moral or no moral;
and as for me, I don't care shucks for the morality of it, nohow.  When I
start in to steal a nigger, or a watermelon, or a Sunday-school book, I
ain't no ways particular how it's done so it's done.  What I want is my
nigger; or what I want is my watermelon; or what I want is my
Sunday-school book; and if a pick's the handiest thing, that's the thing
I'm a-going to dig that nigger or that watermelon or that Sunday-school
book out with; and I don't give a dead rat what the authorities thinks
about it nuther."

"Well," he says, "there's excuse for picks and letting-on in a case like
this; if it warn't so, I wouldn't approve of it, nor I wouldn't stand by
and see the rules broke--because right is right, and wrong is wrong, and
a body ain't got no business doing wrong when he ain't ignorant and knows
better.  It might answer for YOU to dig Jim out with a pick, WITHOUT any
letting on, because you don't know no better; but it wouldn't for me,
because I do know better.  Gimme a case-knife."

He had his own by him, but I handed him mine.  He flung it down, and
says:

"Gimme a CASE-KNIFE."

I didn't know just what to do--but then I thought.  I scratched around
amongst the old tools, and got a pickaxe and give it to him, and he took
it and went to work, and never said a word.

He was always just that particular.  Full of principle.

So then I got a shovel, and then we picked and shoveled, turn about, and
made the fur fly.  We stuck to it about a half an hour, which was as long
as we could stand up; but we had a good deal of a hole to show for it.
When I got up stairs I looked out at the window and see Tom doing his
level best with the lightning-rod, but he couldn't come it, his hands was
so sore.  At last he says:

"It ain't no use, it can't be done.  What you reckon I better do?  Can't
you think of no way?"

"Yes," I says, "but I reckon it ain't regular.  Come up the stairs, and
let on it's a lightning-rod."

So he done it.

Next day Tom stole a pewter spoon and a brass candlestick in the house,
for to make some pens for Jim out of, and six tallow candles; and I hung
around the nigger cabins and laid for a chance, and stole three tin
plates.  Tom says it wasn't enough; but I said nobody wouldn't ever see
the plates that Jim throwed out, because they'd fall in the dog-fennel
and jimpson weeds under the window-hole--then we could tote them back and
he could use them over again.  So Tom was satisfied.  Then he says:

"Now, the thing to study out is, how to get the things to Jim."

"Take them in through the hole," I says, "when we get it done."

He only just looked scornful, and said something about nobody ever heard
of such an idiotic idea, and then he went to studying.  By and by he said
he had ciphered out two or three ways, but there warn't no need to decide
on any of them yet.  Said we'd got to post Jim first.

That night we went down the lightning-rod a little after ten, and took
one of the candles along, and listened under the window-hole, and heard
Jim snoring; so we pitched it in, and it didn't wake him.  Then we
whirled in with the pick and shovel, and in about two hours and a half
the job was done.  We crept in under Jim's bed and into the cabin, and
pawed around and found the candle and lit it, and stood over Jim awhile,
and found him looking hearty and healthy, and then we woke him up gentle
and gradual.  He was so glad to see us he most cried; and called us
honey, and all the pet names he could think of; and was for having us
hunt up a cold-chisel to cut the chain off of his leg with right away,
and clearing out without losing any time.  But Tom he showed him how
unregular it would be, and set down and told him all about our plans, and
how we could alter them in a minute any time there was an alarm; and not
to be the least afraid, because we would see he got away, SURE.  So Jim
he said it was all right, and we set there and talked over old times
awhile, and then Tom asked a lot of questions, and when Jim told him
Uncle Silas come in every day or two to pray with him, and Aunt Sally
come in to see if he was comfortable and had plenty to eat, and both of
them was kind as they could be, Tom says:

"NOW I know how to fix it.  We'll send you some things by them."

I said, "Don't do nothing of the kind; it's one of the most jackass ideas
I ever struck;" but he never paid no attention to me; went right on.  It
was his way when he'd got his plans set.

So he told Jim how we'd have to smuggle in the rope-ladder pie and other
large things by Nat, the nigger that fed him, and he must be on the
lookout, and not be surprised, and not let Nat see him open them; and we
would put small things in uncle's coat-pockets and he must steal them
out; and we would tie things to aunt's apron-strings or put them in her
apron-pocket, if we got a chance; and told him what they would be and
what they was for.  And told him how to keep a journal on the shirt with
his blood, and all that. He told him everything.  Jim he couldn't see no
sense in the most of it, but he allowed we was white folks and knowed
better than him; so he was satisfied, and said he would do it all just as
Tom said.

Jim had plenty corn-cob pipes and tobacco; so we had a right down good
sociable time; then we crawled out through the hole, and so home to bed,
with hands that looked like they'd been chawed.  Tom was in high spirits.
He said it was the best fun he ever had in his life, and the most
intellectural; and said if he only could see his way to it we would keep
it up all the rest of our lives and leave Jim to our children to get out;
for he believed Jim would come to like it better and better the more he
got used to it.  He said that in that way it could be strung out to as
much as eighty year, and would be the best time on record.  And he said
it would make us all celebrated that had a hand in it.

In the morning we went out to the woodpile and chopped up the brass
candlestick into handy sizes, and Tom put them and the pewter spoon in
his pocket.  Then we went to the nigger cabins, and while I got Nat's
notice off, Tom shoved a piece of candlestick into the middle of a
corn-pone that was in Jim's pan, and we went along with Nat to see how it
would work, and it just worked noble; when Jim bit into it it most mashed
all his teeth out; and there warn't ever anything could a worked better.
Tom said so himself. Jim he never let on but what it was only just a
piece of rock or something like that that's always getting into bread,
you know; but after that he never bit into nothing but what he jabbed his
fork into it in three or four places first.

And whilst we was a-standing there in the dimmish light, here comes a
couple of the hounds bulging in from under Jim's bed; and they kept on
piling in till there was eleven of them, and there warn't hardly room in
there to get your breath.  By jings, we forgot to fasten that lean-to
door!  The nigger Nat he only just hollered "Witches" once, and keeled
over on to the floor amongst the dogs, and begun to groan like he was
dying.  Tom jerked the door open and flung out a slab of Jim's meat, and
the dogs went for it, and in two seconds he was out himself and back
again and shut the door, and I knowed he'd fixed the other door too.
Then he went to work on the nigger, coaxing him and petting him, and
asking him if he'd been imagining he saw something again.  He raised up,
and blinked his eyes around, and says:

"Mars Sid, you'll say I's a fool, but if I didn't b'lieve I see most a
million dogs, er devils, er some'n, I wisht I may die right heah in dese
tracks.  I did, mos' sholy.  Mars Sid, I FELT um--I FELT um, sah; dey was
all over me.  Dad fetch it, I jis' wisht I could git my han's on one er
dem witches jis' wunst--on'y jis' wunst--it's all I'd ast.  But mos'ly I
wisht dey'd lemme 'lone, I does."

Tom says:

"Well, I tell you what I think.  What makes them come here just at this
runaway nigger's breakfast-time?  It's because they're hungry; that's the
reason.  You make them a witch pie; that's the thing for YOU to do."

"But my lan', Mars Sid, how's I gwyne to make 'm a witch pie?  I doan'
know how to make it.  I hain't ever hearn er sich a thing b'fo'."

"Well, then, I'll have to make it myself."

"Will you do it, honey?--will you?  I'll wusshup de groun' und' yo' foot,
I will!"

"All right, I'll do it, seeing it's you, and you've been good to us and
showed us the runaway nigger.  But you got to be mighty careful.  When we
come around, you turn your back; and then whatever we've put in the pan,
don't you let on you see it at all.  And don't you look when Jim unloads
the pan--something might happen, I don't know what.  And above all, don't
you HANDLE the witch-things."

"HANNEL 'm, Mars Sid?  What IS you a-talkin' 'bout?  I wouldn' lay de
weight er my finger on um, not f'r ten hund'd thous'n billion dollars, I
wouldn't."




CHAPTER XXXVII.

THAT was all fixed.  So then we went away and went to the rubbage-pile in
the back yard, where they keep the old boots, and rags, and pieces of
bottles, and wore-out tin things, and all such truck, and scratched
around and found an old tin washpan, and stopped up the holes as well as
we could, to bake the pie in, and took it down cellar and stole it full
of flour and started for breakfast, and found a couple of shingle-nails
that Tom said would be handy for a prisoner to scrabble his name and
sorrows on the dungeon walls with, and dropped one of them in Aunt
Sally's apron-pocket which was hanging on a chair, and t'other we stuck
in the band of Uncle Silas's hat, which was on the bureau, because we
heard the children say their pa and ma was going to the runaway nigger's
house this morning, and then went to breakfast, and Tom dropped the
pewter spoon in Uncle Silas's coat-pocket, and Aunt Sally wasn't come
yet, so we had to wait a little while.

And when she come she was hot and red and cross, and couldn't hardly wait
for the blessing; and then she went to sluicing out coffee with one hand
and cracking the handiest child's head with her thimble with the other,
and says:

"I've hunted high and I've hunted low, and it does beat all what HAS
become of your other shirt."

My heart fell down amongst my lungs and livers and things, and a hard
piece of corn-crust started down my throat after it and got met on the
road with a cough, and was shot across the table, and took one of the
children in the eye and curled him up like a fishing-worm, and let a cry
out of him the size of a warwhoop, and Tom he turned kinder blue around
the gills, and it all amounted to a considerable state of things for
about a quarter of a minute or as much as that, and I would a sold out
for half price if there was a bidder.  But after that we was all right
again--it was the sudden surprise of it that knocked us so kind of cold.
Uncle Silas he says:

"It's most uncommon curious, I can't understand it.  I know perfectly
well I took it OFF, because--"

"Because you hain't got but one ON.  Just LISTEN at the man!  I know you
took it off, and know it by a better way than your wool-gethering memory,
too, because it was on the clo's-line yesterday--I see it there myself.
But it's gone, that's the long and the short of it, and you'll just have
to change to a red flann'l one till I can get time to make a new one.
And it 'll be the third I've made in two years.  It just keeps a body on
the jump to keep you in shirts; and whatever you do manage to DO with 'm
all is more'n I can make out.  A body 'd think you WOULD learn to take
some sort of care of 'em at your time of life."

"I know it, Sally, and I do try all I can.  But it oughtn't to be
altogether my fault, because, you know, I don't see them nor have nothing
to do with them except when they're on me; and I don't believe I've ever
lost one of them OFF of me."

"Well, it ain't YOUR fault if you haven't, Silas; you'd a done it if you
could, I reckon.  And the shirt ain't all that's gone, nuther.  Ther's a
spoon gone; and THAT ain't all.  There was ten, and now ther's only nine.
The calf got the shirt, I reckon, but the calf never took the spoon,
THAT'S certain."

"Why, what else is gone, Sally?"

"Ther's six CANDLES gone--that's what.  The rats could a got the candles,
and I reckon they did; I wonder they don't walk off with the whole place,
the way you're always going to stop their holes and don't do it; and if
they warn't fools they'd sleep in your hair, Silas--YOU'D never find it
out; but you can't lay the SPOON on the rats, and that I know."

"Well, Sally, I'm in fault, and I acknowledge it; I've been remiss; but I
won't let to-morrow go by without stopping up them holes."

"Oh, I wouldn't hurry; next year 'll do.  Matilda Angelina Araminta
PHELPS!"

Whack comes the thimble, and the child snatches her claws out of the
sugar-bowl without fooling around any.  Just then the nigger woman steps
on to the passage, and says:

"Missus, dey's a sheet gone."

"A SHEET gone!  Well, for the land's sake!"

"I'll stop up them holes to-day," says Uncle Silas, looking sorrowful.

"Oh, DO shet up!--s'pose the rats took the SHEET?  WHERE'S it gone,
Lize?"

"Clah to goodness I hain't no notion, Miss' Sally.  She wuz on de
clo'sline yistiddy, but she done gone:  she ain' dah no mo' now."

"I reckon the world IS coming to an end.  I NEVER see the beat of it in
all my born days.  A shirt, and a sheet, and a spoon, and six can--"

"Missus," comes a young yaller wench, "dey's a brass cannelstick miss'n."

"Cler out from here, you hussy, er I'll take a skillet to ye!"

Well, she was just a-biling.  I begun to lay for a chance; I reckoned I
would sneak out and go for the woods till the weather moderated.  She
kept a-raging right along, running her insurrection all by herself, and
everybody else mighty meek and quiet; and at last Uncle Silas, looking
kind of foolish, fishes up that spoon out of his pocket.  She stopped,
with her mouth open and her hands up; and as for me, I wished I was in
Jeruslem or somewheres. But not long, because she says:

"It's JUST as I expected.  So you had it in your pocket all the time; and
like as not you've got the other things there, too.  How'd it get there?"

"I reely don't know, Sally," he says, kind of apologizing, "or you know I
would tell.  I was a-studying over my text in Acts Seventeen before
breakfast, and I reckon I put it in there, not noticing, meaning to put
my Testament in, and it must be so, because my Testament ain't in; but
I'll go and see; and if the Testament is where I had it, I'll know I
didn't put it in, and that will show that I laid the Testament down and
took up the spoon, and--"

"Oh, for the land's sake!  Give a body a rest!  Go 'long now, the whole
kit and biling of ye; and don't come nigh me again till I've got back my
peace of mind."

I'd a heard her if she'd a said it to herself, let alone speaking it out;
and I'd a got up and obeyed her if I'd a been dead.  As we was passing
through the setting-room the old man he took up his hat, and the
shingle-nail fell out on the floor, and he just merely picked it up and
laid it on the mantel-shelf, and never said nothing, and went out.  Tom
see him do it, and remembered about the spoon, and says:

"Well, it ain't no use to send things by HIM no more, he ain't reliable."
Then he says:  "But he done us a good turn with the spoon, anyway,
without knowing it, and so we'll go and do him one without HIM knowing
it--stop up his rat-holes."

There was a noble good lot of them down cellar, and it took us a whole
hour, but we done the job tight and good and shipshape.  Then we heard
steps on the stairs, and blowed out our light and hid; and here comes the
old man, with a candle in one hand and a bundle of stuff in t'other,
looking as absent-minded as year before last.  He went a mooning around,
first to one rat-hole and then another, till he'd been to them all.  Then
he stood about five minutes, picking tallow-drip off of his candle and
thinking.  Then he turns off slow and dreamy towards the stairs, saying:

"Well, for the life of me I can't remember when I done it.  I could show
her now that I warn't to blame on account of the rats.  But never mind
--let it go.  I reckon it wouldn't do no good."

And so he went on a-mumbling up stairs, and then we left.  He was a
mighty nice old man.  And always is.

Tom was a good deal bothered about what to do for a spoon, but he said
we'd got to have it; so he took a think.  When he had ciphered it out he
told me how we was to do; then we went and waited around the spoon-basket
till we see Aunt Sally coming, and then Tom went to counting the spoons
and laying them out to one side, and I slid one of them up my sleeve, and
Tom says:

"Why, Aunt Sally, there ain't but nine spoons YET."

She says:

"Go 'long to your play, and don't bother me.  I know better, I counted 'm
myself."

"Well, I've counted them twice, Aunty, and I can't make but nine."

She looked out of all patience, but of course she come to count--anybody
would.

"I declare to gracious ther' AIN'T but nine!" she says.  "Why, what in
the world--plague TAKE the things, I'll count 'm again."

So I slipped back the one I had, and when she got done counting, she
says:

"Hang the troublesome rubbage, ther's TEN now!" and she looked huffy and
bothered both.  But Tom says:

"Why, Aunty, I don't think there's ten."

"You numskull, didn't you see me COUNT 'm?"

"I know, but--"

"Well, I'll count 'm AGAIN."

So I smouched one, and they come out nine, same as the other time.  Well,
she WAS in a tearing way--just a-trembling all over, she was so mad.  But
she counted and counted till she got that addled she'd start to count in
the basket for a spoon sometimes; and so, three times they come out
right, and three times they come out wrong.  Then she grabbed up the
basket and slammed it across the house and knocked the cat galley-west;
and she said cle'r out and let her have some peace, and if we come
bothering around her again betwixt that and dinner she'd skin us.  So we
had the odd spoon, and dropped it in her apron-pocket whilst she was
a-giving us our sailing orders, and Jim got it all right, along with her
shingle nail, before noon.  We was very well satisfied with this
business, and Tom allowed it was worth twice the trouble it took, because
he said NOW she couldn't ever count them spoons twice alike again to save
her life; and wouldn't believe she'd counted them right if she DID; and
said that after she'd about counted her head off for the next three days
he judged she'd give it up and offer to kill anybody that wanted her to
ever count them any more.

So we put the sheet back on the line that night, and stole one out of her
closet; and kept on putting it back and stealing it again for a couple of
days till she didn't know how many sheets she had any more, and she
didn't CARE, and warn't a-going to bullyrag the rest of her soul out
about it, and wouldn't count them again not to save her life; she druther
die first.

So we was all right now, as to the shirt and the sheet and the spoon and
the candles, by the help of the calf and the rats and the mixed-up
counting; and as to the candlestick, it warn't no consequence, it would
blow over by and by.

But that pie was a job; we had no end of trouble with that pie.  We fixed
it up away down in the woods, and cooked it there; and we got it done at
last, and very satisfactory, too; but not all in one day; and we had to
use up three wash-pans full of flour before we got through, and we got
burnt pretty much all over, in places, and eyes put out with the smoke;
because, you see, we didn't want nothing but a crust, and we couldn't
prop it up right, and she would always cave in.  But of course we thought
of the right way at last--which was to cook the ladder, too, in the
pie.  So then we laid in with Jim the second night, and tore up the sheet
all in little strings and twisted them together, and long before daylight
we had a lovely rope that you could a hung a person with.  We let on it
took nine months to make it.

And in the forenoon we took it down to the woods, but it wouldn't go into
the pie.  Being made of a whole sheet, that way, there was rope enough
for forty pies if we'd a wanted them, and plenty left over for soup, or
sausage, or anything you choose.  We could a had a whole dinner.

But we didn't need it.  All we needed was just enough for the pie,
and so we throwed the rest away.  We didn't cook none of the pies in the
wash-pan--afraid the solder would melt; but Uncle Silas he had a noble
brass warming-pan which he thought considerable of, because it belonged
to one of his ancesters with a long wooden handle that come over from
England with William the Conqueror in the Mayflower or one of them early
ships and was hid away up garret with a lot of other old pots and things
that was valuable, not on account of being any account, because they
warn't, but on account of them being relicts, you know, and we snaked her
out, private, and took her down there, but she failed on the first pies,
because we didn't know how, but she come up smiling on the last one.  We
took and lined her with dough, and set her in the coals, and loaded her
up with rag rope, and put on a dough roof, and shut down the lid, and put
hot embers on top, and stood off five foot, with the long handle, cool
and comfortable, and in fifteen minutes she turned out a pie that was a
satisfaction to look at. But the person that et it would want to fetch a
couple of kags of toothpicks along, for if that rope ladder wouldn't
cramp him down to business I don't know nothing what I'm talking about,
and lay him in enough stomach-ache to last him till next time, too.

Nat didn't look when we put the witch pie in Jim's pan; and we put the
three tin plates in the bottom of the pan under the vittles; and so Jim
got everything all right, and as soon as he was by himself he busted into
the pie and hid the rope ladder inside of his straw tick, and scratched
some marks on a tin plate and throwed it out of the window-hole.




CHAPTER XXXVIII.

MAKING them pens was a distressid tough job, and so was the saw; and Jim
allowed the inscription was going to be the toughest of all.  That's the
one which the prisoner has to scrabble on the wall.  But he had to have
it; Tom said he'd GOT to; there warn't no case of a state prisoner not
scrabbling his inscription to leave behind, and his coat of arms.

"Look at Lady Jane Grey," he says; "look at Gilford Dudley; look at old
Northumberland!  Why, Huck, s'pose it IS considerble trouble?--what you
going to do?--how you going to get around it?  Jim's GOT to do his
inscription and coat of arms.  They all do."

Jim says:

"Why, Mars Tom, I hain't got no coat o' arm; I hain't got nuffn but dish
yer ole shirt, en you knows I got to keep de journal on dat."

"Oh, you don't understand, Jim; a coat of arms is very different."

"Well," I says, "Jim's right, anyway, when he says he ain't got no coat
of arms, because he hain't."

"I reckon I knowed that," Tom says, "but you bet he'll have one before he
goes out of this--because he's going out RIGHT, and there ain't going to
be no flaws in his record."

So whilst me and Jim filed away at the pens on a brickbat apiece, Jim
a-making his'n out of the brass and I making mine out of the spoon, Tom
set to work to think out the coat of arms.  By and by he said he'd struck
so many good ones he didn't hardly know which to take, but there was one
which he reckoned he'd decide on.  He says:

"On the scutcheon we'll have a bend OR in the dexter base, a saltire
MURREY in the fess, with a dog, couchant, for common charge, and under
his foot a chain embattled, for slavery, with a chevron VERT in a chief
engrailed, and three invected lines on a field AZURE, with the nombril
points rampant on a dancette indented; crest, a runaway nigger, SABLE,
with his bundle over his shoulder on a bar sinister; and a couple of
gules for supporters, which is you and me; motto, MAGGIORE FRETTA, MINORE
OTTO.  Got it out of a book--means the more haste the less speed."

"Geewhillikins," I says, "but what does the rest of it mean?"

"We ain't got no time to bother over that," he says; "we got to dig in
like all git-out."

"Well, anyway," I says, "what's SOME of it?  What's a fess?"

"A fess--a fess is--YOU don't need to know what a fess is.  I'll show him
how to make it when he gets to it."

"Shucks, Tom," I says, "I think you might tell a person.  What's a bar
sinister?"

"Oh, I don't know.  But he's got to have it.  All the nobility does."

That was just his way.  If it didn't suit him to explain a thing to you,
he wouldn't do it.  You might pump at him a week, it wouldn't make no
difference.

He'd got all that coat of arms business fixed, so now he started in to
finish up the rest of that part of the work, which was to plan out a
mournful inscription--said Jim got to have one, like they all done.  He
made up a lot, and wrote them out on a paper, and read them off, so:

1.  Here a captive heart busted. 2.  Here a poor prisoner, forsook by the
world and friends, fretted his sorrowful life. 3.  Here a lonely heart
broke, and a worn spirit went to its rest, after thirty-seven years of
solitary captivity. 4.  Here, homeless and friendless, after thirty-seven
years of bitter captivity, perished a noble stranger, natural son of
Louis XIV.

Tom's voice trembled whilst he was reading them, and he most broke down.
When he got done he couldn't no way make up his mind which one for Jim to
scrabble on to the wall, they was all so good; but at last he allowed he
would let him scrabble them all on.  Jim said it would take him a year to
scrabble such a lot of truck on to the logs with a nail, and he didn't
know how to make letters, besides; but Tom said he would block them out
for him, and then he wouldn't have nothing to do but just follow the
lines.  Then pretty soon he says:

"Come to think, the logs ain't a-going to do; they don't have log walls
in a dungeon:  we got to dig the inscriptions into a rock.  We'll fetch a
rock."

Jim said the rock was worse than the logs; he said it would take him such
a pison long time to dig them into a rock he wouldn't ever get out.  But
Tom said he would let me help him do it.  Then he took a look to see how
me and Jim was getting along with the pens.  It was most pesky tedious
hard work and slow, and didn't give my hands no show to get well of the
sores, and we didn't seem to make no headway, hardly; so Tom says:

"I know how to fix it.  We got to have a rock for the coat of arms and
mournful inscriptions, and we can kill two birds with that same rock.
There's a gaudy big grindstone down at the mill, and we'll smouch it, and
carve the things on it, and file out the pens and the saw on it, too."

It warn't no slouch of an idea; and it warn't no slouch of a grindstone
nuther; but we allowed we'd tackle it.  It warn't quite midnight yet, so
we cleared out for the mill, leaving Jim at work.  We smouched the
grindstone, and set out to roll her home, but it was a most nation tough
job. Sometimes, do what we could, we couldn't keep her from falling over,
and she come mighty near mashing us every time.  Tom said she was going
to get one of us, sure, before we got through.  We got her half way; and
then we was plumb played out, and most drownded with sweat.  We see it
warn't no use; we got to go and fetch Jim. So he raised up his bed and
slid the chain off of the bed-leg, and wrapt it round and round his neck,
and we crawled out through our hole and down there, and Jim and me laid
into that grindstone and walked her along like nothing; and Tom
superintended.  He could out-superintend any boy I ever see.  He knowed
how to do everything.

Our hole was pretty big, but it warn't big enough to get the grindstone
through; but Jim he took the pick and soon made it big enough.  Then Tom
marked out them things on it with the nail, and set Jim to work on them,
with the nail for a chisel and an iron bolt from the rubbage in the
lean-to for a hammer, and told him to work till the rest of his candle
quit on him, and then he could go to bed, and hide the grindstone under
his straw tick and sleep on it.  Then we helped him fix his chain back on
the bed-leg, and was ready for bed ourselves.  But Tom thought of
something, and says:

"You got any spiders in here, Jim?"

"No, sah, thanks to goodness I hain't, Mars Tom."

"All right, we'll get you some."

"But bless you, honey, I doan' WANT none.  I's afeard un um.  I jis' 's
soon have rattlesnakes aroun'."

Tom thought a minute or two, and says:

"It's a good idea.  And I reckon it's been done.  It MUST a been done; it
stands to reason.  Yes, it's a prime good idea.  Where could you keep
it?"

"Keep what, Mars Tom?"

"Why, a rattlesnake."

"De goodness gracious alive, Mars Tom!  Why, if dey was a rattlesnake to
come in heah I'd take en bust right out thoo dat log wall, I would, wid
my head."

"Why, Jim, you wouldn't be afraid of it after a little.  You could tame
it."

"TAME it!"

"Yes--easy enough.  Every animal is grateful for kindness and petting,
and they wouldn't THINK of hurting a person that pets them.  Any book
will tell you that.  You try--that's all I ask; just try for two or three
days. Why, you can get him so, in a little while, that he'll love you; and
sleep with you; and won't stay away from you a minute; and will let you
wrap him round your neck and put his head in your mouth."

"PLEASE, Mars Tom--DOAN' talk so!  I can't STAN' it!  He'd LET me shove
his head in my mouf--fer a favor, hain't it?  I lay he'd wait a pow'ful
long time 'fo' I AST him.  En mo' en dat, I doan' WANT him to sleep wid
me."

"Jim, don't act so foolish.  A prisoner's GOT to have some kind of a dumb
pet, and if a rattlesnake hain't ever been tried, why, there's more glory
to be gained in your being the first to ever try it than any other way
you could ever think of to save your life."

"Why, Mars Tom, I doan' WANT no sich glory.  Snake take 'n bite Jim's
chin off, den WHAH is de glory?  No, sah, I doan' want no sich doin's."

"Blame it, can't you TRY?  I only WANT you to try--you needn't keep it up
if it don't work."

"But de trouble all DONE ef de snake bite me while I's a tryin' him.
Mars Tom, I's willin' to tackle mos' anything 'at ain't onreasonable, but
ef you en Huck fetches a rattlesnake in heah for me to tame, I's gwyne to
LEAVE, dat's SHORE."

"Well, then, let it go, let it go, if you're so bull-headed about it.  We
can get you some garter-snakes, and you can tie some buttons on their
tails, and let on they're rattlesnakes, and I reckon that 'll have to
do."

"I k'n stan' DEM, Mars Tom, but blame' 'f I couldn' get along widout um,
I tell you dat.  I never knowed b'fo' 't was so much bother and trouble
to be a prisoner."

"Well, it ALWAYS is when it's done right.  You got any rats around here?"

"No, sah, I hain't seed none."

"Well, we'll get you some rats."

"Why, Mars Tom, I doan' WANT no rats.  Dey's de dadblamedest creturs to
'sturb a body, en rustle roun' over 'im, en bite his feet, when he's
tryin' to sleep, I ever see.  No, sah, gimme g'yarter-snakes, 'f I's got
to have 'm, but doan' gimme no rats; I hain' got no use f'r um, skasely."

"But, Jim, you GOT to have 'em--they all do.  So don't make no more fuss
about it.  Prisoners ain't ever without rats.  There ain't no instance of
it.  And they train them, and pet them, and learn them tricks, and they
get to be as sociable as flies.  But you got to play music to them.  You
got anything to play music on?"

"I ain' got nuffn but a coase comb en a piece o' paper, en a juice-harp;
but I reck'n dey wouldn' take no stock in a juice-harp."

"Yes they would.  THEY don't care what kind of music 'tis.  A jews-harp's
plenty good enough for a rat.  All animals like music--in a prison they
dote on it.  Specially, painful music; and you can't get no other kind
out of a jews-harp.  It always interests them; they come out to see
what's the matter with you.  Yes, you're all right; you're fixed very
well.  You want to set on your bed nights before you go to sleep, and
early in the mornings, and play your jews-harp; play 'The Last Link is
Broken'--that's the thing that 'll scoop a rat quicker 'n anything else;
and when you've played about two minutes you'll see all the rats, and the
snakes, and spiders, and things begin to feel worried about you, and
come.  And they'll just fairly swarm over you, and have a noble good
time."

"Yes, DEY will, I reck'n, Mars Tom, but what kine er time is JIM havin'?
Blest if I kin see de pint.  But I'll do it ef I got to.  I reck'n I
better keep de animals satisfied, en not have no trouble in de house."

Tom waited to think it over, and see if there wasn't nothing else; and
pretty soon he says:

"Oh, there's one thing I forgot.  Could you raise a flower here, do you
reckon?"

"I doan know but maybe I could, Mars Tom; but it's tolable dark in heah,
en I ain' got no use f'r no flower, nohow, en she'd be a pow'ful sight o'
trouble."

"Well, you try it, anyway.  Some other prisoners has done it."

"One er dem big cat-tail-lookin' mullen-stalks would grow in heah, Mars
Tom, I reck'n, but she wouldn't be wuth half de trouble she'd coss."

"Don't you believe it.  We'll fetch you a little one and you plant it in
the corner over there, and raise it.  And don't call it mullen, call it
Pitchiola--that's its right name when it's in a prison.  And you want to
water it with your tears."

"Why, I got plenty spring water, Mars Tom."

"You don't WANT spring water; you want to water it with your tears.  It's
the way they always do."

"Why, Mars Tom, I lay I kin raise one er dem mullen-stalks twyste wid
spring water whiles another man's a START'N one wid tears."

"That ain't the idea.  You GOT to do it with tears."

"She'll die on my han's, Mars Tom, she sholy will; kase I doan' skasely
ever cry."

So Tom was stumped.  But he studied it over, and then said Jim would have
to worry along the best he could with an onion.  He promised he would go
to the nigger cabins and drop one, private, in Jim's coffee-pot, in the
morning. Jim said he would "jis' 's soon have tobacker in his coffee;"
and found so much fault with it, and with the work and bother of raising
the mullen, and jews-harping the rats, and petting and flattering up the
snakes and spiders and things, on top of all the other work he had to do
on pens, and inscriptions, and journals, and things, which made it more
trouble and worry and responsibility to be a prisoner than anything he
ever undertook, that Tom most lost all patience with him; and said he was
just loadened down with more gaudier chances than a prisoner ever had in
the world to make a name for himself, and yet he didn't know enough to
appreciate them, and they was just about wasted on him.  So Jim he was
sorry, and said he wouldn't behave so no more, and then me and Tom shoved
for bed.




CHAPTER XXXIX.

IN the morning we went up to the village and bought a wire rat-trap and
fetched it down, and unstopped the best rat-hole, and in about an hour we
had fifteen of the bulliest kind of ones; and then we took it and put it
in a safe place under Aunt Sally's bed.  But while we was gone for
spiders little Thomas Franklin Benjamin Jefferson Elexander Phelps found
it there, and opened the door of it to see if the rats would come out,
and they did; and Aunt Sally she come in, and when we got back she was
a-standing on top of the bed raising Cain, and the rats was doing what
they could to keep off the dull times for her.  So she took and dusted us
both with the hickry, and we was as much as two hours catching another
fifteen or sixteen, drat that meddlesome cub, and they warn't the
likeliest, nuther, because the first haul was the pick of the flock.
I never see a likelier lot of rats than what that first haul was.

We got a splendid stock of sorted spiders, and bugs, and frogs, and
caterpillars, and one thing or another; and we like to got a hornet's
nest, but we didn't.  The family was at home.  We didn't give it right
up, but stayed with them as long as we could; because we allowed we'd
tire them out or they'd got to tire us out, and they done it.  Then we
got allycumpain and rubbed on the places, and was pretty near all right
again, but couldn't set down convenient.  And so we went for the snakes,
and grabbed a couple of dozen garters and house-snakes, and put them in a
bag, and put it in our room, and by that time it was supper-time, and a
rattling good honest day's work:  and hungry?--oh, no, I reckon not!  And
there warn't a blessed snake up there when we went back--we didn't half
tie the sack, and they worked out somehow, and left.  But it didn't
matter much, because they was still on the premises somewheres.  So we
judged we could get some of them again.  No, there warn't no real
scarcity of snakes about the house for a considerable spell.  You'd see
them dripping from the rafters and places every now and then; and they
generly landed in your plate, or down the back of your neck, and most of
the time where you didn't want them.  Well, they was handsome and
striped, and there warn't no harm in a million of them; but that never
made no difference to Aunt Sally; she despised snakes, be the breed what
they might, and she couldn't stand them no way you could fix it; and
every time one of them flopped down on her, it didn't make no difference
what she was doing, she would just lay that work down and light out.  I
never see such a woman.  And you could hear her whoop to Jericho.  You
couldn't get her to take a-holt of one of them with the tongs.  And if
she turned over and found one in bed she would scramble out and lift a
howl that you would think the house was afire.  She disturbed the old man
so that he said he could most wish there hadn't ever been no snakes
created.  Why, after every last snake had been gone clear out of the
house for as much as a week Aunt Sally warn't over it yet; she warn't
near over it; when she was setting thinking about something you could
touch her on the back of her neck with a feather and she would jump right
out of her stockings.  It was very curious.  But Tom said all women was
just so.  He said they was made that way for some reason or other.

We got a licking every time one of our snakes come in her way, and she
allowed these lickings warn't nothing to what she would do if we ever
loaded up the place again with them.  I didn't mind the lickings, because
they didn't amount to nothing; but I minded the trouble we had to lay in
another lot.  But we got them laid in, and all the other things; and you
never see a cabin as blithesome as Jim's was when they'd all swarm out
for music and go for him.  Jim didn't like the spiders, and the spiders
didn't like Jim; and so they'd lay for him, and make it mighty warm for
him.  And he said that between the rats and the snakes and the grindstone
there warn't no room in bed for him, skasely; and when there was, a body
couldn't sleep, it was so lively, and it was always lively, he said,
because THEY never all slept at one time, but took turn about, so when
the snakes was asleep the rats was on deck, and when the rats turned in
the snakes come on watch, so he always had one gang under him, in his
way, and t'other gang having a circus over him, and if he got up to hunt
a new place the spiders would take a chance at him as he crossed over.
He said if he ever got out this time he wouldn't ever be a prisoner
again, not for a salary.

Well, by the end of three weeks everything was in pretty good shape.  The
shirt was sent in early, in a pie, and every time a rat bit Jim he would
get up and write a little in his journal whilst the ink was fresh; the
pens was made, the inscriptions and so on was all carved on the
grindstone; the bed-leg was sawed in two, and we had et up the sawdust,
and it give us a most amazing stomach-ache.  We reckoned we was all going
to die, but didn't.  It was the most undigestible sawdust I ever see; and
Tom said the same.  But as I was saying, we'd got all the work done now,
at last; and we was all pretty much fagged out, too, but mainly Jim.  The
old man had wrote a couple of times to the plantation below Orleans to
come and get their runaway nigger, but hadn't got no answer, because
there warn't no such plantation; so he allowed he would advertise Jim in
the St. Louis and New Orleans papers; and when he mentioned the St. Louis
ones it give me the cold shivers, and I see we hadn't no time to lose.
So Tom said, now for the nonnamous letters.

"What's them?"  I says.

"Warnings to the people that something is up.  Sometimes it's done one
way, sometimes another.  But there's always somebody spying around that
gives notice to the governor of the castle.  When Louis XVI. was going to
light out of the Tooleries, a servant-girl done it.  It's a very good way,
and so is the nonnamous letters.  We'll use them both.  And it's usual
for the prisoner's mother to change clothes with him, and she stays in,
and he slides out in her clothes.  We'll do that, too."

"But looky here, Tom, what do we want to WARN anybody for that
something's up?  Let them find it out for themselves--it's their
lookout."

"Yes, I know; but you can't depend on them.  It's the way they've acted
from the very start--left us to do EVERYTHING.  They're so confiding and
mullet-headed they don't take notice of nothing at all.  So if we don't
GIVE them notice there won't be nobody nor nothing to interfere with us,
and so after all our hard work and trouble this escape 'll go off
perfectly flat; won't amount to nothing--won't be nothing TO it."

"Well, as for me, Tom, that's the way I'd like."

"Shucks!" he says, and looked disgusted.  So I says:

"But I ain't going to make no complaint.  Any way that suits you suits
me. What you going to do about the servant-girl?"

"You'll be her.  You slide in, in the middle of the night, and hook that
yaller girl's frock."

"Why, Tom, that 'll make trouble next morning; because, of course, she
prob'bly hain't got any but that one."

"I know; but you don't want it but fifteen minutes, to carry the
nonnamous letter and shove it under the front door."

"All right, then, I'll do it; but I could carry it just as handy in my
own togs."

"You wouldn't look like a servant-girl THEN, would you?"

"No, but there won't be nobody to see what I look like, ANYWAY."

"That ain't got nothing to do with it.  The thing for us to do is just to
do our DUTY, and not worry about whether anybody SEES us do it or not.
Hain't you got no principle at all?"

"All right, I ain't saying nothing; I'm the servant-girl.  Who's Jim's
mother?"

"I'm his mother.  I'll hook a gown from Aunt Sally."

"Well, then, you'll have to stay in the cabin when me and Jim leaves."

"Not much.  I'll stuff Jim's clothes full of straw and lay it on his bed
to represent his mother in disguise, and Jim 'll take the nigger woman's
gown off of me and wear it, and we'll all evade together.  When a
prisoner of style escapes it's called an evasion.  It's always called so
when a king escapes, f'rinstance.  And the same with a king's son; it
don't make no difference whether he's a natural one or an unnatural one."

So Tom he wrote the nonnamous letter, and I smouched the yaller wench's
frock that night, and put it on, and shoved it under the front door, the
way Tom told me to.  It said:

Beware.  Trouble is brewing.  Keep a sharp lookout. UNKNOWN FRIEND.

Next night we stuck a picture, which Tom drawed in blood, of a skull and
crossbones on the front door; and next night another one of a coffin on
the back door.  I never see a family in such a sweat.  They couldn't a
been worse scared if the place had a been full of ghosts laying for them
behind everything and under the beds and shivering through the air.  If a
door banged, Aunt Sally she jumped and said "ouch!" if anything fell, she
jumped and said "ouch!" if you happened to touch her, when she warn't
noticing, she done the same; she couldn't face noway and be satisfied,
because she allowed there was something behind her every time--so she was
always a-whirling around sudden, and saying "ouch," and before she'd got
two-thirds around she'd whirl back again, and say it again; and she was
afraid to go to bed, but she dasn't set up.  So the thing was working
very well, Tom said; he said he never see a thing work more satisfactory.
He said it showed it was done right.

So he said, now for the grand bulge!  So the very next morning at the
streak of dawn we got another letter ready, and was wondering what we
better do with it, because we heard them say at supper they was going to
have a nigger on watch at both doors all night.  Tom he went down the
lightning-rod to spy around; and the nigger at the back door was asleep,
and he stuck it in the back of his neck and come back.  This letter said:

Don't betray me, I wish to be your friend.  There is a desprate gang of
cutthroats from over in the Indian Territory going to steal your runaway
nigger to-night, and they have been trying to scare you so as you will
stay in the house and not bother them.  I am one of the gang, but have
got religgion and wish to quit it and lead an honest life again, and will
betray the helish design. They will sneak down from northards, along the
fence, at midnight exact, with a false key, and go in the nigger's cabin
to get him. I am to be off a piece and blow a tin horn if I see any
danger; but stead of that I will BA like a sheep soon as they get in and
not blow at all; then whilst they are getting his chains loose, you slip
there and lock them in, and can kill them at your leasure.  Don't do
anything but just the way I am telling you, if you do they will suspicion
something and raise whoop-jamboreehoo. I do not wish any reward but to
know I have done the right thing. UNKNOWN FRIEND.




CHAPTER XL.

WE was feeling pretty good after breakfast, and took my canoe and went
over the river a-fishing, with a lunch, and had a good time, and took a
look at the raft and found her all right, and got home late to supper,
and found them in such a sweat and worry they didn't know which end they
was standing on, and made us go right off to bed the minute we was done
supper, and wouldn't tell us what the trouble was, and never let on a
word about the new letter, but didn't need to, because we knowed as much
about it as anybody did, and as soon as we was half up stairs and her
back was turned we slid for the cellar cupboard and loaded up a good
lunch and took it up to our room and went to bed, and got up about
half-past eleven, and Tom put on Aunt Sally's dress that he stole and
was going to start with the lunch, but says:

"Where's the butter?"

"I laid out a hunk of it," I says, "on a piece of a corn-pone."

"Well, you LEFT it laid out, then--it ain't here."

"We can get along without it," I says.

"We can get along WITH it, too," he says; "just you slide down cellar and
fetch it.  And then mosey right down the lightning-rod and come along.
I'll go and stuff the straw into Jim's clothes to represent his mother in
disguise, and be ready to BA like a sheep and shove soon as you get
there."

So out he went, and down cellar went I. The hunk of butter, big as a
person's fist, was where I had left it, so I took up the slab of
corn-pone with it on, and blowed out my light, and started up stairs very
stealthy, and got up to the main floor all right, but here comes Aunt
Sally with a candle, and I clapped the truck in my hat, and clapped my
hat on my head, and the next second she see me; and she says:

"You been down cellar?"

"Yes'm."

"What you been doing down there?"

"Noth'n."

"NOTH'N!"

"No'm."

"Well, then, what possessed you to go down there this time of night?"

"I don't know 'm."

"You don't KNOW?  Don't answer me that way. Tom, I want to know what you
been DOING down there."

"I hain't been doing a single thing, Aunt Sally, I hope to gracious if I
have."

I reckoned she'd let me go now, and as a generl thing she would; but I
s'pose there was so many strange things going on she was just in a sweat
about every little thing that warn't yard-stick straight; so she says,
very decided:

"You just march into that setting-room and stay there till I come.  You
been up to something you no business to, and I lay I'll find out what it
is before I'M done with you."

So she went away as I opened the door and walked into the setting-room.
My, but there was a crowd there!  Fifteen farmers, and every one of them
had a gun.  I was most powerful sick, and slunk to a chair and set down.
They was setting around, some of them talking a little, in a low voice,
and all of them fidgety and uneasy, but trying to look like they warn't;
but I knowed they was, because they was always taking off their hats, and
putting them on, and scratching their heads, and changing their seats,
and fumbling with their buttons.  I warn't easy myself, but I didn't take
my hat off, all the same.

I did wish Aunt Sally would come, and get done with me, and lick me, if
she wanted to, and let me get away and tell Tom how we'd overdone this
thing, and what a thundering hornet's-nest we'd got ourselves into, so we
could stop fooling around straight off, and clear out with Jim before
these rips got out of patience and come for us.

At last she come and begun to ask me questions, but I COULDN'T answer
them straight, I didn't know which end of me was up; because these men
was in such a fidget now that some was wanting to start right NOW and lay
for them desperadoes, and saying it warn't but a few minutes to midnight;
and others was trying to get them to hold on and wait for the
sheep-signal; and here was Aunty pegging away at the questions, and me
a-shaking all over and ready to sink down in my tracks I was that scared;
and the place getting hotter and hotter, and the butter beginning to melt
and run down my neck and behind my ears; and pretty soon, when one of
them says, "I'M for going and getting in the cabin FIRST and right NOW,
and catching them when they come," I most dropped; and a streak of butter
come a-trickling down my forehead, and Aunt Sally she see it, and turns
white as a sheet, and says:

"For the land's sake, what IS the matter with the child?  He's got the
brain-fever as shore as you're born, and they're oozing out!"

And everybody runs to see, and she snatches off my hat, and out comes the
bread and what was left of the butter, and she grabbed me, and hugged me,
and says:

"Oh, what a turn you did give me! and how glad and grateful I am it ain't
no worse; for luck's against us, and it never rains but it pours, and
when I see that truck I thought we'd lost you, for I knowed by the color
and all it was just like your brains would be if--Dear, dear, whyd'nt you
TELL me that was what you'd been down there for, I wouldn't a cared.  Now
cler out to bed, and don't lemme see no more of you till morning!"

I was up stairs in a second, and down the lightning-rod in another one,
and shinning through the dark for the lean-to.  I couldn't hardly get my
words out, I was so anxious; but I told Tom as quick as I could we must
jump for it now, and not a minute to lose--the house full of men, yonder,
with guns!

His eyes just blazed; and he says:

"No!--is that so?  AIN'T it bully!  Why, Huck, if it was to do over
again, I bet I could fetch two hundred!  If we could put it off till--"

"Hurry!  HURRY!"  I says.  "Where's Jim?"

"Right at your elbow; if you reach out your arm you can touch him.  He's
dressed, and everything's ready.  Now we'll slide out and give the
sheep-signal."

But then we heard the tramp of men coming to the door, and heard them
begin to fumble with the pad-lock, and heard a man say:

"I TOLD you we'd be too soon; they haven't come--the door is locked.
Here, I'll lock some of you into the cabin, and you lay for 'em in the
dark and kill 'em when they come; and the rest scatter around a piece,
and listen if you can hear 'em coming."

So in they come, but couldn't see us in the dark, and most trod on us
whilst we was hustling to get under the bed.  But we got under all right,
and out through the hole, swift but soft--Jim first, me next, and Tom
last, which was according to Tom's orders.  Now we was in the lean-to,
and heard trampings close by outside.  So we crept to the door, and Tom
stopped us there and put his eye to the crack, but couldn't make out
nothing, it was so dark; and whispered and said he would listen for the
steps to get further, and when he nudged us Jim must glide out first, and
him last.  So he set his ear to the crack and listened, and listened, and
listened, and the steps a-scraping around out there all the time; and at
last he nudged us, and we slid out, and stooped down, not breathing, and
not making the least noise, and slipped stealthy towards the fence in
Injun file, and got to it all right, and me and Jim over it; but Tom's
britches catched fast on a splinter on the top rail, and then he hear the
steps coming, so he had to pull loose, which snapped the splinter and
made a noise; and as he dropped in our tracks and started somebody sings
out:

"Who's that?  Answer, or I'll shoot!"

But we didn't answer; we just unfurled our heels and shoved.  Then there
was a rush, and a BANG, BANG, BANG! and the bullets fairly whizzed around
us! We heard them sing out:

"Here they are!  They've broke for the river!  After 'em, boys, and turn
loose the dogs!"

So here they come, full tilt.  We could hear them because they wore boots
and yelled, but we didn't wear no boots and didn't yell.  We was in the
path to the mill; and when they got pretty close on to us we dodged into
the bush and let them go by, and then dropped in behind them.  They'd had
all the dogs shut up, so they wouldn't scare off the robbers; but by this
time somebody had let them loose, and here they come, making powwow
enough for a million; but they was our dogs; so we stopped in our tracks
till they catched up; and when they see it warn't nobody but us, and no
excitement to offer them, they only just said howdy, and tore right ahead
towards the shouting and clattering; and then we up-steam again, and
whizzed along after them till we was nearly to the mill, and then struck
up through the bush to where my canoe was tied, and hopped in and pulled
for dear life towards the middle of the river, but didn't make no more
noise than we was obleeged to. Then we struck out, easy and comfortable,
for the island where my raft was; and we could hear them yelling and
barking at each other all up and down the bank, till we was so far away
the sounds got dim and died out.  And when we stepped on to the raft I
says:

"NOW, old Jim, you're a free man again, and I bet you won't ever be a
slave no more."

"En a mighty good job it wuz, too, Huck.  It 'uz planned beautiful, en it
'uz done beautiful; en dey ain't NOBODY kin git up a plan dat's mo'
mixed-up en splendid den what dat one wuz."

We was all glad as we could be, but Tom was the gladdest of all because
he had a bullet in the calf of his leg.

When me and Jim heard that we didn't feel so brash as what we did before.
It was hurting him considerable, and bleeding; so we laid him in the
wigwam and tore up one of the duke's shirts for to bandage him, but he
says:

"Gimme the rags; I can do it myself.  Don't stop now; don't fool around
here, and the evasion booming along so handsome; man the sweeps, and set
her loose!  Boys, we done it elegant!--'deed we did.  I wish WE'D a had
the handling of Louis XVI., there wouldn't a been no 'Son of Saint Louis,
ascend to heaven!' wrote down in HIS biography; no, sir, we'd a whooped
him over the BORDER--that's what we'd a done with HIM--and done it just
as slick as nothing at all, too.  Man the sweeps--man the sweeps!"

But me and Jim was consulting--and thinking.  And after we'd thought a
minute, I says:

"Say it, Jim."

So he says:

"Well, den, dis is de way it look to me, Huck.  Ef it wuz HIM dat 'uz
bein' sot free, en one er de boys wuz to git shot, would he say, 'Go on
en save me, nemmine 'bout a doctor f'r to save dis one?'  Is dat like
Mars Tom Sawyer?  Would he say dat?  You BET he wouldn't!  WELL, den, is
JIM gywne to say it?  No, sah--I doan' budge a step out'n dis place 'dout
a DOCTOR, not if it's forty year!"

I knowed he was white inside, and I reckoned he'd say what he did say--so
it was all right now, and I told Tom I was a-going for a doctor.  He
raised considerable row about it, but me and Jim stuck to it and wouldn't
budge; so he was for crawling out and setting the raft loose himself; but
we wouldn't let him.  Then he give us a piece of his mind, but it didn't
do no good.

So when he sees me getting the canoe ready, he says:

"Well, then, if you're bound to go, I'll tell you the way to do when you
get to the village.  Shut the door and blindfold the doctor tight and
fast, and make him swear to be silent as the grave, and put a purse full
of gold in his hand, and then take and lead him all around the back
alleys and everywheres in the dark, and then fetch him here in the canoe,
in a roundabout way amongst the islands, and search him and take his
chalk away from him, and don't give it back to him till you get him back
to the village, or else he will chalk this raft so he can find it again.
It's the way they all do."

So I said I would, and left, and Jim was to hide in the woods when he see
the doctor coming till he was gone again.




CHAPTER XLI.

THE doctor was an old man; a very nice, kind-looking old man when I got
him up.  I told him me and my brother was over on Spanish Island hunting
yesterday afternoon, and camped on a piece of a raft we found, and about
midnight he must a kicked his gun in his dreams, for it went off and shot
him in the leg, and we wanted him to go over there and fix it and not say
nothing about it, nor let anybody know, because we wanted to come home
this evening and surprise the folks.

"Who is your folks?" he says.

"The Phelpses, down yonder."

"Oh," he says.  And after a minute, he says:

"How'd you say he got shot?"

"He had a dream," I says, "and it shot him."

"Singular dream," he says.

So he lit up his lantern, and got his saddle-bags, and we started.  But
when he sees the canoe he didn't like the look of her--said she was big
enough for one, but didn't look pretty safe for two.  I says:

"Oh, you needn't be afeard, sir, she carried the three of us easy
enough."

"What three?"

"Why, me and Sid, and--and--and THE GUNS; that's what I mean."

"Oh," he says.

But he put his foot on the gunnel and rocked her, and shook his head, and
said he reckoned he'd look around for a bigger one.  But they was all
locked and chained; so he took my canoe, and said for me to wait till he
come back, or I could hunt around further, or maybe I better go down home
and get them ready for the surprise if I wanted to.  But I said I didn't;
so I told him just how to find the raft, and then he started.

I struck an idea pretty soon.  I says to myself, spos'n he can't fix that
leg just in three shakes of a sheep's tail, as the saying is? spos'n it
takes him three or four days?  What are we going to do?--lay around there
till he lets the cat out of the bag?  No, sir; I know what I'LL do.  I'll
wait, and when he comes back if he says he's got to go any more I'll get
down there, too, if I swim; and we'll take and tie him, and keep him, and
shove out down the river; and when Tom's done with him we'll give him
what it's worth, or all we got, and then let him get ashore.

So then I crept into a lumber-pile to get some sleep; and next time I
waked up the sun was away up over my head!  I shot out and went for the
doctor's house, but they told me he'd gone away in the night some time or
other, and warn't back yet.  Well, thinks I, that looks powerful bad for
Tom, and I'll dig out for the island right off.  So away I shoved, and
turned the corner, and nearly rammed my head into Uncle Silas's stomach!
He says:

"Why, TOM!  Where you been all this time, you rascal?"

"I hain't been nowheres," I says, "only just hunting for the runaway
nigger--me and Sid."

"Why, where ever did you go?" he says.  "Your aunt's been mighty uneasy."

"She needn't," I says, "because we was all right.  We followed the men
and the dogs, but they outrun us, and we lost them; but we thought we
heard them on the water, so we got a canoe and took out after them and
crossed over, but couldn't find nothing of them; so we cruised along
up-shore till we got kind of tired and beat out; and tied up the canoe
and went to sleep, and never waked up till about an hour ago; then we
paddled over here to hear the news, and Sid's at the post-office to see
what he can hear, and I'm a-branching out to get something to eat for us,
and then we're going home."

So then we went to the post-office to get "Sid"; but just as I
suspicioned, he warn't there; so the old man he got a letter out of the
office, and we waited awhile longer, but Sid didn't come; so the old man
said, come along, let Sid foot it home, or canoe it, when he got done
fooling around--but we would ride.  I couldn't get him to let me stay and
wait for Sid; and he said there warn't no use in it, and I must come
along, and let Aunt Sally see we was all right.

When we got home Aunt Sally was that glad to see me she laughed and cried
both, and hugged me, and give me one of them lickings of hern that don't
amount to shucks, and said she'd serve Sid the same when he come.

And the place was plum full of farmers and farmers' wives, to dinner; and
such another clack a body never heard.  Old Mrs. Hotchkiss was the worst;
her tongue was a-going all the time.  She says:

"Well, Sister Phelps, I've ransacked that-air cabin over, an' I b'lieve
the nigger was crazy.  I says to Sister Damrell--didn't I, Sister
Damrell?--s'I, he's crazy, s'I--them's the very words I said.  You all
hearn me: he's crazy, s'I; everything shows it, s'I.  Look at that-air
grindstone, s'I; want to tell ME't any cretur 't's in his right mind 's a
goin' to scrabble all them crazy things onto a grindstone, s'I?  Here
sich 'n' sich a person busted his heart; 'n' here so 'n' so pegged along
for thirty-seven year, 'n' all that--natcherl son o' Louis somebody, 'n'
sich everlast'n rubbage.  He's plumb crazy, s'I; it's what I says in the
fust place, it's what I says in the middle, 'n' it's what I says last 'n'
all the time--the nigger's crazy--crazy 's Nebokoodneezer, s'I."

"An' look at that-air ladder made out'n rags, Sister Hotchkiss," says old
Mrs. Damrell; "what in the name o' goodness COULD he ever want of--"

"The very words I was a-sayin' no longer ago th'n this minute to Sister
Utterback, 'n' she'll tell you so herself.  Sh-she, look at that-air rag
ladder, sh-she; 'n' s'I, yes, LOOK at it, s'I--what COULD he a-wanted of
it, s'I.  Sh-she, Sister Hotchkiss, sh-she--"

"But how in the nation'd they ever GIT that grindstone IN there, ANYWAY?
'n' who dug that-air HOLE? 'n' who--"

"My very WORDS, Brer Penrod!  I was a-sayin'--pass that-air sasser o'
m'lasses, won't ye?--I was a-sayin' to Sister Dunlap, jist this minute,
how DID they git that grindstone in there, s'I.  Without HELP, mind you
--'thout HELP!  THAT'S wher 'tis.  Don't tell ME, s'I; there WUZ help,
s'I; 'n' ther' wuz a PLENTY help, too, s'I; ther's ben a DOZEN a-helpin'
that nigger, 'n' I lay I'd skin every last nigger on this place but I'D
find out who done it, s'I; 'n' moreover, s'I--"

"A DOZEN says you!--FORTY couldn't a done every thing that's been done.
Look at them case-knife saws and things, how tedious they've been made;
look at that bed-leg sawed off with 'm, a week's work for six men; look
at that nigger made out'n straw on the bed; and look at--"

"You may WELL say it, Brer Hightower!  It's jist as I was a-sayin' to
Brer Phelps, his own self.  S'e, what do YOU think of it, Sister
Hotchkiss, s'e? Think o' what, Brer Phelps, s'I?  Think o' that bed-leg
sawed off that a way, s'e?  THINK of it, s'I?  I lay it never sawed
ITSELF off, s'I--somebody SAWED it, s'I; that's my opinion, take it or
leave it, it mayn't be no 'count, s'I, but sich as 't is, it's my
opinion, s'I, 'n' if any body k'n start a better one, s'I, let him DO it,
s'I, that's all.  I says to Sister Dunlap, s'I--"

"Why, dog my cats, they must a ben a house-full o' niggers in there every
night for four weeks to a done all that work, Sister Phelps.  Look at
that shirt--every last inch of it kivered over with secret African writ'n
done with blood!  Must a ben a raft uv 'm at it right along, all the
time, amost.  Why, I'd give two dollars to have it read to me; 'n' as for
the niggers that wrote it, I 'low I'd take 'n' lash 'm t'll--"

"People to HELP him, Brother Marples!  Well, I reckon you'd THINK so if
you'd a been in this house for a while back.  Why, they've stole
everything they could lay their hands on--and we a-watching all the time,
mind you. They stole that shirt right off o' the line! and as for that
sheet they made the rag ladder out of, ther' ain't no telling how many
times they DIDN'T steal that; and flour, and candles, and candlesticks,
and spoons, and the old warming-pan, and most a thousand things that I
disremember now, and my new calico dress; and me and Silas and my Sid and
Tom on the constant watch day AND night, as I was a-telling you, and not
a one of us could catch hide nor hair nor sight nor sound of them; and
here at the last minute, lo and behold you, they slides right in under
our noses and fools us, and not only fools US but the Injun Territory
robbers too, and actuly gets AWAY with that nigger safe and sound, and
that with sixteen men and twenty-two dogs right on their very heels at
that very time!  I tell you, it just bangs anything I ever HEARD of.
Why, SPERITS couldn't a done better and been no smarter. And I reckon
they must a BEEN sperits--because, YOU know our dogs, and ther' ain't no
better; well, them dogs never even got on the TRACK of 'm once!  You
explain THAT to me if you can!--ANY of you!"

"Well, it does beat--"

"Laws alive, I never--"

"So help me, I wouldn't a be--"

"HOUSE-thieves as well as--"

"Goodnessgracioussakes, I'd a ben afeard to live in sich a--"

"'Fraid to LIVE!--why, I was that scared I dasn't hardly go to bed, or
get up, or lay down, or SET down, Sister Ridgeway.  Why, they'd steal the
very--why, goodness sakes, you can guess what kind of a fluster I was
in by the time midnight come last night.  I hope to gracious if I warn't
afraid they'd steal some o' the family!  I was just to that pass I didn't
have no reasoning faculties no more.  It looks foolish enough NOW, in the
daytime; but I says to myself, there's my two poor boys asleep, 'way up
stairs in that lonesome room, and I declare to goodness I was that uneasy
't I crep' up there and locked 'em in!  I DID.  And anybody would.
Because, you know, when you get scared that way, and it keeps running on,
and getting worse and worse all the time, and your wits gets to addling,
and you get to doing all sorts o' wild things, and by and by you think to
yourself, spos'n I was a boy, and was away up there, and the door ain't
locked, and you--" She stopped, looking kind of wondering, and then she
turned her head around slow, and when her eye lit on me--I got up and
took a walk.

Says I to myself, I can explain better how we come to not be in that room
this morning if I go out to one side and study over it a little.  So I
done it.  But I dasn't go fur, or she'd a sent for me.  And when it was
late in the day the people all went, and then I come in and told her the
noise and shooting waked up me and "Sid," and the door was locked, and we
wanted to see the fun, so we went down the lightning-rod, and both of us
got hurt a little, and we didn't never want to try THAT no more.  And
then I went on and told her all what I told Uncle Silas before; and then
she said she'd forgive us, and maybe it was all right enough anyway, and
about what a body might expect of boys, for all boys was a pretty
harum-scarum lot as fur as she could see; and so, as long as no harm
hadn't come of it, she judged she better put in her time being grateful
we was alive and well and she had us still, stead of fretting over what
was past and done.  So then she kissed me, and patted me on the head, and
dropped into a kind of a brown study; and pretty soon jumps up, and says:

"Why, lawsamercy, it's most night, and Sid not come yet!  What HAS become
of that boy?"

I see my chance; so I skips up and says:

"I'll run right up to town and get him," I says.

"No you won't," she says.  "You'll stay right wher' you are; ONE'S enough
to be lost at a time.  If he ain't here to supper, your uncle 'll go."

Well, he warn't there to supper; so right after supper uncle went.

He come back about ten a little bit uneasy; hadn't run across Tom's
track. Aunt Sally was a good DEAL uneasy; but Uncle Silas he said there
warn't no occasion to be--boys will be boys, he said, and you'll see this
one turn up in the morning all sound and right.  So she had to be
satisfied.  But she said she'd set up for him a while anyway, and keep a
light burning so he could see it.

And then when I went up to bed she come up with me and fetched her
candle, and tucked me in, and mothered me so good I felt mean, and like I
couldn't look her in the face; and she set down on the bed and talked
with me a long time, and said what a splendid boy Sid was, and didn't
seem to want to ever stop talking about him; and kept asking me every now
and then if I reckoned he could a got lost, or hurt, or maybe drownded,
and might be laying at this minute somewheres suffering or dead, and she
not by him to help him, and so the tears would drip down silent, and I
would tell her that Sid was all right, and would be home in the morning,
sure; and she would squeeze my hand, or maybe kiss me, and tell me to say
it again, and keep on saying it, because it done her good, and she was in
so much trouble.  And when she was going away she looked down in my eyes
so steady and gentle, and says:

"The door ain't going to be locked, Tom, and there's the window and the
rod; but you'll be good, WON'T you?  And you won't go?  For MY sake."

Laws knows I WANTED to go bad enough to see about Tom, and was all
intending to go; but after that I wouldn't a went, not for kingdoms.

But she was on my mind and Tom was on my mind, so I slept very restless.
And twice I went down the rod away in the night, and slipped around
front, and see her setting there by her candle in the window with her
eyes towards the road and the tears in them; and I wished I could do
something for her, but I couldn't, only to swear that I wouldn't never do
nothing to grieve her any more.  And the third time I waked up at dawn,
and slid down, and she was there yet, and her candle was most out, and
her old gray head was resting on her hand, and she was asleep.




CHAPTER XLII.

THE old man was uptown again before breakfast, but couldn't get no track
of Tom; and both of them set at the table thinking, and not saying
nothing, and looking mournful, and their coffee getting cold, and not
eating anything. And by and by the old man says:

"Did I give you the letter?"

"What letter?"

"The one I got yesterday out of the post-office."

"No, you didn't give me no letter."

"Well, I must a forgot it."

So he rummaged his pockets, and then went off somewheres where he had
laid it down, and fetched it, and give it to her.  She says:

"Why, it's from St. Petersburg--it's from Sis."

I allowed another walk would do me good; but I couldn't stir.  But before
she could break it open she dropped it and run--for she see something.
And so did I. It was Tom Sawyer on a mattress; and that old doctor; and
Jim, in HER calico dress, with his hands tied behind him; and a lot of
people.  I hid the letter behind the first thing that come handy, and
rushed.  She flung herself at Tom, crying, and says:

"Oh, he's dead, he's dead, I know he's dead!"

And Tom he turned his head a little, and muttered something or other,
which showed he warn't in his right mind; then she flung up her hands,
and says:

"He's alive, thank God!  And that's enough!" and she snatched a kiss of
him, and flew for the house to get the bed ready, and scattering orders
right and left at the niggers and everybody else, as fast as her tongue
could go, every jump of the way.

I followed the men to see what they was going to do with Jim; and the old
doctor and Uncle Silas followed after Tom into the house.  The men was
very huffy, and some of them wanted to hang Jim for an example to all the
other niggers around there, so they wouldn't be trying to run away like
Jim done, and making such a raft of trouble, and keeping a whole family
scared most to death for days and nights.  But the others said, don't do
it, it wouldn't answer at all; he ain't our nigger, and his owner would
turn up and make us pay for him, sure.  So that cooled them down a
little, because the people that's always the most anxious for to hang a
nigger that hain't done just right is always the very ones that ain't the
most anxious to pay for him when they've got their satisfaction out of
him.

They cussed Jim considerble, though, and give him a cuff or two side the
head once in a while, but Jim never said nothing, and he never let on to
know me, and they took him to the same cabin, and put his own clothes on
him, and chained him again, and not to no bed-leg this time, but to a big
staple drove into the bottom log, and chained his hands, too, and both
legs, and said he warn't to have nothing but bread and water to eat after
this till his owner come, or he was sold at auction because he didn't
come in a certain length of time, and filled up our hole, and said a
couple of farmers with guns must stand watch around about the cabin every
night, and a bulldog tied to the door in the daytime; and about this time
they was through with the job and was tapering off with a kind of generl
good-bye cussing, and then the old doctor comes and takes a look, and
says:

"Don't be no rougher on him than you're obleeged to, because he ain't a
bad nigger.  When I got to where I found the boy I see I couldn't cut the
bullet out without some help, and he warn't in no condition for me to
leave to go and get help; and he got a little worse and a little worse,
and after a long time he went out of his head, and wouldn't let me come
a-nigh him any more, and said if I chalked his raft he'd kill me, and no
end of wild foolishness like that, and I see I couldn't do anything at
all with him; so I says, I got to have HELP somehow; and the minute I
says it out crawls this nigger from somewheres and says he'll help, and
he done it, too, and done it very well.  Of course I judged he must be a
runaway nigger, and there I WAS! and there I had to stick right straight
along all the rest of the day and all night.  It was a fix, I tell you!
I had a couple of patients with the chills, and of course I'd of liked to
run up to town and see them, but I dasn't, because the nigger might get
away, and then I'd be to blame; and yet never a skiff come close enough
for me to hail.  So there I had to stick plumb until daylight this
morning; and I never see a nigger that was a better nuss or faithfuller,
and yet he was risking his freedom to do it, and was all tired out, too,
and I see plain enough he'd been worked main hard lately.  I liked the
nigger for that; I tell you, gentlemen, a nigger like that is worth a
thousand dollars--and kind treatment, too.  I had everything I needed,
and the boy was doing as well there as he would a done at home--better,
maybe, because it was so quiet; but there I WAS, with both of 'm on my
hands, and there I had to stick till about dawn this morning; then some
men in a skiff come by, and as good luck would have it the nigger was
setting by the pallet with his head propped on his knees sound asleep; so
I motioned them in quiet, and they slipped up on him and grabbed him and
tied him before he knowed what he was about, and we never had no trouble.
And the boy being in a kind of a flighty sleep, too, we muffled the oars
and hitched the raft on, and towed her over very nice and quiet, and the
nigger never made the least row nor said a word from the start.  He ain't
no bad nigger, gentlemen; that's what I think about him."

Somebody says:

"Well, it sounds very good, doctor, I'm obleeged to say."

Then the others softened up a little, too, and I was mighty thankful to
that old doctor for doing Jim that good turn; and I was glad it was
according to my judgment of him, too; because I thought he had a good
heart in him and was a good man the first time I see him.  Then they all
agreed that Jim had acted very well, and was deserving to have some
notice took of it, and reward.  So every one of them promised, right out
and hearty, that they wouldn't cuss him no more.

Then they come out and locked him up.  I hoped they was going to say he
could have one or two of the chains took off, because they was rotten
heavy, or could have meat and greens with his bread and water; but they
didn't think of it, and I reckoned it warn't best for me to mix in, but I
judged I'd get the doctor's yarn to Aunt Sally somehow or other as soon
as I'd got through the breakers that was laying just ahead of me
--explanations, I mean, of how I forgot to mention about Sid being shot
when I was telling how him and me put in that dratted night paddling
around hunting the runaway nigger.

But I had plenty time.  Aunt Sally she stuck to the sick-room all day and
all night, and every time I see Uncle Silas mooning around I dodged him.

Next morning I heard Tom was a good deal better, and they said Aunt Sally
was gone to get a nap.  So I slips to the sick-room, and if I found him
awake I reckoned we could put up a yarn for the family that would wash.
But he was sleeping, and sleeping very peaceful, too; and pale, not
fire-faced the way he was when he come.  So I set down and laid for him
to wake.  In about half an hour Aunt Sally comes gliding in, and there I
was, up a stump again!  She motioned me to be still, and set down by me,
and begun to whisper, and said we could all be joyful now, because all
the symptoms was first-rate, and he'd been sleeping like that for ever so
long, and looking better and peacefuller all the time, and ten to one
he'd wake up in his right mind.

So we set there watching, and by and by he stirs a bit, and opened his
eyes very natural, and takes a look, and says:

"Hello!--why, I'm at HOME!  How's that?  Where's the raft?"

"It's all right," I says.

"And JIM?"

"The same," I says, but couldn't say it pretty brash.  But he never
noticed, but says:

"Good!  Splendid!  NOW we're all right and safe! Did you tell Aunty?"

I was going to say yes; but she chipped in and says:  "About what, Sid?"

"Why, about the way the whole thing was done."

"What whole thing?"

"Why, THE whole thing.  There ain't but one; how we set the runaway
nigger free--me and Tom."

"Good land!  Set the run--What IS the child talking about!  Dear, dear,
out of his head again!"

"NO, I ain't out of my HEAD; I know all what I'm talking about.  We DID
set him free--me and Tom.  We laid out to do it, and we DONE it.  And we
done it elegant, too."  He'd got a start, and she never checked him up,
just set and stared and stared, and let him clip along, and I see it
warn't no use for ME to put in.  "Why, Aunty, it cost us a power of work
--weeks of it--hours and hours, every night, whilst you was all asleep.
And we had to steal candles, and the sheet, and the shirt, and your
dress, and spoons, and tin plates, and case-knives, and the warming-pan,
and the grindstone, and flour, and just no end of things, and you can't
think what work it was to make the saws, and pens, and inscriptions, and
one thing or another, and you can't think HALF the fun it was.  And we
had to make up the pictures of coffins and things, and nonnamous letters
from the robbers, and get up and down the lightning-rod, and dig the hole
into the cabin, and made the rope ladder and send it in cooked up in a
pie, and send in spoons and things to work with in your apron pocket--"

"Mercy sakes!"

"--and load up the cabin with rats and snakes and so on, for company for
Jim; and then you kept Tom here so long with the butter in his hat that
you come near spiling the whole business, because the men come before we
was out of the cabin, and we had to rush, and they heard us and let drive
at us, and I got my share, and we dodged out of the path and let them go
by, and when the dogs come they warn't interested in us, but went for the
most noise, and we got our canoe, and made for the raft, and was all
safe, and Jim was a free man, and we done it all by ourselves, and WASN'T
it bully, Aunty!"

"Well, I never heard the likes of it in all my born days!  So it was YOU,
you little rapscallions, that's been making all this trouble, and turned
everybody's wits clean inside out and scared us all most to death.  I've
as good a notion as ever I had in my life to take it out o' you this very
minute.  To think, here I've been, night after night, a--YOU just get
well once, you young scamp, and I lay I'll tan the Old Harry out o' both
o' ye!"

But Tom, he WAS so proud and joyful, he just COULDN'T hold in, and his
tongue just WENT it--she a-chipping in, and spitting fire all along, and
both of them going it at once, like a cat convention; and she says:

"WELL, you get all the enjoyment you can out of it NOW, for mind I tell
you if I catch you meddling with him again--"

"Meddling with WHO?"  Tom says, dropping his smile and looking surprised.

"With WHO?  Why, the runaway nigger, of course.  Who'd you reckon?"

Tom looks at me very grave, and says:

"Tom, didn't you just tell me he was all right?  Hasn't he got away?"

"HIM?" says Aunt Sally; "the runaway nigger?  'Deed he hasn't.  They've
got him back, safe and sound, and he's in that cabin again, on bread and
water, and loaded down with chains, till he's claimed or sold!"

Tom rose square up in bed, with his eye hot, and his nostrils opening and
shutting like gills, and sings out to me:

"They hain't no RIGHT to shut him up!  SHOVE!--and don't you lose a
minute.  Turn him loose! he ain't no slave; he's as free as any cretur
that walks this earth!"

"What DOES the child mean?"

"I mean every word I SAY, Aunt Sally, and if somebody don't go, I'LL go.
I've knowed him all his life, and so has Tom, there.  Old Miss Watson
died two months ago, and she was ashamed she ever was going to sell him
down the river, and SAID so; and she set him free in her will."

"Then what on earth did YOU want to set him free for, seeing he was
already free?"

"Well, that IS a question, I must say; and just like women!  Why, I
wanted the ADVENTURE of it; and I'd a waded neck-deep in blood to
--goodness alive, AUNT POLLY!"

If she warn't standing right there, just inside the door, looking as
sweet and contented as an angel half full of pie, I wish I may never!

Aunt Sally jumped for her, and most hugged the head off of her, and cried
over her, and I found a good enough place for me under the bed, for it
was getting pretty sultry for us, seemed to me.  And I peeped out, and in
a little while Tom's Aunt Polly shook herself loose and stood there
looking across at Tom over her spectacles--kind of grinding him into the
earth, you know.  And then she says:

"Yes, you BETTER turn y'r head away--I would if I was you, Tom."

"Oh, deary me!" says Aunt Sally; "IS he changed so?  Why, that ain't TOM,
it's Sid; Tom's--Tom's--why, where is Tom?  He was here a minute ago."

"You mean where's Huck FINN--that's what you mean!  I reckon I hain't
raised such a scamp as my Tom all these years not to know him when I SEE
him.  That WOULD be a pretty howdy-do.  Come out from under that bed,
Huck Finn."

So I done it.  But not feeling brash.

Aunt Sally she was one of the mixed-upest-looking persons I ever see
--except one, and that was Uncle Silas, when he come in and they told it
all to him.  It kind of made him drunk, as you may say, and he didn't
know nothing at all the rest of the day, and preached a prayer-meeting
sermon that night that gave him a rattling ruputation, because the oldest
man in the world couldn't a understood it.  So Tom's Aunt Polly, she told
all about who I was, and what; and I had to up and tell how I was in such
a tight place that when Mrs. Phelps took me for Tom Sawyer--she chipped
in and says, "Oh, go on and call me Aunt Sally, I'm used to it now, and
'tain't no need to change"--that when Aunt Sally took me for Tom Sawyer I
had to stand it--there warn't no other way, and I knowed he wouldn't
mind, because it would be nuts for him, being a mystery, and he'd make an
adventure out of it, and be perfectly satisfied.  And so it turned out,
and he let on to be Sid, and made things as soft as he could for me.

And his Aunt Polly she said Tom was right about old Miss Watson setting
Jim free in her will; and so, sure enough, Tom Sawyer had gone and took
all that trouble and bother to set a free nigger free! and I couldn't
ever understand before, until that minute and that talk, how he COULD
help a body set a nigger free with his bringing-up.

Well, Aunt Polly she said that when Aunt Sally wrote to her that Tom and
SID had come all right and safe, she says to herself:

"Look at that, now!  I might have expected it, letting him go off that
way without anybody to watch him.  So now I got to go and trapse all the
way down the river, eleven hundred mile, and find out what that creetur's
up to THIS time, as long as I couldn't seem to get any answer out of you
about it."

"Why, I never heard nothing from you," says Aunt Sally.

"Well, I wonder!  Why, I wrote you twice to ask you what you could mean
by Sid being here."

"Well, I never got 'em, Sis."

Aunt Polly she turns around slow and severe, and says:

"You, Tom!"

"Well--WHAT?" he says, kind of pettish.

"Don t you what ME, you impudent thing--hand out them letters."

"What letters?"

"THEM letters.  I be bound, if I have to take a-holt of you I'll--"

"They're in the trunk.  There, now.  And they're just the same as they
was when I got them out of the office.  I hain't looked into them, I
hain't touched them.  But I knowed they'd make trouble, and I thought if
you warn't in no hurry, I'd--"

"Well, you DO need skinning, there ain't no mistake about it.  And I
wrote another one to tell you I was coming; and I s'pose he--"

"No, it come yesterday; I hain't read it yet, but IT'S all right, I've
got that one."

I wanted to offer to bet two dollars she hadn't, but I reckoned maybe it
was just as safe to not to.  So I never said nothing.




CHAPTER THE LAST

THE first time I catched Tom private I asked him what was his idea, time
of the evasion?--what it was he'd planned to do if the evasion worked all
right and he managed to set a nigger free that was already free before?
And he said, what he had planned in his head from the start, if we got
Jim out all safe, was for us to run him down the river on the raft, and
have adventures plumb to the mouth of the river, and then tell him about
his being free, and take him back up home on a steamboat, in style, and
pay him for his lost time, and write word ahead and get out all the
niggers around, and have them waltz him into town with a torchlight
procession and a brass-band, and then he would be a hero, and so would
we.  But I reckoned it was about as well the way it was.

We had Jim out of the chains in no time, and when Aunt Polly and Uncle
Silas and Aunt Sally found out how good he helped the doctor nurse Tom,
they made a heap of fuss over him, and fixed him up prime, and give him
all he wanted to eat, and a good time, and nothing to do.  And we had him
up to the sick-room, and had a high talk; and Tom give Jim forty dollars
for being prisoner for us so patient, and doing it up so good, and Jim
was pleased most to death, and busted out, and says:

"DAH, now, Huck, what I tell you?--what I tell you up dah on Jackson
islan'?  I TOLE you I got a hairy breas', en what's de sign un it; en I
TOLE you I ben rich wunst, en gwineter to be rich AGIN; en it's come
true; en heah she is!  DAH, now! doan' talk to ME--signs is SIGNS, mine I
tell you; en I knowed jis' 's well 'at I 'uz gwineter be rich agin as I's
a-stannin' heah dis minute!"

And then Tom he talked along and talked along, and says, le's all three
slide out of here one of these nights and get an outfit, and go for
howling adventures amongst the Injuns, over in the Territory, for a
couple of weeks or two; and I says, all right, that suits me, but I ain't
got no money for to buy the outfit, and I reckon I couldn't get none from
home, because it's likely pap's been back before now, and got it all away
from Judge Thatcher and drunk it up.

"No, he hain't," Tom says; "it's all there yet--six thousand dollars and
more; and your pap hain't ever been back since.  Hadn't when I come away,
anyhow."

Jim says, kind of solemn:

"He ain't a-comin' back no mo', Huck."

I says:

"Why, Jim?"

"Nemmine why, Huck--but he ain't comin' back no mo."

But I kept at him; so at last he says:

"Doan' you 'member de house dat was float'n down de river, en dey wuz a
man in dah, kivered up, en I went in en unkivered him and didn' let you
come in?  Well, den, you kin git yo' money when you wants it, kase dat
wuz him."

Tom's most well now, and got his bullet around his neck on a watch-guard
for a watch, and is always seeing what time it is, and so there ain't
nothing more to write about, and I am rotten glad of it, because if I'd a
knowed what a trouble it was to make a book I wouldn't a tackled it, and
ain't a-going to no more.  But I reckon I got to light out for the
Territory ahead of the rest, because Aunt Sally she's going to adopt me
and sivilize me, and I can't stand it.  I been there before.

THE END. YOURS TRULY, HUCK FINN.




End of the Project Gutenberg EBook of Adventures of Huckleberry Finn,
Complete, by Mark Twain (Samuel Clemens)

*** END OF THIS PROJECT GUTENBERG EBOOK HUCKLEBERRY FINN ***

***** This file should be named 76.txt or 76.zip *****
This and all associated files of various formats will be found in:
        http://www.gutenberg.net/7/76/

Produced by David Widger. Previous editions produced by Ron Burkey
and Internet Wiretap


Updated editions will replace the previous one--the old editions
will be renamed.

Creating the works from public domain print editions means that no
one owns a United States copyright in these works, so the Foundation
(and you!) can copy and distribute it in the United States without
permission and without paying copyright royalties.  Special rules,
set forth in the General Terms of Use part of this license, apply to
copying and distributing Project Gutenberg-tm electronic works to
protect the PROJECT GUTENBERG-tm concept and trademark.  Project
Gutenberg is a registered trademark, and may not be used if you
charge for the eBooks, unless you receive specific permission.  If you
do not charge anything for copies of this eBook, complying with the
rules is very easy.  You may use this eBook for nearly any purpose
such as creation of derivative works, reports, performances and
research.  They may be modified and printed and given away--you may do
practically ANYTHING with public domain eBooks.  Redistribution is
subject to the trademark license, especially commercial
redistribution.



*** START: FULL LICENSE ***

THE FULL PROJECT GUTENBERG LICENSE
PLEASE READ THIS BEFORE YOU DISTRIBUTE OR USE THIS WORK

To protect the Project Gutenberg-tm mission of promoting the free
distribution of electronic works, by using or distributing this work
(or any other work associated in any way with the phrase "Project
Gutenberg"), you agree to comply with all the terms of the Full Project
Gutenberg-tm License (available with this file or online at
http://gutenberg.net/license).


Section 1.  General Terms of Use and Redistributing Project Gutenberg-tm
electronic works

1.A.  By reading or using any part of this Project Gutenberg-tm
electronic work, you indicate that you have read, understand, agree to
and accept all the terms of this license and intellectual property
(trademark/copyright) agreement.  If you do not agree to abide by all
the terms of this agreement, you must cease using and return or destroy
all copies of Project Gutenberg-tm electronic works in your possession.
If you paid a fee for obtaining a copy of or access to a Project
Gutenberg-tm electronic work and you do not agree to be bound by the
terms of this agreement, you may obtain a refund from the person or
entity to whom you paid the fee as set forth in paragraph 1.E.8.

1.B.  "Project Gutenberg" is a registered trademark.  It may only be
used on or associated in any way with an electronic work by people who
agree to be bound by the terms of this agreement.  There are a few
things that you can do with most Project Gutenberg-tm electronic works
even without complying with the full terms of this agreement.  See
paragraph 1.C below.  There are a lot of things you can do with Project
Gutenberg-tm electronic works if you follow the terms of this agreement
and help preserve free future access to Project Gutenberg-tm electronic
works.  See paragraph 1.E below.

1.C.  The Project Gutenberg Literary Archive Foundation ("the Foundation"
or PGLAF), owns a compilation copyright in the collection of Project
Gutenberg-tm electronic works.  Nearly all the individual works in the
collection are in the public domain in the United States.  If an
individual work is in the public domain in the United States and you are
located in the United States, we do not claim a right to prevent you from
copying, distributing, performing, displaying or creating derivative
works based on the work as long as all references to Project Gutenberg
are removed.  Of course, we hope that you will support the Project
Gutenberg-tm mission of promoting free access to electronic works by
freely sharing Project Gutenberg-tm works in compliance with the terms of
this agreement for keeping the Project Gutenberg-tm name associated with
the work.  You can easily comply with the terms of this agreement by
keeping this work in the same format with its attached full Project
Gutenberg-tm License when you share it without charge with others.

1.D.  The copyright laws of the place where you are located also govern
what you can do with this work.  Copyright laws in most countries are in
a constant state of change.  If you are outside the United States, check
the laws of your country in addition to the terms of this agreement
before downloading, copying, displaying, performing, distributing or
creating derivative works based on this work or any other Project
Gutenberg-tm work.  The Foundation makes no representations concerning
the copyright status of any work in any country outside the United
States.

1.E.  Unless you have removed all references to Project Gutenberg:

1.E.1.  The following sentence, with active links to, or other immediate
access to, the full Project Gutenberg-tm License must appear prominently
whenever any copy of a Project Gutenberg-tm work (any work on which the
phrase "Project Gutenberg" appears, or with which the phrase "Project
Gutenberg" is associated) is accessed, displayed, performed, viewed,
copied or distributed:

This eBook is for the use of anyone anywhere at no cost and with
almost no restrictions whatsoever.  You may copy it, give it away or
re-use it under the terms of the Project Gutenberg License included
with this eBook or online at www.gutenberg.net

1.E.2.  If an individual Project Gutenberg-tm electronic work is derived
from the public domain (does not contain a notice indicating that it is
posted with permission of the copyright holder), the work can be copied
and distributed to anyone in the United States without paying any fees
or charges.  If you are redistributing or providing access to a work
with the phrase "Project Gutenberg" associated with or appearing on the
work, you must comply either with the requirements of paragraphs 1.E.1
through 1.E.7 or obtain permission for the use of the work and the
Project Gutenberg-tm trademark as set forth in paragraphs 1.E.8 or
1.E.9.

1.E.3.  If an individual Project Gutenberg-tm electronic work is posted
with the permission of the copyright holder, your use and distribution
must comply with both paragraphs 1.E.1 through 1.E.7 and any additional
terms imposed by the copyright holder.  Additional terms will be linked
to the Project Gutenberg-tm License for all works posted with the
permission of the copyright holder found at the beginning of this work.

1.E.4.  Do not unlink or detach or remove the full Project Gutenberg-tm
License terms from this work, or any files containing a part of this
work or any other work associated with Project Gutenberg-tm.

1.E.5.  Do not copy, display, perform, distribute or redistribute this
electronic work, or any part of this electronic work, without
prominently displaying the sentence set forth in paragraph 1.E.1 with
active links or immediate access to the full terms of the Project
Gutenberg-tm License.

1.E.6.  You may convert to and distribute this work in any binary,
compressed, marked up, nonproprietary or proprietary form, including any
word processing or hypertext form.  However, if you provide access to or
distribute copies of a Project Gutenberg-tm work in a format other than
"Plain Vanilla ASCII" or other format used in the official version
posted on the official Project Gutenberg-tm web site (www.gutenberg.net),
you must, at no additional cost, fee or expense to the user, provide a
copy, a means of exporting a copy, or a means of obtaining a copy upon
request, of the work in its original "Plain Vanilla ASCII" or other
form.  Any alternate format must include the full Project Gutenberg-tm
License as specified in paragraph 1.E.1.

1.E.7.  Do not charge a fee for access to, viewing, displaying,
performing, copying or distributing any Project Gutenberg-tm works
unless you comply with paragraph 1.E.8 or 1.E.9.

1.E.8.  You may charge a reasonable fee for copies of or providing
access to or distributing Project Gutenberg-tm electronic works provided
that

- You pay a royalty fee of 20% of the gross profits you derive from
     the use of Project Gutenberg-tm works calculated using the method
     you already use to calculate your applicable taxes.  The fee is
     owed to the owner of the Project Gutenberg-tm trademark, but he
     has agreed to donate royalties under this paragraph to the
     Project Gutenberg Literary Archive Foundation.  Royalty payments
     must be paid within 60 days following each date on which you
     prepare (or are legally required to prepare) your periodic tax
     returns.  Royalty payments should be clearly marked as such and
     sent to the Project Gutenberg Literary Archive Foundation at the
     address specified in Section 4, "Information about donations to
     the Project Gutenberg Literary Archive Foundation."

- You provide a full refund of any money paid by a user who notifies
     you in writing (or by e-mail) within 30 days of receipt that s/he
     does not agree to the terms of the full Project Gutenberg-tm
     License.  You must require such a user to return or
     destroy all copies of the works possessed in a physical medium
     and discontinue all use of and all access to other copies of
     Project Gutenberg-tm works.

- You provide, in accordance with paragraph 1.F.3, a full refund of any
     money paid for a work or a replacement copy, if a defect in the
     electronic work is discovered and reported to you within 90 days
     of receipt of the work.

- You comply with all other terms of this agreement for free
     distribution of Project Gutenberg-tm works.

1.E.9.  If you wish to charge a fee or distribute a Project Gutenberg-tm
electronic work or group of works on different terms than are set
forth in this agreement, you must obtain permission in writing from
both the Project Gutenberg Literary Archive Foundation and Michael
Hart, the owner of the Project Gutenberg-tm trademark.  Contact the
Foundation as set forth in Section 3 below.

1.F.

1.F.1.  Project Gutenberg volunteers and employees expend considerable
effort to identify, do copyright research on, transcribe and proofread
public domain works in creating the Project Gutenberg-tm
collection.  Despite these efforts, Project Gutenberg-tm electronic
works, and the medium on which they may be stored, may contain
"Defects," such as, but not limited to, incomplete, inaccurate or
corrupt data, transcription errors, a copyright or other intellectual
property infringement, a defective or damaged disk or other medium, a
computer virus, or computer codes that damage or cannot be read by
your equipment.

1.F.2.  LIMITED WARRANTY, DISCLAIMER OF DAMAGES - Except for the "Right
of Replacement or Refund" described in paragraph 1.F.3, the Project
Gutenberg Literary Archive Foundation, the owner of the Project
Gutenberg-tm trademark, and any other party distributing a Project
Gutenberg-tm electronic work under this agreement, disclaim all
liability to you for damages, costs and expenses, including legal
fees.  YOU AGREE THAT YOU HAVE NO REMEDIES FOR NEGLIGENCE, STRICT
LIABILITY, BREACH OF WARRANTY OR BREACH OF CONTRACT EXCEPT THOSE
PROVIDED IN PARAGRAPH F3.  YOU AGREE THAT THE FOUNDATION, THE
TRADEMARK OWNER, AND ANY DISTRIBUTOR UNDER THIS AGREEMENT WILL NOT BE
LIABLE TO YOU FOR ACTUAL, DIRECT, INDIRECT, CONSEQUENTIAL, PUNITIVE OR
INCIDENTAL DAMAGES EVEN IF YOU GIVE NOTICE OF THE POSSIBILITY OF SUCH
DAMAGE.

1.F.3.  LIMITED RIGHT OF REPLACEMENT OR REFUND - If you discover a
defect in this electronic work within 90 days of receiving it, you can
receive a refund of the money (if any) you paid for it by sending a
written explanation to the person you received the work from.  If you
received the work on a physical medium, you must return the medium with
your written explanation.  The person or entity that provided you with
the defective work may elect to provide a replacement copy in lieu of a
refund.  If you received the work electronically, the person or entity
providing it to you may choose to give you a second opportunity to
receive the work electronically in lieu of a refund.  If the second copy
is also defective, you may demand a refund in writing without further
opportunities to fix the problem.

1.F.4.  Except for the limited right of replacement or refund set forth
in paragraph 1.F.3, this work is provided to you 'AS-IS' WITH NO OTHER
WARRANTIES OF ANY KIND, EXPRESS OR IMPLIED, INCLUDING BUT NOT LIMITED TO
WARRANTIES OF MERCHANTIBILITY OR FITNESS FOR ANY PURPOSE.

1.F.5.  Some states do not allow disclaimers of certain implied
warranties or the exclusion or limitation of certain types of damages.
If any disclaimer or limitation set forth in this agreement violates the
law of the state applicable to this agreement, the agreement shall be
interpreted to make the maximum disclaimer or limitation permitted by
the applicable state law.  The invalidity or unenforceability of any
provision of this agreement shall not void the remaining provisions.

1.F.6.  INDEMNITY - You agree to indemnify and hold the Foundation, the
trademark owner, any agent or employee of the Foundation, anyone
providing copies of Project Gutenberg-tm electronic works in accordance
with this agreement, and any volunteers associated with the production,
promotion and distribution of Project Gutenberg-tm electronic works,
harmless from all liability, costs and expenses, including legal fees,
that arise directly or indirectly from any of the following which you do
or cause to occur: (a) distribution of this or any Project Gutenberg-tm
work, (b) alteration, modification, or additions or deletions to any
Project Gutenberg-tm work, and (c) any Defect you cause.


Section  2.  Information about the Mission of Project Gutenberg-tm

Project Gutenberg-tm is synonymous with the free distribution of
electronic works in formats readable by the widest variety of computers
including obsolete, old, middle-aged and new computers.  It exists
because of the efforts of hundreds of volunteers and donations from
people in all walks of life.

Volunteers and financial support to provide volunteers with the
assistance they need, is critical to reaching Project Gutenberg-tm's
goals and ensuring that the Project Gutenberg-tm collection will
remain freely available for generations to come.  In 2001, the Project
Gutenberg Literary Archive Foundation was created to provide a secure
and permanent future for Project Gutenberg-tm and future generations.
To learn more about the Project Gutenberg Literary Archive Foundation
and how your efforts and donations can help, see Sections 3 and 4
and the Foundation web page at http://www.pglaf.org.


Section 3.  Information about the Project Gutenberg Literary Archive
Foundation

The Project Gutenberg Literary Archive Foundation is a non profit
501(c)(3) educational corporation organized under the laws of the
state of Mississippi and granted tax exempt status by the Internal
Revenue Service.  The Foundation's EIN or federal tax identification
number is 64-6221541.  Its 501(c)(3) letter is posted at
http://pglaf.org/fundraising.  Contributions to the Project Gutenberg
Literary Archive Foundation are tax deductible to the full extent
permitted by U.S. federal laws and your state's laws.

The Foundation's principal office is located at 4557 Melan Dr. S.
Fairbanks, AK, 99712., but its volunteers and employees are scattered
throughout numerous locations.  Its business office is located at
809 North 1500 West, Salt Lake City, UT 84116, (801) 596-1887, email
business@pglaf.org.  Email contact links and up to date contact
information can be found at the Foundation's web site and official
page at http://pglaf.org

For additional contact information:
     Dr. Gregory B. Newby
     Chief Executive and Director
     gbnewby@pglaf.org


Section 4.  Information about Donations to the Project Gutenberg
Literary Archive Foundation

Project Gutenberg-tm depends upon and cannot survive without wide
spread public support and donations to carry out its mission of
increasing the number of public domain and licensed works that can be
freely distributed in machine readable form accessible by the widest
array of equipment including outdated equipment.  Many small donations
($1 to $5,000) are particularly important to maintaining tax exempt
status with the IRS.

The Foundation is committed to complying with the laws regulating
charities and charitable donations in all 50 states of the United
States.  Compliance requirements are not uniform and it takes a
considerable effort, much paperwork and many fees to meet and keep up
with these requirements.  We do not solicit donations in locations
where we have not received written confirmation of compliance.  To
SEND DONATIONS or determine the status of compliance for any
particular state visit http://pglaf.org

While we cannot and do not solicit contributions from states where we
have not met the solicitation requirements, we know of no prohibition
against accepting unsolicited donations from donors in such states who
approach us with offers to donate.

International donations are gratefully accepted, but we cannot make
any statements concerning tax treatment of donations received from
outside the United States.  U.S. laws alone swamp our small staff.

Please check the Project Gutenberg Web pages for current donation
methods and addresses.  Donations are accepted in a number of other
ways including including checks, online payments and credit card
donations.  To donate, please visit: http://pglaf.org/donate


Section 5.  General Information About Project Gutenberg-tm electronic
works.

Professor Michael S. Hart is the originator of the Project Gutenberg-tm
concept of a library of electronic works that could be freely shared
with anyone.  For thirty years, he produced and distributed Project
Gutenberg-tm eBooks with only a loose network of volunteer support.


Project Gutenberg-tm eBooks are often created from several printed
editions, all of which are confirmed as Public Domain in the U.S.
unless a copyright notice is included.  Thus, we do not necessarily
keep eBooks in compliance with any particular paper edition.


Most people start at our Web site which has the main PG search facility:

     http://www.gutenberg.net

This Web site includes information about Project Gutenberg-tm,
including how to make donations to the Project Gutenberg Literary
Archive Foundation, how to help produce our new eBooks, and how to
subscribe to our email newsletter to hear about new eBooks.
